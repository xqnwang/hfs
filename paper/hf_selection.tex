% Options for packages loaded elsewhere
\PassOptionsToPackage{unicode}{hyperref}
\PassOptionsToPackage{hyphens}{url}
\PassOptionsToPackage{dvipsnames,svgnames,x11names}{xcolor}
%
\documentclass[
  11pt]{article}

\usepackage{amsmath,amssymb}
\usepackage{iftex}
\ifPDFTeX
  \usepackage[T1]{fontenc}
  \usepackage[utf8]{inputenc}
  \usepackage{textcomp} % provide euro and other symbols
\else % if luatex or xetex
  \usepackage{unicode-math}
  \defaultfontfeatures{Scale=MatchLowercase}
  \defaultfontfeatures[\rmfamily]{Ligatures=TeX,Scale=1}
\fi
\usepackage{lmodern}
\ifPDFTeX\else  
    % xetex/luatex font selection
\fi
% Use upquote if available, for straight quotes in verbatim environments
\IfFileExists{upquote.sty}{\usepackage{upquote}}{}
\IfFileExists{microtype.sty}{% use microtype if available
  \usepackage[]{microtype}
  \UseMicrotypeSet[protrusion]{basicmath} % disable protrusion for tt fonts
}{}
\makeatletter
\@ifundefined{KOMAClassName}{% if non-KOMA class
  \IfFileExists{parskip.sty}{%
    \usepackage{parskip}
  }{% else
    \setlength{\parindent}{0pt}
    \setlength{\parskip}{6pt plus 2pt minus 1pt}}
}{% if KOMA class
  \KOMAoptions{parskip=half}}
\makeatother
\usepackage{xcolor}
\setlength{\emergencystretch}{3em} % prevent overfull lines
\setcounter{secnumdepth}{5}
% Make \paragraph and \subparagraph free-standing
\ifx\paragraph\undefined\else
  \let\oldparagraph\paragraph
  \renewcommand{\paragraph}[1]{\oldparagraph{#1}\mbox{}}
\fi
\ifx\subparagraph\undefined\else
  \let\oldsubparagraph\subparagraph
  \renewcommand{\subparagraph}[1]{\oldsubparagraph{#1}\mbox{}}
\fi


\providecommand{\tightlist}{%
  \setlength{\itemsep}{0pt}\setlength{\parskip}{0pt}}\usepackage{longtable,booktabs,array}
\usepackage{calc} % for calculating minipage widths
% Correct order of tables after \paragraph or \subparagraph
\usepackage{etoolbox}
\makeatletter
\patchcmd\longtable{\par}{\if@noskipsec\mbox{}\fi\par}{}{}
\makeatother
% Allow footnotes in longtable head/foot
\IfFileExists{footnotehyper.sty}{\usepackage{footnotehyper}}{\usepackage{footnote}}
\makesavenoteenv{longtable}
\usepackage{graphicx}
\makeatletter
\def\maxwidth{\ifdim\Gin@nat@width>\linewidth\linewidth\else\Gin@nat@width\fi}
\def\maxheight{\ifdim\Gin@nat@height>\textheight\textheight\else\Gin@nat@height\fi}
\makeatother
% Scale images if necessary, so that they will not overflow the page
% margins by default, and it is still possible to overwrite the defaults
% using explicit options in \includegraphics[width, height, ...]{}
\setkeys{Gin}{width=\maxwidth,height=\maxheight,keepaspectratio}
% Set default figure placement to htbp
\makeatletter
\def\fps@figure{htbp}
\makeatother

\addtolength{\oddsidemargin}{-.5in}%
\addtolength{\evensidemargin}{-1in}%
\addtolength{\textwidth}{1in}%
\addtolength{\textheight}{1.7in}%
\addtolength{\topmargin}{-1in}%
\usepackage{todonotes}
\usepackage{bm}
\usepackage{mathptmx}
\usepackage{amsmath} % Ensure amsmath is included for \eqref
\usepackage[]{natbib}
\usepackage[margin = 2.5cm]{geometry}
\allowdisplaybreaks
\sloppy
%\clubpenalty = 10000
%\widowpenalty = 10000
%\brokenpenalty = 10000
\setlength{\parskip}{0pt}
\setlength{\parindent}{15pt}
\expandafter\def\expandafter\normalsize\expandafter{%
  \normalsize  
  \setlength\abovedisplayskip{1ex}
  \setlength\belowdisplayskip{1ex}
  \setlength\abovedisplayshortskip{1ex}
  \setlength\belowdisplayshortskip{1ex}
}
\setlength{\bibsep}{0pt plus 0.3ex}
\usepackage{microtype}
\usepackage[onlyrefs]{mathtools}
\setlength{\tabcolsep}{0.11cm}
\usepackage{threeparttable}
\makeatletter
\g@addto@macro\TPT@defaults{\linespread{1}\selectfont}
\def\fps@table{!tbh}
\def\fps@figure{!tbh}
\makeatother

%% CAPTIONS
\usepackage{setspace}
\usepackage{caption}
\DeclareCaptionStyle{italic}[justification=centering]
 {labelfont={bf},textfont={it},labelsep=colon}
\captionsetup[figure]{style=italic,format=hang,singlelinecheck=true,font=singlespacing}
\captionsetup[table]{style=italic,format=hang,singlelinecheck=true,font=singlespacing}
\usepackage{booktabs}
\usepackage{longtable}
\usepackage{array}
\usepackage{multirow}
\usepackage{wrapfig}
\usepackage{float}
\usepackage{colortbl}
\usepackage{pdflscape}
\usepackage{tabu}
\usepackage{threeparttable}
\usepackage{threeparttablex}
\usepackage[normalem]{ulem}
\usepackage{makecell}
\usepackage{xcolor}
\makeatletter
\@ifpackageloaded{caption}{}{\usepackage{caption}}
\AtBeginDocument{%
\ifdefined\contentsname
  \renewcommand*\contentsname{Table of contents}
\else
  \newcommand\contentsname{Table of contents}
\fi
\ifdefined\listfigurename
  \renewcommand*\listfigurename{List of Figures}
\else
  \newcommand\listfigurename{List of Figures}
\fi
\ifdefined\listtablename
  \renewcommand*\listtablename{List of Tables}
\else
  \newcommand\listtablename{List of Tables}
\fi
\ifdefined\figurename
  \renewcommand*\figurename{Figure}
\else
  \newcommand\figurename{Figure}
\fi
\ifdefined\tablename
  \renewcommand*\tablename{Table}
\else
  \newcommand\tablename{Table}
\fi
}
\@ifpackageloaded{float}{}{\usepackage{float}}
\floatstyle{ruled}
\@ifundefined{c@chapter}{\newfloat{codelisting}{h}{lop}}{\newfloat{codelisting}{h}{lop}[chapter]}
\floatname{codelisting}{Listing}
\newcommand*\listoflistings{\listof{codelisting}{List of Listings}}
\usepackage{amsthm}
\theoremstyle{plain}
\newtheorem{proposition}{Proposition}[section]
\theoremstyle{remark}
\AtBeginDocument{\renewcommand*{\proofname}{Proof}}
\newtheorem*{remark}{Remark}
\newtheorem*{solution}{Solution}
\newtheorem{refremark}{Remark}[section]
\newtheorem{refsolution}{Solution}[section]
\makeatother
\makeatletter
\makeatother
\makeatletter
\@ifpackageloaded{caption}{}{\usepackage{caption}}
\@ifpackageloaded{subcaption}{}{\usepackage{subcaption}}
\makeatother
\ifLuaTeX
  \usepackage{selnolig}  % disable illegal ligatures
\fi
\usepackage[]{natbib}
\bibliographystyle{agsm}
\usepackage{bookmark}

\IfFileExists{xurl.sty}{\usepackage{xurl}}{} % add URL line breaks if available
\urlstyle{same} % disable monospaced font for URLs
\hypersetup{
  pdftitle={Optimal forecast reconciliation with time series selection},
  pdfauthor={Xiaoqian Wang; Rob J Hyndman; Shanika L Wickramasuriya},
  pdfkeywords={Forecasting, Hierarchical time series, Linear forecast
reconciliation, Variable selection, Integer programming},
  colorlinks=true,
  linkcolor={blue},
  filecolor={Maroon},
  citecolor={Blue},
  urlcolor={Blue},
  pdfcreator={LaTeX via pandoc}}


\begin{document}


\def\spacingset#1{\renewcommand{\baselinestretch}%
{#1}\small\normalsize} \spacingset{1}

\renewcommand*{\arraystretch}{0.5} % Specify row height in a table globally

%%%%%%%%%%%%%%%%%%%%%%%%%%%%%%%%%%%%%%%%%%%%%%%%%%%%%%%%%%%%%%%%%%%%%%%%%%%%%%

\date{July 24, 2024}
\title{\bf Optimal forecast reconciliation with time series selection}
\author{
Xiaoqian Wang\thanks{Corresponding author.} \vspace{0.2em}\\
Department of Econometrics \& Business Statistics, Monash
University \vspace{0.2em}\\
and \vspace{0.2em}\\Rob J Hyndman \vspace{0.2em}\\
Department of Econometrics \& Business Statistics, Monash
University \vspace{0.2em}\\
and \vspace{0.2em}\\Shanika L Wickramasuriya \vspace{0.2em}\\
Department of Econometrics \& Business Statistics, Monash
University \vspace{0.2em}\\
}
\maketitle

\bigskip
\bigskip
\begin{abstract}
Forecast reconciliation ensures forecasts of time series in a hierarchy
adhere to aggregation constraints, enabling aligned decision making.
While forecast reconciliation can enhance overall accuracy in
hierarchical or grouped structures, the most substantial improvements
occur in series with initially poor-performing base forecasts.
Nevertheless, certain series may experience deteriorations in reconciled
forecasts. In practical settings, series in a structure often exhibit
poor base forecasts due to model misspecification or low
forecastability. To prevent their negative impact, we propose two
categories of forecast reconciliation methods that incorporate time
series selection based on out-of-sample and in-sample information,
respectively. These methods keep ``poor'' base forecasts unused in
forming reconciled forecasts, while adjusting weights allocated to the
remaining series accordingly when generating bottom-level reconciled
forecasts. Additionally, our methods ameliorate disparities stemming
from varied estimates of the base forecast error covariance matrix,
alleviating challenges associated with estimator selection. Empirical
evaluations through two simulation studies and applications using
Australian labour force and domestic tourism data demonstrate improved
forecast accuracy, particularly evident in higher aggregation levels,
longer forecast horizons, and cases involving model misspecification.
\end{abstract}

\noindent%
{\it Keywords:} Forecasting, Hierarchical time series, Linear forecast
reconciliation, Variable selection, Integer programming
\vfill

\newpage
%\spacingset{1.8} % DON'T change the spacing!
\setstretch{1.5}
\section{Introduction}\label{sec-introduction}

Forecast reconciliation is a post-processing method that ensures
forecasts of multivariate time series adhere to known linear constraints
\citep{Hyndman2011-sd}. For example, the sum of regional unemployment
forecasts should be equal to the national unemployment forecast.

\citet{Hyndman2011-sd} introduced optimal forecast reconciliation,
whereby ``base'' forecasts of all series are generated independently,
and then adjusted to satisfy the constraints, leading to a set of
coherent reconciled forecasts. Subsequent research has extended and
developed the idea in the context of cross-sectional data
\citep{Hyndman2016-cz, Wickramasuriya2019-fc, Panagiotelis2021-mf},
temporal data \citep{Athanasopoulos2017-jj}, and cross-temporal data
\citep{Di_Fonzo2023-vo}. \citet{Athanasopoulos2023-sm} provided a
comprehensive introduction to the forecast reconciliation literature.

Reconciliation is known to improve overall forecast accuracy in
collections of time series with aggregation constraints. On average,
when the base forecasts are unbiased, the mean squared reconciled
forecast error from the minimum trace reconciliation method
\citep{Wickramasuriya2019-fc} is lower than that from the base forecasts
\citep{Wickramasuriya2021-am}. Most of the improvements attributed to
reconciliation are observed in series with initially poor-performing
base forecasts \citep{Athanasopoulos2017-jj}. In practice, it is not
uncommon for some series to have poor base forecasts due to challenges
such as model misspecification or low signal-to-noise ratio (SNR). In
such cases, it may be advantageous to exclude the worst base forecasts
when performing reconciliation. This is the motivation for our proposed
methods.

First, we propose forecast reconciliation methods that incorporate time
series selection based on out-of-sample information, assuming unbiased
base forecasts. We formulate this as an optimization problem, using
diverse penalty functions to control the number of nonzero column
entries in the weighting matrix for linear forecast reconciliation. We
show that the number of selected time series is at least equal to the
number of series at the bottom level, and we can reconstruct the entire
structure by aggregating/disaggregating the selected series. Second, we
relax the unbiasedness assumption and introduce an additional
reconciliation method with selection, utilizing in-sample observations
and their fitted values. This enables us to use the in-sample
reconciliation performance for selection purposes. In this case, it is
possible that fewer than the number of series at the bottom level are
used for reconciliation. In an extreme scenario, the solution may
resemble the traditional top-down approach. Through simulation
experiments and two empirical applications, we demonstrate that our
proposed methods guarantee coherent forecasts that outperform or match
their respective benchmark methods.
\todo[inline]{Reconsider the wording.} The improvements are particularly
pronounced when focusing on higher aggregation levels, longer forecast
horizons, and cases of model misspecification. A remarkable feature of
the proposed methods is their ability to diminish disparities arising
from using different estimates of the base forecast error covariance
matrix, thereby mitigating challenges associated with estimator
selection, which is a prominent concern in the field of forecast
reconciliation research.
\todo[inline]{What about other concerns in forecast reconciliation research? It would be good to summarize these and address areas where the proposed methodology could support such concerns.}

The remainder of the paper is structured as follows.
Section~\ref{sec-preliminaries} presents the notations and a review of
linear forecast reconciliation methods. Section~\ref{sec-methodology}
introduces our proposed methods to achieve time series selection in
reconciliation, and provides some theoretical insights.
Section~\ref{sec-simulations} and Section~\ref{sec-applications} show
the results from simulations and two real-world datasets, respectively,
followed by concluding remarks in Section~\ref{sec-conclusion}. The R
code for reproducing the results is available at
\url{https://github.com/xqnwang/hfs}.

\section{Preliminaries}\label{sec-preliminaries}

\subsection{Notation}\label{notation}

A \emph{hierarchical time series} is an \(n\)-dimensional multivariate
time series that adheres to known linear constraints. Let
\(\bm{y}_t \in \mathbb{R}^n\) be a vector comprising observations from
all time series in the hierarchy at time \(t\), and
\(\bm{b}_t \in \mathbb{R}^{n_b}\) be a vector comprising observations of
only the most disaggregated (``bottom-level'') time series at time
\(t\). The full hierarchy at time \(t\) can be written as \[
\bm{y}_t = \bm{S}\bm{b}_t,
\] for \(t=1,2,\ldots,T\), where \(T\) is the length of the time series,
and \(\bm{S}\) is an \(n \times n_b\) \emph{summing matrix} that defines
the aggregation constraints. We can write the summing matrix as
\(\bm{S} = \left[\begin{array}{c}\bm{A} \\ \bm{I}_{n_b}\end{array}\right]\),
where \(\bm{A}\) is an \(n_a \times n_b\) \emph{aggregation matrix} with
\(n = n_a + n_b\), and \(\bm{I}_{n_b}\) is an \(n_b\)-dimensional
identity matrix.

\begin{figure}[!t]

\centering{

\includegraphics[width=0.35\textwidth,height=\textheight]{figs/hts_example.pdf}

}

\caption{\label{fig-hts}An example of a two-level hierarchical time
series.}

\end{figure}%

For example, Figure~\ref{fig-hts} shows a simple hierarchy with
\(n = 7\), \(n_b = 4\), \(n_a = 3\),
\(\bm{y}_t = [y_{\text{Total},t}, y_{\text{A},t}, y_{\text{B},t}, y_{\text{AA},t}, y_{\text{AB},t}, y_{\text{BA},t}, y_{\text{BB},t}]^{\prime}\),
\(\bm{b}_t = [y_{\text{AA},t}, y_{\text{AB},t}, y_{\text{BA},t}, y_{\text{BB},t}]^{\prime}\),
and \[
\bm{S} = \left[
\begin{array}{cccc}
1 & 1 & 1 & 1 \\
1 & 1 & 0 & 0 \\
0 & 0 & 1 & 1 \\
\multicolumn{4}{c}{\bm{I}_4}
\end{array}\right].
\] The notation is general enough to include aggregation constraints
that are non-hierarchical. Please refer to \citet{Hyndman2021-fo} for
further details.

Hierarchical forecasting methods have been extensively applied across
diverse domains. For instance, forecast reconciliation is widely
implemented in tourism data \citep{Athanasopoulos2009-ps}, where
hierarchical time series arise due to geographic divisions. Total
overnight trips for a whole nation can be disaggregated to states, and
further subdivided into regions. In the context of a grocery retailer,
the total sales of the ``food'' category can be subdivided into various
subcategories and subsequently into distinct items
\citep{Zhang2023-op, Hollyman2021-un}. In electricity load forecasting,
consumption is measured using smart meters which naturally fall within a
comprehensive geographic hierarchy \citep{Taieb2021-tc}. For additional
interesting application examples, please refer to
\citet{Athanasopoulos2023-sm}.

\subsection{Linear forecast
reconciliation}\label{linear-forecast-reconciliation}

Let \(\hat{\bm{y}}_{T+h \mid T} \in \mathbb{R}^n\) be a vector of
\(h\)-step-ahead \emph{base forecasts} for all time series in the
structure, given observations up to and including time \(T\), and
stacked in the same order as \(\bm{y}_t\). We can use any method to
generate these forecasts, but in general they will not be coherent
(i.e., they won't satisfy the aggregation constraints). Let
\(\tilde{\bm{y}}_{T+h \mid T} \in \mathbb{R}^n\) denote a vector of
\(h\)-step-ahead \emph{reconciled forecasts} given by
\begin{equation}\phantomsection\label{eq-lr}{
\tilde{\bm{y}}_{T+h \mid T} = \bm{S}\bm{G}_h\hat{\bm{y}}_{T+h \mid T},
}\end{equation} where \(\bm{G}_h\) is an \(n_b \times n\)
\emph{weighting matrix} and \(\bm{S}\) is an \(n \times n_b\)
\emph{summing matrix}.

In general, forecast reconciliation methods consider the loss function
given by \begin{equation}\phantomsection\label{eq-loss}{
\begin{aligned}
& \mathrm{E}\left[\left\|\bm{y}_{T+h}-\tilde{\bm{y}}_{T+h \mid T}\right\|_2^2 \mid \bm{I}_T\right] \\
& =\underbrace{\left\|\bm{S G_h}\left(\mathrm{E}\left[\hat{\bm{y}}_{T+h \mid T} \mid \bm{I}_T\right]-\mathrm{E}\left[\bm{y}_{T+h} \mid \bm{I}_T\right]\right)+(\bm{S}-\bm{S G_h S}) \mathrm{E}\left[\bm{b}_{T+h} \mid \bm{I}_T\right]\right\|_2^2}_{\text{bias}} +\underbrace{\operatorname{Tr}\left(\operatorname{Var}\left[\bm{y}_{T+h}-\tilde{\bm{y}}_{T+h \mid T} \mid \bm{I}_T\right]\right)}_{\text{variance}},
\end{aligned}
}\end{equation} which includes two parts in its decomposition, bias and
variance of the reconciled forecasts \(\tilde{\bm{y}}_{T+h \mid T}\)
\citep{Ben_Taieb2019-be}.

\subsubsection*{Minimum trace
reconciliation}\label{minimum-trace-reconciliation}
\addcontentsline{toc}{subsubsection}{Minimum trace reconciliation}

Let
\(\hat{\bm{e}}_{t+h \mid t} = \bm{y}_{t+h} - \hat{\bm{y}}_{t+h \mid t}\)
denote the \(h\)-step-ahead in-sample \emph{base forecast errors}, and
\(\tilde{\bm{e}}_{t+h \mid t} = \bm{y}_{t+h} - \tilde{\bm{y}}_{t+h \mid t}\)
denote the \(h\)-step-ahead \emph{reconciled forecast errors}, for
\(t = 1,2,\ldots,T-h\). Assuming the base forecasts are unbiased and
imposing the constraint \(\bm{G_h S}=\bm{I_{n_b}}\) to preserve the
unbiasedness of the reconciled forecasts, \citet{Wickramasuriya2019-fc}
eliminated out the bias term in Equation \eqref{eq-loss} and formulated
the reconciliation problem as minimizing the trace (MinT) of the
\(h\)-step-ahead covariance matrix of the reconciled forecast errors,
\(\operatorname{Var}(\tilde{\bm{e}}_{T+h \mid T})\). This leads to the
unique solution given by \begin{equation}\phantomsection\label{eq-mint}{
\bm{G}_h=\left(\bm{S}^{\prime} \bm{W}_h^{-1} \bm{S}\right)^{-1} \bm{S}^{\prime} \bm{W}_h^{-1},
}\end{equation} where \(\bm{W}_h\) is the positive definite covariance
matrix of the \(h\)-step-ahead base forecast errors.

The MinT problem can be reformulated as a least squares problem with
linear constraints: \begin{equation}\phantomsection\label{eq-mint_op}{
\min_{\tilde{\bm{y}}_{T+h \mid T}} \quad \frac{1}{2}(\hat{\bm{y}}_{T+h \mid T}-\tilde{\bm{y}}_{T+h \mid T})^{\prime} \bm{W}_{h}^{-1}(\hat{\bm{y}}_{T+h \mid T}-\tilde{\bm{y}}_{T+h \mid T})
 \qquad \text{s.t.} \quad \tilde{\bm{y}}_{T+h \mid T}=\bm{S}\tilde{\bm{b}}_{T+h \mid T},
}\end{equation} where
\(\tilde{\bm{b}}_{T+h \mid T} \in \mathbb{R}^{n_b}\) comprises the
\(h\)-step-ahead bottom-level reconciled forecasts, made at time \(T\).
The intuition behind MinT reconciliation is that \textbf{the larger the
estimated variance of the base forecast errors, the larger the range of
adjustments permitted for forecast reconciliation}.

It is challenging to estimate \(\bm{W}_h\), especially for \(h > 1\). It
is common to assume \(\bm{W}_h = k_h\bm{W}_1\), \(\forall h\), where
\(k_h > 0\); then the MinT solution of \(\bm{G}\) does not change with
the forecast horizon, \(h\). Hence, we will drop the subscript \(h\) for
ease of exposition. Table~\ref{tbl-bench} lists the most popularly used
candidate estimators for \(\bm{W}_h\). In principle, all optimization
methods could use either an in-sample or out-of-sample approach. The
methods discussed here are considered ``out-of-sample'' as they use
genuine forecasts, \(\hat{\bm{y}}_{T+h \mid T}\), rather than fitted
values, in the optimization problem.

\begin{table}[!h]

\caption{\label{tbl-bench}Forecast reconciliation methods for which
different estimators of \(\bm{W}_h\) are used.}

\centering{

\centering
\begin{threeparttable}
\begin{tabular}{p{0.7\linewidth}r}
\toprule
Reconciliation method & $\bm{W}_h\propto$\\
\midrule
\textbf{OLS} \citep{Hyndman2011-sd} & $\bm{I}$\\
\textbf{WLSs} \citep{Athanasopoulos2017-jj} & $\operatorname{diag}(\bm{S} \bm{1})$\\
\textbf{WLSv} \citep{Hyndman2016-cz} & $\operatorname{Diag}(\hat{\bm{W}}_1)$\\
\textbf{MinT} \citep{Wickramasuriya2019-fc} & $\hat{\bm{W}}_1$\\
\textbf{MinTs} \citep{Wickramasuriya2019-fc} & $\lambda\operatorname{Diag}(\hat{\bm{W}}_1) + (1-\lambda)\hat{\bm{W}}_1$\\
\bottomrule
\end{tabular}
\begin{tablenotes}
\item Note: $\bm{1}$ is a vector of 1s of size $n_b$, $\operatorname{diag}(\cdot)$ constructs a diagonal matrix using a given vector, $\hat{\bm{W}}_1$ denotes the unbiased covariance estimator based on the in-sample one-step-ahead base forecast errors (i.e., residuals), and $\operatorname{Diag}(\cdot)$ forms a diagonal matrix using the diagonal elements of the input matrix.
\end{tablenotes}
\end{threeparttable}

}

\end{table}%

\subsubsection*{Relaxation of the unbiasedness
assumptions}\label{relaxation-of-the-unbiasedness-assumptions}
\addcontentsline{toc}{subsubsection}{Relaxation of the unbiasedness
assumptions}

\citet{Ben_Taieb2019-be} proposed a reconciliation method relaxing the
assumption of unbiasedness. Their goal was to achieve a tradeoff between
bias and variance by directly minimizing the mean squared reconciled
forecast errors, as described in Equation \eqref{eq-loss}. Specifically,
by expanding the training window incrementally, one observation at a
time, they formulated the reconciliation problem as a regularized
empirical risk minimization (RERM) problem given by \[
\min_{\bm{G}_h} \frac{1}{(T-T_1-h+1)n}\left\|\bm{Y}_{h}^{*}-\hat{\bm{Y}}_{h}^{*} \bm{G}_{h}^{\prime} \bm{S}^{\prime}\right\|_F^2+\lambda\|\operatorname{vec}( \bm{G}_h)\|_1,
\] where \(T_1\) denotes the minimum number of observations used for
model training, \(\left\| \cdot \right\|_F\) is the Frobenius norm,
\(\|\cdot\|_1\) is the \(L_1\) norm, \(\operatorname{vec}(\cdot)\)
denotes the vectorization of a matrix (stacking the columns of the
matrix),
\(\bm{Y}_{h}^{*}=\left[\bm{y}_{T_1+h}, \ldots, \bm{y}_T\right]^{\prime}\),
\(\hat{\bm{Y}}_{h}^{*}=\left[\hat{\bm{y}}_{T_1+h \mid T_1}, \ldots, \hat{\bm{y}}_{T \mid T-h}\right]^{\prime}\),
and \(\lambda \geq 0\) is a regularization parameter.

When \(\lambda = 0\), the problem reduces to an empirical risk
minimization (ERM) problem without regularization. Assuming that the
series in the structure are jointly weakly stationary and
\(\hat{\bm{Y}}_{h}^{*\prime}\hat{\bm{Y}}_{h}^{*}\) is invertible, it has
a closed-form solution given by \[
\hat{\bm{G}}_h = \bm{B}_{h}^{*\prime}\hat{\bm{Y}}_{h}^{*}\left(\hat{\bm{Y}}_{h}^{*\prime}\hat{\bm{Y}}_{h}^{*}\right)^{-1},
\] where
\(\bm{B}_{h}^{*}=\left[\bm{b}_{T_1+h}, \ldots, \bm{b}_T\right]^{\prime}\).
If \(\hat{\bm{Y}}_{h}^{*\prime}\hat{\bm{Y}}_{h}^{*}\) is not invertible,
\citet{Ben_Taieb2019-be} suggested using a generalized inverse. When
\(\lambda > 0\), imposing the \(L_1\) penalty on \(\bm{G}_h\) will
introduce sparsity and reduce estimation variance, albeit at the cost of
introducing some bias.

Relaxing the assumption of unbiasedness of base forecasts,
\citet{Wickramasuriya2021-am} proposed an empirical MinT
(\textbf{EMinT}) solution by minimizing the trace of the covariance
matrix of the reconciled forecast errors,
\(\operatorname{Var}(\tilde{\bm{e}}_{T+h \mid T})\). Assuming the series
are jointly weakly stationary, the solution is given by \[
\hat{\bm{G}}_{h} = \bm{B}_{h}^{\prime}\hat{\bm{Y}}_{h}\left(\hat{\bm{Y}}_{h}^{\prime}\hat{\bm{Y}}_{h}\right)^{-1},
\] where
\(\bm{B}_{h}=\left[\bm{b}_{h}, \ldots, \bm{b}_T\right]^{\prime}\), and
\(\hat{\bm{Y}}_{h}=\left[\hat{\bm{y}}_{h \mid 0}, \ldots, \hat{\bm{y}}_{T \mid T-h}\right]^{\prime}\).

The difference between EMinT and ERM lies in the data sources. EMinT is
an ``in-sample'' method in the sense that \(\hat{\bm{Y}}_{h}\) are
predictions in the form of fitted values, while ERM (and also RERM) is
an ``out-of-sample'' method, with \(\hat{\bm{Y}}_{h}^{*}\) being genuine
forecasts generated on a holdout validation set. Both EMinT and ERM
consider an estimate of \(\bm{G}\) that changes over the forecast
horizon, which is why we keep the subscript \(h\) here.

In practical settings, some series in a structure could have poor base
forecasts due to model misspecification or low forecastability.
Specifically, within a hierarchical structure, the influence of
unforeseen events may prompt a forecaster to make a bad decision,
leading to the use of a misspecified forecasting model for a specific
time series and, consequently, yielding inferior forecasts. Lower-level
time series are normally characterized by less apparent trend and
seasonality, large intermittence, and volatility, rendering them more
challenging to predict and resulting in poor forecasts.

A challenge in forecast reconciliation arises when some base forecasts
perform poorly, as the role of the weighting matrix \(\bm{G}\) is to
assimilate \emph{all} base forecasts and map them into bottom-level
disaggregated forecasts, which are subsequently summed by \(\bm{S}\).
While the RERM method proposed by \citet{Ben_Taieb2019-be} introduces
sparsity by shrinking some elements of \(\bm{G}\) towards zero, it
remains incapable of mitigating the adverse impact of underperforming
base forecasts. Moreover, the method is time-consuming because it uses
expanding windows to recursively generate out-of-sample base forecasts.

In addition to \citet{Ben_Taieb2019-be}, several other contributions
have incorporated diverse forms of shrinkage or penalization in forecast
reconciliation methodologies. For example, \citet{Pang2022-hi}
introduced a group Lasso penalty on weights assigned to clusters
artificially added in a hierarchy to select ideal clusters. Their
objective function focuses on a new hierarchical structure encompassing
geographic and data cluster hierarchies, while disregarding forecast
errors associated with zero-weighted clusters. Furthermore, they derive
the optimal weight vector and optimal bottom level forecasts by solving
the objective successively, leading to a time-consuming method that does
not permanently mitigate the negative impact of poorly performing
clusters on reconciliation performance. To address the insufficient
emphasis on coherence in machine learning methods,
\citet{Mishchenko2019-as} and \citet{Gleason2020-fo} included a
regularization term to penalize forecast incoherence. However, these
soft constraints do not ensure coherence. \citet{Nystrup2020-te} and
\citet{Nystrup2021-di} considered the autocorrelation in forecast errors
and used a shrinkage estimator or eigendecomposition of the
cross-correlation matrix, effectively overcoming estimation
inefficiencies in approximating \(\bm{W}\) within a temporal hierarchy.
Nonetheless, none of the aforementioned contributions achieve time
series selection in forecast reconciliation, failing to alleviate their
adverse impact on forecast performance, while maintaining consideration
for forecast errors across the entire initial hierarchy.

We therefore propose two new forecast reconciliation methods involving
time series selection: constrained out-of-sample (under the unbiasedness
assumption) and unconstrained in-sample (without the unbiasedness
assumption). These methods aim to address the negative effect of some
poor base forecasts on the overall performance of the reconciled
forecasts. Additionally, through the incorporation of regularization in
the objective function, our method improves reconciliation outcomes
produced with a ``poor'' choice of \(\bm{W}\).

\section{Forecast reconciliation with time series
selection}\label{sec-methodology}

In this section, we introduce our methods for forecast reconciliation
while automatically achieving time series selection.
Section~\ref{sec-constrained} introduces constrained ``out-of-sample''
reconciliation methods, formulated based on genuine forecasts, while
Section~\ref{sec-unconstrained} presents an unconstrained ``in-sample''
reconciliation method, where the problem is formulated using in-sample
observations and predictions in the form of fitted values.

\subsection{Series selection under the unbiasedness
assumption}\label{sec-constrained}

As \(\bm{S}\) is fixed and \(\hat{\bm{y}}_{T+h \mid T}\) is given,
\(\bm{G}_h\) determines the linear reconciliation performance, as shown
in Equation \eqref{eq-lr}. We drop the subscript \(h\) here as we assume
\(\bm{W}\) and \(\bm{G}\) do not vary with the forecast horizon. A
natural way to remove forecasts of some series is by controlling the
number of nonzero column entries in \(\bm{G}\). This leads to a
generalization of the MinT optimization problem with an additional
penalty term: \begin{equation}\phantomsection\label{eq-op_u}{
\min_{\bm{G}} \quad \frac{1}{2}\left(\hat{\bm{y}}-\bm{SG}\hat{\bm{y}}\right)^{\prime} \bm{W}^{-1}\left(\hat{\bm{y}}-\bm{SG}\hat{\bm{y}}\right)
+ \lambda\mathfrak{g}(\bm{G}) \qquad \text{s.t.} \quad \bm{G}\bm{S}=\bm{I},
}\end{equation} where \(\hat{\bm{y}}:=\hat{\bm{y}}_{T+1 \mid T}\),
\(\mathfrak{g}(\cdot)\) penalizes the columns of \(\bm{G}\) towards
zero, and \(\lambda\) is a penalty parameter. The methods developed
within this framework are ``out‑of‑sample'' in the sense that
\(\hat{\bm{y}}\) are genuine one-step-ahead forecasts. This can be
considered \emph{a grouped variable selection problem}, with each group
corresponding to a column of \(\bm{G}\). The constraint,
\(\bm{G}\bm{S}=\bm{I}\), ensures that the reconciled forecasts are
unbiased if the base forecasts are unbiased. When \(\lambda = 0\), the
problem reduces to the MinT optimization problem in Equation
\eqref{eq-mint_op} with a closed-form solution given by Equation
\eqref{eq-mint}.

\begin{proposition}[]\protect\hypertarget{prp-1}{}\label{prp-1}

If the assumption that forecast reconciliation preserves unbiasedness is
imposed by enforcing \(\bm{GS}=\bm{I}\), then the number of nonzero
column entries of \(\hat{\bm{G}}\) (the solution to Equation
\eqref{eq-op_u}) will be no less than \(n_b\). Moreover, the constraint
\(\bm{GS}=\bm{I}\) enforces that the selected columns of
\(\hat{\bm{G}}\) will correspond to variables that can ``restore'' the
hierarchy.

\end{proposition}

\begin{proof}
Let \(\bm{X}_{\cdot \mathbb{S}} \in \mathbb{R}^{r \times |\mathbb{S}|}\)
denote the submatrix of the \(r \times c\) matrix \(\bm{X}\) with the
columns indexed by the set \(\mathbb{S}\), where \(|\mathbb{S}|\) is the
cardinality of the set \(\mathbb{S}\). Similarly, let
\(\bm{X}_{\mathbb{S}\cdot} \in \mathbb{R}^{|\mathbb{S}| \times c}\)
denote the submatrix of \(\bm{X}\) with the rows indexed by
\(\mathbb{S}\). If \(\mathbb{S}\) is the set of indices of nonzero
columns in the solution \(\hat{\bm{G}}\) to Equation \eqref{eq-op_u},
then the following equations hold: \[
\hat{\bm{G}}\bm{S} = \hat{\bm{G}}_{\cdot \mathbb{S}}\bm{S}_{\mathbb{S}\cdot} = \bm{I}_{n_b},
\qquad\text{and}\qquad
\min \left(\operatorname{rank}(\hat{\bm{G}}_{\cdot \mathbb{S}}), \operatorname{rank}(\bm{S}_{\mathbb{S}\cdot})\right) \geq \operatorname{rank}(\bm{I}_{n_b})=n_b.
\] This indicates that the number of nonzero columns of \(\hat{\bm{G}}\)
should be no less than \(n_b\), i.e., \(|\mathbb{S}| \geq n_b\).

Moreover, we have
\(\operatorname{rank}(\bm{S}_{\mathbb{S}\cdot}) = n_b\) because
\(\operatorname{rank}(\bm{S}_{\mathbb{S}\cdot}) \leq n_b\), given that
\(\bm{S}\) has \(n_b\) columns. If the solution to Equation
\eqref{eq-op_u} yields a \(\hat{\bm{G}}\) with exactly \(n_b\) nonzero
columns (i.e., \(|\mathbb{S}|=n_b\)), then \(\bm{S}_{\mathbb{S}\cdot}\)
is a full rank square matrix and thus invertible. Applying Theorem 2 in
\citet{Zhang2023-op}, \(\bm{y}_{\mathbb{S}\cdot}\) is valid for
constructing the full hierarchy using nothing but the information
embedded in the aggregation constraints. If the solution yields a
\(\hat{\bm{G}}\) with more than \(n_b\) nonzero columns, we should be
able to identify more than one subset
\(\mathbb{S}^* \subset \mathbb{S}\) with \(|\mathbb{S}^*|=n_b\) to
construct an invertible square matrix \(\bm{S}_{\mathbb{S}^{*}\cdot}\)
and thereby restore the full hierarchy using the valid
\(\bm{y}_{\mathbb{S}^{*}\cdot}\). Therefore, the constraint
\(\bm{GS}=\bm{I}\) ensures that the selected columns of \(\hat{\bm{G}}\)
correspond to variables that can restore the full hierarchy.
\end{proof}

For example, for the simple hierarchy shown in Figure~\ref{fig-hts}, the
selected columns of \(\hat{\bm{G}}\) will be at least \(n_b=4\). Our
constrained reconciliation methods might simultaneously zero out the
columns of \(\bm{G}\) corresponding to series AA and BA, but not to
series AA and AB.

\begin{proposition}[]\protect\hypertarget{prp-2}{}\label{prp-2}

The optimization problem in Equation \eqref{eq-op_u} can be reformulated
as a least squares problem with regularization and linear equality
constraint as follows:
\begin{equation}\phantomsection\label{eq-op_u_reg}{
\begin{aligned}
& \min_{\operatorname{vec}(\bm{G})} \quad \frac{1}{2}\left(\hat{\bm{y}}-\left(\hat{\bm{y}}^{\prime} \otimes \bm{S}\right) \operatorname{vec}(\bm{G})\right)^{\prime} \bm{W}^{-1}\left(\hat{\bm{y}}-\left(\hat{\bm{y}}^{\prime} \otimes \bm{S}\right) \operatorname{vec}(\bm{G})\right) + \lambda\mathfrak{g}\left(\operatorname{vec}(\bm{G})\right) \\
& \text{s.t.} \quad \left(\bm{S}^{\prime} \otimes \bm{I}_{n_b}\right) \operatorname{vec}(\bm{G})=\operatorname{vec}(\bm{I}_{n_b}),
\end{aligned}
}\end{equation} which is characterized as a high-dimensional problem in
which the number of features, denoted as \(p = n_b \times n\), is much
larger than the number of observations, \(n\).

\end{proposition}

\begin{proof}
We have\vspace*{-0.4cm}\enlargethispage{0.4cm} \[
\begin{aligned}
& \bm{SG}\hat{\bm{y}} = \operatorname{vec}\left(\bm{SG}\hat{\bm{y}}\right) = \left(\hat{\bm{y}}^{\prime} \otimes \bm{S}\right) \operatorname{vec}(\bm{G}), \\
& \operatorname{vec}\left(\bm{G}\bm{S}\right) = \operatorname{vec}\left(\bm{I}_{n_b}\bm{G}\bm{S}\right) = \left(\bm{S}^{\prime} \otimes \bm{I}_{n_b}\right) \operatorname{vec}(\bm{G}).
\end{aligned}
\] Substituting these into Equation \eqref{eq-op_u}, the previous
problem now takes the form of a regression problem with an additional
regularization term and an equality constraint on the coefficients, as
shown in Equation \eqref{eq-op_u_reg}.
\end{proof}

Next, we present three classes of regularizations that allow forecast
reconciliation with series selection, resulting in three optimization
problems: (i) group best-subset selection with ridge regularization,
(ii) intuitive method with \(L_0\) regularization, and (iii) group lasso
method.

\subsubsection*{Group best-subset selection with ridge
regularization}\label{group-best-subset-selection-with-ridge-regularization}
\addcontentsline{toc}{subsubsection}{Group best-subset selection with
ridge regularization}

In a high-dimensional context with \(p \gg n\), it is common to assume
that the true regression coefficient (i.e.,
\(\operatorname{vec}(\bm{G})\) in our problem) is sparse. We apply a
combination of \(L_0\) and \(L_2\) regularization to control the nonzero
column entries in \(\bm{G}\):
\begin{equation}\phantomsection\label{eq-subset}{ \begin{aligned}
\min_{\operatorname{vec}(\bm{G})} \quad & \frac{1}{2}\left(\hat{\bm{y}}-\left(\hat{\bm{y}}^{\prime} \otimes \bm{S}\right) \operatorname{vec}(\bm{G})\right)^{\prime} \bm{W}^{-1}\left(\hat{\bm{y}}-\left(\hat{\bm{y}}^{\prime} \otimes \bm{S}\right) \operatorname{vec}(\bm{G})\right) + \lambda_0 \sum_{j=1}^n 1\left(\bm{G}_{\cdot j} \neq \bm{0}\right) + \lambda_2 \left\|\operatorname{vec}\left(\bm{G}\right)\right\|_2^2 \\
\text{s.t.} \quad & \left(\bm{S}^{\prime} \otimes \bm{I}_{n_b}\right) \operatorname{vec}(\bm{G})=\operatorname{vec}(\bm{I}_{n_b}),
\end{aligned}
}\end{equation} where \(1(\cdot)\) is the indicator function,
\(\lambda_0 \geq 0\) controls the number of nonzero columns of
\(\bm{G}\), \(\lambda_2 \geq 0\) controls the strength of the ridge
regularization, and \(\|\cdot\|_2\) is the \(L_2\) norm. In a
hierarchical or grouped time series context,
\(\operatorname{vec}(\bm{G})\) has an inherent non-overlapping grouping
structure, wherein each group corresponds to a single column of
\(\bm{G}\), each of size \(n_b\). Hence, we call this reconciliation
method \emph{group best-subset selection with ridge regularization}. In
the results that follow, we label the \textbf{Subset} method differently
based on various \(\bm{W}\) estimators, referring to them as
\textbf{OLS-subset}, \textbf{WLSs-subset}, \textbf{WLSv-subset},
\textbf{MinT-subset}, and \textbf{MinTs-subset}, respectively.

The inclusion of the ridge term in Equation \eqref{eq-subset} is
motivated by earlier work on best-subset selection
\citep[e.g.,][]{Hazimeh2020-xd, Mazumder2022-hx}, which suggests that
additional ridge regularization helps mitigate the poor predictive
performance of best-subset selection method in low SNR regimes.

We present a Big-M based mixed integer programming (MIP) formulation for
the problem in Equation \eqref{eq-subset}:
\begin{equation}\phantomsection\label{eq-subset_mip}{
\begin{aligned}
\min_{\operatorname{vec}(\bm{G}), \bm{z}, \check{\bm{e}}, \bm{g}^{+}} & \frac{1}{2}\check{\bm{e}}^{\prime} \bm{W}^{-1}\check{\bm{e}} + \lambda_0 \sum_{j=1}^n z_j + \lambda_2 \bm{g}^{+\prime}\bm{g}^{+} \\
\text{s.t.} \quad & \left(\bm{S}^{\prime} \otimes \bm{I}_{n_b}\right) \operatorname{vec}(\bm{G})=\operatorname{vec}\left(\bm{I}_{n_b}\right) \\
& \hat{\bm{y}}-\left(\hat{\bm{y}}^{\prime} \otimes \bm{S}\right)\operatorname{vec}(\bm{G}) = \check{\bm{e}} \\
& \sum_{i=1}^{n_b} g_{i + (j-1) n_b}^{+} \leqslant \mathcal{M} z_j, \quad j \in[n] \\
& \bm{g}^{+} \geqslant \operatorname{vec}(\bm{G}) \\
& \bm{g}^{+} \geqslant-\operatorname{vec}(\bm{G}) \\
& z_j \in\{0,1\}, \quad j \in[n],
\end{aligned}
}\end{equation} where \(\mathcal{M}\) is a Big-M parameter (specified
a-priori) that is sufficiently large that the optimal solution to
Equation \eqref{eq-subset_mip}, \(\bm{g}^{+*}\), satisfies
\(\max_{j \in [n]}\sum_{i=1}^{n_b} g_{i + (j-1) n_b}^{+} \leqslant \mathcal{M}\).
The binary variable \(z_j=0\) implies that \(\bm{G}_{\cdot j}=\bm{0}\),
and \(z_j=1\) implies that
\(\sum_{i=1}^{n_b} g_{i + (j-1) n_b}^{+} \leqslant \mathcal{M}\). Such
Big-M formulations are commonly used in MIP problems to model relations
between discrete and continuous variables, and have been recently
explored in regression with \(L_0\) regularization
\citep{Bertsimas2016-ig}. The problem is a mixed integer quadratic
program (MIQP) that can be solved using commercial MIP solvers, e.g.,
Gurobi and CPLEX.

\textbf{Parameter tuning.} To avoid computationally expensive
cross-validation, we tune the parameters to minimize the sum of squared
reconciled forecast errors on the truncated training set, comprising
only the \(\max\{h, s\}\) observations closest to the forecast origin,
where \(s\) is the seasonal period for seasonal data and \(s=T\) for
non-seasonal data. Let
\(\lambda_{0}^{1} = \frac{1}{2}\left(\hat{\bm{y}}-\tilde{\bm{y}}^{\text{bench}}\right)^{\prime} \bm{W}^{-1}\left(\hat{\bm{y}}-\tilde{\bm{y}}^{\text{bench}}\right)\),
which captures the scale of the first term in the objective function,
where \(\tilde{\bm{y}}^{\text{bench}}\) is a vector of reconciled
forecasts obtained using Equation \eqref{eq-mint} with the same
estimator of \(\bm{W}\), and define
\(\lambda_{0}^{k} = 0.0001\lambda_{0}^{1}\). For the parameter
\(\lambda_0\), we consider a grid of \(k+1\) values,
\(\{\lambda_{0}^{1},\dots,\lambda_{0}^{k}, 0\}\), where
\(\lambda_{0}^{j} = \lambda_{0}^{1}\left(\lambda_{0}^{k} / \lambda_{0}^{1}\right)^{(j-1) / (k-1)}\)
for \(j \in [k]\). So \(\lambda_{0}^{1},\dots,\lambda_{0}^{k}\) is a
sequence decreasing on the log scale. We use a grid of six values for
the parameter \(\lambda_2\),
\(\{0, 10^{-2}, 10^{-1}, 10^{0}, 10^{1}, 10^{2}\}\). Thus, we tune over
a two-dimensional grid of \((k+1) \times 6\) values to find the optimal
combination of \(\lambda_0\) and \(\lambda_2\).

\textbf{Computation details.} The MIQP problem in Equation
\eqref{eq-subset_mip} is NP-hard and computationally intensive.
\citet{Bertsimas2016-ig} showed that commercial MIP solvers are capable
of tackling problem instances for \(p\) up to a thousand. To address
larger instances, there has been impressive work on developing MIP-based
approaches for solving \(L_0\)-regularized regression problem; e.g.,
\citet{Bertsimas2016-ig}, \citet{Hazimeh2020-xd}, and
\citet{Hazimeh2022-hc}. However, it is challenging to extend these
approaches to accommodate additional constraints in the optimization
problem. Despite potential challenges in handling large instances with
commercial MIP solvers, in our experiments, we use Gurobi to solve
Equation \eqref{eq-subset_mip} by configuring parameters such as MIPGap
= \(0.001\) and TimeLimit = \(600\) seconds for cases with \(p > 1000\).
This allows to terminate the solver before reaching the global optimum
and return a suboptimal solution instead. This strategy is motivated by
our need to consider numerous parameter candidates, and the final
solution will be validated against the training set, which prevents the
use of a poor estimate of \(\bm{G}\).

\subsubsection*{\texorpdfstring{Intuitive method with \(L_0\)
regularization}{Intuitive method with L\_0 regularization}}\label{intuitive-method-with-l_0-regularization}
\addcontentsline{toc}{subsubsection}{Intuitive method with \(L_0\)
regularization}

Instead of estimating the entire matrix \(\bm{G}\) as above, we leverage
the MinT solution in Equation \eqref{eq-mint} to streamline the
optimization problem under consideration. Specifically, we define
\(\bar{\bm{S}} = \bm{A}\bm{S}\), where
\(\bm{A} = \operatorname{diag}(\bm{z})\) is an \(n \times n\) diagonal
matrix, and \(\bm{z}\) is an \(n\)-dimensional vector with elements
either equal to 0 or 1. Taking the MinT solution in Equation
\eqref{eq-mint}, we have
\(\bar{\bm{G}} = (\bm{S}^{\prime}\bm{A}^{\prime}\bm{W}^{-1}\bm{A}\bm{S})^{-1}\bm{S}^{\prime}\bm{A}^{\prime}\bm{W}^{-1}\).
Given fixed \(\bm{S}\) and estimation of \(\bm{W}\), \(\bar{\bm{G}}\) is
entirely determined by \(\bm{A}\). Thus, when the \(j\)th diagonal
element of \(\bm{A}\) is zero, the \(j\)th column of \(\bar{\bm{G}}\)
becomes entirely composed of zeros. Therefore, the optimization problem
can be reduced to an integer quadratic programming (IQP) problem where
all of the variables are restricted to being integers: \begin{align*}
\min_{\bm{A}} \quad & \frac{1}{2}\left(\hat{\bm{y}}-\bm{S}\bar{\bm{G}}\hat{\bm{y}}\right)^{\prime} \bm{W}^{-1}\left(\hat{\bm{y}}-\bm{S}\bar{\bm{G}}\hat{\bm{y}}\right) + \lambda_0 \sum_{j=1}^n \bm{A}_{jj} \\
\text{s.t.} \quad & \bar{\bm{G}} = (\bm{S}^{\prime}\bm{A}^{\prime}\bm{W}^{-1}\bm{A}\bm{S})^{-1}\bm{S}^{\prime}\bm{A}^{\prime}\bm{W}^{-1} \qquad\text{and}\qquad \bar{\bm{G}}\bm{S} = \bm{I},
\end{align*} where \(\lambda_0 \geq 0\) controls the number of nonzero
diagonal elements in \(\bm{A}\), consequently affecting the number of
nonzero columns (i.e., selected time series) in \(\bm{G}\). We call this
reconciliation method the \emph{intuitive method with} \(L_0\)
\emph{regularization}. In the results that follow, we label the
\textbf{Intuitive} method differently based on various estimators for
\(\bm{W}\), referring to them as \textbf{OLS-intuitive},
\textbf{WLSs-intuitive}, \textbf{WLSv-intuitive},
\textbf{MinT-intuitive}, and \textbf{MinTs-intuitive}, respectively.

We should note that implementing grouped variable selection with this
optimization problem can be challenging because it imposes restrictions
\(\bar{\bm{G}}\) to ensure it adheres rigorously to the analytical
solution of MinT while making the selection. Therefore, the resulting
solution tends to be dense and may not have zero columns.

To ensure the invertibility of
\(\bm{S}^{\prime}\bm{A}^{\prime}\bm{W}^{-1}\bm{A}\bm{S}\), and make the
problem compatible with Gurobi, we reformulate the problem as
\begin{equation}\phantomsection\label{eq-intuitive_mip}{
\begin{aligned}
\min_{\bm{A},\bar{\bm{G}},\bm{C},\check{\bm{e}},\bm{z}} \quad & \frac{1}{2}\check{\bm{e}}^{\prime} \bm{W}^{-1}\check{\bm{e}} + \lambda_0 \sum_{j=1}^n z_j \\
\text{s.t.} \quad & \bar{\bm{G}}\bm{S} = \bm{I} \\
& \hat{\bm{y}}-\left(\hat{\bm{y}}^{\prime} \otimes \bm{S}\right)\operatorname{vec}(\bar{\bm{G}}) = \check{\bm{e}} \\
& \bar{\bm{G}}\bm{A}\bm{S} = \bm{I} \\
& \bar{\bm{G}} = \bm{C}\bm{S}^{\prime}\bm{A}^{\prime}\bm{W}^{-1} \\
& z_j \in\{0,1\}, \quad j \in[n].
\end{aligned}
}\end{equation}

\textbf{Parameter tuning.} Similar to the setup in the group best-subset
selection, we select the tuning parameter, \(\lambda_0\), by minimizing
the sum of squared reconciled forecast errors on a truncated training
set, comprising only the \(\max\{h, s\}\) observations that occurred
prior to the forecast origin. Let
\(\lambda_{0}^{1} = \frac{1}{2}\left(\hat{\bm{y}}-\tilde{\bm{y}}^{\text{bench}}\right)^{\prime} \bm{W}^{-1}\left(\hat{\bm{y}}-\tilde{\bm{y}}^{\text{bench}}\right)\),
and \(\lambda_{0}^{k} = 0.0001\lambda_{0}^{1}\), the collection of
candidate values for \(\lambda_0\) we consider is
\(\{\lambda_{0}^{1},\dots,\lambda_{0}^{k}, 0\}\), where
\(\lambda_{0}^{j} = \lambda_{0}^{1}\left(\lambda_{0}^{k} / \lambda_{0}^{1}\right)^{(j-1) / (k-1)}\)
for \(j \in [k]\).

\textbf{Computation details.} Following a setup akin to that in the
group best-subset selection, we employ Gurobi to solve Equation
\eqref{eq-intuitive_mip} by configuring parameters such as MIPGap =
\(0.001\) and TimeLimit = \(600\) seconds for problems with
\(p > 1000\).

\subsubsection*{Group lasso method}\label{group-lasso-method}
\addcontentsline{toc}{subsubsection}{Group lasso method}

Lasso is another popular method for the selection and estimation of
parameters in the context of linear regression. \citet{Yuan2006-mw}
introduced the group lasso method that can be used when there is a
grouped structure among the variables. Here, we consider \emph{a group
lasso problem under the unbiasedness assumption} given by
\begin{equation}\phantomsection\label{eq-lasso}{
\begin{aligned}
\min_{\bm{G}} \quad & \frac{1}{2}\left(\hat{\bm{y}}-\left(\hat{\bm{y}}^{\prime} \otimes \bm{S}\right) \operatorname{vec}(\bm{G})\right)^{\prime} \bm{W}^{-1}\left(\hat{\bm{y}}-\left(\hat{\bm{y}}^{\prime} \otimes \bm{S}\right) \operatorname{vec}(\bm{G})\right) + \lambda \sum_{j=1}^n w_j \left\|\bm{G}_{\cdot j}\right\|_2 \\
\text{s.t.} \quad & \left(\bm{S}^{\prime} \otimes \bm{I}_{n_b}\right) \operatorname{vec}(\bm{G})=\operatorname{vec}\left(\bm{I}_{n_b}\right),
\end{aligned}
}\end{equation} where \(\lambda \geq 0\) is a tuning parameter,
\(w_j \neq 0\) is the penalty weight assigned in \(\bm{G}_{\cdot j}\) to
make the model more flexible, and the second term in the objective is
the penalty function that is intermediate between the \(L_1\)-penalty
that is used in the lasso and the \(L_2\)-penalty that is used in ridge
regression. In the results that follow, we label the \textbf{Lasso}
method based on various estimators for \(\bm{W}\), referring to them as
\textbf{OLS-lasso}, \textbf{WLSs-lasso}, \textbf{WLSv-lasso},
\textbf{MinT-lasso}, and \textbf{MinTs-lasso}, respectively.

Next, we present the second order cone programming (SOCP) formulation
for the group lasso based estimators given by
\begin{equation}\phantomsection\label{eq-lasso_socp}{
\begin{aligned}
\min_{\operatorname{vec}(\bm{G}), \check{\bm{e}}, \bm{g}^{+}} & \frac{1}{2}\check{\bm{e}}^{\prime} \bm{W}_h^{-1}\check{\bm{e}} + \lambda \sum_{j=1}^n w_j c_j \\
\text{s.t.} \quad & \left(\bm{S}^{\prime} \otimes \bm{I}_{n_b}\right) \operatorname{vec}(\bm{G})=\operatorname{vec}\left(\bm{I}_{n_b}\right) \\
& \hat{\bm{y}}-\left(\hat{\bm{y}}^{\prime} \otimes \bm{S}\right) \operatorname{vec}(\bm{G}) = \check{\bm{e}} \\
& c_j = \sqrt{\sum_{i=1}^{n_b} g_{i + (j-1) n_b}^{+2}}, \quad j \in[n].
\end{aligned}
}\end{equation} Equation \eqref{eq-lasso_socp} includes additional
auxiliary variables \(c_j \in \mathbb{R}_{\geq 0}\), \(j \in [n]\), and
second order cone constraints,
\(c_j = \sqrt{\sum_{i=1}^{n_b} g_{i + (j-1) n_b}^{+2}}\) for
\(j \in[n]\).

Compared to the previous two methods above, the group lasso method is
computationally friendlier. Nonetheless, \citet{Hazimeh2023-ie}
demonstrated, both empirically and theoretically, that the group
\(L_0\)-regularized method exhibits advantages over its group lasso
counterpart across a range of regimes. Group lasso can either be highly
dense or possess non-zero coefficients that are overly shrunk. This
issue becomes more pronounced when the groups are correlated with each
other, as group lasso tends to retain all correlated groups instead of
seeking a more concise model.

\textbf{Penalty weights and parameter tuning.} In the context of group
lasso, the default choice for the penalty weight, \(w_j\), is
\(\sqrt{p_j}\), where \(p_j\) is the size of each group (in our case,
\(p_j = n_b\)). In our experiments, we allocate different penalty
weights to each group using
\(w_j = 1/\|\bm{G}_{\cdot j}^{\text{bench}}\|_2\), which allows us to
account for variations in scale across different time series in the
structure.

We compute the group lasso over \(k+1\) values of the tuning parameter
\(\lambda\), and select the parameter by optimizing the sum of squared
reconciled forecast errors on a truncated training set, consisting only
of \(\max\{h, s\}\) observations occurred prior to the forecast origin.
The collection of candidate values for \(\lambda\) is
\(\{\lambda^{1},\dots,\lambda^{k}, 0\}\), where
\(\lambda^{1} = \max_{j=1, \ldots, n}\big\|-\big((\hat{\bm{y}}^{\prime} \otimes \bm{S})_{\cdot j^{*}}\big)^{\prime} \bm{W}^{-1} \hat{\bm{y}}\big\|_2 / w_j\),
\(\lambda^{k} = 0.0001\lambda^{1}\), and
\(\lambda^{j} = \lambda^{1}(\lambda^{k} / \lambda^{1})^{(j-1) / (k-1)}\)
for \(j \in [k]\).

\begin{proposition}[]\protect\hypertarget{prp-3}{}\label{prp-3}

Ignoring the constraint \(\bm{G_h S}=\bm{I_{n_b}}\), we define
\(\lambda^{1}\) as the smallest \(\lambda\) value such that all
predictors in the group lasso problem have zero coefficients. Then we
have \[
\lambda^{1} = \max_{j=1, \ldots, n}\big\|-\big((\hat{\bm{y}}^{\prime} \otimes \bm{S})_{\cdot j^{*}}\big)^{\prime} \bm{W}^{-1} \hat{\bm{y}}\big\|_2 / w_j,
\] where \(j^{*}\) denotes the column index of
\(\hat{\bm{y}}^{\prime} \otimes \bm{S}\) that corresponds to the \(j\)th
column of \(\bm{G}\).

\end{proposition}

\begin{proof}
Denote \(\bm{\beta} = \operatorname{vec}(\bm{G})\), and the first term
in the objective of Equation \eqref{eq-lasso} as
\(L\left(\bm{\beta} \mid \bm{D}\right)\), where \(\bm{D}\) is the
working data
\(\{\hat{\bm{y}} , \hat{\bm{y}}^{\prime} \otimes \bm{S}\}\). Ignoring
the constraint \(\bm{G_h S}=\bm{I_{n_b}}\), we define \(\lambda^{1}\) as
the smallest \(\lambda\) value such that all predictors in the group
lasso problem have zero coefficients, i.e., the solution at
\(\lambda^{1}\) is \(\hat{\bm{\beta}}^{1}=\bm{0}\). (Note that there is
no intercept in our problem.) Under the Karush-Kuhn-Tucker conditions,
we have \[
\lambda^{1}
 = \max_{j=1, \ldots, n}\big\|\big[\nabla L(\hat{\bm{\beta}}^{1} \mid \bm{D})\big]^{(j)}\big\|_2 / w_j
 = \max_{j=1, \ldots, n}\big\|-\big((\hat{\bm{y}}^{\prime} \otimes \bm{S})_{\cdot j^{*}}\big)^{\prime} \bm{W}^{-1} \hat{\bm{y}}\big\|_2 / w_j.
\]
\end{proof}

\textbf{Computation details.} Due to the incorporation of the
constraint, we can not directly use some open-source packages designed
for group lasso. Consequently, we employ Gurobi to solve the SOCP
problem, configuring it by setting OptimalityTol = \(0.0001\).

\subsection{Series selection relaxing the unbiasedness
assumption}\label{sec-unconstrained}

In this section, we relax the unbiasedness assumption, and introduce a
reconciliation method with selection that relies on in-sample
observations and fitted values. Let
\(\bm{Y} \in \mathbb{R}^{T \times n}\) denote a matrix comprising
observations from all time series on the training set in the structure,
and \(\hat{\bm{Y}} \in \mathbb{R}^{T \times n}\) denote a matrix of
in-sample one-step-ahead forecasts (i.e., fitted values) for all time
series. The proposed \emph{empirical group lasso} method considers the
optimization problem \[
\min_{\bm{G}} \quad \frac{1}{2 T} \left\|\bm{Y}-\hat{\bm{Y}} \bm{G}^{\prime} \bm{S}^{\prime}\right\|_F^2 + \lambda \sum_{j=1}^n w_j \left\|\bm{G}_{\cdot j}\right\|_2,
\] where \(\lambda \geq 0\) is a tuning parameter, \(w_j \neq 0\) is the
penalty weight assigned in \(\bm{G}_{\cdot j}\) to make a more flexible
model. We rewrite the problem as \[
\min_{\operatorname{vec}(\bm{G})} \quad \frac{1}{2 T} \left\|\operatorname{vec}(\bm{Y})-(\bm{S} \otimes \hat{\bm{Y}}) \operatorname{vec}\left(\bm{G}^{\prime}\right)\right\|_2^2 + \lambda \sum_{j=1}^n w_j \left\|\bm{G}_{\cdot j}\right\|_2,
\] which becomes a standard group lasso problem, with
\(\operatorname{vec}(\bm{Y})\) serving as the dependent variable and
\(\bm{S} \otimes \hat{\bm{Y}}\) as the covariate matrix. We denote this
as \textbf{Elasso} in the results that follow.

Similar to EMinT, Elasso operates as an ``in-sample'' method because
\(\hat{\bm{Y}}\) are predictions in the form of fitted values (i.e.,
\(\bm{Y}\) is in training data when base forecasts are computed). Thus,
both EMinT and Elasso require only a single round of model training.
Although Subset, Intuitive, and Lasso methods introduced in
Section~\ref{sec-constrained} necessitate genuine forecasts, they still
only require one round of model training and a single forecasting phase.
To ensure a fair comparison, we will exclude the ``out-of-sample'' RERM
and ERM methods from our simulation studies and empirical results due to
their iterative approach of generating out-of-sample forecasts using
expanding windows, which is highly time-consuming.

Relaxing the unbiasedness assumption may result in fewer non-zero column
entries in the \(\bm{G}\) solution than the number of series at the
bottom level. This differs from constrained reconciliation methods
detailed in Section~\ref{sec-constrained}. In an extreme scenario, the
solution may take the form of a top-down
\(\bm{G}_{TD}=[\bm{p} \mid \bm{O}_{n_b \times (n-1)}]\), where only the
column corresponding to the top level (most aggregated level) retains
non-zero values, and \(\bm{p} = (p_1, p_2, \ldots, p_{n_b})\) is a
proportionality vector obtained based on in-sample reconciled forecast
errors.

We also explored the empirical version of group best-subset selection
with ridge regularization and the intuitive method with \(L_0\)
regularization in which we omit the unbiasedness assumption. It is worth
mentioning that \citet{Hazimeh2023-ie} presented an algorithmic
framework for formulating the group \(L_0\) problem with ridge
regularization and provided the \textbf{L0Group} Python package for
implementation. However, our experiments showed that this algorithm can
not terminate within five hours for typical instances with
\(p \sim 10^4\). Therefore, in this paper, we only present the empirical
group lasso method for series selection without the unbiasedness
assumption.

\textbf{Penalty weights and parameter tuning.} Similar to the setup in
the group lasso method, we assign different penalty weights to each
group by setting \(w_j = 1/\|\bm{G}_{\cdot j}^{\text{OLS}}\|_2\), where
\(\bm{G}^{\text{OLS}}\) is the solution obtained by the OLS estimator of
\(\bm{W}\). Given a fixed tuning parameter, we solve the target
optimization problem by considering the initial \(T-T_v\) observations,
where \(T_v = \max\{h, s\}\) for seasonal time series and
\(T_v = \lfloor \frac{1}{10}T \rfloor\) for non-seasonal time series.
Then the tuning parameter, \(\lambda\), is selected by minimizing the
sum of squared reconciled forecast errors on a truncated training set,
comprising only the \(T_v\) observations closest to the forecast origin.
Specifically, for \(\lambda\) values, we consider
\(\{\lambda^{1},\dots,\lambda^{k}, 0\}\), where
\(\lambda^{1} = \max_{j=1, \ldots, n}\left\|-\frac{1}{N}\left(\left(\bm{S} \otimes \hat{\bm{Y}}\right)_{\cdot j*}\right)^{\prime} \operatorname{vec}(\bm{Y})\right\|_2 / w_j\),
\(\lambda^{k} = 0.0001\lambda^{1}\), and
\(\lambda^{j} = \lambda^{1}\left(\lambda^{k} / \lambda^{1}\right)^{(j-1) / (k-1)}\)
for \(j \in [k]\). Following the derivation in the proof of
Proposition~\ref{prp-3}, \(\lambda^{1}\) is the smallest \(\lambda\)
value such that all predictors in the empirical group lasso problem have
zero coefficients, i.e., \(\bm{G} = \bm{O}\). Note that we need to
resolve the optimization problem based on the whole training set by
using the optimal tuning parameter to obtain the final solution.

\textbf{Computation details.} While there are open-source packages
available for solving group lasso problems, they are still relatively
slow when handling large instances. For example, given a specific value
for the parameter, \(\lambda\), our experiments observed that, using the
\textbf{gglasso} R package, we can not obtain a solution within five
hours for typical instances with \(p \sim 10^4\). Instead, we use Gurobi
to solve the problem using the SOCP formulation for the empirical group
lasso which aligns with Equation \eqref{eq-lasso_socp} but omits the
constraint.

\section{Monte Carlo simulations}\label{sec-simulations}

To assess the proposed reconciliation methods with time series selection
outlined in Section~\ref{sec-methodology}, we carry out two simulations
with different designs. Both simulations consider a hierarchy comprising
two levels of aggregation, as shown in Figure~\ref{fig-hts}.
Specifically, the structure has four series at the bottom level, and
seven series in total; i.e., \(n_b = 4\), and \(n = 7\). The
bottom-level series are first generated and then summed to obtain
aggregated series at higher levels.

Section~\ref{sec-sim1} considers a setup where the bottom-level series
are generated using a structural time series model, but model
misspecification exists for some series within the structure.
Section~\ref{sec-sim2} explores the impact of the correlation between
series on the performance of reconciled forecasts.

\subsection{Setup 1: Exploring the effect of model
misspecification}\label{sec-sim1}

We follow a simulation setup similar to \citet{Wickramasuriya2019-fc},
assuming that the bottom-level time series are generated using the basic
structural time series model \[
\bm{b}_t=\bm{\mu}_t+\bm{\gamma}_t+\bm{\eta}_t,
\] where \(\bm{\mu}_t\) and \(\bm{\gamma}_t\) are trend and seasonal
components defined by \begin{align*}
\bm{\mu}_t & =\bm{\mu}_{t-1}+\bm{v}_t+\bm{\varrho}_t, &&& \bm{\varrho}_t & \sim \mathcal{N}\left(\bm{0}, \sigma_{\varrho}^2 \bm{I}_4\right), \\
\bm{v}_t & =\bm{v}_{t-1}+\bm{\zeta}_t, &&& \bm{\zeta}_t & \sim \mathcal{N}\left(\bm{0}, \sigma_\zeta^2 \bm{I}_4\right), \\
\bm{\gamma}_t & =-\sum_{i=1}^{s-1} \bm{\gamma}_{t-i}+\bm{\omega}_t, &&& \bm{\omega}_t & \sim \mathcal{N}\left(\bm{0}, \sigma_\omega^2 \bm{I}_4\right),
\end{align*} \(\bm{\varrho}_t\), \(\bm{\zeta}_t\), and \(\bm{\omega}_t\)
are error terms independent of each other and over time, and
\(\bm{\eta}_t\) is generated independently from an
\(\text{ARIMA}(p,0,q)\) process, where \(p\) and \(q\) take values of
\(0\) or \(1\) with equal probability. Coefficients in the ARIMA process
are randomly sampled from a uniform distribution within the range
\([0.5, 0.7]\), and the contemporaneous error covariance matrix is given
by \[
\left[\begin{array}{llll}
5 & 3 & 2 & 1 \\
3 & 4 & 2 & 1 \\
2 & 2 & 5 & 3 \\
1 & 1 & 3 & 4
\end{array}\right],
\] which enables correlations among time series in a hierarchical
structure.

We set \(s = 4\) for quarterly data with error variances
\(\sigma_{\varrho}^2=2\), \(\sigma_\zeta^2=0.007\), and
\(\sigma_\omega^2=7\). Initial values for \(\bm{\mu}_0\), \(\bm{v}_0\),
\(\bm{\gamma}_0\), \(\bm{\gamma}_1\), and \(\bm{\gamma}_2\) are
generated independently from a multivariate normal distribution with
zero mean and identity covariance matrix. For each bottom-level series,
we generate a total of \(T+h = 180\) observations, with the last
\(h = 16\) observations forming the test set. The bottom-level series
are aggregated for data at higher levels. This process is repeated
\(500\) times.

We use ETS models to generate base forecasts in the hierarchy with the
default settings from the \textbf{forecast} R package
\citep{Hyndman2023-fc}. To introduce model misspecification, we
deliberately undermine in-sample and out-of-sample forecasts (i.e.,
fitted values and base forecasts) for specific time series in three
scenarios. In each scenario, a 1.5 multiplier is applied to in-sample
and out-of-sample forecasts for a single series; i.e., series AA at the
bottom level, series A at the middle level, and series Total at the top
level, resulting in Scenarios A--C, respectively.

Table~\ref{tbl-s1-rmse}, Table~\ref{tbl-s2-rmse}, and
Table~\ref{tbl-s3-rmse} summarize the results, each reporting the
average root mean squared error (RMSE) for each level as well as the
whole structure (denoted as \emph{Average}). The \emph{Base} row shows
average RMSE of base forecasts, specifically ETS forecasts in this
setup. Entries below report the percentage decrease (negative) or
increase (positive) in average RMSE of reconciled forecasts compared to
base forecasts. The \emph{BU} row uses a ``bottom-up'' approach,
aggregating bottom-level base forecasts to form higher level forecasts.
The \emph{OLS}, \emph{WLS}, \emph{WLSv}, \emph{MinT}, and \emph{MinTs}
rows all use the MinT method with different estimators of \(\bm{W}\) (as
per Table~\ref{tbl-bench}) being used. The \emph{OLS-}, \emph{WLS-},
\emph{WLSv-}, \emph{MinT-}, and \emph{MinTs-} rows use the three methods
introduced in Section~\ref{sec-constrained} with same different
estimators of \(\bm{W}\) being used. The \emph{EMinT} row uses the
empirical MinT method, while the \emph{Elasso} row uses the empirical
group lasso method proposed in Section~\ref{sec-unconstrained}. Notably,
in each scenario, the largest improvements occur at the hierarchical
level with model misspecification, while slightly deteriorating the
performance at other levels.

\begin{table}

\caption{\label{tbl-s1-rmse}Out-of-sample forecast results for the
simulated data in Scenario A, Setup 1.}

\centering{

\centering
\resizebox{\linewidth}{!}{
\fontsize{11}{13}\selectfont
\begin{threeparttable}
\begin{tabular}{lrrrrrrrrrrrrrrrr}
\toprule
\multicolumn{1}{c}{} & \multicolumn{4}{c}{Top} & \multicolumn{4}{c}{Middle} & \multicolumn{4}{c}{Bottom} & \multicolumn{4}{c}{Average} \\
\cmidrule(l{3pt}r{3pt}){2-5} \cmidrule(l{3pt}r{3pt}){6-9} \cmidrule(l{3pt}r{3pt}){10-13} \cmidrule(l{3pt}r{3pt}){14-17}
Method & h=1 & 1--4 & 1--8 & 1--16 & h=1 & 1--4 & 1--8 & 1--16 & h=1 & 1--4 & 1--8 & 1--16 & h=1 & 1--4 & 1--8 & 1--16\\
\midrule
Base & 9.6 & 10.7 & 12.6 & 15.6 & 6.3 & 7.3 & 8.6 & 10.8 & 6.4 & 7.5 & 8.3 & 9.8 & 6.8 & 7.9 & 9.0 & 10.9\\
BU & 57.8 & 68.5 & 53.7 & 38.9 & 58.2 & 61.8 & 48.1 & 34.4 & 0.0 & 0.0 & 0.0 & 0.0 & 27.0 & 29.6 & 23.8 & 17.7\\
\midrule
OLS & 0.6 & 2.2 & 1.8 & 1.4 & 7.1 & 6.4 & 4.6 & 3.1 & --7.6 & --8.6 & --8.2 & --7.3 & --2.1 & --2.5 & --2.7 & --2.6\\
\cellcolor[HTML]{e6e3e3}{OLS-subset} & \cellcolor[HTML]{e6e3e3}{0.6} & \cellcolor[HTML]{e6e3e3}{\textbf{ 1.8}} & \cellcolor[HTML]{e6e3e3}{\textbf{ 1.5}} & \cellcolor[HTML]{e6e3e3}{\textbf{ 1.3}} & \cellcolor[HTML]{e6e3e3}{7.2} & \cellcolor[HTML]{e6e3e3}{\textbf{ 5.2}} & \cellcolor[HTML]{e6e3e3}{\textbf{ 3.8}} & \cellcolor[HTML]{e6e3e3}{\textbf{ 2.6}} & \cellcolor[HTML]{e6e3e3}{\textbf{ --8.3}} & \cellcolor[HTML]{e6e3e3}{\textbf{--12.9}} & \cellcolor[HTML]{e6e3e3}{\textbf{--11.6}} & \cellcolor[HTML]{e6e3e3}{\textbf{ --9.9}} & \cellcolor[HTML]{e6e3e3}{\textbf{ --2.4}} & \cellcolor[HTML]{e6e3e3}{\textbf{ --5.2}} & \cellcolor[HTML]{e6e3e3}{\textbf{ --4.8}} & \cellcolor[HTML]{e6e3e3}{\textbf{ --4.1}}\\
\cellcolor[HTML]{e6e3e3}{OLS-intuitive} & \cellcolor[HTML]{e6e3e3}{0.8} & \cellcolor[HTML]{e6e3e3}{2.6} & \cellcolor[HTML]{e6e3e3}{2.1} & \cellcolor[HTML]{e6e3e3}{1.8} & \cellcolor[HTML]{e6e3e3}{7.5} & \cellcolor[HTML]{e6e3e3}{\textbf{ 6.1}} & \cellcolor[HTML]{e6e3e3}{\textbf{ 4.4}} & \cellcolor[HTML]{e6e3e3}{\textbf{ 3.0}} & \cellcolor[HTML]{e6e3e3}{\textbf{ --9.0}} & \cellcolor[HTML]{e6e3e3}{\textbf{--12.8}} & \cellcolor[HTML]{e6e3e3}{\textbf{--11.6}} & \cellcolor[HTML]{e6e3e3}{\textbf{ --9.9}} & \cellcolor[HTML]{e6e3e3}{\textbf{ --2.7}} & \cellcolor[HTML]{e6e3e3}{\textbf{ --4.8}} & \cellcolor[HTML]{e6e3e3}{\textbf{ --4.5}} & \cellcolor[HTML]{e6e3e3}{\textbf{ --3.8}}\\
\cellcolor[HTML]{e6e3e3}{OLS-lasso} & \cellcolor[HTML]{e6e3e3}{0.6} & \cellcolor[HTML]{e6e3e3}{2.2} & \cellcolor[HTML]{e6e3e3}{1.8} & \cellcolor[HTML]{e6e3e3}{1.6} & \cellcolor[HTML]{e6e3e3}{7.4} & \cellcolor[HTML]{e6e3e3}{6.7} & \cellcolor[HTML]{e6e3e3}{4.8} & \cellcolor[HTML]{e6e3e3}{3.2} & \cellcolor[HTML]{e6e3e3}{--7.6} & \cellcolor[HTML]{e6e3e3}{--8.5} & \cellcolor[HTML]{e6e3e3}{--8.1} & \cellcolor[HTML]{e6e3e3}{--7.2} & \cellcolor[HTML]{e6e3e3}{--2.0} & \cellcolor[HTML]{e6e3e3}{--2.4} & \cellcolor[HTML]{e6e3e3}{--2.6} & \cellcolor[HTML]{e6e3e3}{--2.5}\\
\midrule
WLSs & 7.3 & 10.6 & 8.1 & 5.9 & 15.6 & 16.0 & 11.8 & 8.0 & --6.9 & --7.8 & --7.4 & --6.4 & 1.9 & 2.0 & 1.0 & 0.2\\
\cellcolor[HTML]{e6e3e3}{WLSs-subset} & \cellcolor[HTML]{e6e3e3}{\textbf{ 5.0}} & \cellcolor[HTML]{e6e3e3}{\textbf{ 5.7}} & \cellcolor[HTML]{e6e3e3}{\textbf{ 4.6}} & \cellcolor[HTML]{e6e3e3}{\textbf{ 3.6}} & \cellcolor[HTML]{e6e3e3}{\textbf{12.3}} & \cellcolor[HTML]{e6e3e3}{\textbf{10.0}} & \cellcolor[HTML]{e6e3e3}{\textbf{ 7.5}} & \cellcolor[HTML]{e6e3e3}{\textbf{ 5.2}} & \cellcolor[HTML]{e6e3e3}{\textbf{ --7.6}} & \cellcolor[HTML]{e6e3e3}{\textbf{--10.5}} & \cellcolor[HTML]{e6e3e3}{\textbf{ --9.6}} & \cellcolor[HTML]{e6e3e3}{\textbf{ --8.2}} & \cellcolor[HTML]{e6e3e3}{\textbf{  0.2}} & \cellcolor[HTML]{e6e3e3}{\textbf{ --2.0}} & \cellcolor[HTML]{e6e3e3}{\textbf{ --2.1}} & \cellcolor[HTML]{e6e3e3}{\textbf{ --2.0}}\\
\cellcolor[HTML]{e6e3e3}{WLSs-intuitive} & \cellcolor[HTML]{e6e3e3}{\textbf{ 7.1}} & \cellcolor[HTML]{e6e3e3}{\textbf{ 9.2}} & \cellcolor[HTML]{e6e3e3}{\textbf{ 7.1}} & \cellcolor[HTML]{e6e3e3}{\textbf{ 5.2}} & \cellcolor[HTML]{e6e3e3}{16.5} & \cellcolor[HTML]{e6e3e3}{\textbf{15.5}} & \cellcolor[HTML]{e6e3e3}{\textbf{11.5}} & \cellcolor[HTML]{e6e3e3}{\textbf{ 7.9}} & \cellcolor[HTML]{e6e3e3}{--6.8} & \cellcolor[HTML]{e6e3e3}{\textbf{ --9.2}} & \cellcolor[HTML]{e6e3e3}{\textbf{ --8.4}} & \cellcolor[HTML]{e6e3e3}{\textbf{ --7.3}} & \cellcolor[HTML]{e6e3e3}{2.1} & \cellcolor[HTML]{e6e3e3}{\textbf{  0.9}} & \cellcolor[HTML]{e6e3e3}{\textbf{  0.1}} & \cellcolor[HTML]{e6e3e3}{\textbf{ --0.4}}\\
\cellcolor[HTML]{e6e3e3}{WLSs-lasso} & \cellcolor[HTML]{e6e3e3}{7.3} & \cellcolor[HTML]{e6e3e3}{\textbf{10.3}} & \cellcolor[HTML]{e6e3e3}{\textbf{ 8.0}} & \cellcolor[HTML]{e6e3e3}{5.9} & \cellcolor[HTML]{e6e3e3}{15.7} & \cellcolor[HTML]{e6e3e3}{16.1} & \cellcolor[HTML]{e6e3e3}{11.8} & \cellcolor[HTML]{e6e3e3}{8.1} & \cellcolor[HTML]{e6e3e3}{\textbf{ --7.0}} & \cellcolor[HTML]{e6e3e3}{--7.8} & \cellcolor[HTML]{e6e3e3}{--7.3} & \cellcolor[HTML]{e6e3e3}{--6.4} & \cellcolor[HTML]{e6e3e3}{1.9} & \cellcolor[HTML]{e6e3e3}{2.0} & \cellcolor[HTML]{e6e3e3}{1.0} & \cellcolor[HTML]{e6e3e3}{0.2}\\
\midrule
WLSv & 1.0 & 2.9 & 2.3 & 1.9 & 4.5 & 4.3 & 3.2 & 2.1 & --25.8 & --26.4 & --22.7 & --18.3 & --12.4 & --12.6 & --10.7 & --8.4\\
\cellcolor[HTML]{e6e3e3}{WLSv-subset} & \cellcolor[HTML]{e6e3e3}{\textcolor{blue}{\textbf{--1.0}}} & \cellcolor[HTML]{e6e3e3}{\textbf{ 0.3}} & \cellcolor[HTML]{e6e3e3}{\textbf{ 0.4}} & \cellcolor[HTML]{e6e3e3}{\textbf{ 0.5}} & \cellcolor[HTML]{e6e3e3}{\textbf{ 0.6}} & \cellcolor[HTML]{e6e3e3}{\textbf{ 0.6}} & \cellcolor[HTML]{e6e3e3}{\textbf{ 0.5}} & \cellcolor[HTML]{e6e3e3}{\textbf{ 0.3}} & \cellcolor[HTML]{e6e3e3}{\textbf{--32.3}} & \cellcolor[HTML]{e6e3e3}{\textbf{--32.2}} & \cellcolor[HTML]{e6e3e3}{\textbf{--27.3}} & \cellcolor[HTML]{e6e3e3}{\textbf{--21.7}} & \cellcolor[HTML]{e6e3e3}{\textbf{--17.3}} & \cellcolor[HTML]{e6e3e3}{\textbf{--17.3}} & \cellcolor[HTML]{e6e3e3}{\textbf{--14.2}} & \cellcolor[HTML]{e6e3e3}{\textbf{--10.9}}\\
\cellcolor[HTML]{e6e3e3}{WLSv-intuitive} & \cellcolor[HTML]{e6e3e3}{\textbf{--0.5}} & \cellcolor[HTML]{e6e3e3}{\textcolor{blue}{\textbf{ 0.2}}} & \cellcolor[HTML]{e6e3e3}{\textcolor{blue}{\textbf{ 0.3}}} & \cellcolor[HTML]{e6e3e3}{\textbf{ 0.5}} & \cellcolor[HTML]{e6e3e3}{\textbf{ 0.9}} & \cellcolor[HTML]{e6e3e3}{\textbf{ 0.7}} & \cellcolor[HTML]{e6e3e3}{\textbf{ 0.5}} & \cellcolor[HTML]{e6e3e3}{\textbf{ 0.3}} & \cellcolor[HTML]{e6e3e3}{\textbf{--32.3}} & \cellcolor[HTML]{e6e3e3}{\textbf{--32.3}} & \cellcolor[HTML]{e6e3e3}{\textbf{--27.4}} & \cellcolor[HTML]{e6e3e3}{\textbf{--21.7}} & \cellcolor[HTML]{e6e3e3}{\textbf{--17.1}} & \cellcolor[HTML]{e6e3e3}{\textbf{--17.3}} & \cellcolor[HTML]{e6e3e3}{\textbf{--14.2}} & \cellcolor[HTML]{e6e3e3}{\textbf{--10.9}}\\
\cellcolor[HTML]{e6e3e3}{WLSv-lasso} & \cellcolor[HTML]{e6e3e3}{\textbf{ 0.4}} & \cellcolor[HTML]{e6e3e3}{\textbf{ 1.5}} & \cellcolor[HTML]{e6e3e3}{\textbf{ 1.5}} & \cellcolor[HTML]{e6e3e3}{\textbf{ 1.4}} & \cellcolor[HTML]{e6e3e3}{\textbf{ 3.0}} & \cellcolor[HTML]{e6e3e3}{\textbf{ 2.5}} & \cellcolor[HTML]{e6e3e3}{\textbf{ 2.0}} & \cellcolor[HTML]{e6e3e3}{\textbf{ 1.3}} & \cellcolor[HTML]{e6e3e3}{\textbf{--28.5}} & \cellcolor[HTML]{e6e3e3}{\textbf{--29.2}} & \cellcolor[HTML]{e6e3e3}{\textbf{--24.9}} & \cellcolor[HTML]{e6e3e3}{\textbf{--19.9}} & \cellcolor[HTML]{e6e3e3}{\textbf{--14.4}} & \cellcolor[HTML]{e6e3e3}{\textbf{--14.9}} & \cellcolor[HTML]{e6e3e3}{\textbf{--12.3}} & \cellcolor[HTML]{e6e3e3}{\textbf{ --9.5}}\\
\midrule
MinT & --0.4 & 0.7 & 0.9 & 0.6 & 0.7 & 0.7 & 0.6 & 0.3 & --32.9 & --33.4 & --28.3 & --22.5 & --17.5 & --17.8 & --14.6 & --11.3\\
\cellcolor[HTML]{e6e3e3}{MinT-subset} & \cellcolor[HTML]{e6e3e3}{\textbf{--0.6}} & \cellcolor[HTML]{e6e3e3}{0.7} & \cellcolor[HTML]{e6e3e3}{\textbf{ 0.8}} & \cellcolor[HTML]{e6e3e3}{0.7} & \cellcolor[HTML]{e6e3e3}{\textbf{ 0.6}} & \cellcolor[HTML]{e6e3e3}{0.8} & \cellcolor[HTML]{e6e3e3}{0.6} & \cellcolor[HTML]{e6e3e3}{0.3} & \cellcolor[HTML]{e6e3e3}{\textbf{--33.0}} & \cellcolor[HTML]{e6e3e3}{--33.1} & \cellcolor[HTML]{e6e3e3}{--28.0} & \cellcolor[HTML]{e6e3e3}{--22.3} & \cellcolor[HTML]{e6e3e3}{\textbf{--17.6}} & \cellcolor[HTML]{e6e3e3}{--17.6} & \cellcolor[HTML]{e6e3e3}{--14.5} & \cellcolor[HTML]{e6e3e3}{--11.2}\\
\cellcolor[HTML]{e6e3e3}{MinT-intuitive} & \cellcolor[HTML]{e6e3e3}{--0.4} & \cellcolor[HTML]{e6e3e3}{0.7} & \cellcolor[HTML]{e6e3e3}{0.9} & \cellcolor[HTML]{e6e3e3}{0.6} & \cellcolor[HTML]{e6e3e3}{0.7} & \cellcolor[HTML]{e6e3e3}{0.7} & \cellcolor[HTML]{e6e3e3}{0.6} & \cellcolor[HTML]{e6e3e3}{0.3} & \cellcolor[HTML]{e6e3e3}{--32.9} & \cellcolor[HTML]{e6e3e3}{--33.4} & \cellcolor[HTML]{e6e3e3}{--28.3} & \cellcolor[HTML]{e6e3e3}{--22.5} & \cellcolor[HTML]{e6e3e3}{--17.5} & \cellcolor[HTML]{e6e3e3}{--17.8} & \cellcolor[HTML]{e6e3e3}{--14.6} & \cellcolor[HTML]{e6e3e3}{--11.3}\\
\cellcolor[HTML]{e6e3e3}{MinT-lasso} & \cellcolor[HTML]{e6e3e3}{\textbf{--0.7}} & \cellcolor[HTML]{e6e3e3}{\textbf{ 0.3}} & \cellcolor[HTML]{e6e3e3}{\textbf{ 0.6}} & \cellcolor[HTML]{e6e3e3}{\textcolor{blue}{\textbf{ 0.4}}} & \cellcolor[HTML]{e6e3e3}{\textcolor{blue}{\textbf{ 0.3}}} & \cellcolor[HTML]{e6e3e3}{\textcolor{blue}{\textbf{ 0.4}}} & \cellcolor[HTML]{e6e3e3}{\textcolor{blue}{\textbf{ 0.4}}} & \cellcolor[HTML]{e6e3e3}{\textcolor{blue}{\textbf{ 0.1}}} & \cellcolor[HTML]{e6e3e3}{\textcolor{blue}{\textbf{--33.2}}} & \cellcolor[HTML]{e6e3e3}{\textcolor{blue}{\textbf{--33.7}}} & \cellcolor[HTML]{e6e3e3}{\textcolor{blue}{\textbf{--28.5}}} & \cellcolor[HTML]{e6e3e3}{\textcolor{blue}{\textbf{--22.6}}} & \cellcolor[HTML]{e6e3e3}{\textcolor{blue}{\textbf{--17.8}}} & \cellcolor[HTML]{e6e3e3}{\textcolor{blue}{\textbf{--18.1}}} & \cellcolor[HTML]{e6e3e3}{\textcolor{blue}{\textbf{--14.8}}} & \cellcolor[HTML]{e6e3e3}{\textcolor{blue}{\textbf{--11.4}}}\\
\midrule
MinTs & --0.9 & 0.6 & 0.7 & 0.5 & 0.6 & 0.6 & 0.5 & 0.2 & --32.9 & --33.5 & --28.3 & --22.5 & --17.6 & --17.9 & --14.6 & --11.3\\
\cellcolor[HTML]{e6e3e3}{MinTs-subset} & \cellcolor[HTML]{e6e3e3}{--0.7} & \cellcolor[HTML]{e6e3e3}{0.9} & \cellcolor[HTML]{e6e3e3}{1.1} & \cellcolor[HTML]{e6e3e3}{1.0} & \cellcolor[HTML]{e6e3e3}{0.7} & \cellcolor[HTML]{e6e3e3}{0.8} & \cellcolor[HTML]{e6e3e3}{0.7} & \cellcolor[HTML]{e6e3e3}{0.4} & \cellcolor[HTML]{e6e3e3}{\textbf{--33.0}} & \cellcolor[HTML]{e6e3e3}{--33.1} & \cellcolor[HTML]{e6e3e3}{--27.9} & \cellcolor[HTML]{e6e3e3}{--22.2} & \cellcolor[HTML]{e6e3e3}{--17.6} & \cellcolor[HTML]{e6e3e3}{--17.5} & \cellcolor[HTML]{e6e3e3}{--14.3} & \cellcolor[HTML]{e6e3e3}{--11.0}\\
\cellcolor[HTML]{e6e3e3}{MinTs-intuitive} & \cellcolor[HTML]{e6e3e3}{--0.9} & \cellcolor[HTML]{e6e3e3}{0.6} & \cellcolor[HTML]{e6e3e3}{0.7} & \cellcolor[HTML]{e6e3e3}{0.5} & \cellcolor[HTML]{e6e3e3}{0.6} & \cellcolor[HTML]{e6e3e3}{0.6} & \cellcolor[HTML]{e6e3e3}{0.5} & \cellcolor[HTML]{e6e3e3}{0.2} & \cellcolor[HTML]{e6e3e3}{--32.9} & \cellcolor[HTML]{e6e3e3}{--33.5} & \cellcolor[HTML]{e6e3e3}{--28.3} & \cellcolor[HTML]{e6e3e3}{--22.5} & \cellcolor[HTML]{e6e3e3}{--17.6} & \cellcolor[HTML]{e6e3e3}{--17.9} & \cellcolor[HTML]{e6e3e3}{--14.6} & \cellcolor[HTML]{e6e3e3}{--11.3}\\
\cellcolor[HTML]{e6e3e3}{MinTs-lasso} & \cellcolor[HTML]{e6e3e3}{--0.9} & \cellcolor[HTML]{e6e3e3}{\textbf{ 0.4}} & \cellcolor[HTML]{e6e3e3}{\textbf{ 0.6}} & \cellcolor[HTML]{e6e3e3}{0.5} & \cellcolor[HTML]{e6e3e3}{0.6} & \cellcolor[HTML]{e6e3e3}{\textcolor{blue}{\textbf{ 0.4}}} & \cellcolor[HTML]{e6e3e3}{\textcolor{blue}{\textbf{ 0.4}}} & \cellcolor[HTML]{e6e3e3}{\textcolor{blue}{\textbf{ 0.1}}} & \cellcolor[HTML]{e6e3e3}{\textcolor{blue}{\textbf{--33.2}}} & \cellcolor[HTML]{e6e3e3}{\textbf{--33.6}} & \cellcolor[HTML]{e6e3e3}{\textbf{--28.4}} & \cellcolor[HTML]{e6e3e3}{\textcolor{blue}{\textbf{--22.6}}} & \cellcolor[HTML]{e6e3e3}{\textbf{--17.7}} & \cellcolor[HTML]{e6e3e3}{\textbf{--18.0}} & \cellcolor[HTML]{e6e3e3}{\textcolor{blue}{\textbf{--14.8}}} & \cellcolor[HTML]{e6e3e3}{\textcolor{blue}{\textbf{--11.4}}}\\
\midrule
EMinT & 2.2 & 2.9 & 2.5 & 1.7 & 2.5 & 2.9 & 2.3 & 1.3 & --31.9 & --32.3 & --27.5 & --22.0 & --15.9 & --16.2 & --13.4 & --10.5\\
\cellcolor[HTML]{e6e3e3}{Elasso} & \cellcolor[HTML]{e6e3e3}{\textbf{ 1.5}} & \cellcolor[HTML]{e6e3e3}{\textbf{ 2.8}} & \cellcolor[HTML]{e6e3e3}{\textbf{ 2.4}} & \cellcolor[HTML]{e6e3e3}{1.7} & \cellcolor[HTML]{e6e3e3}{\textbf{ 2.1}} & \cellcolor[HTML]{e6e3e3}{\textbf{ 2.8}} & \cellcolor[HTML]{e6e3e3}{2.3} & \cellcolor[HTML]{e6e3e3}{1.3} & \cellcolor[HTML]{e6e3e3}{\textbf{--32.1}} & \cellcolor[HTML]{e6e3e3}{--32.2} & \cellcolor[HTML]{e6e3e3}{--27.4} & \cellcolor[HTML]{e6e3e3}{--21.9} & \cellcolor[HTML]{e6e3e3}{\textbf{--16.3}} & \cellcolor[HTML]{e6e3e3}{--16.2} & \cellcolor[HTML]{e6e3e3}{--13.3} & \cellcolor[HTML]{e6e3e3}{--10.5}\\
\bottomrule
\end{tabular}
\begin{tablenotes}[para]
\item Note: The Base row shows the average RMSE of the base forecasts. Entries below this row indicate the percentage decrease (negative) or increase (positive) in the average RMSE of the reconciled forecasts compared to the base forecasts. The entries with the lowest values in each column are highlighted in blue. In each panel, the proposed methods are indicated with a gray background, and methods that outperform the benchmark method are marked in bold.
\end{tablenotes}
\end{threeparttable}}

}

\end{table}%

Focusing on the results of benchmark reconciliation methods, we find
that the BU approach performs the best in Scenarios B and C but ranks as
the worst overall in Scenario A. This is not surprising, as bottom-level
base forecasts are deteriorated in Scenario A, while higher-level base
forecasts are deteriorated in Scenarios B and C. Moreover, WLSv, MinT,
and MinTs perform especially well in Setup 1, benefiting from their
ability to consider in-sample covariance of base forecast errors,
allowing for a larger range of adjustments in reconciliation for base
forecasts with higher estimated error variance. EMinT also provides
accurate reconciled forecasts in our setup, where the in-sample
forecasts for specific series are intentionally undermined, a situation
that can be detected by the in-sample information based EMinT method.
However, OLS and WLSs significantly underperform other benchmark methods
in this simulation design.

In all three scenarios, our proposed methods consistently produce either
improved or comparable reconciled forecasts compared to their respective
benchmarks. The improvements are particularly pronounced when using OLS
and WLSs estimators of \(\bm{W}\) in the benchmark methods, which do not
take into account the in-sample covariance of base forecast errors. One
advantage of using our proposed forecast reconciliation methods with
selection is their ability to reduce the difference introduced by using
different estimates of \(\bm{W}\), thereby mitigating the risk of
estimator selection. In some cases, such as Scenarios B and C, we can
align the forecast accuracy achieved using different estimators, and
make them approach the best results we can obtain. Dropping the
unbiasedness assumption, Elasso performs similarly to EMinT overall
while achieving improvements at the top level, which is typically the
aspect of greatest concern to practitioners.

\begin{table}

\caption{\label{tbl-s1-selection}Proportion of time series being
selected after using the proposed reconciliation methods with selection
in Scenario A, Setup 1.}

\centering{

\centering\begingroup\fontsize{11}{13}\selectfont

\begin{threeparttable}
\begin{tabular}{llrrrrrr>{}r}
\toprule
  & Top & A & B & AA & AB & BA & BB & Summary\\
\midrule
OLS-subset & 0.52 & 0.79 & 0.57 & 0.79 & 1 & 0.91 & 0.85 & \includegraphics[width=0.47in, height=0.1in]{/Users/xwan0362/Git/hfs/paper/_figs/s1_OLS-subset.png}\\
OLS-intuitive & 0.80 & 0.90 & 0.81 & 0.80 & 1 & 0.85 & 0.86 & \includegraphics[width=0.47in, height=0.1in]{/Users/xwan0362/Git/hfs/paper/_figs/s1_OLS-intuitive.png}\\
OLS-lasso & 0.90 & 1.00 & 0.68 & 1.00 & 1 & 1.00 & 1.00 & \includegraphics[width=0.47in, height=0.1in]{/Users/xwan0362/Git/hfs/paper/_figs/s1_OLS-lasso.png}\\
\midrule
WLSs-subset & 0.85 & 0.91 & 0.86 & 0.90 & 1 & 0.97 & 0.97 & \includegraphics[width=0.47in, height=0.1in]{/Users/xwan0362/Git/hfs/paper/_figs/s1_WLSs-subset.png}\\
WLSs-intuitive & 0.92 & 0.95 & 0.67 & 0.92 & 1 & 0.92 & 0.95 & \includegraphics[width=0.47in, height=0.1in]{/Users/xwan0362/Git/hfs/paper/_figs/s1_WLSs-intuitive.png}\\
WLSs-lasso & 0.72 & 1.00 & 0.72 & 1.00 & 1 & 1.00 & 1.00 & \includegraphics[width=0.47in, height=0.1in]{/Users/xwan0362/Git/hfs/paper/_figs/s1_WLSs-lasso.png}\\
\midrule
WLSv-subset & 0.50 & 0.62 & 0.42 & 0.19 & 1 & 0.81 & 0.87 & \includegraphics[width=0.47in, height=0.1in]{/Users/xwan0362/Git/hfs/paper/_figs/s1_WLSv-subset.png}\\
WLSv-intuitive & 0.59 & 0.55 & 0.49 & 0.17 & 1 & 0.76 & 0.86 & \includegraphics[width=0.47in, height=0.1in]{/Users/xwan0362/Git/hfs/paper/_figs/s1_WLSv-intuitive.png}\\
WLSv-lasso & 0.40 & 1.00 & 0.41 & 0.77 & 1 & 1.00 & 1.00 & \includegraphics[width=0.47in, height=0.1in]{/Users/xwan0362/Git/hfs/paper/_figs/s1_WLSv-lasso.png}\\
\midrule
MinT-subset & 0.66 & 0.90 & 0.61 & 0.72 & 1 & 0.91 & 0.93 & \includegraphics[width=0.47in, height=0.1in]{/Users/xwan0362/Git/hfs/paper/_figs/s1_MinT-subset.png}\\
MinT-intuitive & 1.00 & 1.00 & 1.00 & 1.00 & 1 & 1.00 & 1.00 & \includegraphics[width=0.47in, height=0.1in]{/Users/xwan0362/Git/hfs/paper/_figs/s1_MinT-intuitive.png}\\
MinT-lasso & 0.80 & 0.96 & 0.84 & 0.72 & 1 & 0.98 & 0.97 & \includegraphics[width=0.47in, height=0.1in]{/Users/xwan0362/Git/hfs/paper/_figs/s1_MinT-lasso.png}\\
\midrule
MinTs-subset & 0.57 & 0.88 & 0.52 & 0.67 & 1 & 0.89 & 0.92 & \includegraphics[width=0.47in, height=0.1in]{/Users/xwan0362/Git/hfs/paper/_figs/s1_MinTs-subset.png}\\
MinTs-intuitive & 1.00 & 1.00 & 1.00 & 1.00 & 1 & 1.00 & 1.00 & \includegraphics[width=0.47in, height=0.1in]{/Users/xwan0362/Git/hfs/paper/_figs/s1_MinTs-intuitive.png}\\
MinTs-lasso & 0.68 & 1.00 & 0.66 & 0.74 & 1 & 1.00 & 1.00 & \includegraphics[width=0.47in, height=0.1in]{/Users/xwan0362/Git/hfs/paper/_figs/s1_MinTs-lasso.png}\\
\midrule
Elasso & 0.82 & 0.63 & 0.69 & 1.00 & 1 & 1.00 & 1.00 & \includegraphics[width=0.47in, height=0.1in]{/Users/xwan0362/Git/hfs/paper/_figs/s1_Elasso.png}\\
\bottomrule
\end{tabular}
\begin{tablenotes}[para]
\item Note: the last column displays a stacked barplot for each method, based on the total number of selected series data from 500 simulation instances, with a darker sub-bar indicating a larger number.
\end{tablenotes}
\end{threeparttable}
\endgroup{}

}

\end{table}%

In addition, we report the proportion of time series being selected from
our proposed methods in 500 simulation instances, as shown in
Table~\ref{tbl-s1-selection}, Table~\ref{tbl-s2-selection}, and
Table~\ref{tbl-s3-selection}. Clearly, our methods select fewer time
series, while enhancing forecast accuracy compared to benchmarks. Subset
methods, in particular, tend to return fewer time series compared to the
Intuitive and Lasso methods, which aligns with our expectations that the
Intuitive and Lasso methods tend to produce dense estimates. Most
importantly, depending on the scenario considered, the time series with
model misspecification has been selected less often than others. For
example, in Scenario A, series AA is expected to be removed while
retaining AB. This allows obtaining series AA via operations such as
A\(-\)AB, Total\(-\)B\(-\)AB, or Total\(-\)AB\(-\)BA\(-\)BB. The results
in Table~\ref{tbl-s1-selection} align with our expectations, showing
frequent exclusion of series AA and consistent selection of AB.

\subsection{Setup 2: Exploring the effect of
correlation}\label{sec-sim2}

We now simulate a hierarchical structure with correlated series, using a
similar simulation to \citet{Wickramasuriya2021-am}, and the same
hierarchical structure as shown in Figure~\ref{fig-hts}. We use a
stationary VAR(1) data generating process for the time series at the
bottom level: \[
\bm{b}_t= \bm{c} + \left[\begin{array}{cc}
\bm{A}_1 & \bm{0} \\
\bm{0} & \bm{A}_2
\end{array}\right] \bm{b}_{t-1} + \bm{\varepsilon}_t,
\] where \(\bm{c}\) is a constant vector with all entries set to \(1\),
\(\bm{A}_1\) and \(\bm{A}_2\) are \(2 \times 2\) matrices with
eigenvalues \(z_{1,2}=0.6[\cos (\pi / 3) \pm i \sin (\pi / 3)]\) and
\(z_{3,4}=0.9[\cos (\pi / 6) \pm i \sin (\pi / 6)]\), respectively,
\(\bm{\varepsilon}_t \sim \mathcal{N}(\bm{0}, \bm{\Sigma})\), where \[
\bm{\Sigma}=\left[\begin{array}{cc}
\bm{\Sigma}_1 & 0 \\0 & \bm{\Sigma}_2
\end{array}\right], \quad\text{and}\quad \bm{\Sigma}_1=\bm{\Sigma}_2=\left[\begin{array}{cc}2 & \sqrt{6} \rho \\\sqrt{6} \rho & 3\end{array}\right],
\] and \(\rho \in \{0, \pm 0.2, \pm 0.4, \pm 0.6, \pm 0.8\}\) controls
the error correlation in the simulated hierarchy.

For each time series at the bottom level, we generate a total of \(101\)
observations, with the last observation serving as the test set, i.e.,
\(T=100\) and \(h=1\). Once again, the data at the higher levels are
obtained by aggregating the bottom-level series. The process is repeated
\(500\) times for each candidate correlation, \(\rho\).

\begin{figure}[!b]

\centering{

\includegraphics{hf_selection_files/figure-pdf/fig-corr-data-1.pdf}

}

\caption{\label{fig-corr-data}An example hierarchical time series and
its in-sample residuals in Setup 2.}

\end{figure}%

For each series, base forecasts are generated from ARMA models. We
identify the best ARMA model using the automated algorithm implemented
in the \textbf{forecast} R package \citep{HK08}. Additionally, when
fitting ARMA models for time series Total, A, and BA, we introduce a
slight bias by omitting the constant term. Figure~\ref{fig-corr-data}
presents an illustrative example of a simulated hierarchical time
series. The left panels depict time plots for each series at different
levels of the structure, while the right panels show the residuals
obtained from forecasting each series using the fitted ARMA model.
Notably, despite our omission of the constant term when fitting ARMA
models to series Total, A, and BA, the residuals derived from the
identified optimal models still exhibit fluctuations around zero and do
not display significant deviations in comparison to the residuals from
other series. This is because the influence of the constant term is
minimal, i.e., it is much smaller compared to the data variability.
Thus, it may be challenging to identify the ``poor'' base forecasts and
exclude them from reconciliation in this setup.

\begin{table}

\caption{\label{tbl-corr-rmse}Out-of-sample forecast results across
various error correlations for simulation in Setup 2.}

\centering{

\centering
\resizebox{\linewidth}{!}{
\fontsize{11}{13}\selectfont
\begin{threeparttable}
\begin{tabular}{lrrrrrrrrrrrrrrrrrrrr}
\toprule
\multicolumn{1}{c}{} & \multicolumn{5}{c}{Top} & \multicolumn{5}{c}{Middle} & \multicolumn{5}{c}{Bottom} & \multicolumn{5}{c}{Average} \\
\cmidrule(l{3pt}r{3pt}){2-6} \cmidrule(l{3pt}r{3pt}){7-11} \cmidrule(l{3pt}r{3pt}){12-16} \cmidrule(l{3pt}r{3pt}){17-21}
Method & $\rho$=--0.8 & --0.4 & 0 & 0.4 & 0.8 & $\rho$=--0.8 & --0.4 & 0 & 0.4 & 0.8 & $\rho$=--0.8 & --0.4 & 0 & 0.4 & 0.8 & $\rho$=--0.8 & --0.4 & 0 & 0.4 & 0.8\\
\midrule
Base & 2.4 & 2.9 & 3.4 & 4.1 & 4.0 & 1.5 & 1.8 & 2.1 & 2.4 & 2.5 & 1.5 & 1.5 & 1.5 & 1.5 & 1.4 & 1.6 & 1.8 & 2.0 & 2.1 & 2.1\\
BU & --17.0 & --9.0 & --6.7 & --7.0 & --7.4 & --6.8 & 0.4 & 4.8 & 5.7 & 2.8 & 0.0 & 0.0 & 0.0 & 0.0 & 0.0 & --5.3 & --1.9 & --0.2 & --0.1 & --1.0\\
\midrule
OLS & --11.0 & --8.2 & --7.7 & --8.2 & --8.0 & --3.5 & --0.7 & 3.1 & 2.5 & 0.8 & 0.7 & --0.6 & --2.0 & --2.3 & --2.1 & --2.8 & --2.4 & --1.8 & --2.4 & --2.7\\
\cellcolor[HTML]{e6e3e3}{OLS-subset} & \cellcolor[HTML]{e6e3e3}{\textbf{--11.4}} & \cellcolor[HTML]{e6e3e3}{\textbf{ --8.4}} & \cellcolor[HTML]{e6e3e3}{\textbf{ --8.1}} & \cellcolor[HTML]{e6e3e3}{\textbf{ --8.4}} & \cellcolor[HTML]{e6e3e3}{\textbf{ --8.8}} & \cellcolor[HTML]{e6e3e3}{\textbf{ --3.7}} & \cellcolor[HTML]{e6e3e3}{--0.7} & \cellcolor[HTML]{e6e3e3}{3.2} & \cellcolor[HTML]{e6e3e3}{2.5} & \cellcolor[HTML]{e6e3e3}{\textbf{ 0.4}} & \cellcolor[HTML]{e6e3e3}{\textbf{ 0.3}} & \cellcolor[HTML]{e6e3e3}{\textbf{--0.8}} & \cellcolor[HTML]{e6e3e3}{--2.0} & \cellcolor[HTML]{e6e3e3}{--1.7} & \cellcolor[HTML]{e6e3e3}{\textbf{--2.6}} & \cellcolor[HTML]{e6e3e3}{\textbf{ --3.2}} & \cellcolor[HTML]{e6e3e3}{\textbf{ --2.5}} & \cellcolor[HTML]{e6e3e3}{\textbf{--1.9}} & \cellcolor[HTML]{e6e3e3}{--2.2} & \cellcolor[HTML]{e6e3e3}{\textbf{--3.2}}\\
\cellcolor[HTML]{e6e3e3}{OLS-intuitive} & \cellcolor[HTML]{e6e3e3}{\textbf{--11.6}} & \cellcolor[HTML]{e6e3e3}{--8.0} & \cellcolor[HTML]{e6e3e3}{\textbf{ --7.8}} & \cellcolor[HTML]{e6e3e3}{--8.0} & \cellcolor[HTML]{e6e3e3}{\textbf{ --8.4}} & \cellcolor[HTML]{e6e3e3}{\textbf{ --3.6}} & \cellcolor[HTML]{e6e3e3}{--0.4} & \cellcolor[HTML]{e6e3e3}{3.7} & \cellcolor[HTML]{e6e3e3}{2.5} & \cellcolor[HTML]{e6e3e3}{\textbf{ 0.3}} & \cellcolor[HTML]{e6e3e3}{\textbf{ 0.6}} & \cellcolor[HTML]{e6e3e3}{--0.2} & \cellcolor[HTML]{e6e3e3}{--1.3} & \cellcolor[HTML]{e6e3e3}{--0.4} & \cellcolor[HTML]{e6e3e3}{--1.5} & \cellcolor[HTML]{e6e3e3}{\textbf{ --3.0}} & \cellcolor[HTML]{e6e3e3}{--2.0} & \cellcolor[HTML]{e6e3e3}{--1.3} & \cellcolor[HTML]{e6e3e3}{--1.6} & \cellcolor[HTML]{e6e3e3}{\textbf{--2.8}}\\
\cellcolor[HTML]{e6e3e3}{OLS-lasso} & \cellcolor[HTML]{e6e3e3}{\textbf{--19.2}} & \cellcolor[HTML]{e6e3e3}{\textbf{ --9.8}} & \cellcolor[HTML]{e6e3e3}{--7.2} & \cellcolor[HTML]{e6e3e3}{\textbf{ --8.7}} & \cellcolor[HTML]{e6e3e3}{\textbf{ --8.2}} & \cellcolor[HTML]{e6e3e3}{\textbf{--10.5}} & \cellcolor[HTML]{e6e3e3}{\textbf{ --1.7}} & \cellcolor[HTML]{e6e3e3}{\textbf{ 2.9}} & \cellcolor[HTML]{e6e3e3}{\textbf{ 2.4}} & \cellcolor[HTML]{e6e3e3}{0.8} & \cellcolor[HTML]{e6e3e3}{\textbf{--0.8}} & \cellcolor[HTML]{e6e3e3}{\textbf{--0.8}} & \cellcolor[HTML]{e6e3e3}{--1.6} & \cellcolor[HTML]{e6e3e3}{--2.3} & \cellcolor[HTML]{e6e3e3}{--2.1} & \cellcolor[HTML]{e6e3e3}{\textbf{ --7.1}} & \cellcolor[HTML]{e6e3e3}{\textbf{ --3.1}} & \cellcolor[HTML]{e6e3e3}{--1.6} & \cellcolor[HTML]{e6e3e3}{\textbf{--2.5}} & \cellcolor[HTML]{e6e3e3}{\textbf{--2.8}}\\
\midrule
WLSs & --16.8 & --11.1 & --9.6 & --10.4 & --10.2 & --8.1 & --2.8 & 1.5 & 1.2 & --0.4 & --0.3 & --1.1 & --2.4 & --2.9 & --2.9 & --5.7 & --3.9 & --3.0 & --3.6 & --4.0\\
\cellcolor[HTML]{e6e3e3}{WLSs-subset} & \cellcolor[HTML]{e6e3e3}{\textbf{--17.3}} & \cellcolor[HTML]{e6e3e3}{\textbf{--11.4}} & \cellcolor[HTML]{e6e3e3}{\textbf{ --9.9}} & \cellcolor[HTML]{e6e3e3}{\textbf{--11.1}} & \cellcolor[HTML]{e6e3e3}{\textbf{--10.8}} & \cellcolor[HTML]{e6e3e3}{\textbf{ --8.3}} & \cellcolor[HTML]{e6e3e3}{--2.8} & \cellcolor[HTML]{e6e3e3}{\textbf{ 1.4}} & \cellcolor[HTML]{e6e3e3}{\textbf{ 0.7}} & \cellcolor[HTML]{e6e3e3}{\textbf{--0.9}} & \cellcolor[HTML]{e6e3e3}{\textbf{--0.7}} & \cellcolor[HTML]{e6e3e3}{\textbf{--1.3}} & \cellcolor[HTML]{e6e3e3}{--2.4} & \cellcolor[HTML]{e6e3e3}{\textbf{--3.2}} & \cellcolor[HTML]{e6e3e3}{\textbf{--3.3}} & \cellcolor[HTML]{e6e3e3}{\textbf{ --6.1}} & \cellcolor[HTML]{e6e3e3}{\textbf{ --4.0}} & \cellcolor[HTML]{e6e3e3}{\textbf{--3.1}} & \cellcolor[HTML]{e6e3e3}{\textbf{--4.1}} & \cellcolor[HTML]{e6e3e3}{\textbf{--4.5}}\\
\cellcolor[HTML]{e6e3e3}{WLSs-intuitive} & \cellcolor[HTML]{e6e3e3}{\textbf{--16.9}} & \cellcolor[HTML]{e6e3e3}{\textbf{--11.5}} & \cellcolor[HTML]{e6e3e3}{\textbf{ --9.8}} & \cellcolor[HTML]{e6e3e3}{--10.0} & \cellcolor[HTML]{e6e3e3}{\textbf{--10.6}} & \cellcolor[HTML]{e6e3e3}{\textbf{ --8.5}} & \cellcolor[HTML]{e6e3e3}{--2.8} & \cellcolor[HTML]{e6e3e3}{\textbf{ 1.4}} & \cellcolor[HTML]{e6e3e3}{1.5} & \cellcolor[HTML]{e6e3e3}{\textbf{--0.7}} & \cellcolor[HTML]{e6e3e3}{\textbf{--0.7}} & \cellcolor[HTML]{e6e3e3}{\textbf{--1.2}} & \cellcolor[HTML]{e6e3e3}{--2.3} & \cellcolor[HTML]{e6e3e3}{--2.7} & \cellcolor[HTML]{e6e3e3}{\textbf{--3.0}} & \cellcolor[HTML]{e6e3e3}{\textbf{ --6.1}} & \cellcolor[HTML]{e6e3e3}{\textbf{ --4.0}} & \cellcolor[HTML]{e6e3e3}{--3.0} & \cellcolor[HTML]{e6e3e3}{--3.3} & \cellcolor[HTML]{e6e3e3}{\textbf{--4.3}}\\
\cellcolor[HTML]{e6e3e3}{WLSs-lasso} & \cellcolor[HTML]{e6e3e3}{\textbf{--18.3}} & \cellcolor[HTML]{e6e3e3}{--11.1} & \cellcolor[HTML]{e6e3e3}{--9.2} & \cellcolor[HTML]{e6e3e3}{\textbf{--10.5}} & \cellcolor[HTML]{e6e3e3}{--9.8} & \cellcolor[HTML]{e6e3e3}{\textbf{ --9.3}} & \cellcolor[HTML]{e6e3e3}{--2.4} & \cellcolor[HTML]{e6e3e3}{\textbf{ 1.4}} & \cellcolor[HTML]{e6e3e3}{1.2} & \cellcolor[HTML]{e6e3e3}{--0.1} & \cellcolor[HTML]{e6e3e3}{\textbf{--0.8}} & \cellcolor[HTML]{e6e3e3}{--1.0} & \cellcolor[HTML]{e6e3e3}{--2.4} & \cellcolor[HTML]{e6e3e3}{--2.9} & \cellcolor[HTML]{e6e3e3}{--2.8} & \cellcolor[HTML]{e6e3e3}{\textbf{ --6.6}} & \cellcolor[HTML]{e6e3e3}{--3.7} & \cellcolor[HTML]{e6e3e3}{--2.9} & \cellcolor[HTML]{e6e3e3}{\textbf{--3.7}} & \cellcolor[HTML]{e6e3e3}{--3.7}\\
\midrule
WLSv & --16.5 & --11.9 & --10.0 & --10.6 & --10.6 & --7.6 & --3.4 & 0.9 & 1.1 & --0.5 & --0.5 & --1.2 & --2.3 & --2.9 & --3.0 & --5.7 & --4.3 & --3.2 & --3.7 & --4.2\\
\cellcolor[HTML]{e6e3e3}{WLSv-subset} & \cellcolor[HTML]{e6e3e3}{\textbf{--16.8}} & \cellcolor[HTML]{e6e3e3}{\textbf{--12.1}} & \cellcolor[HTML]{e6e3e3}{--9.8} & \cellcolor[HTML]{e6e3e3}{\textbf{--10.8}} & \cellcolor[HTML]{e6e3e3}{\textbf{--10.7}} & \cellcolor[HTML]{e6e3e3}{\textbf{ --7.8}} & \cellcolor[HTML]{e6e3e3}{\textbf{ --3.5}} & \cellcolor[HTML]{e6e3e3}{1.1} & \cellcolor[HTML]{e6e3e3}{1.2} & \cellcolor[HTML]{e6e3e3}{\textbf{--1.0}} & \cellcolor[HTML]{e6e3e3}{\textbf{--1.1}} & \cellcolor[HTML]{e6e3e3}{\textbf{--1.3}} & \cellcolor[HTML]{e6e3e3}{--2.2} & \cellcolor[HTML]{e6e3e3}{--2.9} & \cellcolor[HTML]{e6e3e3}{\textbf{--3.2}} & \cellcolor[HTML]{e6e3e3}{\textbf{ --6.1}} & \cellcolor[HTML]{e6e3e3}{\textbf{ --4.4}} & \cellcolor[HTML]{e6e3e3}{--3.0} & \cellcolor[HTML]{e6e3e3}{--3.7} & \cellcolor[HTML]{e6e3e3}{\textbf{--4.4}}\\
\cellcolor[HTML]{e6e3e3}{WLSv-intuitive} & \cellcolor[HTML]{e6e3e3}{\textbf{--17.6}} & \cellcolor[HTML]{e6e3e3}{\textbf{--12.6}} & \cellcolor[HTML]{e6e3e3}{\textbf{--10.1}} & \cellcolor[HTML]{e6e3e3}{--10.5} & \cellcolor[HTML]{e6e3e3}{--10.6} & \cellcolor[HTML]{e6e3e3}{\textbf{ --8.7}} & \cellcolor[HTML]{e6e3e3}{\textbf{ --3.8}} & \cellcolor[HTML]{e6e3e3}{\textbf{ 0.7}} & \cellcolor[HTML]{e6e3e3}{1.1} & \cellcolor[HTML]{e6e3e3}{\textbf{--0.8}} & \cellcolor[HTML]{e6e3e3}{\textbf{--1.9}} & \cellcolor[HTML]{e6e3e3}{\textbf{--1.5}} & \cellcolor[HTML]{e6e3e3}{--2.3} & \cellcolor[HTML]{e6e3e3}{\textbf{--3.0}} & \cellcolor[HTML]{e6e3e3}{--3.0} & \cellcolor[HTML]{e6e3e3}{\textbf{ --7.0}} & \cellcolor[HTML]{e6e3e3}{\textbf{ --4.7}} & \cellcolor[HTML]{e6e3e3}{\textbf{--3.3}} & \cellcolor[HTML]{e6e3e3}{--3.7} & \cellcolor[HTML]{e6e3e3}{\textbf{--4.3}}\\
\cellcolor[HTML]{e6e3e3}{WLSv-lasso} & \cellcolor[HTML]{e6e3e3}{\textbf{--19.8}} & \cellcolor[HTML]{e6e3e3}{--11.6} & \cellcolor[HTML]{e6e3e3}{--9.7} & \cellcolor[HTML]{e6e3e3}{--10.5} & \cellcolor[HTML]{e6e3e3}{--10.6} & \cellcolor[HTML]{e6e3e3}{\textbf{--10.5}} & \cellcolor[HTML]{e6e3e3}{--3.0} & \cellcolor[HTML]{e6e3e3}{1.2} & \cellcolor[HTML]{e6e3e3}{1.2} & \cellcolor[HTML]{e6e3e3}{--0.5} & \cellcolor[HTML]{e6e3e3}{\textbf{--1.2}} & \cellcolor[HTML]{e6e3e3}{--1.1} & \cellcolor[HTML]{e6e3e3}{--2.2} & \cellcolor[HTML]{e6e3e3}{--2.9} & \cellcolor[HTML]{e6e3e3}{--3.0} & \cellcolor[HTML]{e6e3e3}{\textbf{ --7.5}} & \cellcolor[HTML]{e6e3e3}{--4.1} & \cellcolor[HTML]{e6e3e3}{--3.0} & \cellcolor[HTML]{e6e3e3}{--3.7} & \cellcolor[HTML]{e6e3e3}{--4.2}\\
\midrule
MinT & --25.4 & --18.8 & --12.4 & \textcolor{blue}{\textbf{--15.3}} & \textcolor{blue}{\textbf{--12.6}} & --15.5 & --7.0 & 0.0 & --2.0 & --2.0 & --4.0 & --4.6 & --4.3 & --5.8 & --5.1 & --11.4 & --8.5 & --5.0 & --7.2 & --6.0\\
\cellcolor[HTML]{e6e3e3}{MinT-subset} & \cellcolor[HTML]{e6e3e3}{--25.4} & \cellcolor[HTML]{e6e3e3}{--18.8} & \cellcolor[HTML]{e6e3e3}{--12.4} & \cellcolor[HTML]{e6e3e3}{\textcolor{blue}{\textbf{--15.3}}} & \cellcolor[HTML]{e6e3e3}{\textcolor{blue}{\textbf{--12.6}}} & \cellcolor[HTML]{e6e3e3}{--15.5} & \cellcolor[HTML]{e6e3e3}{--7.0} & \cellcolor[HTML]{e6e3e3}{0.0} & \cellcolor[HTML]{e6e3e3}{--2.0} & \cellcolor[HTML]{e6e3e3}{--2.0} & \cellcolor[HTML]{e6e3e3}{--4.0} & \cellcolor[HTML]{e6e3e3}{--4.6} & \cellcolor[HTML]{e6e3e3}{--4.3} & \cellcolor[HTML]{e6e3e3}{--5.8} & \cellcolor[HTML]{e6e3e3}{--5.1} & \cellcolor[HTML]{e6e3e3}{--11.4} & \cellcolor[HTML]{e6e3e3}{--8.5} & \cellcolor[HTML]{e6e3e3}{--5.0} & \cellcolor[HTML]{e6e3e3}{--7.2} & \cellcolor[HTML]{e6e3e3}{--6.0}\\
\cellcolor[HTML]{e6e3e3}{MinT-intuitive} & \cellcolor[HTML]{e6e3e3}{--25.4} & \cellcolor[HTML]{e6e3e3}{--18.8} & \cellcolor[HTML]{e6e3e3}{--12.4} & \cellcolor[HTML]{e6e3e3}{\textcolor{blue}{\textbf{--15.3}}} & \cellcolor[HTML]{e6e3e3}{\textcolor{blue}{\textbf{--12.6}}} & \cellcolor[HTML]{e6e3e3}{--15.5} & \cellcolor[HTML]{e6e3e3}{--7.0} & \cellcolor[HTML]{e6e3e3}{0.0} & \cellcolor[HTML]{e6e3e3}{--2.0} & \cellcolor[HTML]{e6e3e3}{--2.0} & \cellcolor[HTML]{e6e3e3}{--4.0} & \cellcolor[HTML]{e6e3e3}{--4.6} & \cellcolor[HTML]{e6e3e3}{--4.3} & \cellcolor[HTML]{e6e3e3}{--5.8} & \cellcolor[HTML]{e6e3e3}{--5.1} & \cellcolor[HTML]{e6e3e3}{--11.4} & \cellcolor[HTML]{e6e3e3}{--8.5} & \cellcolor[HTML]{e6e3e3}{--5.0} & \cellcolor[HTML]{e6e3e3}{--7.2} & \cellcolor[HTML]{e6e3e3}{--6.0}\\
\cellcolor[HTML]{e6e3e3}{MinT-lasso} & \cellcolor[HTML]{e6e3e3}{--25.4} & \cellcolor[HTML]{e6e3e3}{--18.8} & \cellcolor[HTML]{e6e3e3}{--12.4} & \cellcolor[HTML]{e6e3e3}{\textcolor{blue}{\textbf{--15.3}}} & \cellcolor[HTML]{e6e3e3}{\textcolor{blue}{\textbf{--12.6}}} & \cellcolor[HTML]{e6e3e3}{--15.5} & \cellcolor[HTML]{e6e3e3}{--7.0} & \cellcolor[HTML]{e6e3e3}{0.0} & \cellcolor[HTML]{e6e3e3}{--2.0} & \cellcolor[HTML]{e6e3e3}{--2.0} & \cellcolor[HTML]{e6e3e3}{--4.0} & \cellcolor[HTML]{e6e3e3}{--4.6} & \cellcolor[HTML]{e6e3e3}{--4.3} & \cellcolor[HTML]{e6e3e3}{--5.8} & \cellcolor[HTML]{e6e3e3}{--5.1} & \cellcolor[HTML]{e6e3e3}{--11.4} & \cellcolor[HTML]{e6e3e3}{--8.5} & \cellcolor[HTML]{e6e3e3}{--5.0} & \cellcolor[HTML]{e6e3e3}{--7.2} & \cellcolor[HTML]{e6e3e3}{--6.0}\\
\midrule
MinTs & --25.4 & --17.7 & --12.1 & --14.2 & --12.5 & --16.1 & --6.8 & --0.8 & --1.6 & \textcolor{blue}{\textbf{--2.4}} & --4.0 & --4.6 & --4.9 & --5.9 & \textcolor{blue}{\textbf{--5.2}} & --11.6 & --8.2 & --5.4 & --6.8 & \textcolor{blue}{\textbf{--6.2}}\\
\cellcolor[HTML]{e6e3e3}{MinTs-subset} & \cellcolor[HTML]{e6e3e3}{--25.2} & \cellcolor[HTML]{e6e3e3}{--17.6} & \cellcolor[HTML]{e6e3e3}{--12.1} & \cellcolor[HTML]{e6e3e3}{--14.2} & \cellcolor[HTML]{e6e3e3}{--12.5} & \cellcolor[HTML]{e6e3e3}{--16.1} & \cellcolor[HTML]{e6e3e3}{--6.8} & \cellcolor[HTML]{e6e3e3}{--0.8} & \cellcolor[HTML]{e6e3e3}{--1.6} & \cellcolor[HTML]{e6e3e3}{\textcolor{blue}{\textbf{--2.4}}} & \cellcolor[HTML]{e6e3e3}{--3.9} & \cellcolor[HTML]{e6e3e3}{--4.6} & \cellcolor[HTML]{e6e3e3}{--4.9} & \cellcolor[HTML]{e6e3e3}{--5.9} & \cellcolor[HTML]{e6e3e3}{\textcolor{blue}{\textbf{--5.2}}} & \cellcolor[HTML]{e6e3e3}{--11.5} & \cellcolor[HTML]{e6e3e3}{--8.2} & \cellcolor[HTML]{e6e3e3}{--5.4} & \cellcolor[HTML]{e6e3e3}{--6.8} & \cellcolor[HTML]{e6e3e3}{\textcolor{blue}{\textbf{--6.2}}}\\
\cellcolor[HTML]{e6e3e3}{MinTs-intuitive} & \cellcolor[HTML]{e6e3e3}{--25.4} & \cellcolor[HTML]{e6e3e3}{--17.7} & \cellcolor[HTML]{e6e3e3}{--12.1} & \cellcolor[HTML]{e6e3e3}{--14.2} & \cellcolor[HTML]{e6e3e3}{--12.5} & \cellcolor[HTML]{e6e3e3}{--16.1} & \cellcolor[HTML]{e6e3e3}{--6.8} & \cellcolor[HTML]{e6e3e3}{--0.8} & \cellcolor[HTML]{e6e3e3}{--1.6} & \cellcolor[HTML]{e6e3e3}{\textcolor{blue}{\textbf{--2.4}}} & \cellcolor[HTML]{e6e3e3}{--4.0} & \cellcolor[HTML]{e6e3e3}{--4.6} & \cellcolor[HTML]{e6e3e3}{--4.9} & \cellcolor[HTML]{e6e3e3}{--5.9} & \cellcolor[HTML]{e6e3e3}{\textcolor{blue}{\textbf{--5.2}}} & \cellcolor[HTML]{e6e3e3}{--11.6} & \cellcolor[HTML]{e6e3e3}{--8.2} & \cellcolor[HTML]{e6e3e3}{--5.4} & \cellcolor[HTML]{e6e3e3}{--6.8} & \cellcolor[HTML]{e6e3e3}{\textcolor{blue}{\textbf{--6.2}}}\\
\cellcolor[HTML]{e6e3e3}{MinTs-lasso} & \cellcolor[HTML]{e6e3e3}{--25.4} & \cellcolor[HTML]{e6e3e3}{--17.6} & \cellcolor[HTML]{e6e3e3}{--12.1} & \cellcolor[HTML]{e6e3e3}{--14.2} & \cellcolor[HTML]{e6e3e3}{--12.5} & \cellcolor[HTML]{e6e3e3}{--16.1} & \cellcolor[HTML]{e6e3e3}{--6.7} & \cellcolor[HTML]{e6e3e3}{--0.8} & \cellcolor[HTML]{e6e3e3}{--1.6} & \cellcolor[HTML]{e6e3e3}{\textcolor{blue}{\textbf{--2.4}}} & \cellcolor[HTML]{e6e3e3}{--4.0} & \cellcolor[HTML]{e6e3e3}{--4.6} & \cellcolor[HTML]{e6e3e3}{--4.9} & \cellcolor[HTML]{e6e3e3}{--5.9} & \cellcolor[HTML]{e6e3e3}{\textcolor{blue}{\textbf{--5.2}}} & \cellcolor[HTML]{e6e3e3}{--11.6} & \cellcolor[HTML]{e6e3e3}{--8.2} & \cellcolor[HTML]{e6e3e3}{--5.4} & \cellcolor[HTML]{e6e3e3}{--6.8} & \cellcolor[HTML]{e6e3e3}{\textcolor{blue}{\textbf{--6.2}}}\\
\midrule
EMinT & \textcolor{blue}{\textbf{--31.2}} & \textcolor{blue}{\textbf{--19.8}} & \textcolor{blue}{\textbf{--12.5}} & --14.1 & --11.1 & \textcolor{blue}{\textbf{--22.9}} & \textcolor{blue}{\textbf{--10.9}} & \textcolor{blue}{\textbf{--2.4}} & \textcolor{blue}{\textbf{--3.2}} & --1.0 & \textcolor{blue}{\textbf{--7.4}} & \textcolor{blue}{\textbf{--7.3}} & \textcolor{blue}{\textbf{--6.9}} & \textcolor{blue}{\textbf{--7.5}} & --5.1 & \textcolor{blue}{\textbf{--16.4}} & \textcolor{blue}{\textbf{--11.2}} & \textcolor{blue}{\textbf{--6.9}} & \textcolor{blue}{\textbf{--7.9}} & --5.3\\
\cellcolor[HTML]{e6e3e3}{Elasso} & \cellcolor[HTML]{e6e3e3}{--31.0} & \cellcolor[HTML]{e6e3e3}{--19.1} & \cellcolor[HTML]{e6e3e3}{--11.1} & \cellcolor[HTML]{e6e3e3}{--13.6} & \cellcolor[HTML]{e6e3e3}{\textbf{--11.2}} & \cellcolor[HTML]{e6e3e3}{--22.7} & \cellcolor[HTML]{e6e3e3}{--9.7} & \cellcolor[HTML]{e6e3e3}{--1.8} & \cellcolor[HTML]{e6e3e3}{--2.4} & \cellcolor[HTML]{e6e3e3}{\textbf{--1.7}} & \cellcolor[HTML]{e6e3e3}{\textcolor{blue}{\textbf{--7.4}}} & \cellcolor[HTML]{e6e3e3}{--7.2} & \cellcolor[HTML]{e6e3e3}{--6.1} & \cellcolor[HTML]{e6e3e3}{--5.7} & \cellcolor[HTML]{e6e3e3}{--3.5} & \cellcolor[HTML]{e6e3e3}{--16.3} & \cellcolor[HTML]{e6e3e3}{--10.6} & \cellcolor[HTML]{e6e3e3}{--6.0} & \cellcolor[HTML]{e6e3e3}{--6.8} & \cellcolor[HTML]{e6e3e3}{--4.9}\\
\bottomrule
\end{tabular}
\begin{tablenotes}[para]
\item Note: The Base row shows the average RMSE of the base forecasts. Entries below this row indicate the percentage decrease (negative) or increase (positive) in the average RMSE of the reconciled forecasts compared to the base forecasts. The entries with the lowest values in each column are highlighted in blue. In each panel, the proposed methods are indicated with a gray background, and methods that outperform the benchmark method are marked in bold.
\end{tablenotes}
\end{threeparttable}}

}

\end{table}%

Table~\ref{tbl-corr-rmse} summarizes the average RMSE of the base
forecasts across various error correlations and the percentage relative
improvements in RMSE achieved by reconciliation methods relative to the
base forecasts. The results show that, for OLS, WLSs, and WLSv
estimators, our proposed methods consistently dominate or are equivalent
to their respective benchmark methods at all levels. We should highlight
the challenge of identifying the ``poor'' base forecasts in this
simulation design, given that the omission of the constant term has
minimal impact relative to the data variability. In addition, we observe
that the MinT and MinTs methods perform especially well and our methods
provide results similar to these benchmark methods. This is attributed
to the use of in-sample covariance by MinT and MinTs, which allows for
large adjustments in reconciliation for base forecasts with high
estimated error variance. Elasso forecasts are slightly worse than
EMinT, possibly due to the difficulty of identifying underperforming
base forecasts in this simulation setup. We have also considered
alternative error correlation values, \(\rho = -0.6, -0.2, 0.2, 0.4\),
for this simulation setting, but to save space, we do not present all
results. The omitted results follow a similar pattern and are available
upon request.

In Table~\ref{tbl-corr-selection-neg} and
Table~\ref{tbl-corr-selection-pos}, we present the proportion of time
series being selected using our proposed methods.
Table~\ref{tbl-corr-selection-neg} shows that, for OLS, WLSs, and WLSv
estimators, Subset and Intuitive methods are able to exclude the series
Total, A, and BA in some instances, in which small biases are introduced
in model fitting, while essentially retaining the remaining series.
Subset methods outperform Intuitive methods in selection. Lasso methods
typically select all bottom-level series, given their tendency to yield
dense estimates, as discussed in Section~\ref{sec-constrained}. Elasso
also selects all bottom-level series. When dealing with a high positive
error correlation, Table~\ref{tbl-corr-selection-pos} shows that our
methods still show potential for selection but it becomes somewhat
challenging to identify and exclude the series that should be omitted in
reconciliation. Hence, our methods are preferred, especially when the
error correlation within the structure is negative.

\begin{table}

\caption{\label{tbl-corr-selection-neg}Proportion of time series being
selected after using the proposed reconciliation methods with selection
in Setup 2, with the error correlation being -0.8.}

\centering{

\centering\begingroup\fontsize{11}{13}\selectfont

\begin{threeparttable}
\begin{tabular}{llrrrrrr>{}r}
\toprule
  & Top & A & B & AA & AB & BA & BB & Summary\\
\midrule
OLS-subset & 0.32 & 0.34 & 0.95 & 0.98 & 1 & 0.74 & 1.00 & \includegraphics[width=0.47in, height=0.1in]{/Users/xwan0362/Git/hfs/paper/_figs/corr_neg_OLS-subset.png}\\
OLS-intuitive & 0.58 & 0.52 & 0.93 & 0.97 & 1 & 0.61 & 0.97 & \includegraphics[width=0.47in, height=0.1in]{/Users/xwan0362/Git/hfs/paper/_figs/corr_neg_OLS-intuitive.png}\\
OLS-lasso & 0.61 & 0.34 & 0.38 & 1.00 & 1 & 1.00 & 1.00 & \includegraphics[width=0.47in, height=0.1in]{/Users/xwan0362/Git/hfs/paper/_figs/corr_neg_OLS-lasso.png}\\
\midrule
WLSs-subset & 0.27 & 0.40 & 0.98 & 1.00 & 1 & 0.73 & 1.00 & \includegraphics[width=0.47in, height=0.1in]{/Users/xwan0362/Git/hfs/paper/_figs/corr_neg_WLSs-subset.png}\\
WLSs-intuitive & 0.49 & 0.57 & 0.96 & 1.00 & 1 & 0.74 & 0.99 & \includegraphics[width=0.47in, height=0.1in]{/Users/xwan0362/Git/hfs/paper/_figs/corr_neg_WLSs-intuitive.png}\\
WLSs-lasso & 0.48 & 0.62 & 0.72 & 1.00 & 1 & 1.00 & 1.00 & \includegraphics[width=0.47in, height=0.1in]{/Users/xwan0362/Git/hfs/paper/_figs/corr_neg_WLSs-lasso.png}\\
\midrule
WLSv-subset & 0.30 & 0.42 & 1.00 & 1.00 & 1 & 0.68 & 1.00 & \includegraphics[width=0.47in, height=0.1in]{/Users/xwan0362/Git/hfs/paper/_figs/corr_neg_WLSv-subset.png}\\
WLSv-intuitive & 0.49 & 0.53 & 0.99 & 1.00 & 1 & 0.47 & 1.00 & \includegraphics[width=0.47in, height=0.1in]{/Users/xwan0362/Git/hfs/paper/_figs/corr_neg_WLSv-intuitive.png}\\
WLSv-lasso & 0.35 & 0.70 & 0.85 & 1.00 & 1 & 1.00 & 1.00 & \includegraphics[width=0.47in, height=0.1in]{/Users/xwan0362/Git/hfs/paper/_figs/corr_neg_WLSv-lasso.png}\\
\midrule
MinT-subset & 1.00 & 1.00 & 1.00 & 1.00 & 1 & 1.00 & 1.00 & \includegraphics[width=0.47in, height=0.1in]{/Users/xwan0362/Git/hfs/paper/_figs/corr_neg_MinT-subset.png}\\
MinT-intuitive & 1.00 & 1.00 & 1.00 & 1.00 & 1 & 1.00 & 1.00 & \includegraphics[width=0.47in, height=0.1in]{/Users/xwan0362/Git/hfs/paper/_figs/corr_neg_MinT-intuitive.png}\\
MinT-lasso & 1.00 & 1.00 & 1.00 & 1.00 & 1 & 1.00 & 1.00 & \includegraphics[width=0.47in, height=0.1in]{/Users/xwan0362/Git/hfs/paper/_figs/corr_neg_MinT-lasso.png}\\
\midrule
MinTs-subset & 0.87 & 0.85 & 1.00 & 1.00 & 1 & 0.85 & 1.00 & \includegraphics[width=0.47in, height=0.1in]{/Users/xwan0362/Git/hfs/paper/_figs/corr_neg_MinTs-subset.png}\\
MinTs-intuitive & 1.00 & 1.00 & 1.00 & 1.00 & 1 & 1.00 & 1.00 & \includegraphics[width=0.47in, height=0.1in]{/Users/xwan0362/Git/hfs/paper/_figs/corr_neg_MinTs-intuitive.png}\\
MinTs-lasso & 0.86 & 0.84 & 1.00 & 1.00 & 1 & 0.85 & 1.00 & \includegraphics[width=0.47in, height=0.1in]{/Users/xwan0362/Git/hfs/paper/_figs/corr_neg_MinTs-lasso.png}\\
\midrule
Elasso & 0.94 & 0.79 & 0.93 & 1.00 & 1 & 1.00 & 1.00 & \includegraphics[width=0.47in, height=0.1in]{/Users/xwan0362/Git/hfs/paper/_figs/corr_neg_Elasso.png}\\
\bottomrule
\end{tabular}
\begin{tablenotes}[para]
\item Note: the last column displays a stacked barplot for each method, based on the total number of selected series data from 500 simulation instances, with a darker sub-bar indicating a larger number.
\end{tablenotes}
\end{threeparttable}
\endgroup{}

}

\end{table}%

\section{Applications}\label{sec-applications}

In this section we describe two empirical applications:
Section~\ref{sec-labour} focuses on a grouped hierarchy built using the
Australian labour force survey data released by the Australian Bureau of
Statistics, while Section~\ref{sec-tourism} considers Australian
domestic tourism flows with a natural geographic hierarchy.

\subsection{Forecasting Australian labour force}\label{sec-labour}

The dataset from the Labour Force Survey was released by the Australian
Bureau of Statistics, and comprises monthly data on the number of
unemployed persons in Australia from January 2010 to July
2023\footnote{The Labour Force Survey data is publicly available at
  \url{https://www.abs.gov.au/statistics/labour/employment-and-unemployment/labour-force-australia-detailed/aug-2023}.}.
To deal with the few missing observations, we use linear interpolation.
Analyzing unemployment data by labour market region and duration of job
search offers valuable insights into regional disparities, and the
structural nuances underlying unemployment. Forecast reconciliation is
crucial in such a case to ensure aligned decision making.

We construct a grouped hierarchy by disaggregating the number of
unemployed persons over two independent attributes, duration of job
search (referred to as \emph{Duration}), and State and Territory
(referred to as \emph{STT} ). At the bottom level, the data are
disaggregated by both attributes. We refer to the bottom level as the
\emph{Duration} \(\times\) \emph{STT} level. Specifically, there are six
different groups of job search duration, under 1 month, 1--3 months,
3--6 months, 6--12 months, 1--2 years, and 2 years and over.
Additionally, the number of unemployed persons in Australia can be
disaggregated by eight states and territories, i.e., NSW (New South
Wales), VIC (Victoria), QLD (Queensland), SA (South Australia), WA
(Western Australia), TAS (Tasmania), NT (Northern Territory), and ACT
(Australian Capital Territory). So the final grouped hierarchy consists
of the top series, six series at the Duration level, eight series at the
STT level, and \(48\) series at the Duration \(\times\) STT level,
giving \(63\) time series in total, each of length \(163\) observations.

\begin{figure}[!t]

\centering{

\includegraphics{hf_selection_files/figure-pdf/fig-labour-data-1.pdf}

}

\caption{\label{fig-labour-data}Australia unemployed persons,
disaggregated by state and territory, and by duration of job search.}

\end{figure}%

The top panel in Figure~\ref{fig-labour-data} shows the total number of
unemployed persons in Australia from January 2010 to July 2023,
representing the top-level series in the hierarchical structure. The
monthly series shows strong seasonality within each year, marked by
prominent peaks occurring every January, attributable to school-leavers.
Lower peaks occur in July, perhaps impacted by the start of the
financial year. Amidst the backdrop of COVID-19's non-essential service
shutdowns and trading restrictions, March and April 2020 saw a notable
surge in unemployment. However, as coronavirus cases dwindled
significantly and restrictions eased in the aftermath, employment made a
remarkable recovery, leading to a subsequent decline in unemployment.
The bottom-left panel displays the breakdown of unemployed individuals
by state and territory, while the bottom-right panel presents the
breakdown by the duration of job search. The plots display diverse and
rich dynamics both within and between different levels of the hierarchy.
For example, there was noticeable growth observed during 2020 for some
states such as NSW, VIC, and QLD, whereas other states did not
experience such significant growth. Additionally, there is a resemblance
in the seasonal patterns between NSW and QLD, while the seasonal pattern
in VIC differs. When comparing the series at the STT level and Duration
level, the seasonal patterns in the Duration-level series are more
consistent and potentially easier to forecast.

We assess the forecast accuracy of base forecasts and various
reconciliation methods through a rolling forecast origin approach. Our
aim is to generate \(1\)- to \(12\)-step-ahead forecasts for each of the
\(63\) series while ensuring coherence. Given the limited data compared
to the forecast horizon, we initiate the process with a training set of
\(139\) observations for each series. The training set is used to select
the optimal ETS model with the automatic algorithm implemented in the
\textbf{forecast} package for R \citep{HK08}. Using these fitted ETS
models, we generate base forecasts, and perform diverse forecast
reconciliation methods. Then we roll the forecast origin forward by one
month and repeat the process, until July 2022. We note that it may be
challenging to identify the series with ``poor'' forecasts due to
structural changes in the data caused by the COVID-19 pandemic, which
affect the accuracy of forecasts across all time series.

\begin{table}

\caption{\label{tbl-labour-rmse-avg}Average out-of-sample forecast
results for Australian labour force data.}

\centering{

\centering
\resizebox{\linewidth}{!}{
\fontsize{11}{13}\selectfont
\begin{threeparttable}
\begin{tabular}{lrrrrrrrrrrrrrrrrrrrr}
\toprule
\multicolumn{1}{c}{} & \multicolumn{4}{c}{Top} & \multicolumn{4}{c}{Duration} & \multicolumn{4}{c}{STT} & \multicolumn{4}{c}{Duration x STT} & \multicolumn{4}{c}{Average} \\
\cmidrule(l{3pt}r{3pt}){2-5} \cmidrule(l{3pt}r{3pt}){6-9} \cmidrule(l{3pt}r{3pt}){10-13} \cmidrule(l{3pt}r{3pt}){14-17} \cmidrule(l{3pt}r{3pt}){18-21}
Method & h=1 & 1--4 & 1--8 & 1--12 & h=1 & 1--4 & 1--8 & 1--12 & h=1 & 1--4 & 1--8 & 1--12 & h=1 & 1--4 & 1--8 & 1--12 & h=1 & 1--4 & 1--8 & 1--12\\
\midrule
Base & 29.4 & 44.9 & 58.6 & 67.6 & 10.1 & 14.2 & 16.3 & 18.1 & 6.6 & 8.4 & 9.9 & 10.7 & 2.3 & 2.9 & 3.1 & 3.3 & 4.0 & 5.3 & 6.1 & 6.6\\
BU & 46.7 & 34.1 & 29.3 & 24.2 & 7.4 & 2.3 & 0.8 & 0.8 & 5.1 & 9.8 & 10.5 & 10.4 & 0.0 & 0.0 & 0.0 & 0.0 & 8.4 & 7.1 & 6.8 & 6.3\\
\midrule
OLS & 2.0 & 1.7 & 1.5 & 1.0 & 0.6 & --4.2 & --4.3 & --3.3 & --0.7 & 0.4 & 0.0 & --0.1 & 1.9 & 0.7 & 0.8 & 0.7 & 1.0 & --0.5 & --0.6 & --0.5\\
\cellcolor[HTML]{e6e3e3}{OLS-subset} & \cellcolor[HTML]{e6e3e3}{2.1} & \cellcolor[HTML]{e6e3e3}{\textbf{ 1.0}} & \cellcolor[HTML]{e6e3e3}{\textbf{--1.2}} & \cellcolor[HTML]{e6e3e3}{\textbf{--2.0}} & \cellcolor[HTML]{e6e3e3}{0.6} & \cellcolor[HTML]{e6e3e3}{--4.1} & \cellcolor[HTML]{e6e3e3}{\textbf{--5.2}} & \cellcolor[HTML]{e6e3e3}{\textbf{--4.4}} & \cellcolor[HTML]{e6e3e3}{\textbf{ --1.0}} & \cellcolor[HTML]{e6e3e3}{0.6} & \cellcolor[HTML]{e6e3e3}{\textbf{ --0.6}} & \cellcolor[HTML]{e6e3e3}{\textbf{--1.2}} & \cellcolor[HTML]{e6e3e3}{1.9} & \cellcolor[HTML]{e6e3e3}{0.8} & \cellcolor[HTML]{e6e3e3}{\textbf{ 0.3}} & \cellcolor[HTML]{e6e3e3}{\textbf{ 0.2}} & \cellcolor[HTML]{e6e3e3}{1.0} & \cellcolor[HTML]{e6e3e3}{--0.5} & \cellcolor[HTML]{e6e3e3}{\textcolor{blue}{\textbf{--1.5}}} & \cellcolor[HTML]{e6e3e3}{\textbf{--1.6}}\\
\cellcolor[HTML]{e6e3e3}{OLS-intuitive} & \cellcolor[HTML]{e6e3e3}{\textbf{--1.3}} & \cellcolor[HTML]{e6e3e3}{\textbf{ 1.5}} & \cellcolor[HTML]{e6e3e3}{\textbf{ 1.0}} & \cellcolor[HTML]{e6e3e3}{\textbf{ 0.2}} & \cellcolor[HTML]{e6e3e3}{\textbf{--0.9}} & \cellcolor[HTML]{e6e3e3}{--3.9} & \cellcolor[HTML]{e6e3e3}{--4.2} & \cellcolor[HTML]{e6e3e3}{--3.2} & \cellcolor[HTML]{e6e3e3}{\textbf{ --1.3}} & \cellcolor[HTML]{e6e3e3}{0.8} & \cellcolor[HTML]{e6e3e3}{0.5} & \cellcolor[HTML]{e6e3e3}{0.6} & \cellcolor[HTML]{e6e3e3}{\textbf{ 1.8}} & \cellcolor[HTML]{e6e3e3}{1.2} & \cellcolor[HTML]{e6e3e3}{1.3} & \cellcolor[HTML]{e6e3e3}{1.2} & \cellcolor[HTML]{e6e3e3}{\textcolor{blue}{\textbf{ 0.1}}} & \cellcolor[HTML]{e6e3e3}{--0.1} & \cellcolor[HTML]{e6e3e3}{--0.3} & \cellcolor[HTML]{e6e3e3}{--0.2}\\
\cellcolor[HTML]{e6e3e3}{OLS-lasso} & \cellcolor[HTML]{e6e3e3}{2.0} & \cellcolor[HTML]{e6e3e3}{1.7} & \cellcolor[HTML]{e6e3e3}{1.5} & \cellcolor[HTML]{e6e3e3}{1.0} & \cellcolor[HTML]{e6e3e3}{0.6} & \cellcolor[HTML]{e6e3e3}{--4.2} & \cellcolor[HTML]{e6e3e3}{--4.3} & \cellcolor[HTML]{e6e3e3}{--3.3} & \cellcolor[HTML]{e6e3e3}{--0.7} & \cellcolor[HTML]{e6e3e3}{0.4} & \cellcolor[HTML]{e6e3e3}{0.0} & \cellcolor[HTML]{e6e3e3}{--0.1} & \cellcolor[HTML]{e6e3e3}{1.9} & \cellcolor[HTML]{e6e3e3}{0.7} & \cellcolor[HTML]{e6e3e3}{0.8} & \cellcolor[HTML]{e6e3e3}{0.7} & \cellcolor[HTML]{e6e3e3}{1.0} & \cellcolor[HTML]{e6e3e3}{--0.5} & \cellcolor[HTML]{e6e3e3}{--0.6} & \cellcolor[HTML]{e6e3e3}{--0.5}\\
\midrule
WLSs & 16.9 & 10.8 & 9.0 & 7.0 & --0.7 & --4.8 & --5.0 & --4.3 & --3.0 & 0.6 & 1.4 & 1.6 & --1.3 & \textcolor{blue}{\textbf{--1.7}} & \textcolor{blue}{\textbf{--1.5}} & \textcolor{blue}{\textbf{--1.6}} & 0.6 & --0.3 & --0.2 & --0.3\\
\cellcolor[HTML]{e6e3e3}{WLSs-subset} & \cellcolor[HTML]{e6e3e3}{\textbf{14.9}} & \cellcolor[HTML]{e6e3e3}{\textbf{ 9.7}} & \cellcolor[HTML]{e6e3e3}{\textbf{ 6.4}} & \cellcolor[HTML]{e6e3e3}{\textbf{ 4.2}} & \cellcolor[HTML]{e6e3e3}{\textcolor{blue}{\textbf{--1.6}}} & \cellcolor[HTML]{e6e3e3}{--3.7} & \cellcolor[HTML]{e6e3e3}{--4.1} & \cellcolor[HTML]{e6e3e3}{--3.4} & \cellcolor[HTML]{e6e3e3}{--2.3} & \cellcolor[HTML]{e6e3e3}{1.5} & \cellcolor[HTML]{e6e3e3}{1.6} & \cellcolor[HTML]{e6e3e3}{\textbf{ 1.0}} & \cellcolor[HTML]{e6e3e3}{--0.6} & \cellcolor[HTML]{e6e3e3}{--0.2} & \cellcolor[HTML]{e6e3e3}{0.5} & \cellcolor[HTML]{e6e3e3}{0.3} & \cellcolor[HTML]{e6e3e3}{0.6} & \cellcolor[HTML]{e6e3e3}{0.6} & \cellcolor[HTML]{e6e3e3}{0.4} & \cellcolor[HTML]{e6e3e3}{0.1}\\
\cellcolor[HTML]{e6e3e3}{WLSs-intuitive} & \cellcolor[HTML]{e6e3e3}{16.9} & \cellcolor[HTML]{e6e3e3}{10.8} & \cellcolor[HTML]{e6e3e3}{9.0} & \cellcolor[HTML]{e6e3e3}{7.0} & \cellcolor[HTML]{e6e3e3}{--0.7} & \cellcolor[HTML]{e6e3e3}{--4.8} & \cellcolor[HTML]{e6e3e3}{--5.0} & \cellcolor[HTML]{e6e3e3}{--4.3} & \cellcolor[HTML]{e6e3e3}{--3.0} & \cellcolor[HTML]{e6e3e3}{0.6} & \cellcolor[HTML]{e6e3e3}{1.4} & \cellcolor[HTML]{e6e3e3}{1.6} & \cellcolor[HTML]{e6e3e3}{--1.3} & \cellcolor[HTML]{e6e3e3}{\textcolor{blue}{\textbf{--1.7}}} & \cellcolor[HTML]{e6e3e3}{\textcolor{blue}{\textbf{--1.5}}} & \cellcolor[HTML]{e6e3e3}{\textcolor{blue}{\textbf{--1.6}}} & \cellcolor[HTML]{e6e3e3}{0.6} & \cellcolor[HTML]{e6e3e3}{--0.3} & \cellcolor[HTML]{e6e3e3}{--0.2} & \cellcolor[HTML]{e6e3e3}{--0.3}\\
\cellcolor[HTML]{e6e3e3}{WLSs-lasso} & \cellcolor[HTML]{e6e3e3}{16.9} & \cellcolor[HTML]{e6e3e3}{10.8} & \cellcolor[HTML]{e6e3e3}{9.0} & \cellcolor[HTML]{e6e3e3}{7.0} & \cellcolor[HTML]{e6e3e3}{--0.7} & \cellcolor[HTML]{e6e3e3}{--4.8} & \cellcolor[HTML]{e6e3e3}{--5.0} & \cellcolor[HTML]{e6e3e3}{--4.3} & \cellcolor[HTML]{e6e3e3}{--3.0} & \cellcolor[HTML]{e6e3e3}{0.6} & \cellcolor[HTML]{e6e3e3}{1.4} & \cellcolor[HTML]{e6e3e3}{1.6} & \cellcolor[HTML]{e6e3e3}{--1.3} & \cellcolor[HTML]{e6e3e3}{\textcolor{blue}{\textbf{--1.7}}} & \cellcolor[HTML]{e6e3e3}{\textcolor{blue}{\textbf{--1.5}}} & \cellcolor[HTML]{e6e3e3}{\textcolor{blue}{\textbf{--1.6}}} & \cellcolor[HTML]{e6e3e3}{0.6} & \cellcolor[HTML]{e6e3e3}{--0.3} & \cellcolor[HTML]{e6e3e3}{--0.2} & \cellcolor[HTML]{e6e3e3}{--0.3}\\
\midrule
WLSv & 15.6 & 9.8 & 8.4 & 6.6 & 1.1 & --5.0 & \textcolor{blue}{\textbf{--5.6}} & --4.5 & --2.6 & --0.5 & --0.1 & 0.2 & \textcolor{blue}{\textbf{--1.5}} & --1.5 & \textcolor{blue}{\textbf{--1.5}} & --1.3 & 0.9 & --0.7 & --0.8 & --0.6\\
\cellcolor[HTML]{e6e3e3}{WLSv-subset} & \cellcolor[HTML]{e6e3e3}{\textbf{10.2}} & \cellcolor[HTML]{e6e3e3}{\textbf{ 5.1}} & \cellcolor[HTML]{e6e3e3}{\textbf{ 2.9}} & \cellcolor[HTML]{e6e3e3}{\textbf{ 1.8}} & \cellcolor[HTML]{e6e3e3}{\textbf{--0.6}} & \cellcolor[HTML]{e6e3e3}{\textcolor{blue}{\textbf{--5.4}}} & \cellcolor[HTML]{e6e3e3}{--5.5} & \cellcolor[HTML]{e6e3e3}{\textcolor{blue}{\textbf{--4.6}}} & \cellcolor[HTML]{e6e3e3}{--1.8} & \cellcolor[HTML]{e6e3e3}{\textbf{--0.8}} & \cellcolor[HTML]{e6e3e3}{\textbf{ --1.1}} & \cellcolor[HTML]{e6e3e3}{\textbf{--1.0}} & \cellcolor[HTML]{e6e3e3}{--1.1} & \cellcolor[HTML]{e6e3e3}{--1.1} & \cellcolor[HTML]{e6e3e3}{--0.7} & \cellcolor[HTML]{e6e3e3}{--0.5} & \cellcolor[HTML]{e6e3e3}{\textbf{ 0.2}} & \cellcolor[HTML]{e6e3e3}{\textcolor{blue}{\textbf{--1.3}}} & \cellcolor[HTML]{e6e3e3}{\textcolor{blue}{\textbf{--1.5}}} & \cellcolor[HTML]{e6e3e3}{\textbf{--1.3}}\\
\cellcolor[HTML]{e6e3e3}{WLSv-intuitive} & \cellcolor[HTML]{e6e3e3}{15.6} & \cellcolor[HTML]{e6e3e3}{9.8} & \cellcolor[HTML]{e6e3e3}{8.4} & \cellcolor[HTML]{e6e3e3}{6.6} & \cellcolor[HTML]{e6e3e3}{1.1} & \cellcolor[HTML]{e6e3e3}{--5.0} & \cellcolor[HTML]{e6e3e3}{\textcolor{blue}{\textbf{--5.6}}} & \cellcolor[HTML]{e6e3e3}{--4.5} & \cellcolor[HTML]{e6e3e3}{--2.6} & \cellcolor[HTML]{e6e3e3}{--0.5} & \cellcolor[HTML]{e6e3e3}{--0.1} & \cellcolor[HTML]{e6e3e3}{0.2} & \cellcolor[HTML]{e6e3e3}{\textcolor{blue}{\textbf{--1.5}}} & \cellcolor[HTML]{e6e3e3}{--1.5} & \cellcolor[HTML]{e6e3e3}{\textcolor{blue}{\textbf{--1.5}}} & \cellcolor[HTML]{e6e3e3}{--1.3} & \cellcolor[HTML]{e6e3e3}{0.9} & \cellcolor[HTML]{e6e3e3}{--0.7} & \cellcolor[HTML]{e6e3e3}{--0.8} & \cellcolor[HTML]{e6e3e3}{--0.6}\\
\cellcolor[HTML]{e6e3e3}{WLSv-lasso} & \cellcolor[HTML]{e6e3e3}{15.6} & \cellcolor[HTML]{e6e3e3}{9.8} & \cellcolor[HTML]{e6e3e3}{8.4} & \cellcolor[HTML]{e6e3e3}{6.6} & \cellcolor[HTML]{e6e3e3}{1.1} & \cellcolor[HTML]{e6e3e3}{--5.0} & \cellcolor[HTML]{e6e3e3}{\textcolor{blue}{\textbf{--5.6}}} & \cellcolor[HTML]{e6e3e3}{--4.5} & \cellcolor[HTML]{e6e3e3}{--2.6} & \cellcolor[HTML]{e6e3e3}{--0.5} & \cellcolor[HTML]{e6e3e3}{--0.1} & \cellcolor[HTML]{e6e3e3}{0.2} & \cellcolor[HTML]{e6e3e3}{\textcolor{blue}{\textbf{--1.5}}} & \cellcolor[HTML]{e6e3e3}{--1.5} & \cellcolor[HTML]{e6e3e3}{\textcolor{blue}{\textbf{--1.5}}} & \cellcolor[HTML]{e6e3e3}{--1.3} & \cellcolor[HTML]{e6e3e3}{0.9} & \cellcolor[HTML]{e6e3e3}{--0.7} & \cellcolor[HTML]{e6e3e3}{--0.8} & \cellcolor[HTML]{e6e3e3}{--0.6}\\
\midrule
MinTs & 9.9 & 5.8 & 6.4 & 5.3 & 0.6 & --5.0 & --5.0 & --3.7 & --3.4 & --2.4 & --1.4 & --0.8 & --0.4 & --0.9 & --0.8 & --0.7 & 0.4 & \textcolor{blue}{\textbf{--1.3}} & --0.9 & --0.5\\
\cellcolor[HTML]{e6e3e3}{MinTs-subset} & \cellcolor[HTML]{e6e3e3}{\textbf{ 9.1}} & \cellcolor[HTML]{e6e3e3}{6.1} & \cellcolor[HTML]{e6e3e3}{6.5} & \cellcolor[HTML]{e6e3e3}{\textbf{ 4.5}} & \cellcolor[HTML]{e6e3e3}{\textbf{ 0.3}} & \cellcolor[HTML]{e6e3e3}{--4.8} & \cellcolor[HTML]{e6e3e3}{\textbf{--5.1}} & \cellcolor[HTML]{e6e3e3}{\textbf{--3.9}} & \cellcolor[HTML]{e6e3e3}{\textbf{ --3.6}} & \cellcolor[HTML]{e6e3e3}{--2.4} & \cellcolor[HTML]{e6e3e3}{--1.3} & \cellcolor[HTML]{e6e3e3}{\textbf{--0.9}} & \cellcolor[HTML]{e6e3e3}{\textbf{--0.6}} & \cellcolor[HTML]{e6e3e3}{--0.8} & \cellcolor[HTML]{e6e3e3}{--0.7} & \cellcolor[HTML]{e6e3e3}{--0.7} & \cellcolor[HTML]{e6e3e3}{\textcolor{blue}{\textbf{ 0.1}}} & \cellcolor[HTML]{e6e3e3}{--1.2} & \cellcolor[HTML]{e6e3e3}{--0.9} & \cellcolor[HTML]{e6e3e3}{\textbf{--0.7}}\\
\cellcolor[HTML]{e6e3e3}{MinTs-intuitive} & \cellcolor[HTML]{e6e3e3}{9.9} & \cellcolor[HTML]{e6e3e3}{5.8} & \cellcolor[HTML]{e6e3e3}{6.4} & \cellcolor[HTML]{e6e3e3}{5.3} & \cellcolor[HTML]{e6e3e3}{0.6} & \cellcolor[HTML]{e6e3e3}{--5.0} & \cellcolor[HTML]{e6e3e3}{--5.0} & \cellcolor[HTML]{e6e3e3}{--3.7} & \cellcolor[HTML]{e6e3e3}{--3.4} & \cellcolor[HTML]{e6e3e3}{--2.4} & \cellcolor[HTML]{e6e3e3}{--1.4} & \cellcolor[HTML]{e6e3e3}{--0.8} & \cellcolor[HTML]{e6e3e3}{--0.4} & \cellcolor[HTML]{e6e3e3}{--0.9} & \cellcolor[HTML]{e6e3e3}{--0.8} & \cellcolor[HTML]{e6e3e3}{--0.7} & \cellcolor[HTML]{e6e3e3}{0.4} & \cellcolor[HTML]{e6e3e3}{\textcolor{blue}{\textbf{--1.3}}} & \cellcolor[HTML]{e6e3e3}{--0.9} & \cellcolor[HTML]{e6e3e3}{--0.5}\\
\cellcolor[HTML]{e6e3e3}{MinTs-lasso} & \cellcolor[HTML]{e6e3e3}{9.9} & \cellcolor[HTML]{e6e3e3}{5.8} & \cellcolor[HTML]{e6e3e3}{6.4} & \cellcolor[HTML]{e6e3e3}{5.3} & \cellcolor[HTML]{e6e3e3}{0.6} & \cellcolor[HTML]{e6e3e3}{--5.0} & \cellcolor[HTML]{e6e3e3}{--5.0} & \cellcolor[HTML]{e6e3e3}{--3.7} & \cellcolor[HTML]{e6e3e3}{--3.4} & \cellcolor[HTML]{e6e3e3}{--2.4} & \cellcolor[HTML]{e6e3e3}{--1.4} & \cellcolor[HTML]{e6e3e3}{--0.8} & \cellcolor[HTML]{e6e3e3}{--0.4} & \cellcolor[HTML]{e6e3e3}{--0.9} & \cellcolor[HTML]{e6e3e3}{--0.8} & \cellcolor[HTML]{e6e3e3}{--0.7} & \cellcolor[HTML]{e6e3e3}{0.4} & \cellcolor[HTML]{e6e3e3}{\textcolor{blue}{\textbf{--1.3}}} & \cellcolor[HTML]{e6e3e3}{--0.9} & \cellcolor[HTML]{e6e3e3}{--0.5}\\
\midrule
EMinT & 43.1 & 17.6 & 9.9 & 10.2 & 36.8 & 25.2 & 27.9 & 24.1 & 16.8 & 15.1 & 6.2 & 6.2 & 32.3 & 27.9 & 29.8 & 27.9 & 31.4 & 23.3 & 21.4 & 19.6\\
\cellcolor[HTML]{e6e3e3}{Elasso} & \cellcolor[HTML]{e6e3e3}{\textcolor{blue}{\textbf{--5.8}}} & \cellcolor[HTML]{e6e3e3}{\textcolor{blue}{\textbf{--2.0}}} & \cellcolor[HTML]{e6e3e3}{\textcolor{blue}{\textbf{--2.5}}} & \cellcolor[HTML]{e6e3e3}{\textcolor{blue}{\textbf{--2.3}}} & \cellcolor[HTML]{e6e3e3}{\textbf{33.5}} & \cellcolor[HTML]{e6e3e3}{\textbf{11.1}} & \cellcolor[HTML]{e6e3e3}{\textbf{ 1.1}} & \cellcolor[HTML]{e6e3e3}{\textbf{--3.6}} & \cellcolor[HTML]{e6e3e3}{\textcolor{blue}{\textbf{--17.4}}} & \cellcolor[HTML]{e6e3e3}{\textcolor{blue}{\textbf{--8.9}}} & \cellcolor[HTML]{e6e3e3}{\textcolor{blue}{\textbf{--10.0}}} & \cellcolor[HTML]{e6e3e3}{\textcolor{blue}{\textbf{--8.8}}} & \cellcolor[HTML]{e6e3e3}{\textbf{20.6}} & \cellcolor[HTML]{e6e3e3}{\textbf{ 6.4}} & \cellcolor[HTML]{e6e3e3}{\textbf{ 2.6}} & \cellcolor[HTML]{e6e3e3}{\textbf{ 0.5}} & \cellcolor[HTML]{e6e3e3}{\textbf{12.7}} & \cellcolor[HTML]{e6e3e3}{\textbf{ 3.4}} & \cellcolor[HTML]{e6e3e3}{\textbf{--1.1}} & \cellcolor[HTML]{e6e3e3}{\textcolor{blue}{\textbf{--2.9}}}\\
\bottomrule
\end{tabular}
\begin{tablenotes}[para]
\item Note: The Base row shows the average RMSE of the base forecasts. Entries below this row indicate the percentage decrease (negative) or increase (positive) in the average RMSE of the reconciled forecasts compared to the base forecasts. The entries with the lowest values in each column are highlighted in blue. In each panel, the proposed methods are indicated with a gray background, and methods that outperform the benchmark method are marked in bold.
\end{tablenotes}
\end{threeparttable}}

}

\end{table}%

The average results are presented in Table~\ref{tbl-labour-rmse-avg}.
The poor performance of the MinT method and associated methods can be
attributed to the poor sample covariance estimator when the sample size
is only slightly larger than the number of series in the structure.
Subset methods using different estimators of \(\bm{G}\) generally
improve forecast accuracy over their benchmark methods, particularly
when focusing on aggregation levels, which are typically of paramount
concern to practitioners. The only exception is the WLSs-subset method,
which returns reduced accuracy for longer horizons. However, it still
demonstrates improvements in top-level forecasts. Moreover, the
Intuitive and Lasso methods almost always yield results identical to the
corresponding benchmark methods, because they tend to provide dense
estimates, and ETS models typically do not result in extremely poor
forecasts. The only exception is OLS-intuitive, which shows improved
forecast accuracy at the top level but deterioration at other levels.
When we drop the unbiasedness assumption, EMinT is the worst-performing
method across all levels because it relies on the assumption that the
series in the hierarchy are jointly weakly stationary, which is
evidently not the case in the application. Elasso significantly improves
the quality of forecasts over EMinT, with the most accurate coherent
forecasts observed at the top level and STT level. Overall, Elasso
performs well for longer forecast horizons, but it is less effective for
one-step-ahead forecasts.

We provide results based on the final test set spanning from August 2022
to July 2023 in Table~\ref{tbl-labour-rmse}. This is the latest
available data, enabling us to use more data for model training and
explore the post-COVID pattern. All Subset methods (using various
estimators of \(\bm{W}\)) produce improved or comparable reconciled
forecasts compared to their benchmarks. The accuracy improvements become
more noticeable for longer forecast horizons. Similar to the average
results in Table~\ref{tbl-labour-rmse-avg}, the Intuitive and Lasso
methods yield results identical to the benchmark methods due to their
tendency to offer dense estimates. Surprisingly, when relaxing the
unbiasedness assumption, Elasso ranks the best and demonstrates
significant improvement over EMinT, and outperforms other methods across
almost all levels except for the top level.

Table~\ref{tbl-labour-info} presents the number of series selected at
each level and the optimal tuning parameter values obtained using
proposed methods. We only show results from Subset and Elasso methods,
as they return the best RMSE results. The scale variation in optimal
parameters for different methods is due to objective function scales.
Table~\ref{tbl-labour-info} shows that all Subset methods exclude some
series. Remarkably, the Elasso method consistently outperforms the
others overall, even though it uses only \(11\) series for forecast
reconciliation. Most of the series at the STT level are removed, while
the majority of series at the Duration level are retained. This aligns
with our data description, highlighting that the seasonal patterns in
the Duration level series are more consistent and potentially easier to
forecast compared to those at the STT level.

\subsection{Forecasting Australian domestic tourism}\label{sec-tourism}

Australian domestic tourism flows are measured as the number of
overnight trips Australians spend away from home. The data are sourced
from the National Visitor Survey and collected through computer-assisted
telephone interviews with approximately \(120,000\) residents aged
\(15\) years and older. The data follow a geographic structure, with
national total tourism flows at the top level, then disaggregated into
seven states and territories (referred to as \emph{State} level
hereafter), further dividing into \(27\) zones, and finally into \(76\)
regions. Thus, \(n_b=76\) and \(n=111\). Each series spans January 1998
to December 2017, with a total of \(240\) observations.

\begin{figure}

\centering{

\includegraphics{hf_selection_files/figure-pdf/fig-tourism-data-1.pdf}

}

\caption{\label{fig-tourism-data}Domestic tourism flows from January
1998 to December 2017 for the whole of Australia as well as the states.}

\end{figure}%

Figure~\ref{fig-tourism-data} shows aggregate tourism flows for
Australia and individual states, revealing pronounced seasonal patterns
across the national total and states, albeit with varying seasonal
patterns among series. Notably, significant growth began around 2010 for
the national total flow and some states such as NSW, VIC, QLD, and WA.
While flows are relatively flat for SA, TAS, and NT. Moreover, the time
plot displays a large decrease in tourism flows for WA in 2016.

Our objective is to forecast tourism flows for each series in the
geographic hierarchy while ensuring coherence across all levels. We use
a rolling forecast origin to evaluate the forecast accuracy of different
methods. We start with a training set of \(216\) months for each series,
and compute base forecasts from optimal ETS models. We then roll the
forecast origin forward, month by month, until December 2016. The base
forecasts are reconciled using our proposed methods and some existing
reconciliation methods.

Table~\ref{tbl-tourism-rmse-avg} reports average RMSE values for base
forecasts generated by ETS models, along with the percentage relative
improvements obtained by each reconciliation method. Similar to
Section~\ref{sec-labour}, MinT and the respective proposed methods are
not considered due to their poor performance. The results show that the
OLS method outperforms other benchmarks like WLSs, WLSv and MinTs,
despite the fact that WLSv and MinTs account for the in-sample
covariance of base forecast errors. This highlights the effectiveness of
the OLS method despite its simplicity.

Overall, the Subset methods outperform their respective benchmark
methods, especially for aggregation levels and longer horizons. The only
exception is the OLS-subset method, which slightly reduces overall
accuracy while improving top-level forecasts. Intuitive and Lasso
methods produce results almost identical to the corresponding
benchmarks, which is not surprising as ETS models rarely yield extremely
poor forecasts, making them challenging to be selected out using methods
that tend to return dense estimates. When we relax the unbiasedness
assumption, EMinT consistently performs the worst across all levels due
to the evident lack of joint weak stationarity among the series in the
hierarchy. The Elasso method shows significant improvement compared to
EMinT, and outperforms other methods across almost all levels except the
bottom level.

\begin{table}

\caption{\label{tbl-tourism-rmse-avg}Average out-of-sample forecast
results for Australian domestic tourism data.}

\centering{

\centering
\resizebox{\linewidth}{!}{
\fontsize{11}{13}\selectfont
\begin{threeparttable}
\begin{tabular}{lrrrrrrrrrrrrrrrrrrrr}
\toprule
\multicolumn{1}{c}{} & \multicolumn{4}{c}{Top} & \multicolumn{4}{c}{State} & \multicolumn{4}{c}{Zone} & \multicolumn{4}{c}{Region} & \multicolumn{4}{c}{Average} \\
\cmidrule(l{3pt}r{3pt}){2-5} \cmidrule(l{3pt}r{3pt}){6-9} \cmidrule(l{3pt}r{3pt}){10-13} \cmidrule(l{3pt}r{3pt}){14-17} \cmidrule(l{3pt}r{3pt}){18-21}
Method & h=1 & 1--4 & 1--8 & 1--12 & h=1 & 1--4 & 1--8 & 1--12 & h=1 & 1--4 & 1--8 & 1--12 & h=1 & 1--4 & 1--8 & 1--12 & h=1 & 1--4 & 1--8 & 1--12\\
\midrule
Base & 1565.8 & 1520.2 & 1548.3 & 1773.1 & 366.0 & 406.1 & 421.4 & 442.0 & 142.5 & 170.8 & 178.5 & 185.3 & 72.1 & 86.4 & 90.9 & 94.4 & 121.2 & 140.0 & 146.2 & 153.6\\
BU & 14.3 & 38.8 & 42.0 & 38.8 & 4.6 & 10.3 & 13.8 & 15.7 & --0.1 & 0.9 & 1.3 & 1.6 & 0.0 & 0.0 & 0.0 & 0.0 & 2.5 & 6.0 & 6.9 & 7.4\\
\midrule
OLS & --0.6 & 1.1 & 1.8 & 1.9 & --1.2 & --1.0 & --1.0 & --1.3 & --2.8 & --4.0 & --4.8 & --5.6 & --0.1 & --0.8 & --1.6 & --2.4 & --1.1 & --1.6 & --2.1 & --2.7\\
\cellcolor[HTML]{e6e3e3}{OLS-subset} & \cellcolor[HTML]{e6e3e3}{--0.6} & \cellcolor[HTML]{e6e3e3}{\textbf{  --1.9}} & \cellcolor[HTML]{e6e3e3}{\textbf{  --4.9}} & \cellcolor[HTML]{e6e3e3}{\textbf{  --3.0}} & \cellcolor[HTML]{e6e3e3}{--1.2} & \cellcolor[HTML]{e6e3e3}{\textbf{ --1.2}} & \cellcolor[HTML]{e6e3e3}{1.5} & \cellcolor[HTML]{e6e3e3}{0.9} & \cellcolor[HTML]{e6e3e3}{--2.6} & \cellcolor[HTML]{e6e3e3}{--0.5} & \cellcolor[HTML]{e6e3e3}{--0.2} & \cellcolor[HTML]{e6e3e3}{--1.1} & \cellcolor[HTML]{e6e3e3}{0.2} & \cellcolor[HTML]{e6e3e3}{2.1} & \cellcolor[HTML]{e6e3e3}{1.7} & \cellcolor[HTML]{e6e3e3}{0.6} & \cellcolor[HTML]{e6e3e3}{--1.0} & \cellcolor[HTML]{e6e3e3}{0.3} & \cellcolor[HTML]{e6e3e3}{0.5} & \cellcolor[HTML]{e6e3e3}{--0.2}\\
\cellcolor[HTML]{e6e3e3}{OLS-intuitive} & \cellcolor[HTML]{e6e3e3}{--0.6} & \cellcolor[HTML]{e6e3e3}{1.1} & \cellcolor[HTML]{e6e3e3}{1.8} & \cellcolor[HTML]{e6e3e3}{1.9} & \cellcolor[HTML]{e6e3e3}{--1.2} & \cellcolor[HTML]{e6e3e3}{--1.0} & \cellcolor[HTML]{e6e3e3}{--1.0} & \cellcolor[HTML]{e6e3e3}{--1.3} & \cellcolor[HTML]{e6e3e3}{--2.8} & \cellcolor[HTML]{e6e3e3}{--4.0} & \cellcolor[HTML]{e6e3e3}{--4.8} & \cellcolor[HTML]{e6e3e3}{--5.6} & \cellcolor[HTML]{e6e3e3}{--0.1} & \cellcolor[HTML]{e6e3e3}{--0.8} & \cellcolor[HTML]{e6e3e3}{--1.6} & \cellcolor[HTML]{e6e3e3}{--2.4} & \cellcolor[HTML]{e6e3e3}{--1.1} & \cellcolor[HTML]{e6e3e3}{--1.6} & \cellcolor[HTML]{e6e3e3}{--2.1} & \cellcolor[HTML]{e6e3e3}{--2.7}\\
\cellcolor[HTML]{e6e3e3}{OLS-lasso} & \cellcolor[HTML]{e6e3e3}{\textbf{  --0.8}} & \cellcolor[HTML]{e6e3e3}{2.1} & \cellcolor[HTML]{e6e3e3}{2.8} & \cellcolor[HTML]{e6e3e3}{2.9} & \cellcolor[HTML]{e6e3e3}{\textbf{ --1.3}} & \cellcolor[HTML]{e6e3e3}{--0.4} & \cellcolor[HTML]{e6e3e3}{0.5} & \cellcolor[HTML]{e6e3e3}{0.3} & \cellcolor[HTML]{e6e3e3}{--2.3} & \cellcolor[HTML]{e6e3e3}{--3.5} & \cellcolor[HTML]{e6e3e3}{--4.2} & \cellcolor[HTML]{e6e3e3}{--4.9} & \cellcolor[HTML]{e6e3e3}{0.0} & \cellcolor[HTML]{e6e3e3}{\textbf{--0.9}} & \cellcolor[HTML]{e6e3e3}{--1.5} & \cellcolor[HTML]{e6e3e3}{--2.2} & \cellcolor[HTML]{e6e3e3}{--1.0} & \cellcolor[HTML]{e6e3e3}{--1.3} & \cellcolor[HTML]{e6e3e3}{--1.5} & \cellcolor[HTML]{e6e3e3}{--2.0}\\
\midrule
WLSs & 4.1 & 16.5 & 19.0 & 18.1 & 0.6 & 2.0 & 4.1 & 5.2 & --2.7 & --3.1 & --3.3 & --3.4 & --0.5 & --1.0 & --1.4 & --1.8 & --0.4 & 0.7 & 1.0 & 1.1\\
\cellcolor[HTML]{e6e3e3}{WLSs-subset} & \cellcolor[HTML]{e6e3e3}{4.1} & \cellcolor[HTML]{e6e3e3}{\textbf{   6.9}} & \cellcolor[HTML]{e6e3e3}{\textbf{   8.9}} & \cellcolor[HTML]{e6e3e3}{\textbf{  10.2}} & \cellcolor[HTML]{e6e3e3}{0.6} & \cellcolor[HTML]{e6e3e3}{\textbf{  1.7}} & \cellcolor[HTML]{e6e3e3}{\textbf{  4.0}} & \cellcolor[HTML]{e6e3e3}{\textbf{  4.3}} & \cellcolor[HTML]{e6e3e3}{--2.7} & \cellcolor[HTML]{e6e3e3}{--3.0} & \cellcolor[HTML]{e6e3e3}{--2.1} & \cellcolor[HTML]{e6e3e3}{--2.1} & \cellcolor[HTML]{e6e3e3}{--0.5} & \cellcolor[HTML]{e6e3e3}{--0.1} & \cellcolor[HTML]{e6e3e3}{--0.1} & \cellcolor[HTML]{e6e3e3}{--0.5} & \cellcolor[HTML]{e6e3e3}{--0.4} & \cellcolor[HTML]{e6e3e3}{\textbf{  0.0}} & \cellcolor[HTML]{e6e3e3}{\textbf{  0.9}} & \cellcolor[HTML]{e6e3e3}{\textbf{  1.0}}\\
\cellcolor[HTML]{e6e3e3}{WLSs-intuitive} & \cellcolor[HTML]{e6e3e3}{4.1} & \cellcolor[HTML]{e6e3e3}{16.5} & \cellcolor[HTML]{e6e3e3}{19.0} & \cellcolor[HTML]{e6e3e3}{18.1} & \cellcolor[HTML]{e6e3e3}{0.6} & \cellcolor[HTML]{e6e3e3}{2.0} & \cellcolor[HTML]{e6e3e3}{4.1} & \cellcolor[HTML]{e6e3e3}{5.2} & \cellcolor[HTML]{e6e3e3}{--2.7} & \cellcolor[HTML]{e6e3e3}{--3.1} & \cellcolor[HTML]{e6e3e3}{--3.3} & \cellcolor[HTML]{e6e3e3}{--3.4} & \cellcolor[HTML]{e6e3e3}{--0.5} & \cellcolor[HTML]{e6e3e3}{--1.0} & \cellcolor[HTML]{e6e3e3}{--1.4} & \cellcolor[HTML]{e6e3e3}{--1.8} & \cellcolor[HTML]{e6e3e3}{--0.4} & \cellcolor[HTML]{e6e3e3}{0.7} & \cellcolor[HTML]{e6e3e3}{1.0} & \cellcolor[HTML]{e6e3e3}{1.1}\\
\cellcolor[HTML]{e6e3e3}{WLSs-lasso} & \cellcolor[HTML]{e6e3e3}{\textbf{   3.6}} & \cellcolor[HTML]{e6e3e3}{17.1} & \cellcolor[HTML]{e6e3e3}{19.5} & \cellcolor[HTML]{e6e3e3}{18.5} & \cellcolor[HTML]{e6e3e3}{\textbf{  0.3}} & \cellcolor[HTML]{e6e3e3}{2.1} & \cellcolor[HTML]{e6e3e3}{4.3} & \cellcolor[HTML]{e6e3e3}{5.5} & \cellcolor[HTML]{e6e3e3}{--2.7} & \cellcolor[HTML]{e6e3e3}{--2.9} & \cellcolor[HTML]{e6e3e3}{--3.2} & \cellcolor[HTML]{e6e3e3}{--3.3} & \cellcolor[HTML]{e6e3e3}{\textbf{--0.6}} & \cellcolor[HTML]{e6e3e3}{--1.0} & \cellcolor[HTML]{e6e3e3}{--1.4} & \cellcolor[HTML]{e6e3e3}{--1.7} & \cellcolor[HTML]{e6e3e3}{\textbf{ --0.5}} & \cellcolor[HTML]{e6e3e3}{0.8} & \cellcolor[HTML]{e6e3e3}{1.1} & \cellcolor[HTML]{e6e3e3}{1.2}\\
\midrule
WLSv & 6.2 & 22.3 & 25.0 & 23.6 & 1.1 & 3.7 & 6.3 & 7.9 & --2.6 & --2.4 & --2.4 & --2.2 & --1.6 & --1.6 & --1.8 & --1.9 & --0.5 & 1.5 & 2.1 & 2.4\\
\cellcolor[HTML]{e6e3e3}{WLSv-subset} & \cellcolor[HTML]{e6e3e3}{\textbf{   0.9}} & \cellcolor[HTML]{e6e3e3}{\textbf{   4.5}} & \cellcolor[HTML]{e6e3e3}{\textbf{   4.9}} & \cellcolor[HTML]{e6e3e3}{\textbf{   5.7}} & \cellcolor[HTML]{e6e3e3}{\textbf{ --1.2}} & \cellcolor[HTML]{e6e3e3}{\textbf{ --0.3}} & \cellcolor[HTML]{e6e3e3}{\textbf{ --0.1}} & \cellcolor[HTML]{e6e3e3}{\textbf{  0.7}} & \cellcolor[HTML]{e6e3e3}{\textbf{ --3.2}} & \cellcolor[HTML]{e6e3e3}{\textbf{ --3.8}} & \cellcolor[HTML]{e6e3e3}{\textbf{ --4.5}} & \cellcolor[HTML]{e6e3e3}{\textbf{ --4.9}} & \cellcolor[HTML]{e6e3e3}{--1.3} & \cellcolor[HTML]{e6e3e3}{--1.2} & \cellcolor[HTML]{e6e3e3}{--1.7} & \cellcolor[HTML]{e6e3e3}{\textbf{--2.3}} & \cellcolor[HTML]{e6e3e3}{\textbf{ --1.6}} & \cellcolor[HTML]{e6e3e3}{\textbf{ --1.2}} & \cellcolor[HTML]{e6e3e3}{\textbf{ --1.6}} & \cellcolor[HTML]{e6e3e3}{\textbf{ --1.7}}\\
\cellcolor[HTML]{e6e3e3}{WLSv-intuitive} & \cellcolor[HTML]{e6e3e3}{6.2} & \cellcolor[HTML]{e6e3e3}{22.3} & \cellcolor[HTML]{e6e3e3}{25.0} & \cellcolor[HTML]{e6e3e3}{23.6} & \cellcolor[HTML]{e6e3e3}{1.1} & \cellcolor[HTML]{e6e3e3}{3.7} & \cellcolor[HTML]{e6e3e3}{6.3} & \cellcolor[HTML]{e6e3e3}{7.9} & \cellcolor[HTML]{e6e3e3}{--2.6} & \cellcolor[HTML]{e6e3e3}{--2.4} & \cellcolor[HTML]{e6e3e3}{--2.4} & \cellcolor[HTML]{e6e3e3}{--2.2} & \cellcolor[HTML]{e6e3e3}{--1.6} & \cellcolor[HTML]{e6e3e3}{--1.6} & \cellcolor[HTML]{e6e3e3}{--1.8} & \cellcolor[HTML]{e6e3e3}{--1.9} & \cellcolor[HTML]{e6e3e3}{--0.5} & \cellcolor[HTML]{e6e3e3}{1.5} & \cellcolor[HTML]{e6e3e3}{2.1} & \cellcolor[HTML]{e6e3e3}{2.4}\\
\cellcolor[HTML]{e6e3e3}{WLSv-lasso} & \cellcolor[HTML]{e6e3e3}{6.2} & \cellcolor[HTML]{e6e3e3}{22.3} & \cellcolor[HTML]{e6e3e3}{25.0} & \cellcolor[HTML]{e6e3e3}{23.6} & \cellcolor[HTML]{e6e3e3}{1.1} & \cellcolor[HTML]{e6e3e3}{3.7} & \cellcolor[HTML]{e6e3e3}{6.3} & \cellcolor[HTML]{e6e3e3}{7.9} & \cellcolor[HTML]{e6e3e3}{--2.6} & \cellcolor[HTML]{e6e3e3}{--2.4} & \cellcolor[HTML]{e6e3e3}{--2.4} & \cellcolor[HTML]{e6e3e3}{--2.2} & \cellcolor[HTML]{e6e3e3}{--1.6} & \cellcolor[HTML]{e6e3e3}{--1.6} & \cellcolor[HTML]{e6e3e3}{--1.8} & \cellcolor[HTML]{e6e3e3}{--1.9} & \cellcolor[HTML]{e6e3e3}{--0.5} & \cellcolor[HTML]{e6e3e3}{1.5} & \cellcolor[HTML]{e6e3e3}{2.1} & \cellcolor[HTML]{e6e3e3}{2.4}\\
\midrule
MinTs & 5.1 & 17.2 & 19.7 & 18.6 & 0.1 & 1.9 & 4.4 & 5.7 & --3.5 & --3.3 & --3.4 & --3.4 & \textcolor{blue}{\textbf{--1.9}} & \textcolor{blue}{\textbf{--2.0}} & \textcolor{blue}{\textbf{--2.4}} & \textcolor{blue}{\textbf{--2.7}} & --1.2 & 0.2 & 0.7 & 0.9\\
\cellcolor[HTML]{e6e3e3}{MinTs-subset} & \cellcolor[HTML]{e6e3e3}{\textbf{   1.8}} & \cellcolor[HTML]{e6e3e3}{\textbf{   1.3}} & \cellcolor[HTML]{e6e3e3}{\textbf{   2.0}} & \cellcolor[HTML]{e6e3e3}{\textbf{   3.2}} & \cellcolor[HTML]{e6e3e3}{\textbf{ --2.2}} & \cellcolor[HTML]{e6e3e3}{\textbf{ --2.1}} & \cellcolor[HTML]{e6e3e3}{\textbf{ --1.3}} & \cellcolor[HTML]{e6e3e3}{\textbf{ --0.7}} & \cellcolor[HTML]{e6e3e3}{\textbf{ --4.2}} & \cellcolor[HTML]{e6e3e3}{\textbf{ --4.5}} & \cellcolor[HTML]{e6e3e3}{\textbf{ --4.9}} & \cellcolor[HTML]{e6e3e3}{\textbf{ --5.4}} & \cellcolor[HTML]{e6e3e3}{--1.5} & \cellcolor[HTML]{e6e3e3}{--1.3} & \cellcolor[HTML]{e6e3e3}{--1.9} & \cellcolor[HTML]{e6e3e3}{--2.5} & \cellcolor[HTML]{e6e3e3}{\textbf{ --2.0}} & \cellcolor[HTML]{e6e3e3}{\textbf{ --2.2}} & \cellcolor[HTML]{e6e3e3}{\textbf{ --2.3}} & \cellcolor[HTML]{e6e3e3}{\textbf{ --2.5}}\\
\cellcolor[HTML]{e6e3e3}{MinTs-intuitive} & \cellcolor[HTML]{e6e3e3}{5.1} & \cellcolor[HTML]{e6e3e3}{17.2} & \cellcolor[HTML]{e6e3e3}{19.7} & \cellcolor[HTML]{e6e3e3}{18.6} & \cellcolor[HTML]{e6e3e3}{0.1} & \cellcolor[HTML]{e6e3e3}{1.9} & \cellcolor[HTML]{e6e3e3}{4.4} & \cellcolor[HTML]{e6e3e3}{5.7} & \cellcolor[HTML]{e6e3e3}{--3.5} & \cellcolor[HTML]{e6e3e3}{--3.3} & \cellcolor[HTML]{e6e3e3}{--3.4} & \cellcolor[HTML]{e6e3e3}{--3.4} & \cellcolor[HTML]{e6e3e3}{\textcolor{blue}{\textbf{--1.9}}} & \cellcolor[HTML]{e6e3e3}{\textcolor{blue}{\textbf{--2.0}}} & \cellcolor[HTML]{e6e3e3}{\textcolor{blue}{\textbf{--2.4}}} & \cellcolor[HTML]{e6e3e3}{\textcolor{blue}{\textbf{--2.7}}} & \cellcolor[HTML]{e6e3e3}{--1.2} & \cellcolor[HTML]{e6e3e3}{0.2} & \cellcolor[HTML]{e6e3e3}{0.7} & \cellcolor[HTML]{e6e3e3}{0.9}\\
\cellcolor[HTML]{e6e3e3}{MinTs-lasso} & \cellcolor[HTML]{e6e3e3}{5.1} & \cellcolor[HTML]{e6e3e3}{17.2} & \cellcolor[HTML]{e6e3e3}{19.7} & \cellcolor[HTML]{e6e3e3}{18.6} & \cellcolor[HTML]{e6e3e3}{0.1} & \cellcolor[HTML]{e6e3e3}{1.9} & \cellcolor[HTML]{e6e3e3}{4.4} & \cellcolor[HTML]{e6e3e3}{5.7} & \cellcolor[HTML]{e6e3e3}{--3.5} & \cellcolor[HTML]{e6e3e3}{--3.3} & \cellcolor[HTML]{e6e3e3}{--3.4} & \cellcolor[HTML]{e6e3e3}{--3.4} & \cellcolor[HTML]{e6e3e3}{\textcolor{blue}{\textbf{--1.9}}} & \cellcolor[HTML]{e6e3e3}{\textcolor{blue}{\textbf{--2.0}}} & \cellcolor[HTML]{e6e3e3}{\textcolor{blue}{\textbf{--2.4}}} & \cellcolor[HTML]{e6e3e3}{\textcolor{blue}{\textbf{--2.7}}} & \cellcolor[HTML]{e6e3e3}{--1.2} & \cellcolor[HTML]{e6e3e3}{0.2} & \cellcolor[HTML]{e6e3e3}{0.7} & \cellcolor[HTML]{e6e3e3}{0.9}\\
\midrule
EMinT & --2.3 & 24.3 & 58.8 & 59.7 & 36.9 & 56.0 & 68.4 & 70.4 & 51.4 & 64.6 & 75.8 & 81.4 & 65.9 & 72.3 & 81.9 & 85.9 & 48.3 & 62.3 & 75.4 & 79.0\\
\cellcolor[HTML]{e6e3e3}{Elasso} & \cellcolor[HTML]{e6e3e3}{\textcolor{blue}{\textbf{ --17.0}}} & \cellcolor[HTML]{e6e3e3}{\textcolor{blue}{\textbf{ --19.4}}} & \cellcolor[HTML]{e6e3e3}{\textcolor{blue}{\textbf{ --19.8}}} & \cellcolor[HTML]{e6e3e3}{\textcolor{blue}{\textbf{ --18.7}}} & \cellcolor[HTML]{e6e3e3}{\textcolor{blue}{\textbf{--21.6}}} & \cellcolor[HTML]{e6e3e3}{\textcolor{blue}{\textbf{--17.3}}} & \cellcolor[HTML]{e6e3e3}{\textcolor{blue}{\textbf{--19.3}}} & \cellcolor[HTML]{e6e3e3}{\textcolor{blue}{\textbf{--19.6}}} & \cellcolor[HTML]{e6e3e3}{\textcolor{blue}{\textbf{ --6.5}}} & \cellcolor[HTML]{e6e3e3}{\textcolor{blue}{\textbf{ --9.4}}} & \cellcolor[HTML]{e6e3e3}{\textcolor{blue}{\textbf{--11.5}}} & \cellcolor[HTML]{e6e3e3}{\textcolor{blue}{\textbf{--12.6}}} & \cellcolor[HTML]{e6e3e3}{\textbf{ 2.2}} & \cellcolor[HTML]{e6e3e3}{\textbf{ 0.4}} & \cellcolor[HTML]{e6e3e3}{\textbf{--1.0}} & \cellcolor[HTML]{e6e3e3}{\textbf{--1.8}} & \cellcolor[HTML]{e6e3e3}{\textcolor{blue}{\textbf{ --7.0}}} & \cellcolor[HTML]{e6e3e3}{\textcolor{blue}{\textbf{ --7.7}}} & \cellcolor[HTML]{e6e3e3}{\textcolor{blue}{\textbf{ --9.2}}} & \cellcolor[HTML]{e6e3e3}{\textcolor{blue}{\textbf{ --9.9}}}\\
\bottomrule
\end{tabular}
\begin{tablenotes}[para]
\item Note: The Base row shows the average RMSE of the base forecasts. Entries below this row indicate the percentage decrease (negative) or increase (positive) in the average RMSE of the reconciled forecasts compared to the base forecasts. The entries with the lowest values in each column are highlighted in blue. In each panel, the proposed methods are indicated with a gray background, and methods that outperform the benchmark method are marked in bold.
\end{tablenotes}
\end{threeparttable}}

}

\end{table}%

Table~\ref{tbl-tourism-rmse} present the results based on the last one
training set spanning from January 2017 to December 2017. The
reconciliation errors across \(111\) series and across four hierarchy
levels are displayed in Figure~\ref{fig-tourism-rmse}. The results show
a similar performance to the average results described above, indicating
relatively high-quality forecasts from Subset and Elasso methods.

\begin{figure}[!t]

\centering{

\includegraphics{hf_selection_files/figure-pdf/fig-tourism-rmse-1.pdf}

}

\caption{\label{fig-tourism-rmse}Average out-of-sample forecasting
performance, measured in terms of RMSE (from 1- to 12-step-ahead), for
each series across different reconciliation methods. Time series are
arranged along the horizontal axis.}

\end{figure}%

Additionally, Table~\ref{tbl-tourism-info} summarizes the number of
series selected using proposed methods as well as the optimal tuning
parameter values identified. Here we focus on the Subset and Elasso
methods since they are useful in the tourism application. Note that the
scale variation in optimal parameters for different methods comes from
to objective function scales. We observe that the OLS-subset and
WLSs-subset methods exclude some series at the State and Zone levels for
reconciliation. In contrast, the WLSv and MinTs methods retain all
series, benefiting from the consideration of in-sample covariance, which
enables larger adjustments made to series with large in-sample forecast
error variances during reconciliation. Nonetheless, the WLSv and MinTs
methods can still enhance the quality of reconciled forecasts due to the
inclusion of shrinkage through ridge regularization. It is surprising
that Elasso performs exceptionally well despite using only \(13\) series
for reconciliation.

\section{Conclusion}\label{sec-conclusion}

In existing forecast reconciliation literature, we map all base
forecasts into bottom-level disaggregated forecasts, which are then
summed to yield coherent forecasts for the entire structure. The mapping
step can be conceptually regarded as a forecast combination. It is
common that the base forecasts for some time series perform poorly. This
may reduce overall reconciliation effectiveness. In this paper, we have
addressed this issue by introducing a selection mechanism to forecast
reconciliation; i.e., incorporating time series selection when
reconciling forecasts, while ensuring coherent forecasts for all series.

Under the unbiasedness assumption, we developed three reconciliation
methods with selection mechanisms to automatically remove some base
forecasts when forming reconciled forecasts. These methods include group
best-subset selection with ridge regularization (Subset), an intuitive
method with \(L_0\) regularization (Intuitive), and a group lasso method
(Lasso). These methods use different penalty functions designed to
penalize the columns of the weighting matrix, \(\bm{G}\), towards zero.
Additionally, we relaxed the unbiasedness assumption and proposed the
empirical group lasso method (Elasso) which selects series based on
in-sample observations and fitted values.

Simulation experiments and two empirical applications demonstrated the
superiority of the proposed methods over existing reconciliation methods
without series selection. When model misspecification was introduced for
some series in the hierarchy, our proposed methods ensured coherent
forecasts that outperformed or, at least, matched their respective
benchmarks in the minimum trace reconciliation framework. In both
empirical applications, where no apparent model misspecification was
present, Subset and Elasso methods were always preferred, especially for
aggregation levels and longer forecast horizons, while Intuitive and
Lasso methods yielded results identical to corresponding benchmarks, as
they tend to provide dense estimates.

A feature of the proposed methods is their ability to reduce the
disparities arising from using different estimates of the base forecast
error covariance matrix, thereby mitigating the challenges associated
with estimator selection, which is a prominent issue within the field of
forecast reconciliation research.

As the number of series grows, solving these problems efficiently
becomes challenging, and the exact computation of these estimators
remains a hurdle. In our study, we have used Gurobi, one of the most
widely used commercial solvers, to address NP-hard MIP problems. Despite
various efforts to develop MIP-based approaches for solving
\(L_0\)-regularized regression problems, extending these methods to
incorporate additional constraints remains a challenge. We leave this
aspect to be addressed in future research.

\section*{Supplementary materials}\label{supplementary-materials}
\addcontentsline{toc}{section}{Supplementary materials}

\textbf{Appendix:} Additional results obtained in
Section~\ref{sec-simulations} and Section~\ref{sec-applications}. (.pdf
file)

\section*{Acknowledgments}\label{acknowledgments}
\addcontentsline{toc}{section}{Acknowledgments}

Rob J Hyndman was supported by the Australian Research Council
Industrial Transformation Training Centre in Optimisation Technologies,
Integrated Methodologies, and Applications (OPTIMA), Project ID
IC200100009.

\section*{References}\label{references}
\addcontentsline{toc}{section}{References}

\renewcommand{\bibsection}{}
\bibliography{references.bib}

\newpage
\appendix
\pagenumbering{arabic}% resets `page` counter to 1
\setcounter{section}{0}
\renewcommand{\thesection}{\Alph{section}}
\renewcommand{\thefigure}{A\arabic{figure}}
\renewcommand{\thetable}{A\arabic{table}}
\setcounter{figure}{0}
\setcounter{table}{0}

\section*{Appendix}\label{appendix}
\addcontentsline{toc}{section}{Appendix}

The section provides additional results obtained in
Section~\ref{sec-simulations} and Section~\ref{sec-applications}.

\begin{table}

\caption{\label{tbl-s2-rmse}Out-of-sample forecast results for the
simulated data in Scenario B, Setup 1.}

\centering{

\centering
\resizebox{\linewidth}{!}{
\fontsize{11}{13}\selectfont
\begin{threeparttable}
\begin{tabular}{lrrrrrrrrrrrrrrrr}
\toprule
\multicolumn{1}{c}{} & \multicolumn{4}{c}{Top} & \multicolumn{4}{c}{Middle} & \multicolumn{4}{c}{Bottom} & \multicolumn{4}{c}{Average} \\
\cmidrule(l{3pt}r{3pt}){2-5} \cmidrule(l{3pt}r{3pt}){6-9} \cmidrule(l{3pt}r{3pt}){10-13} \cmidrule(l{3pt}r{3pt}){14-17}
Method & h=1 & 1--4 & 1--8 & 1--16 & h=1 & 1--4 & 1--8 & 1--16 & h=1 & 1--4 & 1--8 & 1--16 & h=1 & 1--4 & 1--8 & 1--16\\
\midrule
Base & 9.6 & 10.7 & 12.6 & 15.6 & 12.1 & 14.4 & 15.3 & 17.0 & 4.2 & 4.9 & 5.9 & 7.5 & 7.2 & 8.5 & 9.6 & 11.4\\
BU & \textcolor{blue}{\textbf{--1.0}} & 0.4 & 0.6 & 0.7 & \textcolor{blue}{\textbf{--47.7}} & \textcolor{blue}{\textbf{--49.6}} & \textcolor{blue}{\textbf{--43.6}} & \textcolor{blue}{\textbf{--36.2}} & \textcolor{blue}{\textbf{ 0.0}} & \textcolor{blue}{\textbf{ 0.0}} & \textcolor{blue}{\textbf{ 0.0}} & \textcolor{blue}{\textbf{ 0.0}} & \textcolor{blue}{\textbf{--23.0}} & \textcolor{blue}{\textbf{--24.0}} & \textcolor{blue}{\textbf{--19.8}} & \textcolor{blue}{\textbf{--15.3}}\\
\midrule
OLS & 8.5 & 13.9 & 10.4 & 7.6 & --28.2 & --29.4 & --26.7 & --23.1 & 22.9 & 23.9 & 17.0 & 11.3 & --4.2 & --3.8 & --4.2 & --4.1\\
\cellcolor[HTML]{e6e3e3}{OLS-subset} & \cellcolor[HTML]{e6e3e3}{\textbf{--0.5}} & \cellcolor[HTML]{e6e3e3}{\textbf{ 0.5}} & \cellcolor[HTML]{e6e3e3}{\textbf{ 0.6}} & \cellcolor[HTML]{e6e3e3}{\textbf{ 0.7}} & \cellcolor[HTML]{e6e3e3}{\textbf{--46.3}} & \cellcolor[HTML]{e6e3e3}{\textbf{--49.0}} & \cellcolor[HTML]{e6e3e3}{\textbf{--43.2}} & \cellcolor[HTML]{e6e3e3}{\textbf{--35.9}} & \cellcolor[HTML]{e6e3e3}{\textbf{ 2.2}} & \cellcolor[HTML]{e6e3e3}{\textbf{ 1.0}} & \cellcolor[HTML]{e6e3e3}{\textbf{ 0.7}} & \cellcolor[HTML]{e6e3e3}{\textbf{ 0.5}} & \cellcolor[HTML]{e6e3e3}{\textbf{--21.5}} & \cellcolor[HTML]{e6e3e3}{\textbf{--23.4}} & \cellcolor[HTML]{e6e3e3}{\textbf{--19.4}} & \cellcolor[HTML]{e6e3e3}{\textbf{--15.0}}\\
\cellcolor[HTML]{e6e3e3}{OLS-intuitive} & \cellcolor[HTML]{e6e3e3}{\textbf{--0.5}} & \cellcolor[HTML]{e6e3e3}{\textbf{ 0.5}} & \cellcolor[HTML]{e6e3e3}{\textbf{ 0.6}} & \cellcolor[HTML]{e6e3e3}{\textbf{ 0.6}} & \cellcolor[HTML]{e6e3e3}{\textbf{--46.5}} & \cellcolor[HTML]{e6e3e3}{\textbf{--49.0}} & \cellcolor[HTML]{e6e3e3}{\textbf{--43.2}} & \cellcolor[HTML]{e6e3e3}{\textbf{--36.0}} & \cellcolor[HTML]{e6e3e3}{\textbf{ 2.2}} & \cellcolor[HTML]{e6e3e3}{\textbf{ 1.2}} & \cellcolor[HTML]{e6e3e3}{\textbf{ 0.7}} & \cellcolor[HTML]{e6e3e3}{\textbf{ 0.5}} & \cellcolor[HTML]{e6e3e3}{\textbf{--21.6}} & \cellcolor[HTML]{e6e3e3}{\textbf{--23.4}} & \cellcolor[HTML]{e6e3e3}{\textbf{--19.4}} & \cellcolor[HTML]{e6e3e3}{\textbf{--15.0}}\\
\cellcolor[HTML]{e6e3e3}{OLS-lasso} & \cellcolor[HTML]{e6e3e3}{\textbf{--0.2}} & \cellcolor[HTML]{e6e3e3}{\textbf{ 1.5}} & \cellcolor[HTML]{e6e3e3}{\textbf{ 1.4}} & \cellcolor[HTML]{e6e3e3}{\textbf{ 1.3}} & \cellcolor[HTML]{e6e3e3}{\textbf{--46.9}} & \cellcolor[HTML]{e6e3e3}{\textbf{--48.9}} & \cellcolor[HTML]{e6e3e3}{\textbf{--43.1}} & \cellcolor[HTML]{e6e3e3}{\textbf{--35.8}} & \cellcolor[HTML]{e6e3e3}{\textbf{ 0.9}} & \cellcolor[HTML]{e6e3e3}{\textbf{ 0.8}} & \cellcolor[HTML]{e6e3e3}{\textbf{ 0.5}} & \cellcolor[HTML]{e6e3e3}{\textbf{ 0.3}} & \cellcolor[HTML]{e6e3e3}{\textbf{--22.1}} & \cellcolor[HTML]{e6e3e3}{\textbf{--23.3}} & \cellcolor[HTML]{e6e3e3}{\textbf{--19.3}} & \cellcolor[HTML]{e6e3e3}{\textbf{--14.9}}\\
\midrule
WLSs & 12.1 & 18.6 & 14.0 & 10.2 & --34.4 & --35.1 & --31.7 & --26.9 & 15.6 & 17.0 & 12.0 & 8.0 & --9.0 & --8.0 & --7.6 & --6.5\\
\cellcolor[HTML]{e6e3e3}{WLSs-subset} & \cellcolor[HTML]{e6e3e3}{\textbf{--0.1}} & \cellcolor[HTML]{e6e3e3}{\textbf{ 1.2}} & \cellcolor[HTML]{e6e3e3}{\textbf{ 1.1}} & \cellcolor[HTML]{e6e3e3}{\textbf{ 1.1}} & \cellcolor[HTML]{e6e3e3}{\textbf{--46.7}} & \cellcolor[HTML]{e6e3e3}{\textbf{--48.8}} & \cellcolor[HTML]{e6e3e3}{\textbf{--43.1}} & \cellcolor[HTML]{e6e3e3}{\textbf{--35.8}} & \cellcolor[HTML]{e6e3e3}{\textbf{ 1.5}} & \cellcolor[HTML]{e6e3e3}{\textbf{ 1.1}} & \cellcolor[HTML]{e6e3e3}{\textbf{ 0.8}} & \cellcolor[HTML]{e6e3e3}{\textbf{ 0.6}} & \cellcolor[HTML]{e6e3e3}{\textbf{--21.8}} & \cellcolor[HTML]{e6e3e3}{\textbf{--23.2}} & \cellcolor[HTML]{e6e3e3}{\textbf{--19.2}} & \cellcolor[HTML]{e6e3e3}{\textbf{--14.8}}\\
\cellcolor[HTML]{e6e3e3}{WLSs-intuitive} & \cellcolor[HTML]{e6e3e3}{\textbf{ 0.0}} & \cellcolor[HTML]{e6e3e3}{\textbf{ 1.2}} & \cellcolor[HTML]{e6e3e3}{\textbf{ 1.0}} & \cellcolor[HTML]{e6e3e3}{\textbf{ 0.9}} & \cellcolor[HTML]{e6e3e3}{\textbf{--46.5}} & \cellcolor[HTML]{e6e3e3}{\textbf{--48.8}} & \cellcolor[HTML]{e6e3e3}{\textbf{--43.1}} & \cellcolor[HTML]{e6e3e3}{\textbf{--35.9}} & \cellcolor[HTML]{e6e3e3}{\textbf{ 1.7}} & \cellcolor[HTML]{e6e3e3}{\textbf{ 1.3}} & \cellcolor[HTML]{e6e3e3}{\textbf{ 0.9}} & \cellcolor[HTML]{e6e3e3}{\textbf{ 0.6}} & \cellcolor[HTML]{e6e3e3}{\textbf{--21.6}} & \cellcolor[HTML]{e6e3e3}{\textbf{--23.1}} & \cellcolor[HTML]{e6e3e3}{\textbf{--19.2}} & \cellcolor[HTML]{e6e3e3}{\textbf{--14.9}}\\
\cellcolor[HTML]{e6e3e3}{WLSs-lasso} & \cellcolor[HTML]{e6e3e3}{\textbf{--0.1}} & \cellcolor[HTML]{e6e3e3}{\textbf{ 1.5}} & \cellcolor[HTML]{e6e3e3}{\textbf{ 1.5}} & \cellcolor[HTML]{e6e3e3}{\textbf{ 1.3}} & \cellcolor[HTML]{e6e3e3}{\textbf{--46.7}} & \cellcolor[HTML]{e6e3e3}{\textbf{--48.9}} & \cellcolor[HTML]{e6e3e3}{\textbf{--43.1}} & \cellcolor[HTML]{e6e3e3}{\textbf{--35.8}} & \cellcolor[HTML]{e6e3e3}{\textbf{ 0.9}} & \cellcolor[HTML]{e6e3e3}{\textbf{ 0.8}} & \cellcolor[HTML]{e6e3e3}{\textbf{ 0.5}} & \cellcolor[HTML]{e6e3e3}{\textbf{ 0.3}} & \cellcolor[HTML]{e6e3e3}{\textbf{--22.0}} & \cellcolor[HTML]{e6e3e3}{\textbf{--23.2}} & \cellcolor[HTML]{e6e3e3}{\textbf{--19.3}} & \cellcolor[HTML]{e6e3e3}{\textbf{--14.9}}\\
\midrule
WLSv & --0.8 & 2.3 & 1.8 & 1.6 & --46.3 & --47.9 & --42.3 & --35.2 & 1.6 & 1.9 & 1.2 & 0.8 & --21.7 & --22.2 & --18.6 & --14.4\\
\cellcolor[HTML]{e6e3e3}{WLSv-subset} & \cellcolor[HTML]{e6e3e3}{--0.7} & \cellcolor[HTML]{e6e3e3}{\textbf{ 1.3}} & \cellcolor[HTML]{e6e3e3}{\textbf{ 1.4}} & \cellcolor[HTML]{e6e3e3}{\textbf{ 1.4}} & \cellcolor[HTML]{e6e3e3}{\textbf{--46.9}} & \cellcolor[HTML]{e6e3e3}{\textbf{--48.7}} & \cellcolor[HTML]{e6e3e3}{\textbf{--42.9}} & \cellcolor[HTML]{e6e3e3}{\textbf{--35.6}} & \cellcolor[HTML]{e6e3e3}{\textbf{ 1.0}} & \cellcolor[HTML]{e6e3e3}{\textbf{ 1.0}} & \cellcolor[HTML]{e6e3e3}{\textbf{ 0.8}} & \cellcolor[HTML]{e6e3e3}{\textbf{ 0.6}} & \cellcolor[HTML]{e6e3e3}{\textbf{--22.2}} & \cellcolor[HTML]{e6e3e3}{\textbf{--23.1}} & \cellcolor[HTML]{e6e3e3}{\textbf{--19.1}} & \cellcolor[HTML]{e6e3e3}{\textbf{--14.7}}\\
\cellcolor[HTML]{e6e3e3}{WLSv-intuitive} & \cellcolor[HTML]{e6e3e3}{--0.4} & \cellcolor[HTML]{e6e3e3}{\textbf{ 1.5}} & \cellcolor[HTML]{e6e3e3}{\textbf{ 1.4}} & \cellcolor[HTML]{e6e3e3}{\textbf{ 1.2}} & \cellcolor[HTML]{e6e3e3}{\textbf{--46.9}} & \cellcolor[HTML]{e6e3e3}{\textbf{--48.6}} & \cellcolor[HTML]{e6e3e3}{\textbf{--42.8}} & \cellcolor[HTML]{e6e3e3}{\textbf{--35.6}} & \cellcolor[HTML]{e6e3e3}{\textbf{ 0.9}} & \cellcolor[HTML]{e6e3e3}{\textbf{ 1.2}} & \cellcolor[HTML]{e6e3e3}{\textbf{ 0.9}} & \cellcolor[HTML]{e6e3e3}{\textbf{ 0.7}} & \cellcolor[HTML]{e6e3e3}{\textbf{--22.2}} & \cellcolor[HTML]{e6e3e3}{\textbf{--23.0}} & \cellcolor[HTML]{e6e3e3}{\textbf{--19.0}} & \cellcolor[HTML]{e6e3e3}{\textbf{--14.7}}\\
\cellcolor[HTML]{e6e3e3}{WLSv-lasso} & \cellcolor[HTML]{e6e3e3}{--0.6} & \cellcolor[HTML]{e6e3e3}{\textbf{ 1.3}} & \cellcolor[HTML]{e6e3e3}{\textbf{ 1.3}} & \cellcolor[HTML]{e6e3e3}{\textbf{ 1.3}} & \cellcolor[HTML]{e6e3e3}{\textbf{--47.2}} & \cellcolor[HTML]{e6e3e3}{\textbf{--48.9}} & \cellcolor[HTML]{e6e3e3}{\textbf{--43.0}} & \cellcolor[HTML]{e6e3e3}{\textbf{--35.7}} & \cellcolor[HTML]{e6e3e3}{\textbf{ 0.6}} & \cellcolor[HTML]{e6e3e3}{\textbf{ 0.8}} & \cellcolor[HTML]{e6e3e3}{\textbf{ 0.5}} & \cellcolor[HTML]{e6e3e3}{\textbf{ 0.4}} & \cellcolor[HTML]{e6e3e3}{\textbf{--22.4}} & \cellcolor[HTML]{e6e3e3}{\textbf{--23.3}} & \cellcolor[HTML]{e6e3e3}{\textbf{--19.2}} & \cellcolor[HTML]{e6e3e3}{\textbf{--14.8}}\\
\midrule
MinT & 0.2 & 0.5 & 0.6 & 0.5 & --47.5 & --49.4 & --43.5 & --36.1 & 1.1 & 0.5 & 0.3 & 0.1 & --22.3 & --23.7 & --19.6 & \textcolor{blue}{\textbf{--15.3}}\\
\cellcolor[HTML]{e6e3e3}{MinT-subset} & \cellcolor[HTML]{e6e3e3}{\textbf{--0.1}} & \cellcolor[HTML]{e6e3e3}{0.8} & \cellcolor[HTML]{e6e3e3}{0.9} & \cellcolor[HTML]{e6e3e3}{0.9} & \cellcolor[HTML]{e6e3e3}{--46.9} & \cellcolor[HTML]{e6e3e3}{--49.1} & \cellcolor[HTML]{e6e3e3}{--43.3} & \cellcolor[HTML]{e6e3e3}{--36.0} & \cellcolor[HTML]{e6e3e3}{1.7} & \cellcolor[HTML]{e6e3e3}{0.9} & \cellcolor[HTML]{e6e3e3}{0.5} & \cellcolor[HTML]{e6e3e3}{0.3} & \cellcolor[HTML]{e6e3e3}{--21.9} & \cellcolor[HTML]{e6e3e3}{--23.4} & \cellcolor[HTML]{e6e3e3}{--19.4} & \cellcolor[HTML]{e6e3e3}{--15.1}\\
\cellcolor[HTML]{e6e3e3}{MinT-intuitive} & \cellcolor[HTML]{e6e3e3}{0.2} & \cellcolor[HTML]{e6e3e3}{0.5} & \cellcolor[HTML]{e6e3e3}{0.6} & \cellcolor[HTML]{e6e3e3}{0.5} & \cellcolor[HTML]{e6e3e3}{--47.5} & \cellcolor[HTML]{e6e3e3}{--49.4} & \cellcolor[HTML]{e6e3e3}{--43.5} & \cellcolor[HTML]{e6e3e3}{--36.1} & \cellcolor[HTML]{e6e3e3}{1.1} & \cellcolor[HTML]{e6e3e3}{0.5} & \cellcolor[HTML]{e6e3e3}{0.3} & \cellcolor[HTML]{e6e3e3}{0.1} & \cellcolor[HTML]{e6e3e3}{--22.3} & \cellcolor[HTML]{e6e3e3}{--23.7} & \cellcolor[HTML]{e6e3e3}{--19.6} & \cellcolor[HTML]{e6e3e3}{\textcolor{blue}{\textbf{--15.3}}}\\
\cellcolor[HTML]{e6e3e3}{MinT-lasso} & \cellcolor[HTML]{e6e3e3}{\textbf{--0.3}} & \cellcolor[HTML]{e6e3e3}{\textbf{ 0.3}} & \cellcolor[HTML]{e6e3e3}{0.6} & \cellcolor[HTML]{e6e3e3}{0.5} & \cellcolor[HTML]{e6e3e3}{\textbf{--47.6}} & \cellcolor[HTML]{e6e3e3}{--49.4} & \cellcolor[HTML]{e6e3e3}{--43.5} & \cellcolor[HTML]{e6e3e3}{--36.1} & \cellcolor[HTML]{e6e3e3}{\textbf{ 0.8}} & \cellcolor[HTML]{e6e3e3}{\textbf{ 0.3}} & \cellcolor[HTML]{e6e3e3}{\textbf{ 0.2}} & \cellcolor[HTML]{e6e3e3}{0.1} & \cellcolor[HTML]{e6e3e3}{\textbf{--22.5}} & \cellcolor[HTML]{e6e3e3}{\textbf{--23.9}} & \cellcolor[HTML]{e6e3e3}{\textbf{--19.7}} & \cellcolor[HTML]{e6e3e3}{\textcolor{blue}{\textbf{--15.3}}}\\
\midrule
MinTs & --0.3 & 0.3 & \textcolor{blue}{\textbf{ 0.4}} & \textcolor{blue}{\textbf{ 0.4}} & --47.6 & --49.5 & \textcolor{blue}{\textbf{--43.6}} & \textcolor{blue}{\textbf{--36.2}} & 0.7 & 0.2 & 0.1 & \textcolor{blue}{\textbf{ 0.0}} & --22.6 & --23.9 & \textcolor{blue}{\textbf{--19.8}} & \textcolor{blue}{\textbf{--15.3}}\\
\cellcolor[HTML]{e6e3e3}{MinTs-subset} & \cellcolor[HTML]{e6e3e3}{\textbf{--0.8}} & \cellcolor[HTML]{e6e3e3}{0.5} & \cellcolor[HTML]{e6e3e3}{0.8} & \cellcolor[HTML]{e6e3e3}{0.8} & \cellcolor[HTML]{e6e3e3}{--47.2} & \cellcolor[HTML]{e6e3e3}{--49.2} & \cellcolor[HTML]{e6e3e3}{--43.4} & \cellcolor[HTML]{e6e3e3}{--36.0} & \cellcolor[HTML]{e6e3e3}{1.0} & \cellcolor[HTML]{e6e3e3}{0.7} & \cellcolor[HTML]{e6e3e3}{0.4} & \cellcolor[HTML]{e6e3e3}{0.3} & \cellcolor[HTML]{e6e3e3}{--22.3} & \cellcolor[HTML]{e6e3e3}{--23.6} & \cellcolor[HTML]{e6e3e3}{--19.5} & \cellcolor[HTML]{e6e3e3}{--15.1}\\
\cellcolor[HTML]{e6e3e3}{MinTs-intuitive} & \cellcolor[HTML]{e6e3e3}{--0.3} & \cellcolor[HTML]{e6e3e3}{0.3} & \cellcolor[HTML]{e6e3e3}{\textcolor{blue}{\textbf{ 0.4}}} & \cellcolor[HTML]{e6e3e3}{\textcolor{blue}{\textbf{ 0.4}}} & \cellcolor[HTML]{e6e3e3}{--47.6} & \cellcolor[HTML]{e6e3e3}{--49.5} & \cellcolor[HTML]{e6e3e3}{\textcolor{blue}{\textbf{--43.6}}} & \cellcolor[HTML]{e6e3e3}{\textcolor{blue}{\textbf{--36.2}}} & \cellcolor[HTML]{e6e3e3}{0.7} & \cellcolor[HTML]{e6e3e3}{0.2} & \cellcolor[HTML]{e6e3e3}{0.1} & \cellcolor[HTML]{e6e3e3}{\textcolor{blue}{\textbf{ 0.0}}} & \cellcolor[HTML]{e6e3e3}{--22.6} & \cellcolor[HTML]{e6e3e3}{--23.9} & \cellcolor[HTML]{e6e3e3}{\textcolor{blue}{\textbf{--19.8}}} & \cellcolor[HTML]{e6e3e3}{\textcolor{blue}{\textbf{--15.3}}}\\
\cellcolor[HTML]{e6e3e3}{MinTs-lasso} & \cellcolor[HTML]{e6e3e3}{\textbf{--0.9}} & \cellcolor[HTML]{e6e3e3}{\textcolor{blue}{\textbf{ 0.2}}} & \cellcolor[HTML]{e6e3e3}{0.5} & \cellcolor[HTML]{e6e3e3}{0.5} & \cellcolor[HTML]{e6e3e3}{\textcolor{blue}{\textbf{--47.7}}} & \cellcolor[HTML]{e6e3e3}{--49.5} & \cellcolor[HTML]{e6e3e3}{\textcolor{blue}{\textbf{--43.6}}} & \cellcolor[HTML]{e6e3e3}{\textcolor{blue}{\textbf{--36.2}}} & \cellcolor[HTML]{e6e3e3}{\textbf{ 0.5}} & \cellcolor[HTML]{e6e3e3}{0.2} & \cellcolor[HTML]{e6e3e3}{0.1} & \cellcolor[HTML]{e6e3e3}{0.1} & \cellcolor[HTML]{e6e3e3}{\textbf{--22.8}} & \cellcolor[HTML]{e6e3e3}{\textcolor{blue}{\textbf{--24.0}}} & \cellcolor[HTML]{e6e3e3}{\textcolor{blue}{\textbf{--19.8}}} & \cellcolor[HTML]{e6e3e3}{\textcolor{blue}{\textbf{--15.3}}}\\
\midrule
EMinT & 2.2 & 2.9 & 2.5 & 1.7 & --46.2 & --48.1 & --42.4 & --35.3 & 3.6 & 2.9 & 2.0 & 1.1 & --20.5 & --21.9 & --18.2 & --14.3\\
\cellcolor[HTML]{e6e3e3}{Elasso} & \cellcolor[HTML]{e6e3e3}{\textbf{ 1.4}} & \cellcolor[HTML]{e6e3e3}{\textbf{ 2.7}} & \cellcolor[HTML]{e6e3e3}{\textbf{ 2.4}} & \cellcolor[HTML]{e6e3e3}{\textbf{ 1.6}} & \cellcolor[HTML]{e6e3e3}{\textbf{--46.4}} & \cellcolor[HTML]{e6e3e3}{\textbf{--48.2}} & \cellcolor[HTML]{e6e3e3}{--42.4} & \cellcolor[HTML]{e6e3e3}{\textbf{--35.4}} & \cellcolor[HTML]{e6e3e3}{\textbf{ 3.1}} & \cellcolor[HTML]{e6e3e3}{3.2} & \cellcolor[HTML]{e6e3e3}{2.1} & \cellcolor[HTML]{e6e3e3}{1.2} & \cellcolor[HTML]{e6e3e3}{\textbf{--20.9}} & \cellcolor[HTML]{e6e3e3}{--21.9} & \cellcolor[HTML]{e6e3e3}{--18.2} & \cellcolor[HTML]{e6e3e3}{--14.3}\\
\bottomrule
\end{tabular}
\begin{tablenotes}[para]
\item Note: The Base row shows the average RMSE of the base forecasts. Entries below this row indicate the percentage decrease (negative) or increase (positive) in the average RMSE of the reconciled forecasts compared to the base forecasts. The entries with the lowest values in each column are highlighted in blue. In each panel, the proposed methods are indicated with a gray background, and methods that outperform the benchmark method are marked in bold.
\end{tablenotes}
\end{threeparttable}}

}

\end{table}%

\begin{table}

\caption{\label{tbl-s3-rmse}Out-of-sample forecast results for the
simulated data in Scenario C, Setup 1.}

\centering{

\centering
\resizebox{\linewidth}{!}{
\fontsize{11}{13}\selectfont
\begin{threeparttable}
\begin{tabular}{lrrrrrrrrrrrrrrrr}
\toprule
\multicolumn{1}{c}{} & \multicolumn{4}{c}{Top} & \multicolumn{4}{c}{Middle} & \multicolumn{4}{c}{Bottom} & \multicolumn{4}{c}{Average} \\
\cmidrule(l{3pt}r{3pt}){2-5} \cmidrule(l{3pt}r{3pt}){6-9} \cmidrule(l{3pt}r{3pt}){10-13} \cmidrule(l{3pt}r{3pt}){14-17}
Method & h=1 & 1--4 & 1--8 & 1--16 & h=1 & 1--4 & 1--8 & 1--16 & h=1 & 1--4 & 1--8 & 1--16 & h=1 & 1--4 & 1--8 & 1--16\\
\midrule
Base & 25.0 & 30.3 & 30.9 & 32.3 & 6.3 & 7.3 & 8.6 & 10.8 & 4.2 & 4.9 & 5.9 & 7.5 & 7.8 & 9.2 & 10.3 & 12.0\\
BU & --62.0 & \textcolor{blue}{\textbf{--64.4}} & \textcolor{blue}{\textbf{--59.0}} & --51.5 & \textcolor{blue}{\textbf{--0.3}} & \textcolor{blue}{\textbf{ 0.0}} & \textcolor{blue}{\textbf{ 0.1}} & \textcolor{blue}{\textbf{ 0.0}} & \textcolor{blue}{\textbf{ 0.0}} & \textcolor{blue}{\textbf{ 0.0}} & \textcolor{blue}{\textbf{ 0.0}} & \textcolor{blue}{\textbf{ 0.0}} & \textcolor{blue}{\textbf{--28.5}} & \textcolor{blue}{\textbf{--30.2}} & \textcolor{blue}{\textbf{--25.3}} & \textcolor{blue}{\textbf{--19.8}}\\
\midrule
OLS & --34.8 & --35.5 & --33.5 & --30.1 & 45.3 & 50.6 & 37.7 & 25.1 & 27.7 & 29.9 & 21.2 & 13.7 & 3.1 & 3.8 & 1.6 & --0.2\\
\cellcolor[HTML]{e6e3e3}{OLS-subset} & \cellcolor[HTML]{e6e3e3}{\textbf{--35.3}} & \cellcolor[HTML]{e6e3e3}{\textbf{--41.9}} & \cellcolor[HTML]{e6e3e3}{\textbf{--39.2}} & \cellcolor[HTML]{e6e3e3}{\textbf{--35.0}} & \cellcolor[HTML]{e6e3e3}{\textbf{43.9}} & \cellcolor[HTML]{e6e3e3}{\textbf{39.5}} & \cellcolor[HTML]{e6e3e3}{\textbf{29.5}} & \cellcolor[HTML]{e6e3e3}{\textbf{19.6}} & \cellcolor[HTML]{e6e3e3}{\textbf{27.1}} & \cellcolor[HTML]{e6e3e3}{\textbf{23.6}} & \cellcolor[HTML]{e6e3e3}{\textbf{16.8}} & \cellcolor[HTML]{e6e3e3}{\textbf{10.9}} & \cellcolor[HTML]{e6e3e3}{\textbf{  2.4}} & \cellcolor[HTML]{e6e3e3}{\textbf{ --3.5}} & \cellcolor[HTML]{e6e3e3}{\textbf{ --4.2}} & \cellcolor[HTML]{e6e3e3}{\textbf{ --4.5}}\\
\cellcolor[HTML]{e6e3e3}{OLS-intuitive} & \cellcolor[HTML]{e6e3e3}{\textbf{--41.2}} & \cellcolor[HTML]{e6e3e3}{\textbf{--49.2}} & \cellcolor[HTML]{e6e3e3}{\textbf{--45.5}} & \cellcolor[HTML]{e6e3e3}{\textbf{--40.0}} & \cellcolor[HTML]{e6e3e3}{\textbf{35.1}} & \cellcolor[HTML]{e6e3e3}{\textbf{26.8}} & \cellcolor[HTML]{e6e3e3}{\textbf{20.3}} & \cellcolor[HTML]{e6e3e3}{\textbf{13.7}} & \cellcolor[HTML]{e6e3e3}{\textbf{21.9}} & \cellcolor[HTML]{e6e3e3}{\textbf{15.9}} & \cellcolor[HTML]{e6e3e3}{\textbf{11.5}} & \cellcolor[HTML]{e6e3e3}{\textbf{ 7.6}} & \cellcolor[HTML]{e6e3e3}{\textbf{ --4.0}} & \cellcolor[HTML]{e6e3e3}{\textbf{--12.2}} & \cellcolor[HTML]{e6e3e3}{\textbf{--10.9}} & \cellcolor[HTML]{e6e3e3}{\textbf{ --9.1}}\\
\cellcolor[HTML]{e6e3e3}{OLS-lasso} & \cellcolor[HTML]{e6e3e3}{\textbf{--61.8}} & \cellcolor[HTML]{e6e3e3}{\textbf{--63.6}} & \cellcolor[HTML]{e6e3e3}{\textbf{--58.1}} & \cellcolor[HTML]{e6e3e3}{\textbf{--50.9}} & \cellcolor[HTML]{e6e3e3}{\textbf{ 0.4}} & \cellcolor[HTML]{e6e3e3}{\textbf{ 1.3}} & \cellcolor[HTML]{e6e3e3}{\textbf{ 1.3}} & \cellcolor[HTML]{e6e3e3}{\textbf{ 0.7}} & \cellcolor[HTML]{e6e3e3}{\textbf{ 0.3}} & \cellcolor[HTML]{e6e3e3}{\textbf{ 0.8}} & \cellcolor[HTML]{e6e3e3}{\textbf{ 0.6}} & \cellcolor[HTML]{e6e3e3}{\textbf{ 0.4}} & \cellcolor[HTML]{e6e3e3}{\textbf{--28.2}} & \cellcolor[HTML]{e6e3e3}{\textbf{--29.3}} & \cellcolor[HTML]{e6e3e3}{\textbf{--24.5}} & \cellcolor[HTML]{e6e3e3}{\textbf{--19.2}}\\
\midrule
WLSs & --50.9 & --52.4 & --48.7 & --43.3 & 17.6 & 20.0 & 14.5 & 9.3 & 9.6 & 11.3 & 7.7 & 4.9 & --16.3 & --16.7 & --14.9 & --12.5\\
\cellcolor[HTML]{e6e3e3}{WLSs-subset} & \cellcolor[HTML]{e6e3e3}{\textbf{--61.8}} & \cellcolor[HTML]{e6e3e3}{\textbf{--63.6}} & \cellcolor[HTML]{e6e3e3}{\textbf{--58.1}} & \cellcolor[HTML]{e6e3e3}{\textbf{--50.7}} & \cellcolor[HTML]{e6e3e3}{\textbf{ 0.3}} & \cellcolor[HTML]{e6e3e3}{\textbf{ 1.4}} & \cellcolor[HTML]{e6e3e3}{\textbf{ 1.4}} & \cellcolor[HTML]{e6e3e3}{\textbf{ 0.9}} & \cellcolor[HTML]{e6e3e3}{\textbf{ 0.3}} & \cellcolor[HTML]{e6e3e3}{\textbf{ 0.9}} & \cellcolor[HTML]{e6e3e3}{\textbf{ 0.7}} & \cellcolor[HTML]{e6e3e3}{\textbf{ 0.6}} & \cellcolor[HTML]{e6e3e3}{\textbf{--28.2}} & \cellcolor[HTML]{e6e3e3}{\textbf{--29.3}} & \cellcolor[HTML]{e6e3e3}{\textbf{--24.4}} & \cellcolor[HTML]{e6e3e3}{\textbf{--19.0}}\\
\cellcolor[HTML]{e6e3e3}{WLSs-intuitive} & \cellcolor[HTML]{e6e3e3}{\textbf{--61.8}} & \cellcolor[HTML]{e6e3e3}{\textbf{--63.8}} & \cellcolor[HTML]{e6e3e3}{\textbf{--58.3}} & \cellcolor[HTML]{e6e3e3}{\textbf{--50.9}} & \cellcolor[HTML]{e6e3e3}{\textbf{ 0.0}} & \cellcolor[HTML]{e6e3e3}{\textbf{ 1.0}} & \cellcolor[HTML]{e6e3e3}{\textbf{ 1.0}} & \cellcolor[HTML]{e6e3e3}{\textbf{ 0.7}} & \cellcolor[HTML]{e6e3e3}{\textbf{ 0.3}} & \cellcolor[HTML]{e6e3e3}{\textbf{ 0.7}} & \cellcolor[HTML]{e6e3e3}{\textbf{ 0.6}} & \cellcolor[HTML]{e6e3e3}{\textbf{ 0.5}} & \cellcolor[HTML]{e6e3e3}{\textbf{--28.3}} & \cellcolor[HTML]{e6e3e3}{\textbf{--29.5}} & \cellcolor[HTML]{e6e3e3}{\textbf{--24.6}} & \cellcolor[HTML]{e6e3e3}{\textbf{--19.2}}\\
\cellcolor[HTML]{e6e3e3}{WLSs-lasso} & \cellcolor[HTML]{e6e3e3}{\textbf{--61.7}} & \cellcolor[HTML]{e6e3e3}{\textbf{--63.5}} & \cellcolor[HTML]{e6e3e3}{\textbf{--58.0}} & \cellcolor[HTML]{e6e3e3}{\textbf{--50.7}} & \cellcolor[HTML]{e6e3e3}{\textbf{ 0.5}} & \cellcolor[HTML]{e6e3e3}{\textbf{ 1.5}} & \cellcolor[HTML]{e6e3e3}{\textbf{ 1.4}} & \cellcolor[HTML]{e6e3e3}{\textbf{ 0.9}} & \cellcolor[HTML]{e6e3e3}{\textbf{ 0.3}} & \cellcolor[HTML]{e6e3e3}{\textbf{ 0.9}} & \cellcolor[HTML]{e6e3e3}{\textbf{ 0.7}} & \cellcolor[HTML]{e6e3e3}{\textbf{ 0.5}} & \cellcolor[HTML]{e6e3e3}{\textbf{--28.1}} & \cellcolor[HTML]{e6e3e3}{\textbf{--29.2}} & \cellcolor[HTML]{e6e3e3}{\textbf{--24.4}} & \cellcolor[HTML]{e6e3e3}{\textbf{--19.1}}\\
\midrule
WLSv & --61.1 & --63.4 & --58.1 & --50.8 & 1.0 & 1.7 & 1.3 & 0.8 & 0.7 & 1.0 & 0.6 & 0.4 & --27.6 & --29.1 & --24.5 & --19.2\\
\cellcolor[HTML]{e6e3e3}{WLSv-subset} & \cellcolor[HTML]{e6e3e3}{\textbf{--61.9}} & \cellcolor[HTML]{e6e3e3}{\textbf{--63.6}} & \cellcolor[HTML]{e6e3e3}{\textbf{--58.2}} & \cellcolor[HTML]{e6e3e3}{\textbf{--50.9}} & \cellcolor[HTML]{e6e3e3}{\textbf{ 0.2}} & \cellcolor[HTML]{e6e3e3}{\textbf{ 1.3}} & \cellcolor[HTML]{e6e3e3}{\textbf{ 1.2}} & \cellcolor[HTML]{e6e3e3}{0.8} & \cellcolor[HTML]{e6e3e3}{\textbf{ 0.1}} & \cellcolor[HTML]{e6e3e3}{\textbf{ 0.8}} & \cellcolor[HTML]{e6e3e3}{0.6} & \cellcolor[HTML]{e6e3e3}{0.5} & \cellcolor[HTML]{e6e3e3}{\textbf{--28.3}} & \cellcolor[HTML]{e6e3e3}{\textbf{--29.3}} & \cellcolor[HTML]{e6e3e3}{--24.5} & \cellcolor[HTML]{e6e3e3}{--19.2}\\
\cellcolor[HTML]{e6e3e3}{WLSv-intuitive} & \cellcolor[HTML]{e6e3e3}{\textbf{--61.8}} & \cellcolor[HTML]{e6e3e3}{\textbf{--63.8}} & \cellcolor[HTML]{e6e3e3}{\textbf{--58.3}} & \cellcolor[HTML]{e6e3e3}{\textbf{--51.0}} & \cellcolor[HTML]{e6e3e3}{\textbf{ 0.0}} & \cellcolor[HTML]{e6e3e3}{\textbf{ 1.1}} & \cellcolor[HTML]{e6e3e3}{\textbf{ 1.1}} & \cellcolor[HTML]{e6e3e3}{\textbf{ 0.6}} & \cellcolor[HTML]{e6e3e3}{\textbf{ 0.1}} & \cellcolor[HTML]{e6e3e3}{\textbf{ 0.6}} & \cellcolor[HTML]{e6e3e3}{\textbf{ 0.5}} & \cellcolor[HTML]{e6e3e3}{0.4} & \cellcolor[HTML]{e6e3e3}{\textbf{--28.4}} & \cellcolor[HTML]{e6e3e3}{\textbf{--29.5}} & \cellcolor[HTML]{e6e3e3}{\textbf{--24.7}} & \cellcolor[HTML]{e6e3e3}{\textbf{--19.3}}\\
\cellcolor[HTML]{e6e3e3}{WLSv-lasso} & \cellcolor[HTML]{e6e3e3}{\textbf{--61.8}} & \cellcolor[HTML]{e6e3e3}{\textbf{--63.9}} & \cellcolor[HTML]{e6e3e3}{\textbf{--58.4}} & \cellcolor[HTML]{e6e3e3}{\textbf{--51.1}} & \cellcolor[HTML]{e6e3e3}{\textbf{ 0.2}} & \cellcolor[HTML]{e6e3e3}{\textbf{ 0.9}} & \cellcolor[HTML]{e6e3e3}{\textbf{ 0.9}} & \cellcolor[HTML]{e6e3e3}{\textbf{ 0.5}} & \cellcolor[HTML]{e6e3e3}{\textbf{ 0.1}} & \cellcolor[HTML]{e6e3e3}{\textbf{ 0.5}} & \cellcolor[HTML]{e6e3e3}{\textbf{ 0.4}} & \cellcolor[HTML]{e6e3e3}{\textbf{ 0.3}} & \cellcolor[HTML]{e6e3e3}{\textbf{--28.3}} & \cellcolor[HTML]{e6e3e3}{\textbf{--29.6}} & \cellcolor[HTML]{e6e3e3}{\textbf{--24.8}} & \cellcolor[HTML]{e6e3e3}{\textbf{--19.4}}\\
\midrule
MinT & --62.1 & --64.3 & --58.9 & \textcolor{blue}{\textbf{--51.6}} & --0.2 & 0.6 & 0.5 & 0.2 & 0.8 & 0.5 & 0.3 & 0.1 & --28.3 & --29.9 & --25.1 & \textcolor{blue}{\textbf{--19.8}}\\
\cellcolor[HTML]{e6e3e3}{MinT-subset} & \cellcolor[HTML]{e6e3e3}{--61.8} & \cellcolor[HTML]{e6e3e3}{--63.7} & \cellcolor[HTML]{e6e3e3}{--58.2} & \cellcolor[HTML]{e6e3e3}{--50.9} & \cellcolor[HTML]{e6e3e3}{0.4} & \cellcolor[HTML]{e6e3e3}{1.2} & \cellcolor[HTML]{e6e3e3}{1.3} & \cellcolor[HTML]{e6e3e3}{0.8} & \cellcolor[HTML]{e6e3e3}{0.8} & \cellcolor[HTML]{e6e3e3}{1.0} & \cellcolor[HTML]{e6e3e3}{0.7} & \cellcolor[HTML]{e6e3e3}{0.5} & \cellcolor[HTML]{e6e3e3}{--28.0} & \cellcolor[HTML]{e6e3e3}{--29.3} & \cellcolor[HTML]{e6e3e3}{--24.5} & \cellcolor[HTML]{e6e3e3}{--19.2}\\
\cellcolor[HTML]{e6e3e3}{MinT-intuitive} & \cellcolor[HTML]{e6e3e3}{--62.1} & \cellcolor[HTML]{e6e3e3}{--64.3} & \cellcolor[HTML]{e6e3e3}{--58.9} & \cellcolor[HTML]{e6e3e3}{\textcolor{blue}{\textbf{--51.6}}} & \cellcolor[HTML]{e6e3e3}{--0.2} & \cellcolor[HTML]{e6e3e3}{0.6} & \cellcolor[HTML]{e6e3e3}{0.5} & \cellcolor[HTML]{e6e3e3}{0.2} & \cellcolor[HTML]{e6e3e3}{0.8} & \cellcolor[HTML]{e6e3e3}{0.5} & \cellcolor[HTML]{e6e3e3}{0.3} & \cellcolor[HTML]{e6e3e3}{0.1} & \cellcolor[HTML]{e6e3e3}{--28.3} & \cellcolor[HTML]{e6e3e3}{--29.9} & \cellcolor[HTML]{e6e3e3}{--25.1} & \cellcolor[HTML]{e6e3e3}{\textcolor{blue}{\textbf{--19.8}}}\\
\cellcolor[HTML]{e6e3e3}{MinT-lasso} & \cellcolor[HTML]{e6e3e3}{--62.1} & \cellcolor[HTML]{e6e3e3}{\textcolor{blue}{\textbf{--64.4}}} & \cellcolor[HTML]{e6e3e3}{--58.9} & \cellcolor[HTML]{e6e3e3}{--51.5} & \cellcolor[HTML]{e6e3e3}{\textcolor{blue}{\textbf{--0.3}}} & \cellcolor[HTML]{e6e3e3}{\textbf{ 0.3}} & \cellcolor[HTML]{e6e3e3}{\textbf{ 0.4}} & \cellcolor[HTML]{e6e3e3}{\textbf{ 0.1}} & \cellcolor[HTML]{e6e3e3}{\textbf{ 0.6}} & \cellcolor[HTML]{e6e3e3}{\textbf{ 0.3}} & \cellcolor[HTML]{e6e3e3}{\textbf{ 0.1}} & \cellcolor[HTML]{e6e3e3}{0.1} & \cellcolor[HTML]{e6e3e3}{\textbf{--28.4}} & \cellcolor[HTML]{e6e3e3}{\textbf{--30.1}} & \cellcolor[HTML]{e6e3e3}{\textbf{--25.2}} & \cellcolor[HTML]{e6e3e3}{\textcolor{blue}{\textbf{--19.8}}}\\
\midrule
MinTs & \textcolor{blue}{\textbf{--62.2}} & \textcolor{blue}{\textbf{--64.4}} & \textcolor{blue}{\textbf{--59.0}} & \textcolor{blue}{\textbf{--51.6}} & \textcolor{blue}{\textbf{--0.3}} & 0.3 & 0.4 & 0.1 & 0.4 & 0.3 & 0.1 & \textcolor{blue}{\textbf{ 0.0}} & \textcolor{blue}{\textbf{--28.5}} & --30.1 & --25.2 & \textcolor{blue}{\textbf{--19.8}}\\
\cellcolor[HTML]{e6e3e3}{MinTs-subset} & \cellcolor[HTML]{e6e3e3}{--62.0} & \cellcolor[HTML]{e6e3e3}{--63.8} & \cellcolor[HTML]{e6e3e3}{--58.4} & \cellcolor[HTML]{e6e3e3}{--51.1} & \cellcolor[HTML]{e6e3e3}{0.4} & \cellcolor[HTML]{e6e3e3}{1.1} & \cellcolor[HTML]{e6e3e3}{1.2} & \cellcolor[HTML]{e6e3e3}{0.7} & \cellcolor[HTML]{e6e3e3}{0.5} & \cellcolor[HTML]{e6e3e3}{0.9} & \cellcolor[HTML]{e6e3e3}{0.7} & \cellcolor[HTML]{e6e3e3}{0.5} & \cellcolor[HTML]{e6e3e3}{--28.2} & \cellcolor[HTML]{e6e3e3}{--29.5} & \cellcolor[HTML]{e6e3e3}{--24.6} & \cellcolor[HTML]{e6e3e3}{--19.3}\\
\cellcolor[HTML]{e6e3e3}{MinTs-intuitive} & \cellcolor[HTML]{e6e3e3}{\textcolor{blue}{\textbf{--62.2}}} & \cellcolor[HTML]{e6e3e3}{\textcolor{blue}{\textbf{--64.4}}} & \cellcolor[HTML]{e6e3e3}{\textcolor{blue}{\textbf{--59.0}}} & \cellcolor[HTML]{e6e3e3}{\textcolor{blue}{\textbf{--51.6}}} & \cellcolor[HTML]{e6e3e3}{\textcolor{blue}{\textbf{--0.3}}} & \cellcolor[HTML]{e6e3e3}{0.3} & \cellcolor[HTML]{e6e3e3}{0.4} & \cellcolor[HTML]{e6e3e3}{0.1} & \cellcolor[HTML]{e6e3e3}{0.4} & \cellcolor[HTML]{e6e3e3}{0.3} & \cellcolor[HTML]{e6e3e3}{0.1} & \cellcolor[HTML]{e6e3e3}{\textcolor{blue}{\textbf{ 0.0}}} & \cellcolor[HTML]{e6e3e3}{\textcolor{blue}{\textbf{--28.5}}} & \cellcolor[HTML]{e6e3e3}{--30.1} & \cellcolor[HTML]{e6e3e3}{--25.2} & \cellcolor[HTML]{e6e3e3}{\textcolor{blue}{\textbf{--19.8}}}\\
\cellcolor[HTML]{e6e3e3}{MinTs-lasso} & \cellcolor[HTML]{e6e3e3}{\textcolor{blue}{\textbf{--62.2}}} & \cellcolor[HTML]{e6e3e3}{\textcolor{blue}{\textbf{--64.4}}} & \cellcolor[HTML]{e6e3e3}{--58.9} & \cellcolor[HTML]{e6e3e3}{--51.5} & \cellcolor[HTML]{e6e3e3}{--0.2} & \cellcolor[HTML]{e6e3e3}{0.3} & \cellcolor[HTML]{e6e3e3}{0.4} & \cellcolor[HTML]{e6e3e3}{0.1} & \cellcolor[HTML]{e6e3e3}{\textbf{ 0.2}} & \cellcolor[HTML]{e6e3e3}{\textbf{ 0.2}} & \cellcolor[HTML]{e6e3e3}{0.1} & \cellcolor[HTML]{e6e3e3}{\textcolor{blue}{\textbf{ 0.0}}} & \cellcolor[HTML]{e6e3e3}{\textcolor{blue}{\textbf{--28.5}}} & \cellcolor[HTML]{e6e3e3}{--30.1} & \cellcolor[HTML]{e6e3e3}{--25.2} & \cellcolor[HTML]{e6e3e3}{\textcolor{blue}{\textbf{--19.8}}}\\
\midrule
EMinT & --60.7 & --63.5 & --58.2 & --51.0 & 2.5 & 2.9 & 2.3 & 1.3 & 3.6 & 2.9 & 2.0 & 1.1 & --26.2 & --28.3 & --23.8 & --18.9\\
\cellcolor[HTML]{e6e3e3}{Elasso} & \cellcolor[HTML]{e6e3e3}{\textbf{--60.9}} & \cellcolor[HTML]{e6e3e3}{\textbf{--63.6}} & \cellcolor[HTML]{e6e3e3}{--58.2} & \cellcolor[HTML]{e6e3e3}{\textbf{--51.1}} & \cellcolor[HTML]{e6e3e3}{\textbf{ 2.3}} & \cellcolor[HTML]{e6e3e3}{\textbf{ 2.8}} & \cellcolor[HTML]{e6e3e3}{2.3} & \cellcolor[HTML]{e6e3e3}{1.3} & \cellcolor[HTML]{e6e3e3}{\textbf{ 3.1}} & \cellcolor[HTML]{e6e3e3}{3.1} & \cellcolor[HTML]{e6e3e3}{2.1} & \cellcolor[HTML]{e6e3e3}{1.2} & \cellcolor[HTML]{e6e3e3}{\textbf{--26.5}} & \cellcolor[HTML]{e6e3e3}{--28.3} & \cellcolor[HTML]{e6e3e3}{--23.8} & \cellcolor[HTML]{e6e3e3}{--18.9}\\
\bottomrule
\end{tabular}
\begin{tablenotes}[para]
\item Note: The Base row shows the average RMSE of the base forecasts. Entries below this row indicate the percentage decrease (negative) or increase (positive) in the average RMSE of the reconciled forecasts compared to the base forecasts. The entries with the lowest values in each column are highlighted in blue. In each panel, the proposed methods are indicated with a gray background, and methods that outperform the benchmark method are marked in bold.
\end{tablenotes}
\end{threeparttable}}

}

\end{table}%

\begin{table}

\caption{\label{tbl-s2-selection}Proportion of time series being
selected after using the proposed reconciliation methods with selection
in Scenario B, Setup 1.}

\centering{

\centering\begingroup\fontsize{11}{13}\selectfont

\begin{threeparttable}
\begin{tabular}{llrrrrrr>{}r}
\toprule
  & Top & A & B & AA & AB & BA & BB & Summary\\
\midrule
OLS-subset & 0.55 & 0.04 & 0.41 & 0.74 & 0.78 & 0.79 & 0.83 & \includegraphics[width=0.47in, height=0.1in]{/Users/xwan0362/Git/hfs/paper/_figs/s2_OLS-subset.png}\\
OLS-intuitive & 0.61 & 0.04 & 0.52 & 0.75 & 0.69 & 0.69 & 0.83 & \includegraphics[width=0.47in, height=0.1in]{/Users/xwan0362/Git/hfs/paper/_figs/s2_OLS-intuitive.png}\\
OLS-lasso & 0.04 & 0.35 & 0.02 & 1.00 & 1.00 & 1.00 & 1.00 & \includegraphics[width=0.47in, height=0.1in]{/Users/xwan0362/Git/hfs/paper/_figs/s2_OLS-lasso.png}\\
\midrule
WLSs-subset & 0.45 & 0.06 & 0.36 & 0.81 & 0.84 & 0.81 & 0.87 & \includegraphics[width=0.47in, height=0.1in]{/Users/xwan0362/Git/hfs/paper/_figs/s2_WLSs-subset.png}\\
WLSs-intuitive & 0.61 & 0.06 & 0.48 & 0.75 & 0.71 & 0.73 & 0.84 & \includegraphics[width=0.47in, height=0.1in]{/Users/xwan0362/Git/hfs/paper/_figs/s2_WLSs-intuitive.png}\\
WLSs-lasso & 0.02 & 0.33 & 0.02 & 1.00 & 1.00 & 1.00 & 1.00 & \includegraphics[width=0.47in, height=0.1in]{/Users/xwan0362/Git/hfs/paper/_figs/s2_WLSs-lasso.png}\\
\midrule
WLSv-subset & 0.54 & 0.29 & 0.46 & 0.91 & 0.94 & 0.86 & 0.89 & \includegraphics[width=0.47in, height=0.1in]{/Users/xwan0362/Git/hfs/paper/_figs/s2_WLSv-subset.png}\\
WLSv-intuitive & 0.59 & 0.32 & 0.53 & 0.82 & 0.86 & 0.77 & 0.86 & \includegraphics[width=0.47in, height=0.1in]{/Users/xwan0362/Git/hfs/paper/_figs/s2_WLSv-intuitive.png}\\
WLSv-lasso & 0.27 & 0.42 & 0.26 & 1.00 & 1.00 & 1.00 & 1.00 & \includegraphics[width=0.47in, height=0.1in]{/Users/xwan0362/Git/hfs/paper/_figs/s2_WLSv-lasso.png}\\
\midrule
MinT-subset & 0.69 & 0.64 & 0.66 & 0.95 & 0.96 & 0.90 & 0.90 & \includegraphics[width=0.47in, height=0.1in]{/Users/xwan0362/Git/hfs/paper/_figs/s2_MinT-subset.png}\\
MinT-intuitive & 1.00 & 1.00 & 1.00 & 1.00 & 1.00 & 1.00 & 1.00 & \includegraphics[width=0.47in, height=0.1in]{/Users/xwan0362/Git/hfs/paper/_figs/s2_MinT-intuitive.png}\\
MinT-lasso & 0.82 & 0.74 & 0.83 & 1.00 & 0.99 & 0.97 & 0.97 & \includegraphics[width=0.47in, height=0.1in]{/Users/xwan0362/Git/hfs/paper/_figs/s2_MinT-lasso.png}\\
\midrule
MinTs-subset & 0.62 & 0.63 & 0.58 & 0.95 & 0.96 & 0.90 & 0.86 & \includegraphics[width=0.47in, height=0.1in]{/Users/xwan0362/Git/hfs/paper/_figs/s2_MinTs-subset.png}\\
MinTs-intuitive & 1.00 & 1.00 & 1.00 & 1.00 & 1.00 & 1.00 & 1.00 & \includegraphics[width=0.47in, height=0.1in]{/Users/xwan0362/Git/hfs/paper/_figs/s2_MinTs-intuitive.png}\\
MinTs-lasso & 0.68 & 0.75 & 0.68 & 1.00 & 1.00 & 1.00 & 1.00 & \includegraphics[width=0.47in, height=0.1in]{/Users/xwan0362/Git/hfs/paper/_figs/s2_MinTs-lasso.png}\\
\midrule
Elasso & 0.78 & 0.95 & 0.68 & 1.00 & 1.00 & 1.00 & 1.00 & \includegraphics[width=0.47in, height=0.1in]{/Users/xwan0362/Git/hfs/paper/_figs/s2_Elasso.png}\\
\bottomrule
\end{tabular}
\begin{tablenotes}[para]
\item Note: the last column displays a stacked barplot for each method, based on the total number of selected series data from 500 simulation instances, with a darker sub-bar indicating a larger number.
\end{tablenotes}
\end{threeparttable}
\endgroup{}

}

\end{table}%

\begin{table}

\caption{\label{tbl-s3-selection}Proportion of time series being
selected after using the proposed reconciliation methods with selection
in Scenario C, Setup 1.}

\centering{

\centering\begingroup\fontsize{11}{13}\selectfont

\begin{threeparttable}
\begin{tabular}{llrrrrrr>{}r}
\toprule
  & Top & A & B & AA & AB & BA & BB & Summary\\
\midrule
OLS-subset & 0.75 & 0.45 & 0.44 & 0.82 & 0.79 & 0.83 & 0.80 & \includegraphics[width=0.47in, height=0.1in]{/Users/xwan0362/Git/hfs/paper/_figs/s3_OLS-subset.png}\\
OLS-intuitive & 0.47 & 0.70 & 0.69 & 0.86 & 0.92 & 0.90 & 0.89 & \includegraphics[width=0.47in, height=0.1in]{/Users/xwan0362/Git/hfs/paper/_figs/s3_OLS-intuitive.png}\\
OLS-lasso & 0.38 & 0.01 & 0.01 & 1.00 & 1.00 & 1.00 & 1.00 & \includegraphics[width=0.47in, height=0.1in]{/Users/xwan0362/Git/hfs/paper/_figs/s3_OLS-lasso.png}\\
\midrule
WLSs-subset & 0.08 & 0.42 & 0.41 & 0.87 & 0.85 & 0.84 & 0.89 & \includegraphics[width=0.47in, height=0.1in]{/Users/xwan0362/Git/hfs/paper/_figs/s3_WLSs-subset.png}\\
WLSs-intuitive & 0.06 & 0.55 & 0.50 & 0.66 & 0.87 & 0.69 & 0.88 & \includegraphics[width=0.47in, height=0.1in]{/Users/xwan0362/Git/hfs/paper/_figs/s3_WLSs-intuitive.png}\\
WLSs-lasso & 0.35 & 0.03 & 0.03 & 1.00 & 1.00 & 1.00 & 1.00 & \includegraphics[width=0.47in, height=0.1in]{/Users/xwan0362/Git/hfs/paper/_figs/s3_WLSs-lasso.png}\\
\midrule
WLSv-subset & 0.31 & 0.67 & 0.65 & 0.88 & 0.90 & 0.91 & 0.90 & \includegraphics[width=0.47in, height=0.1in]{/Users/xwan0362/Git/hfs/paper/_figs/s3_WLSv-subset.png}\\
WLSv-intuitive & 0.34 & 0.63 & 0.60 & 0.80 & 0.89 & 0.84 & 0.87 & \includegraphics[width=0.47in, height=0.1in]{/Users/xwan0362/Git/hfs/paper/_figs/s3_WLSv-intuitive.png}\\
WLSv-lasso & 0.45 & 0.35 & 0.36 & 1.00 & 1.00 & 1.00 & 1.00 & \includegraphics[width=0.47in, height=0.1in]{/Users/xwan0362/Git/hfs/paper/_figs/s3_WLSv-lasso.png}\\
\midrule
MinT-subset & 0.69 & 0.78 & 0.80 & 0.91 & 0.91 & 0.91 & 0.91 & \includegraphics[width=0.47in, height=0.1in]{/Users/xwan0362/Git/hfs/paper/_figs/s3_MinT-subset.png}\\
MinT-intuitive & 1.00 & 1.00 & 1.00 & 1.00 & 1.00 & 1.00 & 1.00 & \includegraphics[width=0.47in, height=0.1in]{/Users/xwan0362/Git/hfs/paper/_figs/s3_MinT-intuitive.png}\\
MinT-lasso & 0.75 & 0.89 & 0.86 & 0.97 & 0.97 & 0.97 & 0.97 & \includegraphics[width=0.47in, height=0.1in]{/Users/xwan0362/Git/hfs/paper/_figs/s3_MinT-lasso.png}\\
\midrule
MinTs-subset & 0.67 & 0.74 & 0.76 & 0.90 & 0.89 & 0.88 & 0.91 & \includegraphics[width=0.47in, height=0.1in]{/Users/xwan0362/Git/hfs/paper/_figs/s3_MinTs-subset.png}\\
MinTs-intuitive & 1.00 & 1.00 & 1.00 & 1.00 & 1.00 & 1.00 & 1.00 & \includegraphics[width=0.47in, height=0.1in]{/Users/xwan0362/Git/hfs/paper/_figs/s3_MinTs-intuitive.png}\\
MinTs-lasso & 0.77 & 0.72 & 0.73 & 1.00 & 1.00 & 1.00 & 1.00 & \includegraphics[width=0.47in, height=0.1in]{/Users/xwan0362/Git/hfs/paper/_figs/s3_MinTs-lasso.png}\\
\midrule
Elasso & 0.95 & 0.64 & 0.64 & 1.00 & 1.00 & 1.00 & 1.00 & \includegraphics[width=0.47in, height=0.1in]{/Users/xwan0362/Git/hfs/paper/_figs/s3_Elasso.png}\\
\bottomrule
\end{tabular}
\begin{tablenotes}[para]
\item Note: the last column displays a stacked barplot for each method, based on the total number of selected series data from 500 simulation instances, with a darker sub-bar indicating a larger number.
\end{tablenotes}
\end{threeparttable}
\endgroup{}

}

\end{table}%

\begin{table}

\caption{\label{tbl-corr-selection-pos}Proportion of time series being
selected after using the proposed reconciliation methods with selection
in Setup 2, with the error correlation being 0.8.}

\centering{

\centering\begingroup\fontsize{11}{13}\selectfont

\begin{threeparttable}
\begin{tabular}{llrrrrrr>{}r}
\toprule
  & Top & A & B & AA & AB & BA & BB & Summary\\
\midrule
OLS-subset & 0.33 & 0.52 & 0.96 & 0.95 & 0.98 & 0.96 & 0.78 & \includegraphics[width=0.47in, height=0.1in]{/Users/xwan0362/Git/hfs/paper/_figs/corr_pos_OLS-subset.png}\\
OLS-intuitive & 0.54 & 0.77 & 0.93 & 0.89 & 0.97 & 0.83 & 0.85 & \includegraphics[width=0.47in, height=0.1in]{/Users/xwan0362/Git/hfs/paper/_figs/corr_pos_OLS-intuitive.png}\\
OLS-lasso & 0.69 & 0.53 & 0.60 & 1.00 & 1.00 & 1.00 & 1.00 & \includegraphics[width=0.47in, height=0.1in]{/Users/xwan0362/Git/hfs/paper/_figs/corr_pos_OLS-lasso.png}\\
\midrule
WLSs-subset & 0.29 & 0.60 & 1.00 & 1.00 & 1.00 & 0.98 & 0.86 & \includegraphics[width=0.47in, height=0.1in]{/Users/xwan0362/Git/hfs/paper/_figs/corr_pos_WLSs-subset.png}\\
WLSs-intuitive & 0.63 & 0.67 & 0.99 & 0.98 & 1.00 & 0.93 & 0.86 & \includegraphics[width=0.47in, height=0.1in]{/Users/xwan0362/Git/hfs/paper/_figs/corr_pos_WLSs-intuitive.png}\\
WLSs-lasso & 0.69 & 0.76 & 0.91 & 1.00 & 1.00 & 1.00 & 1.00 & \includegraphics[width=0.47in, height=0.1in]{/Users/xwan0362/Git/hfs/paper/_figs/corr_pos_WLSs-lasso.png}\\
\midrule
WLSv-subset & 0.32 & 0.55 & 1.00 & 1.00 & 1.00 & 0.99 & 0.76 & \includegraphics[width=0.47in, height=0.1in]{/Users/xwan0362/Git/hfs/paper/_figs/corr_pos_WLSv-subset.png}\\
WLSv-intuitive & 0.58 & 0.56 & 1.00 & 1.00 & 0.98 & 1.00 & 0.75 & \includegraphics[width=0.47in, height=0.1in]{/Users/xwan0362/Git/hfs/paper/_figs/corr_pos_WLSv-intuitive.png}\\
WLSv-lasso & 0.77 & 0.84 & 0.99 & 1.00 & 1.00 & 1.00 & 1.00 & \includegraphics[width=0.47in, height=0.1in]{/Users/xwan0362/Git/hfs/paper/_figs/corr_pos_WLSv-lasso.png}\\
\midrule
MinT-subset & 1.00 & 1.00 & 1.00 & 1.00 & 1.00 & 1.00 & 1.00 & \includegraphics[width=0.47in, height=0.1in]{/Users/xwan0362/Git/hfs/paper/_figs/corr_pos_MinT-subset.png}\\
MinT-intuitive & 1.00 & 1.00 & 1.00 & 1.00 & 1.00 & 1.00 & 1.00 & \includegraphics[width=0.47in, height=0.1in]{/Users/xwan0362/Git/hfs/paper/_figs/corr_pos_MinT-intuitive.png}\\
MinT-lasso & 1.00 & 1.00 & 1.00 & 1.00 & 1.00 & 1.00 & 1.00 & \includegraphics[width=0.47in, height=0.1in]{/Users/xwan0362/Git/hfs/paper/_figs/corr_pos_MinT-lasso.png}\\
\midrule
MinTs-subset & 1.00 & 1.00 & 1.00 & 1.00 & 1.00 & 1.00 & 1.00 & \includegraphics[width=0.47in, height=0.1in]{/Users/xwan0362/Git/hfs/paper/_figs/corr_pos_MinTs-subset.png}\\
MinTs-intuitive & 1.00 & 1.00 & 1.00 & 1.00 & 1.00 & 1.00 & 1.00 & \includegraphics[width=0.47in, height=0.1in]{/Users/xwan0362/Git/hfs/paper/_figs/corr_pos_MinTs-intuitive.png}\\
MinTs-lasso & 1.00 & 1.00 & 1.00 & 1.00 & 1.00 & 1.00 & 1.00 & \includegraphics[width=0.47in, height=0.1in]{/Users/xwan0362/Git/hfs/paper/_figs/corr_pos_MinTs-lasso.png}\\
\midrule
Elasso & 0.73 & 0.65 & 0.98 & 0.98 & 0.86 & 1.00 & 0.99 & \includegraphics[width=0.47in, height=0.1in]{/Users/xwan0362/Git/hfs/paper/_figs/corr_pos_Elasso.png}\\
\bottomrule
\end{tabular}
\begin{tablenotes}[para]
\item Note: the last column displays a stacked barplot for each method, based on the total number of selected series data from 500 simulation instances, with a darker sub-bar indicating a larger number.
\end{tablenotes}
\end{threeparttable}
\endgroup{}

}

\end{table}%

\begin{table}

\caption{\label{tbl-labour-rmse}Out-of-sample forecast results on a
single test set (from August 2022 to July 2023) for Australian labour
force data.}

\centering{

\centering
\resizebox{\linewidth}{!}{
\fontsize{11}{13}\selectfont
\begin{threeparttable}
\begin{tabular}{lrrrrrrrrrrrrrrrrrrrr}
\toprule
\multicolumn{1}{c}{} & \multicolumn{4}{c}{Top} & \multicolumn{4}{c}{Duration} & \multicolumn{4}{c}{STT} & \multicolumn{4}{c}{Duration x STT} & \multicolumn{4}{c}{Average} \\
\cmidrule(l{3pt}r{3pt}){2-5} \cmidrule(l{3pt}r{3pt}){6-9} \cmidrule(l{3pt}r{3pt}){10-13} \cmidrule(l{3pt}r{3pt}){14-17} \cmidrule(l{3pt}r{3pt}){18-21}
Method & h=1 & 1--4 & 1--8 & 1--12 & h=1 & 1--4 & 1--8 & 1--12 & h=1 & 1--4 & 1--8 & 1--12 & h=1 & 1--4 & 1--8 & 1--12 & h=1 & 1--4 & 1--8 & 1--12\\
\midrule
Base & 18.5 & 13.6 & 18.3 & 28.3 & 11.8 & 12.7 & 13.9 & 16.9 & 6.7 & 6.0 & 6.0 & 6.3 & 2.3 & 2.6 & 2.7 & 2.9 & 4.1 & 4.1 & 4.4 & 5.1\\
BU & \textcolor{blue}{\textbf{--81.5}} & 33.4 & --19.9 & --45.0 & \textcolor{blue}{\textbf{--30.7}} & --9.2 & --7.9 & --10.1 & --12.9 & --10.4 & --13.4 & --13.5 & 0.0 & 0.0 & 0.0 & 0.0 & \textcolor{blue}{\textbf{--17.1}} & --2.8 & --5.9 & --9.3\\
\midrule
OLS & --16.2 & --14.2 & --13.4 & --10.4 & 2.5 & --2.6 & --2.7 & --0.6 & --1.8 & --0.9 & --1.9 & 0.3 & 6.7 & 5.1 & 5.1 & 4.9 & 2.1 & 0.7 & 0.4 & 1.1\\
\cellcolor[HTML]{e6e3e3}{OLS-subset} & \cellcolor[HTML]{e6e3e3}{\textbf{--17.0}} & \cellcolor[HTML]{e6e3e3}{--2.1} & \cellcolor[HTML]{e6e3e3}{\textbf{--31.2}} & \cellcolor[HTML]{e6e3e3}{\textbf{--38.4}} & \cellcolor[HTML]{e6e3e3}{\textbf{  2.0}} & \cellcolor[HTML]{e6e3e3}{--1.7} & \cellcolor[HTML]{e6e3e3}{\textbf{ --5.0}} & \cellcolor[HTML]{e6e3e3}{\textbf{ --2.7}} & \cellcolor[HTML]{e6e3e3}{\textbf{ --2.7}} & \cellcolor[HTML]{e6e3e3}{\textbf{ --4.2}} & \cellcolor[HTML]{e6e3e3}{\textbf{ --8.6}} & \cellcolor[HTML]{e6e3e3}{\textbf{ --7.3}} & \cellcolor[HTML]{e6e3e3}{6.7} & \cellcolor[HTML]{e6e3e3}{5.2} & \cellcolor[HTML]{e6e3e3}{\textbf{ 3.3}} & \cellcolor[HTML]{e6e3e3}{\textbf{ 3.7}} & \cellcolor[HTML]{e6e3e3}{\textbf{  1.7}} & \cellcolor[HTML]{e6e3e3}{1.1} & \cellcolor[HTML]{e6e3e3}{\textbf{ --3.5}} & \cellcolor[HTML]{e6e3e3}{\textbf{ --3.8}}\\
\cellcolor[HTML]{e6e3e3}{OLS-intuitive} & \cellcolor[HTML]{e6e3e3}{\textbf{--79.6}} & \cellcolor[HTML]{e6e3e3}{\textbf{--23.9}} & \cellcolor[HTML]{e6e3e3}{\textbf{--31.9}} & \cellcolor[HTML]{e6e3e3}{\textbf{--32.1}} & \cellcolor[HTML]{e6e3e3}{\textbf{--13.0}} & \cellcolor[HTML]{e6e3e3}{0.4} & \cellcolor[HTML]{e6e3e3}{--0.8} & \cellcolor[HTML]{e6e3e3}{0.3} & \cellcolor[HTML]{e6e3e3}{\textbf{ --8.9}} & \cellcolor[HTML]{e6e3e3}{4.9} & \cellcolor[HTML]{e6e3e3}{7.0} & \cellcolor[HTML]{e6e3e3}{13.2} & \cellcolor[HTML]{e6e3e3}{\textbf{  6.3}} & \cellcolor[HTML]{e6e3e3}{12.3} & \cellcolor[HTML]{e6e3e3}{12.4} & \cellcolor[HTML]{e6e3e3}{11.6} & \cellcolor[HTML]{e6e3e3}{\textbf{ --8.5}} & \cellcolor[HTML]{e6e3e3}{5.6} & \cellcolor[HTML]{e6e3e3}{4.7} & \cellcolor[HTML]{e6e3e3}{4.4}\\
\cellcolor[HTML]{e6e3e3}{OLS-lasso} & \cellcolor[HTML]{e6e3e3}{--16.2} & \cellcolor[HTML]{e6e3e3}{--14.2} & \cellcolor[HTML]{e6e3e3}{--13.4} & \cellcolor[HTML]{e6e3e3}{--10.4} & \cellcolor[HTML]{e6e3e3}{2.5} & \cellcolor[HTML]{e6e3e3}{--2.6} & \cellcolor[HTML]{e6e3e3}{--2.7} & \cellcolor[HTML]{e6e3e3}{--0.6} & \cellcolor[HTML]{e6e3e3}{--1.8} & \cellcolor[HTML]{e6e3e3}{--0.9} & \cellcolor[HTML]{e6e3e3}{--1.9} & \cellcolor[HTML]{e6e3e3}{0.3} & \cellcolor[HTML]{e6e3e3}{6.7} & \cellcolor[HTML]{e6e3e3}{5.1} & \cellcolor[HTML]{e6e3e3}{5.1} & \cellcolor[HTML]{e6e3e3}{4.9} & \cellcolor[HTML]{e6e3e3}{2.1} & \cellcolor[HTML]{e6e3e3}{0.7} & \cellcolor[HTML]{e6e3e3}{0.4} & \cellcolor[HTML]{e6e3e3}{1.1}\\
\midrule
WLSs & --60.6 & --29.4 & --44.0 & --38.6 & --12.0 & --7.6 & --6.9 & --5.9 & --6.5 & --8.1 & --9.4 & --8.0 & 3.2 & 1.7 & 1.7 & 1.6 & --7.7 & --4.4 & --5.8 & --5.9\\
\cellcolor[HTML]{e6e3e3}{WLSs-subset} & \cellcolor[HTML]{e6e3e3}{\textbf{--61.6}} & \cellcolor[HTML]{e6e3e3}{--22.4} & \cellcolor[HTML]{e6e3e3}{\textcolor{blue}{\textbf{--47.3}}} & \cellcolor[HTML]{e6e3e3}{\textcolor{blue}{\textbf{--50.4}}} & \cellcolor[HTML]{e6e3e3}{--12.0} & \cellcolor[HTML]{e6e3e3}{\textbf{--8.0}} & \cellcolor[HTML]{e6e3e3}{\textbf{--10.5}} & \cellcolor[HTML]{e6e3e3}{\textbf{ --7.8}} & \cellcolor[HTML]{e6e3e3}{\textbf{ --6.6}} & \cellcolor[HTML]{e6e3e3}{\textbf{--10.7}} & \cellcolor[HTML]{e6e3e3}{\textbf{--14.3}} & \cellcolor[HTML]{e6e3e3}{\textbf{--12.8}} & \cellcolor[HTML]{e6e3e3}{3.2} & \cellcolor[HTML]{e6e3e3}{5.6} & \cellcolor[HTML]{e6e3e3}{4.1} & \cellcolor[HTML]{e6e3e3}{5.9} & \cellcolor[HTML]{e6e3e3}{\textbf{ --7.8}} & \cellcolor[HTML]{e6e3e3}{--2.8} & \cellcolor[HTML]{e6e3e3}{\textbf{ --6.7}} & \cellcolor[HTML]{e6e3e3}{\textbf{ --6.4}}\\
\cellcolor[HTML]{e6e3e3}{WLSs-intuitive} & \cellcolor[HTML]{e6e3e3}{--60.6} & \cellcolor[HTML]{e6e3e3}{--29.4} & \cellcolor[HTML]{e6e3e3}{--44.0} & \cellcolor[HTML]{e6e3e3}{--38.6} & \cellcolor[HTML]{e6e3e3}{--12.0} & \cellcolor[HTML]{e6e3e3}{--7.6} & \cellcolor[HTML]{e6e3e3}{--6.9} & \cellcolor[HTML]{e6e3e3}{--5.9} & \cellcolor[HTML]{e6e3e3}{--6.5} & \cellcolor[HTML]{e6e3e3}{--8.1} & \cellcolor[HTML]{e6e3e3}{--9.4} & \cellcolor[HTML]{e6e3e3}{--8.0} & \cellcolor[HTML]{e6e3e3}{3.2} & \cellcolor[HTML]{e6e3e3}{1.7} & \cellcolor[HTML]{e6e3e3}{1.7} & \cellcolor[HTML]{e6e3e3}{1.6} & \cellcolor[HTML]{e6e3e3}{--7.7} & \cellcolor[HTML]{e6e3e3}{--4.4} & \cellcolor[HTML]{e6e3e3}{--5.8} & \cellcolor[HTML]{e6e3e3}{--5.9}\\
\cellcolor[HTML]{e6e3e3}{WLSs-lasso} & \cellcolor[HTML]{e6e3e3}{--60.6} & \cellcolor[HTML]{e6e3e3}{--29.4} & \cellcolor[HTML]{e6e3e3}{--44.0} & \cellcolor[HTML]{e6e3e3}{--38.6} & \cellcolor[HTML]{e6e3e3}{--12.0} & \cellcolor[HTML]{e6e3e3}{--7.6} & \cellcolor[HTML]{e6e3e3}{--6.9} & \cellcolor[HTML]{e6e3e3}{--5.9} & \cellcolor[HTML]{e6e3e3}{--6.5} & \cellcolor[HTML]{e6e3e3}{--8.1} & \cellcolor[HTML]{e6e3e3}{--9.4} & \cellcolor[HTML]{e6e3e3}{--8.0} & \cellcolor[HTML]{e6e3e3}{3.2} & \cellcolor[HTML]{e6e3e3}{1.7} & \cellcolor[HTML]{e6e3e3}{1.7} & \cellcolor[HTML]{e6e3e3}{1.6} & \cellcolor[HTML]{e6e3e3}{--7.7} & \cellcolor[HTML]{e6e3e3}{--4.4} & \cellcolor[HTML]{e6e3e3}{--5.8} & \cellcolor[HTML]{e6e3e3}{--5.9}\\
\midrule
WLSv & --60.6 & --29.1 & --41.4 & --36.6 & --14.5 & --8.7 & --5.6 & --4.8 & --3.3 & --6.8 & --8.0 & --7.0 & 5.5 & 2.6 & 2.6 & 3.1 & --6.7 & --4.0 & --4.5 & --4.6\\
\cellcolor[HTML]{e6e3e3}{WLSv-subset} & \cellcolor[HTML]{e6e3e3}{--51.6} & \cellcolor[HTML]{e6e3e3}{\textcolor{blue}{\textbf{--32.7}}} & \cellcolor[HTML]{e6e3e3}{--36.6} & \cellcolor[HTML]{e6e3e3}{--29.6} & \cellcolor[HTML]{e6e3e3}{\textbf{--18.3}} & \cellcolor[HTML]{e6e3e3}{\textbf{--9.8}} & \cellcolor[HTML]{e6e3e3}{\textbf{--10.5}} & \cellcolor[HTML]{e6e3e3}{\textbf{--10.9}} & \cellcolor[HTML]{e6e3e3}{--1.1} & \cellcolor[HTML]{e6e3e3}{--4.3} & \cellcolor[HTML]{e6e3e3}{\textbf{ --8.1}} & \cellcolor[HTML]{e6e3e3}{\textbf{ --7.3}} & \cellcolor[HTML]{e6e3e3}{\textbf{  2.5}} & \cellcolor[HTML]{e6e3e3}{\textbf{ 2.1}} & \cellcolor[HTML]{e6e3e3}{\textbf{ 2.3}} & \cellcolor[HTML]{e6e3e3}{\textbf{ 1.8}} & \cellcolor[HTML]{e6e3e3}{\textbf{ --7.9}} & \cellcolor[HTML]{e6e3e3}{\textbf{--4.3}} & \cellcolor[HTML]{e6e3e3}{\textbf{ --5.8}} & \cellcolor[HTML]{e6e3e3}{\textbf{ --6.5}}\\
\cellcolor[HTML]{e6e3e3}{WLSv-intuitive} & \cellcolor[HTML]{e6e3e3}{--60.6} & \cellcolor[HTML]{e6e3e3}{--29.1} & \cellcolor[HTML]{e6e3e3}{--41.4} & \cellcolor[HTML]{e6e3e3}{--36.6} & \cellcolor[HTML]{e6e3e3}{--14.5} & \cellcolor[HTML]{e6e3e3}{--8.7} & \cellcolor[HTML]{e6e3e3}{--5.6} & \cellcolor[HTML]{e6e3e3}{--4.8} & \cellcolor[HTML]{e6e3e3}{--3.3} & \cellcolor[HTML]{e6e3e3}{--6.8} & \cellcolor[HTML]{e6e3e3}{--8.0} & \cellcolor[HTML]{e6e3e3}{--7.0} & \cellcolor[HTML]{e6e3e3}{5.5} & \cellcolor[HTML]{e6e3e3}{2.6} & \cellcolor[HTML]{e6e3e3}{2.6} & \cellcolor[HTML]{e6e3e3}{3.1} & \cellcolor[HTML]{e6e3e3}{--6.7} & \cellcolor[HTML]{e6e3e3}{--4.0} & \cellcolor[HTML]{e6e3e3}{--4.5} & \cellcolor[HTML]{e6e3e3}{--4.6}\\
\cellcolor[HTML]{e6e3e3}{WLSv-lasso} & \cellcolor[HTML]{e6e3e3}{--60.6} & \cellcolor[HTML]{e6e3e3}{--29.1} & \cellcolor[HTML]{e6e3e3}{--41.4} & \cellcolor[HTML]{e6e3e3}{--36.6} & \cellcolor[HTML]{e6e3e3}{--14.5} & \cellcolor[HTML]{e6e3e3}{--8.7} & \cellcolor[HTML]{e6e3e3}{--5.6} & \cellcolor[HTML]{e6e3e3}{--4.8} & \cellcolor[HTML]{e6e3e3}{--3.3} & \cellcolor[HTML]{e6e3e3}{--6.8} & \cellcolor[HTML]{e6e3e3}{--8.0} & \cellcolor[HTML]{e6e3e3}{--7.0} & \cellcolor[HTML]{e6e3e3}{5.5} & \cellcolor[HTML]{e6e3e3}{2.6} & \cellcolor[HTML]{e6e3e3}{2.6} & \cellcolor[HTML]{e6e3e3}{3.1} & \cellcolor[HTML]{e6e3e3}{--6.7} & \cellcolor[HTML]{e6e3e3}{--4.0} & \cellcolor[HTML]{e6e3e3}{--4.5} & \cellcolor[HTML]{e6e3e3}{--4.6}\\
\midrule
MinTs & --27.0 & --21.1 & --22.9 & --21.8 & --9.1 & --7.9 & --6.7 & --4.6 & --3.6 & --9.0 & --10.5 & --7.6 & 7.7 & 3.7 & 3.0 & 3.4 & --1.9 & --3.3 & --3.9 & --3.1\\
\cellcolor[HTML]{e6e3e3}{MinTs-subset} & \cellcolor[HTML]{e6e3e3}{\textbf{--41.4}} & \cellcolor[HTML]{e6e3e3}{--9.3} & \cellcolor[HTML]{e6e3e3}{--17.8} & \cellcolor[HTML]{e6e3e3}{\textbf{--45.1}} & \cellcolor[HTML]{e6e3e3}{\textbf{--12.2}} & \cellcolor[HTML]{e6e3e3}{--5.0} & \cellcolor[HTML]{e6e3e3}{\textbf{ --8.2}} & \cellcolor[HTML]{e6e3e3}{\textbf{ --6.8}} & \cellcolor[HTML]{e6e3e3}{\textbf{ --6.1}} & \cellcolor[HTML]{e6e3e3}{\textbf{ --9.8}} & \cellcolor[HTML]{e6e3e3}{--7.1} & \cellcolor[HTML]{e6e3e3}{\textbf{ --9.3}} & \cellcolor[HTML]{e6e3e3}{\textbf{  5.3}} & \cellcolor[HTML]{e6e3e3}{4.7} & \cellcolor[HTML]{e6e3e3}{3.8} & \cellcolor[HTML]{e6e3e3}{3.6} & \cellcolor[HTML]{e6e3e3}{\textbf{ --5.3}} & \cellcolor[HTML]{e6e3e3}{--1.5} & \cellcolor[HTML]{e6e3e3}{--3.0} & \cellcolor[HTML]{e6e3e3}{\textbf{ --6.1}}\\
\cellcolor[HTML]{e6e3e3}{MinTs-intuitive} & \cellcolor[HTML]{e6e3e3}{--27.0} & \cellcolor[HTML]{e6e3e3}{--21.1} & \cellcolor[HTML]{e6e3e3}{--22.9} & \cellcolor[HTML]{e6e3e3}{--21.8} & \cellcolor[HTML]{e6e3e3}{--9.1} & \cellcolor[HTML]{e6e3e3}{--7.9} & \cellcolor[HTML]{e6e3e3}{--6.7} & \cellcolor[HTML]{e6e3e3}{--4.6} & \cellcolor[HTML]{e6e3e3}{--3.6} & \cellcolor[HTML]{e6e3e3}{--9.0} & \cellcolor[HTML]{e6e3e3}{--10.5} & \cellcolor[HTML]{e6e3e3}{--7.6} & \cellcolor[HTML]{e6e3e3}{7.7} & \cellcolor[HTML]{e6e3e3}{3.7} & \cellcolor[HTML]{e6e3e3}{3.0} & \cellcolor[HTML]{e6e3e3}{3.4} & \cellcolor[HTML]{e6e3e3}{--1.9} & \cellcolor[HTML]{e6e3e3}{--3.3} & \cellcolor[HTML]{e6e3e3}{--3.9} & \cellcolor[HTML]{e6e3e3}{--3.1}\\
\cellcolor[HTML]{e6e3e3}{MinTs-lasso} & \cellcolor[HTML]{e6e3e3}{--27.0} & \cellcolor[HTML]{e6e3e3}{--21.1} & \cellcolor[HTML]{e6e3e3}{--22.9} & \cellcolor[HTML]{e6e3e3}{--21.8} & \cellcolor[HTML]{e6e3e3}{--9.1} & \cellcolor[HTML]{e6e3e3}{--7.9} & \cellcolor[HTML]{e6e3e3}{--6.7} & \cellcolor[HTML]{e6e3e3}{--4.6} & \cellcolor[HTML]{e6e3e3}{--3.6} & \cellcolor[HTML]{e6e3e3}{--9.0} & \cellcolor[HTML]{e6e3e3}{--10.5} & \cellcolor[HTML]{e6e3e3}{--7.6} & \cellcolor[HTML]{e6e3e3}{7.7} & \cellcolor[HTML]{e6e3e3}{3.7} & \cellcolor[HTML]{e6e3e3}{3.0} & \cellcolor[HTML]{e6e3e3}{3.4} & \cellcolor[HTML]{e6e3e3}{--1.9} & \cellcolor[HTML]{e6e3e3}{--3.3} & \cellcolor[HTML]{e6e3e3}{--3.9} & \cellcolor[HTML]{e6e3e3}{--3.1}\\
\midrule
EMinT & --60.4 & --14.0 & 1.4 & --29.9 & --6.0 & 12.0 & 10.7 & --6.7 & 16.7 & --0.9 & --12.4 & \textcolor{blue}{\textbf{--21.0}} & 23.3 & 17.2 & 16.7 & 10.1 & 7.7 & 10.8 & 9.0 & --3.7\\
\cellcolor[HTML]{e6e3e3}{Elasso} & \cellcolor[HTML]{e6e3e3}{--4.2} & \cellcolor[HTML]{e6e3e3}{--3.3} & \cellcolor[HTML]{e6e3e3}{\textbf{--22.3}} & \cellcolor[HTML]{e6e3e3}{--8.0} & \cellcolor[HTML]{e6e3e3}{\textbf{--19.7}} & \cellcolor[HTML]{e6e3e3}{\textcolor{blue}{\textbf{--9.9}}} & \cellcolor[HTML]{e6e3e3}{\textcolor{blue}{\textbf{--19.9}}} & \cellcolor[HTML]{e6e3e3}{\textcolor{blue}{\textbf{--25.3}}} & \cellcolor[HTML]{e6e3e3}{\textcolor{blue}{\textbf{--24.6}}} & \cellcolor[HTML]{e6e3e3}{\textcolor{blue}{\textbf{--24.3}}} & \cellcolor[HTML]{e6e3e3}{\textcolor{blue}{\textbf{--22.6}}} & \cellcolor[HTML]{e6e3e3}{--14.6} & \cellcolor[HTML]{e6e3e3}{\textcolor{blue}{\textbf{--10.8}}} & \cellcolor[HTML]{e6e3e3}{\textcolor{blue}{\textbf{--3.8}}} & \cellcolor[HTML]{e6e3e3}{\textcolor{blue}{\textbf{--0.2}}} & \cellcolor[HTML]{e6e3e3}{\textcolor{blue}{\textbf{--4.9}}} & \cellcolor[HTML]{e6e3e3}{\textbf{--15.7}} & \cellcolor[HTML]{e6e3e3}{\textcolor{blue}{\textbf{--9.3}}} & \cellcolor[HTML]{e6e3e3}{\textcolor{blue}{\textbf{--11.4}}} & \cellcolor[HTML]{e6e3e3}{\textcolor{blue}{\textbf{--13.2}}}\\
\bottomrule
\end{tabular}
\begin{tablenotes}[para]
\item Note: The Base row shows the average RMSE of the base forecasts. Entries below this row indicate the percentage decrease (negative) or increase (positive) in the average RMSE of the reconciled forecasts compared to the base forecasts. The entries with the lowest values in each column are highlighted in blue. In each panel, the proposed methods are indicated with a gray background, and methods that outperform the benchmark method are marked in bold.
\end{tablenotes}
\end{threeparttable}}

}

\end{table}%

\begin{table}

\caption{\label{tbl-labour-info}Number of time series selected using
different proposed methods and the optimal parameter values identified
in the labour application, considering a single test set (from August
2022 to July 2023). The None row shows the original number of series in
the structure.}

\centering{

\centering\begingroup\fontsize{10}{12}\selectfont

\begin{tabular}{lrrrrrrrr}
\toprule
\multicolumn{1}{c}{} & \multicolumn{5}{c}{Number of time series retained} & \multicolumn{3}{c}{Optimal parameters} \\
\cmidrule(l{3pt}r{3pt}){2-6} \cmidrule(l{3pt}r{3pt}){7-9}
  & Top & Duration & STT & Duration x STT & Total & $\lambda$ & $\lambda_0$ & $\lambda_2$\\
\midrule
None & 1 & 6 & 8 & 48 & 63 & - & - & -\\
OLS-subset & 0 & 5 & 1 & 48 & 54 & - & 4.16 & 1.00\\
WLSs-subset & 0 & 5 & 1 & 46 & 52 & - & 0.38 & 0.10\\
WLSv-subset & 1 & 5 & 7 & 48 & 61 & - & 0.51 & 1.00\\
MinTs-subset & 0 & 1 & 1 & 47 & 49 & - & 0.03 & 0.01\\
Elasso & 1 & 5 & 2 & 3 & 11 & 213.59 & - & -\\
\bottomrule
\end{tabular}
\endgroup{}

}

\end{table}%

\begin{table}

\caption{\label{tbl-tourism-rmse}Out-of-sample forecast results on a
single test set (from January 2017 to December 2017) for Australian
domestic tourism data.}

\centering{

\centering
\resizebox{\linewidth}{!}{
\fontsize{11}{13}\selectfont
\begin{threeparttable}
\begin{tabular}{lrrrrrrrrrrrrrrrrrrrr}
\toprule
\multicolumn{1}{c}{} & \multicolumn{4}{c}{Top} & \multicolumn{4}{c}{State} & \multicolumn{4}{c}{Zone} & \multicolumn{4}{c}{Region} & \multicolumn{4}{c}{Average} \\
\cmidrule(l{3pt}r{3pt}){2-5} \cmidrule(l{3pt}r{3pt}){6-9} \cmidrule(l{3pt}r{3pt}){10-13} \cmidrule(l{3pt}r{3pt}){14-17} \cmidrule(l{3pt}r{3pt}){18-21}
Method & h=1 & 1--4 & 1--8 & 1--12 & h=1 & 1--4 & 1--8 & 1--12 & h=1 & 1--4 & 1--8 & 1--12 & h=1 & 1--4 & 1--8 & 1--12 & h=1 & 1--4 & 1--8 & 1--12\\
\midrule
Base & 1158.2 & 716.6 & 1279.5 & 1907.6 & 452.7 & 323.3 & 349.9 & 424.8 & 165.5 & 163.6 & 160.7 & 179.7 & 100.8 & 89.4 & 88.2 & 94.1 & 148.3 & 127.9 & 133.1 & 152.1\\
BU & 89.1 & 132.8 & 53.4 & 42.0 & --4.6 & 10.3 & 17.0 & 19.7 & 1.1 & --2.4 & 0.4 & 1.0 & 0.0 & 0.0 & 0.0 & 0.0 & 5.7 & 7.6 & 7.6 & 8.5\\
\midrule
OLS & --4.7 & --0.4 & 0.5 & 1.4 & --3.0 & --3.9 & --1.6 & --1.5 & --2.1 & --4.2 & --5.6 & --7.5 & 1.0 & --0.4 & --1.9 & --3.2 & --1.0 & --2.1 & --2.7 & --3.6\\
\cellcolor[HTML]{e6e3e3}{OLS-subset} & \cellcolor[HTML]{e6e3e3}{--4.7} & \cellcolor[HTML]{e6e3e3}{8.0} & \cellcolor[HTML]{e6e3e3}{\textbf{  --1.4}} & \cellcolor[HTML]{e6e3e3}{\textbf{ --14.1}} & \cellcolor[HTML]{e6e3e3}{--3.0} & \cellcolor[HTML]{e6e3e3}{5.5} & \cellcolor[HTML]{e6e3e3}{0.3} & \cellcolor[HTML]{e6e3e3}{\textbf{ --7.9}} & \cellcolor[HTML]{e6e3e3}{--2.1} & \cellcolor[HTML]{e6e3e3}{--1.5} & \cellcolor[HTML]{e6e3e3}{--3.7} & \cellcolor[HTML]{e6e3e3}{\textbf{ --8.7}} & \cellcolor[HTML]{e6e3e3}{1.0} & \cellcolor[HTML]{e6e3e3}{1.7} & \cellcolor[HTML]{e6e3e3}{--0.1} & \cellcolor[HTML]{e6e3e3}{--2.3} & \cellcolor[HTML]{e6e3e3}{--1.0} & \cellcolor[HTML]{e6e3e3}{1.7} & \cellcolor[HTML]{e6e3e3}{--1.2} & \cellcolor[HTML]{e6e3e3}{\textbf{ --6.5}}\\
\cellcolor[HTML]{e6e3e3}{OLS-intuitive} & \cellcolor[HTML]{e6e3e3}{--4.7} & \cellcolor[HTML]{e6e3e3}{--0.4} & \cellcolor[HTML]{e6e3e3}{0.5} & \cellcolor[HTML]{e6e3e3}{1.4} & \cellcolor[HTML]{e6e3e3}{--3.0} & \cellcolor[HTML]{e6e3e3}{--3.9} & \cellcolor[HTML]{e6e3e3}{--1.6} & \cellcolor[HTML]{e6e3e3}{--1.5} & \cellcolor[HTML]{e6e3e3}{--2.1} & \cellcolor[HTML]{e6e3e3}{--4.2} & \cellcolor[HTML]{e6e3e3}{--5.6} & \cellcolor[HTML]{e6e3e3}{--7.5} & \cellcolor[HTML]{e6e3e3}{1.0} & \cellcolor[HTML]{e6e3e3}{--0.4} & \cellcolor[HTML]{e6e3e3}{--1.9} & \cellcolor[HTML]{e6e3e3}{--3.2} & \cellcolor[HTML]{e6e3e3}{--1.0} & \cellcolor[HTML]{e6e3e3}{--2.1} & \cellcolor[HTML]{e6e3e3}{--2.7} & \cellcolor[HTML]{e6e3e3}{--3.6}\\
\cellcolor[HTML]{e6e3e3}{OLS-lasso} & \cellcolor[HTML]{e6e3e3}{--4.7} & \cellcolor[HTML]{e6e3e3}{--0.4} & \cellcolor[HTML]{e6e3e3}{0.5} & \cellcolor[HTML]{e6e3e3}{1.4} & \cellcolor[HTML]{e6e3e3}{--3.0} & \cellcolor[HTML]{e6e3e3}{--3.9} & \cellcolor[HTML]{e6e3e3}{--1.6} & \cellcolor[HTML]{e6e3e3}{--1.5} & \cellcolor[HTML]{e6e3e3}{--2.1} & \cellcolor[HTML]{e6e3e3}{--4.2} & \cellcolor[HTML]{e6e3e3}{--5.6} & \cellcolor[HTML]{e6e3e3}{--7.5} & \cellcolor[HTML]{e6e3e3}{1.0} & \cellcolor[HTML]{e6e3e3}{--0.4} & \cellcolor[HTML]{e6e3e3}{--1.9} & \cellcolor[HTML]{e6e3e3}{--3.2} & \cellcolor[HTML]{e6e3e3}{--1.0} & \cellcolor[HTML]{e6e3e3}{--2.1} & \cellcolor[HTML]{e6e3e3}{--2.7} & \cellcolor[HTML]{e6e3e3}{--3.6}\\
\midrule
WLSs & 25.1 & 55.2 & 20.8 & 19.1 & --15.8 & --5.0 & 3.5 & 6.2 & --5.9 & --5.4 & --4.7 & --5.0 & --0.2 & --0.8 & --1.6 & --2.2 & --3.0 & --0.1 & 0.3 & 0.9\\
\cellcolor[HTML]{e6e3e3}{WLSs-subset} & \cellcolor[HTML]{e6e3e3}{25.1} & \cellcolor[HTML]{e6e3e3}{\textbf{ 18.7}} & \cellcolor[HTML]{e6e3e3}{\textbf{   0.8}} & \cellcolor[HTML]{e6e3e3}{\textbf{  --7.8}} & \cellcolor[HTML]{e6e3e3}{--15.8} & \cellcolor[HTML]{e6e3e3}{--2.7} & \cellcolor[HTML]{e6e3e3}{\textbf{ --2.1}} & \cellcolor[HTML]{e6e3e3}{\textbf{ --6.2}} & \cellcolor[HTML]{e6e3e3}{--5.9} & \cellcolor[HTML]{e6e3e3}{--4.1} & \cellcolor[HTML]{e6e3e3}{\textbf{ --4.8}} & \cellcolor[HTML]{e6e3e3}{\textbf{ --8.5}} & \cellcolor[HTML]{e6e3e3}{--0.2} & \cellcolor[HTML]{e6e3e3}{0.3} & \cellcolor[HTML]{e6e3e3}{--1.0} & \cellcolor[HTML]{e6e3e3}{\textbf{--2.5}} & \cellcolor[HTML]{e6e3e3}{--3.0} & \cellcolor[HTML]{e6e3e3}{\textbf{ --0.6}} & \cellcolor[HTML]{e6e3e3}{\textbf{ --2.1}} & \cellcolor[HTML]{e6e3e3}{\textbf{ --5.5}}\\
\cellcolor[HTML]{e6e3e3}{WLSs-intuitive} & \cellcolor[HTML]{e6e3e3}{25.1} & \cellcolor[HTML]{e6e3e3}{55.2} & \cellcolor[HTML]{e6e3e3}{20.8} & \cellcolor[HTML]{e6e3e3}{19.1} & \cellcolor[HTML]{e6e3e3}{--15.8} & \cellcolor[HTML]{e6e3e3}{--5.0} & \cellcolor[HTML]{e6e3e3}{3.5} & \cellcolor[HTML]{e6e3e3}{6.2} & \cellcolor[HTML]{e6e3e3}{--5.9} & \cellcolor[HTML]{e6e3e3}{--5.4} & \cellcolor[HTML]{e6e3e3}{--4.7} & \cellcolor[HTML]{e6e3e3}{--5.0} & \cellcolor[HTML]{e6e3e3}{--0.2} & \cellcolor[HTML]{e6e3e3}{--0.8} & \cellcolor[HTML]{e6e3e3}{--1.6} & \cellcolor[HTML]{e6e3e3}{--2.2} & \cellcolor[HTML]{e6e3e3}{--3.0} & \cellcolor[HTML]{e6e3e3}{--0.1} & \cellcolor[HTML]{e6e3e3}{0.3} & \cellcolor[HTML]{e6e3e3}{0.9}\\
\cellcolor[HTML]{e6e3e3}{WLSs-lasso} & \cellcolor[HTML]{e6e3e3}{25.1} & \cellcolor[HTML]{e6e3e3}{55.2} & \cellcolor[HTML]{e6e3e3}{20.8} & \cellcolor[HTML]{e6e3e3}{19.1} & \cellcolor[HTML]{e6e3e3}{--15.8} & \cellcolor[HTML]{e6e3e3}{--5.0} & \cellcolor[HTML]{e6e3e3}{3.5} & \cellcolor[HTML]{e6e3e3}{6.2} & \cellcolor[HTML]{e6e3e3}{--5.9} & \cellcolor[HTML]{e6e3e3}{--5.4} & \cellcolor[HTML]{e6e3e3}{--4.7} & \cellcolor[HTML]{e6e3e3}{--5.0} & \cellcolor[HTML]{e6e3e3}{--0.2} & \cellcolor[HTML]{e6e3e3}{--0.8} & \cellcolor[HTML]{e6e3e3}{--1.6} & \cellcolor[HTML]{e6e3e3}{--2.2} & \cellcolor[HTML]{e6e3e3}{--3.0} & \cellcolor[HTML]{e6e3e3}{--0.1} & \cellcolor[HTML]{e6e3e3}{0.3} & \cellcolor[HTML]{e6e3e3}{0.9}\\
\midrule
WLSv & 38.2 & 76.2 & 29.6 & 25.6 & --17.4 & --3.1 & 7.0 & 9.9 & --5.0 & --4.3 & --3.1 & --3.2 & --4.2 & --1.6 & --1.8 & --2.1 & --3.9 & 1.3 & 2.0 & 2.8\\
\cellcolor[HTML]{e6e3e3}{WLSv-subset} & \cellcolor[HTML]{e6e3e3}{38.2} & \cellcolor[HTML]{e6e3e3}{\textbf{ 34.5}} & \cellcolor[HTML]{e6e3e3}{\textbf{  10.7}} & \cellcolor[HTML]{e6e3e3}{\textbf{   8.5}} & \cellcolor[HTML]{e6e3e3}{--17.4} & \cellcolor[HTML]{e6e3e3}{\textbf{ --8.8}} & \cellcolor[HTML]{e6e3e3}{\textbf{ --0.8}} & \cellcolor[HTML]{e6e3e3}{\textbf{  1.4}} & \cellcolor[HTML]{e6e3e3}{--5.0} & \cellcolor[HTML]{e6e3e3}{\textbf{ --5.5}} & \cellcolor[HTML]{e6e3e3}{\textbf{ --5.3}} & \cellcolor[HTML]{e6e3e3}{\textbf{ --6.7}} & \cellcolor[HTML]{e6e3e3}{--4.1} & \cellcolor[HTML]{e6e3e3}{\textbf{ --2.0}} & \cellcolor[HTML]{e6e3e3}{\textbf{--2.6}} & \cellcolor[HTML]{e6e3e3}{\textbf{--3.4}} & \cellcolor[HTML]{e6e3e3}{--3.9} & \cellcolor[HTML]{e6e3e3}{\textbf{ --2.3}} & \cellcolor[HTML]{e6e3e3}{\textbf{ --2.0}} & \cellcolor[HTML]{e6e3e3}{\textbf{ --2.2}}\\
\cellcolor[HTML]{e6e3e3}{WLSv-intuitive} & \cellcolor[HTML]{e6e3e3}{38.2} & \cellcolor[HTML]{e6e3e3}{76.2} & \cellcolor[HTML]{e6e3e3}{29.6} & \cellcolor[HTML]{e6e3e3}{25.6} & \cellcolor[HTML]{e6e3e3}{--17.4} & \cellcolor[HTML]{e6e3e3}{--3.1} & \cellcolor[HTML]{e6e3e3}{7.0} & \cellcolor[HTML]{e6e3e3}{9.9} & \cellcolor[HTML]{e6e3e3}{--5.0} & \cellcolor[HTML]{e6e3e3}{--4.3} & \cellcolor[HTML]{e6e3e3}{--3.1} & \cellcolor[HTML]{e6e3e3}{--3.2} & \cellcolor[HTML]{e6e3e3}{--4.2} & \cellcolor[HTML]{e6e3e3}{--1.6} & \cellcolor[HTML]{e6e3e3}{--1.8} & \cellcolor[HTML]{e6e3e3}{--2.1} & \cellcolor[HTML]{e6e3e3}{--3.9} & \cellcolor[HTML]{e6e3e3}{1.3} & \cellcolor[HTML]{e6e3e3}{2.0} & \cellcolor[HTML]{e6e3e3}{2.8}\\
\cellcolor[HTML]{e6e3e3}{WLSv-lasso} & \cellcolor[HTML]{e6e3e3}{38.2} & \cellcolor[HTML]{e6e3e3}{76.2} & \cellcolor[HTML]{e6e3e3}{29.6} & \cellcolor[HTML]{e6e3e3}{25.6} & \cellcolor[HTML]{e6e3e3}{--17.4} & \cellcolor[HTML]{e6e3e3}{--3.1} & \cellcolor[HTML]{e6e3e3}{7.0} & \cellcolor[HTML]{e6e3e3}{9.9} & \cellcolor[HTML]{e6e3e3}{--5.0} & \cellcolor[HTML]{e6e3e3}{--4.3} & \cellcolor[HTML]{e6e3e3}{--3.1} & \cellcolor[HTML]{e6e3e3}{--3.2} & \cellcolor[HTML]{e6e3e3}{--4.2} & \cellcolor[HTML]{e6e3e3}{--1.6} & \cellcolor[HTML]{e6e3e3}{--1.8} & \cellcolor[HTML]{e6e3e3}{--2.1} & \cellcolor[HTML]{e6e3e3}{--3.9} & \cellcolor[HTML]{e6e3e3}{1.3} & \cellcolor[HTML]{e6e3e3}{2.0} & \cellcolor[HTML]{e6e3e3}{2.8}\\
\midrule
MinTs & 20.6 & 53.6 & 21.6 & 19.0 & \textcolor{blue}{\textbf{--22.2}} & --7.2 & 3.5 & 6.3 & \textcolor{blue}{\textbf{--12.1}} & --6.6 & --5.1 & --5.3 & \textcolor{blue}{\textbf{ --5.3}} & --2.6 & --2.8 & --3.1 & --8.6 & --1.8 & --0.3 & 0.4\\
\cellcolor[HTML]{e6e3e3}{MinTs-subset} & \cellcolor[HTML]{e6e3e3}{20.6} & \cellcolor[HTML]{e6e3e3}{\textbf{ 20.0}} & \cellcolor[HTML]{e6e3e3}{\textbf{   6.4}} & \cellcolor[HTML]{e6e3e3}{\textbf{   5.6}} & \cellcolor[HTML]{e6e3e3}{\textcolor{blue}{\textbf{--22.2}}} & \cellcolor[HTML]{e6e3e3}{\textcolor{blue}{\textbf{--11.3}}} & \cellcolor[HTML]{e6e3e3}{\textbf{ --2.5}} & \cellcolor[HTML]{e6e3e3}{\textbf{ --0.1}} & \cellcolor[HTML]{e6e3e3}{\textcolor{blue}{\textbf{--12.1}}} & \cellcolor[HTML]{e6e3e3}{\textbf{ --7.5}} & \cellcolor[HTML]{e6e3e3}{\textbf{ --6.4}} & \cellcolor[HTML]{e6e3e3}{\textbf{ --7.8}} & \cellcolor[HTML]{e6e3e3}{\textcolor{blue}{\textbf{ --5.3}}} & \cellcolor[HTML]{e6e3e3}{\textcolor{blue}{\textbf{ --2.9}}} & \cellcolor[HTML]{e6e3e3}{\textcolor{blue}{\textbf{--3.2}}} & \cellcolor[HTML]{e6e3e3}{\textcolor{blue}{\textbf{--3.9}}} & \cellcolor[HTML]{e6e3e3}{--8.6} & \cellcolor[HTML]{e6e3e3}{\textcolor{blue}{\textbf{ --4.5}}} & \cellcolor[HTML]{e6e3e3}{\textcolor{blue}{\textbf{ --3.2}}} & \cellcolor[HTML]{e6e3e3}{\textbf{ --3.3}}\\
\cellcolor[HTML]{e6e3e3}{MinTs-intuitive} & \cellcolor[HTML]{e6e3e3}{20.6} & \cellcolor[HTML]{e6e3e3}{53.6} & \cellcolor[HTML]{e6e3e3}{21.6} & \cellcolor[HTML]{e6e3e3}{19.0} & \cellcolor[HTML]{e6e3e3}{\textcolor{blue}{\textbf{--22.2}}} & \cellcolor[HTML]{e6e3e3}{--7.2} & \cellcolor[HTML]{e6e3e3}{3.5} & \cellcolor[HTML]{e6e3e3}{6.3} & \cellcolor[HTML]{e6e3e3}{\textcolor{blue}{\textbf{--12.1}}} & \cellcolor[HTML]{e6e3e3}{--6.6} & \cellcolor[HTML]{e6e3e3}{--5.1} & \cellcolor[HTML]{e6e3e3}{--5.3} & \cellcolor[HTML]{e6e3e3}{\textcolor{blue}{\textbf{ --5.3}}} & \cellcolor[HTML]{e6e3e3}{--2.6} & \cellcolor[HTML]{e6e3e3}{--2.8} & \cellcolor[HTML]{e6e3e3}{--3.1} & \cellcolor[HTML]{e6e3e3}{--8.6} & \cellcolor[HTML]{e6e3e3}{--1.8} & \cellcolor[HTML]{e6e3e3}{--0.3} & \cellcolor[HTML]{e6e3e3}{0.4}\\
\cellcolor[HTML]{e6e3e3}{MinTs-lasso} & \cellcolor[HTML]{e6e3e3}{20.6} & \cellcolor[HTML]{e6e3e3}{53.6} & \cellcolor[HTML]{e6e3e3}{21.6} & \cellcolor[HTML]{e6e3e3}{19.0} & \cellcolor[HTML]{e6e3e3}{\textcolor{blue}{\textbf{--22.2}}} & \cellcolor[HTML]{e6e3e3}{--7.2} & \cellcolor[HTML]{e6e3e3}{3.5} & \cellcolor[HTML]{e6e3e3}{6.3} & \cellcolor[HTML]{e6e3e3}{\textcolor{blue}{\textbf{--12.1}}} & \cellcolor[HTML]{e6e3e3}{--6.6} & \cellcolor[HTML]{e6e3e3}{--5.1} & \cellcolor[HTML]{e6e3e3}{--5.3} & \cellcolor[HTML]{e6e3e3}{\textcolor{blue}{\textbf{ --5.3}}} & \cellcolor[HTML]{e6e3e3}{--2.6} & \cellcolor[HTML]{e6e3e3}{--2.8} & \cellcolor[HTML]{e6e3e3}{--3.1} & \cellcolor[HTML]{e6e3e3}{--8.6} & \cellcolor[HTML]{e6e3e3}{--1.8} & \cellcolor[HTML]{e6e3e3}{--0.3} & \cellcolor[HTML]{e6e3e3}{0.4}\\
\midrule
EMinT & 116.5 & 97.8 & --15.8 & --13.7 & 149.4 & 114.5 & 63.5 & 47.5 & 108.4 & 68.4 & 60.6 & 54.2 & 122.1 & 103.1 & 90.2 & 78.2 & 123.2 & 93.9 & 67.9 & 55.5\\
\cellcolor[HTML]{e6e3e3}{Elasso} & \cellcolor[HTML]{e6e3e3}{\textcolor{blue}{\textbf{ --84.5}}} & \cellcolor[HTML]{e6e3e3}{\textcolor{blue}{\textbf{--50.4}}} & \cellcolor[HTML]{e6e3e3}{\textcolor{blue}{\textbf{ --16.3}}} & \cellcolor[HTML]{e6e3e3}{\textcolor{blue}{\textbf{ --16.4}}} & \cellcolor[HTML]{e6e3e3}{\textbf{--18.3}} & \cellcolor[HTML]{e6e3e3}{\textbf{  0.6}} & \cellcolor[HTML]{e6e3e3}{\textcolor{blue}{\textbf{ --9.0}}} & \cellcolor[HTML]{e6e3e3}{\textcolor{blue}{\textbf{--11.4}}} & \cellcolor[HTML]{e6e3e3}{\textbf{ --7.8}} & \cellcolor[HTML]{e6e3e3}{\textcolor{blue}{\textbf{ --8.8}}} & \cellcolor[HTML]{e6e3e3}{\textcolor{blue}{\textbf{ --7.5}}} & \cellcolor[HTML]{e6e3e3}{\textcolor{blue}{\textbf{--10.4}}} & \cellcolor[HTML]{e6e3e3}{\textbf{  2.9}} & \cellcolor[HTML]{e6e3e3}{\textbf{  1.6}} & \cellcolor[HTML]{e6e3e3}{\textbf{ 4.1}} & \cellcolor[HTML]{e6e3e3}{\textbf{ 0.3}} & \cellcolor[HTML]{e6e3e3}{\textcolor{blue}{\textbf{--10.2}}} & \cellcolor[HTML]{e6e3e3}{\textbf{ --4.4}} & \cellcolor[HTML]{e6e3e3}{\textcolor{blue}{\textbf{ --3.2}}} & \cellcolor[HTML]{e6e3e3}{\textcolor{blue}{\textbf{ --6.7}}}\\
\bottomrule
\end{tabular}
\begin{tablenotes}[para]
\item Note: The Base row shows the average RMSE of the base forecasts. Entries below this row indicate the percentage decrease (negative) or increase (positive) in the average RMSE of the reconciled forecasts compared to the base forecasts. The entries with the lowest values in each column are highlighted in blue. In each panel, the proposed methods are indicated with a gray background, and methods that outperform the benchmark method are marked in bold.
\end{tablenotes}
\end{threeparttable}}

}

\end{table}%

\begin{table}

\caption{\label{tbl-tourism-info}Number of time series selected using
different proposed methods and the optimal parameter values identified
in the tourism application, considering a single test set (from January
2017 to December 2017). The None row shows the original number of series
in the structure.}

\centering{

\centering\begingroup\fontsize{10}{12}\selectfont

\begin{tabular}{lrrrrrrrr}
\toprule
\multicolumn{1}{c}{} & \multicolumn{5}{c}{Number of time series retained} & \multicolumn{3}{c}{Optimal parameters} \\
\cmidrule(l{3pt}r{3pt}){2-6} \cmidrule(l{3pt}r{3pt}){7-9}
  & Top & State & Zone & Region & Total & $\lambda$ & $\lambda_0$ & $\lambda_2$\\
\midrule
None & 1 & 7 & 27 & 76 & 111 & - & - & -\\
OLS-subset & 1 & 2 & 13 & 76 & 92 & - & 27.98 & 10.00\\
WLSs-subset & 1 & 1 & 15 & 76 & 93 & - & 18.73 & 10.00\\
WLSv-subset & 1 & 7 & 27 & 76 & 111 & - & 0.03 & 0.01\\
MinTs-subset & 1 & 7 & 27 & 76 & 111 & - & 0.05 & 0.01\\
Elasso & 1 & 4 & 0 & 8 & 13 & 71759.21 & - & -\\
\bottomrule
\end{tabular}
\endgroup{}

}

\end{table}%




\end{document}
