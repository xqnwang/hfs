\documentclass[11pt,a4paper,]{article}
\usepackage{lmodern}

\usepackage{multirow}
\usepackage{amssymb,amsmath}
\usepackage{ifxetex,ifluatex}
\usepackage{fixltx2e} % provides \textsubscript
\ifnum 0\ifxetex 1\fi\ifluatex 1\fi=0 % if pdftex
  \usepackage[T1]{fontenc}
  \usepackage[utf8]{inputenc}
\else % if luatex or xelatex
  \usepackage{unicode-math}
  \defaultfontfeatures{Ligatures=TeX,Scale=MatchLowercase}
\fi
% use upquote if available, for straight quotes in verbatim environments
\IfFileExists{upquote.sty}{\usepackage{upquote}}{}
% use microtype if available
\IfFileExists{microtype.sty}{%
\usepackage[]{microtype}
\UseMicrotypeSet[protrusion]{basicmath} % disable protrusion for tt fonts
}{}
\PassOptionsToPackage{hyphens}{url} % url is loaded by hyperref
\usepackage[unicode=true]{hyperref}
\PassOptionsToPackage{usenames,dvipsnames}{color} % color is loaded by hyperref
\hypersetup{
            pdftitle={Optimal forecast reconciliation with time series selection},
            pdfkeywords={Coherent, Hierarchical time series, Grouped
time series, Linear forecast reconciliation, Optimization problem},
            colorlinks=true,
            linkcolor=blue,
            citecolor=Blue,
            urlcolor=Blue,
            breaklinks=true}
\urlstyle{same}  % don't use monospace font for urls
\usepackage{geometry}
\geometry{left=2.5cm,right=2.5cm,top=2.5cm,bottom=2.5cm}
\usepackage[style=authoryear-comp,]{biblatex}
\addbibresource{references.bib}
\usepackage{longtable,booktabs}
% Fix footnotes in tables (requires footnote package)
\IfFileExists{footnote.sty}{\usepackage{footnote}\makesavenoteenv{long table}}{}
\usepackage{graphicx,grffile}
\makeatletter
\def\maxwidth{\ifdim\Gin@nat@width>\linewidth\linewidth\else\Gin@nat@width\fi}
\def\maxheight{\ifdim\Gin@nat@height>\textheight\textheight\else\Gin@nat@height\fi}
\makeatother
% Scale images if necessary, so that they will not overflow the page
% margins by default, and it is still possible to overwrite the defaults
% using explicit options in \includegraphics[width, height, ...]{}
\setkeys{Gin}{width=\maxwidth,height=\maxheight,keepaspectratio}
\IfFileExists{parskip.sty}{%
\usepackage{parskip}
}{% else
\setlength{\parindent}{0pt}
\setlength{\parskip}{6pt plus 2pt minus 1pt}
}
\setlength{\emergencystretch}{3em}  % prevent overfull lines
\providecommand{\tightlist}{%
  \setlength{\itemsep}{0pt}\setlength{\parskip}{0pt}}
\setcounter{secnumdepth}{5}

% set default figure placement to htbp
\makeatletter
\def\fps@figure{htbp}
\makeatother


\title{Optimal forecast reconciliation with time series selection}

%% MONASH STUFF

%% CAPTIONS
\RequirePackage{caption}
\DeclareCaptionStyle{italic}[justification=centering]
 {labelfont={bf},textfont={it},labelsep=colon}
\captionsetup[figure]{style=italic,format=hang,singlelinecheck=true}
\captionsetup[table]{style=italic,format=hang,singlelinecheck=true}

%% FONT
\usepackage[tabular,lf]{sourcesanspro}
\RequirePackage{mathpazo}

%% HEADERS AND FOOTERS
\RequirePackage{fancyhdr}
\pagestyle{fancy}
\rfoot{\Large\sffamily\raisebox{-0.1cm}{\textbf{\thepage}}}
\makeatletter
\lhead{\textsf{\expandafter{\@title}}}
\makeatother
\rhead{}
\cfoot{}
\setlength{\headheight}{15pt}
\renewcommand{\headrulewidth}{0.4pt}
\renewcommand{\footrulewidth}{0.4pt}
\fancypagestyle{plain}{%
\fancyhf{} % clear all header and footer fields
\fancyfoot[C]{\sffamily\thepage} % except the center
\renewcommand{\headrulewidth}{0pt}
\renewcommand{\footrulewidth}{0pt}}

%% MATHS
\RequirePackage{bm,amsmath}
\allowdisplaybreaks

%% GRAPHICS
\RequirePackage{graphicx}
\setcounter{topnumber}{2}
\setcounter{bottomnumber}{2}
\setcounter{totalnumber}{4}
\renewcommand{\topfraction}{0.85}
\renewcommand{\bottomfraction}{0.85}
\renewcommand{\textfraction}{0.15}
\renewcommand{\floatpagefraction}{0.8}

%\RequirePackage[section]{placeins}

%% SECTION TITLES
\RequirePackage[compact,sf,bf]{titlesec}
\titleformat{\section}[block]
  {\fontsize{15}{17}\bfseries\sffamily}
  {\thesection}
  {0.4em}{}
\titleformat{\subsection}[block]
  {\fontsize{12}{14}\bfseries\sffamily}
  {\thesubsection}
  {0.4em}{}
\titlespacing{\section}{0pt}{*5}{*1}
\titlespacing{\subsection}{0pt}{*2}{*0.2}


%% TITLE PAGE
\def\Date{\number\day}
\def\Month{\ifcase\month\or
 January\or February\or March\or April\or May\or June\or
 July\or August\or September\or October\or November\or December\fi}
\def\Year{\number\year}

\makeatletter
\def\wp#1{\gdef\@wp{#1}}\def\@wp{??/??}
\def\jel#1{\gdef\@jel{#1}}\def\@jel{??}
\def\showjel{{\large\textsf{\textbf{JEL classification:}}~\@jel}}
\def\nojel{\def\showjel{}}
\def\addresses#1{\gdef\@addresses{#1}}\def\@addresses{??}
\def\cover{{\sffamily\setcounter{page}{0}
        \thispagestyle{empty}
        \placefig{2}{1.5}{width=5cm}{_extensions/wp/monash2}
        \placefig{16.9}{1.5}{width=2.1cm}{_extensions/wp/MBSportrait}
        \begin{textblock}{4}(16.9,4)ISSN 1440-771X\end{textblock}
        \begin{textblock}{7}(12.7,27.9)\hfill
        \includegraphics[height=0.7cm]{_extensions/wp/AACSB}~~~
        \includegraphics[height=0.7cm]{_extensions/wp/EQUIS}~~~
        \includegraphics[height=0.7cm]{_extensions/wp/AMBA}
        \end{textblock}
        \vspace*{2cm}
        \begin{center}\Large
        Department of Econometrics and Business Statistics\\[.5cm]
        \footnotesize http://monash.edu/business/ebs/research/publications
        \end{center}\vspace{2cm}
        \begin{center}
        \fbox{\parbox{14cm}{\begin{onehalfspace}\centering\Huge\vspace*{0.3cm}
                \textsf{\textbf{\expandafter{\@title}}}\vspace{1cm}\par
                \LARGE\@author\end{onehalfspace}
        }}
        \end{center}
        \vfill
                \begin{center}\Large
                \Month~\Year\\[1cm]
                Working Paper \@wp
        \end{center}\vspace*{2cm}}}
\def\pageone{{\sffamily\setstretch{1}%
        \thispagestyle{empty}%
        \vbox to \textheight{%
        \raggedright\baselineskip=1.2cm
     {\fontsize{24.88}{30}\sffamily\textbf{\expandafter{\@title}}}
        \vspace{2cm}\par
        \hspace{1cm}\parbox{14cm}{\sffamily\large\@addresses}\vspace{1cm}\vfill
        \hspace{1cm}{\large\Date~\Month~\Year}\\[1cm]
        \hspace{1cm}\showjel\vss}}}
\def\blindtitle{{\sffamily
     \thispagestyle{plain}\raggedright\baselineskip=1.2cm
     {\fontsize{24.88}{30}\sffamily\textbf{\expandafter{\@title}}}\vspace{1cm}\par
        }}
\def\titlepage{{\cover\newpage\pageone\newpage\blindtitle}}

\def\blind{\def\titlepage{{\blindtitle}}\let\maketitle\blindtitle}
\def\titlepageonly{\def\titlepage{{\pageone\end{document}}}}
\def\nocover{\def\titlepage{{\pageone\newpage\blindtitle}}\let\maketitle\titlepage}
\let\maketitle\titlepage
\makeatother

%% SPACING
\RequirePackage{setspace}
\spacing{1.5}

%% LINE AND PAGE BREAKING
\sloppy
\clubpenalty = 10000
\widowpenalty = 10000
\brokenpenalty = 10000
\RequirePackage{microtype}

%% PARAGRAPH BREAKS
\setlength{\parskip}{1.4ex}
\setlength{\parindent}{0em}

%% HYPERLINKS
\RequirePackage{xcolor} % Needed for links
\definecolor{darkblue}{rgb}{0,0,.6}
\RequirePackage{url}

\makeatletter
\@ifpackageloaded{hyperref}{}{\RequirePackage{hyperref}}
\makeatother
\hypersetup{
     citecolor=0 0 0,
     breaklinks=true,
     bookmarksopen=true,
     bookmarksnumbered=true,
     linkcolor=darkblue,
     urlcolor=blue,
     citecolor=darkblue,
     colorlinks=true}

%% KEYWORDS
\newenvironment{keywords}{\par\vspace{0.5cm}\noindent{\sffamily\textbf{Keywords:}}}{\vspace{0.25cm}\par\hrule\vspace{0.5cm}\par}

%% ABSTRACT
\renewenvironment{abstract}{\begin{minipage}{\textwidth}\parskip=1.4ex\noindent
\hrule\vspace{0.1cm}\par{\sffamily\textbf{\abstractname}}\newline}
  {\end{minipage}}


\usepackage[T1]{fontenc}
\usepackage[utf8]{inputenc}

\usepackage[showonlyrefs]{mathtools}
\usepackage[no-weekday]{eukdate}

%% BIBLIOGRAPHY

\makeatletter
\@ifpackageloaded{biblatex}{}{\usepackage[style=authoryear-comp, backend=biber, natbib=true]{biblatex}}
\makeatother
\ExecuteBibliographyOptions{bibencoding=utf8,minnames=1,maxnames=3, maxbibnames=99,dashed=false,terseinits=true,giveninits=true,uniquename=false,uniquelist=false,doi=false, isbn=false,url=true,sortcites=false, date=year}

\DeclareFieldFormat{url}{\texttt{\url{#1}}}
\DeclareFieldFormat[article]{pages}{#1}
\DeclareFieldFormat[inproceedings]{pages}{\lowercase{pp.}#1}
\DeclareFieldFormat[incollection]{pages}{\lowercase{pp.}#1}
\DeclareFieldFormat[article]{volume}{\mkbibbold{#1}}
\DeclareFieldFormat[article]{number}{\mkbibparens{#1}}
\DeclareFieldFormat[article]{title}{\MakeCapital{#1}}
\DeclareFieldFormat[inproceedings]{title}{#1}
\DeclareFieldFormat{shorthandwidth}{#1}
% No dot before number of articles
\usepackage{xpatch}
\xpatchbibmacro{volume+number+eid}{\setunit*{\adddot}}{}{}{}
% Remove In: for an article.
\renewbibmacro{in:}{%
  \ifentrytype{article}{}{%
  \printtext{\bibstring{in}\intitlepunct}}}

\makeatletter
\DeclareDelimFormat[cbx@textcite]{nameyeardelim}{\addspace}
\makeatother
\renewcommand*{\finalnamedelim}{%
  %\ifnumgreater{\value{liststop}}{2}{\finalandcomma}{}% there really should be no funny Oxford comma business here
  \addspace\&\space}

\wp{no/yr}
\nojel

\RequirePackage[absolute,overlay]{textpos}
\setlength{\TPHorizModule}{1cm}
\setlength{\TPVertModule}{1cm}
\def\placefig#1#2#3#4{\begin{textblock}{.1}(#1,#2)\rlap{\includegraphics[#3]{#4}}\end{textblock}}




\author{Xiaoqian~Wang, Rob J~Hyndman, Shanika L~Wickramasuriya}
\addresses{\textbf{Xiaoqian Wang}\newline
Monash University, VIC 3800, Australia\newline
{Email: xiaoqian.wang@monash.edu}\newline Corresponding author\newline\\[0.5cm]
\textbf{Rob J Hyndman}\newline
Monash University, VIC 3800, Australia\newline
{Email: rob.hyndman@monash.edu}\\[0.5cm]
\textbf{Shanika L Wickramasuriya}\newline
Monash University, VIC 3145, Australia\newline
{Email: shanika.wickramasuriya@monash.edu}\\[0.5cm]
}

\date{\sf\Date~\Month~\Year}
\makeatletter
 \lfoot{\sf Wang, Hyndman, Wickramasuriya: \@date}
\makeatother

\usepackage{booktabs}
\usepackage{longtable}
\usepackage{array}
\usepackage{multirow}
\usepackage{wrapfig}
\usepackage{float}
\usepackage{colortbl}
\usepackage{pdflscape}
\usepackage{tabu}
\usepackage{threeparttable}
\usepackage{threeparttablex}
\usepackage[normalem]{ulem}
\usepackage{makecell}
\usepackage{xcolor}
\makeatletter
\makeatother
\makeatletter
\makeatother
\makeatletter
\@ifpackageloaded{caption}{}{\usepackage{caption}}
\AtBeginDocument{%
\ifdefined\contentsname
  \renewcommand*\contentsname{Table of contents}
\else
  \newcommand\contentsname{Table of contents}
\fi
\ifdefined\listfigurename
  \renewcommand*\listfigurename{List of Figures}
\else
  \newcommand\listfigurename{List of Figures}
\fi
\ifdefined\listtablename
  \renewcommand*\listtablename{List of Tables}
\else
  \newcommand\listtablename{List of Tables}
\fi
\ifdefined\figurename
  \renewcommand*\figurename{Figure}
\else
  \newcommand\figurename{Figure}
\fi
\ifdefined\tablename
  \renewcommand*\tablename{Table}
\else
  \newcommand\tablename{Table}
\fi
}
\@ifpackageloaded{float}{}{\usepackage{float}}
\floatstyle{ruled}
\@ifundefined{c@chapter}{\newfloat{codelisting}{h}{lop}}{\newfloat{codelisting}{h}{lop}[chapter]}
\floatname{codelisting}{Listing}
\newcommand*\listoflistings{\listof{codelisting}{List of Listings}}
\makeatother
\makeatletter
\@ifpackageloaded{caption}{}{\usepackage{caption}}
\@ifpackageloaded{subcaption}{}{\usepackage{subcaption}}
\makeatother
\makeatletter
\@ifpackageloaded{tcolorbox}{}{\usepackage[skins,breakable]{tcolorbox}}
\makeatother
\makeatletter
\@ifundefined{shadecolor}{\definecolor{shadecolor}{rgb}{.97, .97, .97}}
\makeatother
\makeatletter
\makeatother
\makeatletter
\makeatother

% Adjust headwidth in case user has changed geometry in header-includes
\renewcommand{\headwidth}{\textwidth}

\begin{document}
\maketitle
\begin{abstract}
Forecast reconciliation ensures forecasts of time series in a hierarchy
adhere to aggregation constraints, enabling aligned decision making.
While forecast reconciliation can improve overall forecast accuracy in
hierarchical or grouped structures, the most substantial improvements
occur in series with initially poor base forecasts, and some series may
still experience a deterioration in reconciled forecasts. However, in
practice, some series in a structure often have poor base forecasts due
to model misspecification or low forecastability. To address this, we
propose two categories of forecast reconciliation methods that
incorporate time series selection based on out-of-sample and in-sample
information, respectively. Our methods keep ``poor'' base forecasts of
some series unused in forming reconciled forecasts, preventing their
negative influence on the reconciled forecasts. This process adjusts the
weights allocated to the remaining series accordingly when generating
bottom-level reconciled forecasts. Additionally, our methods can reduce
disparities arising from using different estimates of the base forecast
error covariance matrix, thus alleviating the challenge of estimator
selection. We evaluate the proposed methods through two simulation
studies and two empirical applications using Australian labour force
data and domestic tourism data, showing improved accuracy compared to
alternative methods, especially for aggregation levels, longer forecast
horizons, and under model misspecification.
\end{abstract}
\begin{keywords}
Coherent, Hierarchical time series, Grouped time series, Linear forecast
reconciliation, Optimization problem
\end{keywords}

\ifdefined\Shaded\renewenvironment{Shaded}{\begin{tcolorbox}[interior hidden, borderline west={3pt}{0pt}{shadecolor}, breakable, frame hidden, boxrule=0pt, sharp corners, enhanced]}{\end{tcolorbox}}\fi

\hypertarget{sec-introduction}{%
\section{Introduction}\label{sec-introduction}}

Hierarchical time series are characterized as a set of time series
organized in a unique hierarchical aggregation structure, while grouped
time series arise when attributes of interest are crossed rather than
nested in the aggregation structure \autocite{Hyndman2016-cz}. For
example, unemployment data is a crucial social and economic indicator
and is a natural example of a grouped time series. The analysis of the
number of unemployment persons in a country is crucial for informed
policymaking and economic research. It is also valuable to examine
attributes such as labour market region and the length of time that
unemployed people have been looking for work. Such a disaggregation
allows us to identify regional disparities, and comprehend the
structural nuances underlying unemployment. Forecasts of time series in
such a hierarchical or grouped structure should adhere to some known
aggregation constraints to maintain coherence, which is a vital aspect
for aligned decision making.

Earlier studies perform forecast reconciliation by focusing only on a
single level of the structure, subsequently aggregating or
disaggregating their forecasts to produce coherent forecasts for other
levels of the structure. These single-level methods typically fall into
three categories, namely bottom-up \autocite{Dunn1976-op}, top-down
\autocite{Gross1990-lg}, and middle-out
\autocite{Athanasopoulos2009-ps}. However, these methods only use
information from a single level while overlooking valuable insights
available at other levels and the intricate relationships in the
structure.

To overcome these limitations, \textcite{Hyndman2011-sd} introduced a
reconciliation approach using a linear regression model based on
forecasts from all series within a given structure, resulting in a
generalized least squares (GLS) solution. This method initially
generates independent base forecasts for all series in the structure and
subsequently adjusts these forecasts to make them coherent. Following
this work, further research has led to the modifications of the least
squares-based reconciliation method in various frameworks, including
cross-sectional data
\autocite{Hyndman2016-cz,Wickramasuriya2019-fc,Panagiotelis2021-mf},
temporal data \autocite{Athanasopoulos2017-jj,Nystrup2020-di}, and
cross-temporal data \autocite{Di_Fonzo2023-vo}. In particular, assuming
unbiased base forecasts, \textcite{Wickramasuriya2019-fc} proposed the
minimum trace method, which formulates the reconciliation problem as
minimizing the trace of the covariance matrix of reconciled forecast
errors. These reconciliation approaches have been demonstrated to
produce coherent and potentially more accurate forecasts compared to
traditional single-level methods in various empirical applications
\autocites[see, for
example,][]{Taieb2021-tc,Panagiotelis2021-mf,Wickramasuriya2023-hn}.
Additionally, \textcite{Van_Erven2015-ir},
\textcite{Wickramasuriya2019-fc}, \textcite{Panagiotelis2021-mf}, and
\textcite{Wickramasuriya2021-am} provided theoretical insights into the
performance of forecast reconciliation methods. Please refer to
\textcite{Athanasopoulos2023-sm} for a comprehensive introduction of
various forecast reconciliation methods.

Reconciliation is well recognized for its ability to improve overall
forecast accuracy in structures with aggregation constraints.
Theoretically, the mean squared reconciled forecast error from the
minimum trace (MinT) reconciliation method for each series in the
structure is lower than that of OLS and base forecasts
\autocite{Wickramasuriya2021-am}. However, it is important to
acknowledge that reconciled forecasts for some series in the structure
may experience deterioration in its forecasting performance. As
demonstrated by \textcite{Athanasopoulos2017-jj}, most of the
improvements attributed to reconciliation are observed in series with
initially poor-performing base forecasts. Thus, they suggested that the
ideal solution would be to combine the most accurate aspects of base
forecasts from each level, aiming to avoid a myopic view from a single
level. In practice, it is not uncommon for some series in a structure to
exhibit poor performance in their base forecasts due to inherent
challenges that are difficult to completely mitigate in real-world
situations. These challenges may include model misspecification or low
forecastability resulting from the absence of discernible patterns (low
signal-to-noise ratio). In such cases, it becomes crucial to exclude
``poor'' base forecasts of some series in a hierarchy when performing
reconciliation, thereby preventing their negative influence on the
reconciled forecasts. This forms the primary objective of this paper.

This paper addresses a few gaps in the field of forecast reconciliation.
First, we propose forecast reconciliation methods that incorporate time
series selection based on out-of-sample information, assuming unbiased
base forecasts. We formulate this as an optimization problem, using
different penalty functions designed to control the number of nonzero
column entries in the weighting matrix for linear forecast
reconciliation. We theoretically show that the number of selected time
series is at least equal to the number of series at the bottom level,
and we can reconstruct the full hierarchical structure by
aggregating/disaggregating the selected series. Second, we relax the
unbiasedness assumption and introduce another reconciliation method with
selection, utilizing in-sample observations and their fitted values.
This allows us to use the in-sample reconciliation performance for
selection purposes. In this case, it may happen that less than the
number of series at the bottom level are used for reconciliation. We
carry out simulation experiments and two empirical applications, showing
that our proposed methods guarantee coherent forecasts that outperform
or, at the very least, match their respective benchmark methods. The
improvements are particularly pronounced when focusing on aggregation
levels, longer forecast horizons, and model misspecification. A
remarkable feature of the proposed methods is their ability to reduce
the disparities arising from using different estimates of the base
forecast error covariance matrix, thereby mitigating the challenges
associated with estimator selection, which is a prominent issue in the
field of forecast reconciliation research.

The remainder of the paper is structured as follows.
Section~\ref{sec-preliminaries} presents the notations and a review of
linear forecast reconciliation methods. Section~\ref{sec-methodology}
introduces our proposed methods to achieve time series selection in
reconciliation, and provides some theoretical insights.
Section~\ref{sec-simulations} and Section~\ref{sec-applications} show
the results from simulations and two real-world datasets, respectively,
followed by concluding remarks in Section~\ref{sec-conclusion}. The R
code for reproducing the results is available at LINK.

\hypertarget{sec-preliminaries}{%
\section{Preliminaries}\label{sec-preliminaries}}

\hypertarget{notation}{%
\subsection{Notation}\label{notation}}

We denote the set \(\{1,\ldots,k\}\) by \([k]\) for any positive integer
\(k\). A \emph{hierarchical time series} can be considered as an
\(n\)-dimensional multivariate time series that adheres to known linear
constraints. Let \(\boldsymbol{y}_t \in \mathbb{R}^n\) be a vector
comprising observations of all time series in the hierarchy at time
\(t\), and \(\boldsymbol{b}_t \in \mathbb{R}^{n_b}\) be a vector
comprising observations of all bottom-level time series at time \(t\).
The full hierarchy at time \(t\) can be written as \[
\boldsymbol{y}_t = \boldsymbol{S}\boldsymbol{b}_t,
\]

for \(t=1,2,\ldots,T\), where \(T\) is the length of the time series,
and \(\boldsymbol{S}\) is an \(n \times n_b\) \emph{summing matrix} that
shows aggregation constraints present in the structure. We can write the
summing matrix as
\(\boldsymbol{S} = \left[\begin{array}{c}\boldsymbol{A} \\ \boldsymbol{I}_{n_b}\end{array}\right]\),
where \(\boldsymbol{A}\) is an \(n_a \times n_b\) \emph{aggregation
matrix} with \(n = n_a + n_b\), and \(\boldsymbol{I}_{n_b}\) is an
\(n_b\)-dimensional identity matrix.

\begin{figure}

{\centering \includegraphics[width=0.4\textwidth,height=\textheight]{figs/hts_example.pdf}

}

\caption{\label{fig-hts}An example of a two-level hierarchical time
series.}

\end{figure}

To clarify these notations, consider the example of a simple hierarchy
in Figure~\ref{fig-hts}. For this two-level hierarchy, we have
\(n = 7\), \(n_b = 4\), \(n_a = 3\),
\(\boldsymbol{y}_t = [y_{\text{Total},t}, y_{\text{A},t}, y_{\text{B},t}, y_{\text{AA},t}, y_{\text{AB},t}, y_{\text{BA},t}, y_{\text{BB},t}]^{\prime}\),
\(\boldsymbol{b}_t = [y_{\text{AA},t}, y_{\text{AB},t}, y_{\text{BA},t}, y_{\text{BB},t}]^{\prime}\),
and \[
\boldsymbol{S} = \left[
\begin{array}{cccc}
1 & 1 & 1 & 1 \\
1 & 1 & 0 & 0 \\
0 & 0 & 1 & 1 \\
\multicolumn{4}{c}{\boldsymbol{I}_4}
\end{array}\right].
\]

When data structure does not naturally disaggregate in a unique
hierarchical manner, we can combine these hierarchical structures to
form a \emph{grouped time series}. Thus, grouped time series can also be
considered as hierarchical time series with more than one grouping
structure. Please refer to \textcite{Hyndman2021-fo} on further details.

\hypertarget{linear-forecast-reconciliation}{%
\subsection{Linear forecast
reconciliation}\label{linear-forecast-reconciliation}}

Let \(\hat{\boldsymbol{y}}_{T+h \mid T} \in \mathbb{R}^n\) be a vector
of \(h\)-step-ahead \emph{base forecasts} for all time series in the
structure, given observations up to and including time \(T\), and
stacked in the same order as \(\boldsymbol{y}_t\). We can use any method
to generate these forecasts, but in general they will not add up
especially when we forecast each series independently.

When forecasting these time series structure, we expect the forecasts to
be \emph{coherent} (i.e., aggregation constraints of the data are
satisfied). Let \(\tilde{\boldsymbol{y}}_{T+h \mid T} \in \mathbb{R}^n\)
denote a vector of \(h\)-step-ahead \emph{reconciled forecasts} which
are coherent by construction, \(\psi\) a \emph{mapping} that reconciles
base forecasts, \(\hat{\boldsymbol{y}}_{T+h \mid T}\). Then we have
\emph{forecast reconciliation}
\(\tilde{\boldsymbol{y}}_{T+h \mid T}=\psi(\hat{\boldsymbol{y}}_{T+h \mid T})\),
which is essentially a post-processing method. In this paper, we focus
on linear forecast reconciliation given by
\begin{equation}\protect\hypertarget{eq-lr}{}{
\tilde{\boldsymbol{y}}_{T+h \mid T} = \boldsymbol{S}\boldsymbol{G}_h\hat{\boldsymbol{y}}_{T+h \mid T},
}\label{eq-lr}\end{equation}

where

\begin{itemize}
\tightlist
\item
  \(\boldsymbol{G}_h\) is an \(n_b \times n\) weighting matrix that maps
  base forecasts into the bottom level. In other words, it combines all
  base forecasts to form reconciled forecasts for bottom-level series.
\item
  \(\boldsymbol{S}\) is an \(n \times n_b\) summing matrix that sums up
  bottom-level reconciled forecasts to produce coherent forecasts of all
  levels. It identifies the linear constraints in a given structure.
\end{itemize}

\hypertarget{minimum-trace-reconciliation}{%
\subsubsection{Minimum trace
reconciliation}\label{minimum-trace-reconciliation}}

Let the \(h\)-step-ahead in-sample \emph{base forecast errors} be
defined as
\(\hat{\boldsymbol{e}}_{t+h \mid t} = \boldsymbol{y}_{t+h} - \hat{\boldsymbol{y}}_{t+h \mid t}\),
and the \(h\)-step-ahead \emph{reconciled forecast errors} be defined as
\(\tilde{\boldsymbol{e}}_{t+h \mid t} = \boldsymbol{y}_{t+h} - \tilde{\boldsymbol{y}}_{t+h \mid t}\)
for \(t = 1,2,\ldots,T-h\). \textcite{Wickramasuriya2019-fc} formulated
a linear reconciliation problem as minimizing the trace (MinT) of the
\(h\)-step-ahead covariance matrix of the reconciled forecast errors,
\(\operatorname{Var}(\tilde{\boldsymbol{e}}_{t+h \mid t})\). Under the
assumption of unbiasedness of base forecasts and reconciled forecasts,
the unique solution of the minimization problem is given by
\begin{equation}\protect\hypertarget{eq-mint}{}{
\boldsymbol{G}_h=\left(\boldsymbol{S}^{\prime} \boldsymbol{W}_h^{-1} \boldsymbol{S}\right)^{-1} \boldsymbol{S}^{\prime} \boldsymbol{W}_h^{-1},
}\label{eq-mint}\end{equation} where \(\boldsymbol{W}_h\) is the
positive definite covariance matrix of the \(h\)-step-ahead base
forecast errors.

The trace minimization problem can be reformulated as a least squares
problem with linear constraints given by
\begin{equation}\protect\hypertarget{eq-mint_op}{}{
\begin{aligned}
& \min _{\tilde{\boldsymbol{y}}_{T+h \mid T}} \quad \frac{1}{2}(\hat{\boldsymbol{y}}_{T+h \mid T}-\tilde{\boldsymbol{y}}_{T+h \mid T})^{\prime} \boldsymbol{W}_{h}^{-1}(\hat{\boldsymbol{y}}_{T+h \mid T}-\tilde{\boldsymbol{y}}_{T+h \mid T}) \\
& \text { s.t. } \quad \tilde{\boldsymbol{y}}_{T+h \mid T}=\boldsymbol{S}\tilde{\boldsymbol{b}}_{T+h \mid T},
\end{aligned}
}\label{eq-mint_op}\end{equation}

where \(\tilde{\boldsymbol{b}}_{T+h \mid T} \in \mathbb{R}^{n_b}\) is
the vector comprising \(h\)-step-ahead bottom-level reconciled
forecasts, made at time \(T\). Focusing on \(\boldsymbol{W}_h\), the
intuition behind the MinT reconciliation is that \textbf{the larger the
estimated variance of the base forecast errors, the larger the range of
adjustments permitted for forecast reconciliation}.

It is challenging to estimate \(\boldsymbol{W}_h\), especially for
\(h > 1\). Assuming that \(\boldsymbol{W}_h = k_h\boldsymbol{W}_1\),
\(\forall h\), where \(k_h > 0\), the MinT solution of
\(\boldsymbol{G}\) does not change with the forecast horizon, \(h\).
Hence, we will drop the subscript \(h\) for the ease of exposition. The
most popularly used candidate estimators for \(\boldsymbol{W}\) in the
forecast reconciliation literature are listed as follows.

\begin{enumerate}
\def\labelenumi{\arabic{enumi}.}
\tightlist
\item
  \(\boldsymbol{W}_{\text{OLS}} \propto \boldsymbol{I}\), is the
  \emph{OLS estimator} proposed by \textcite{Hyndman2011-sd}, and
  assumes that the base forecast errors are uncorrelated and
  equivariant. We denote this as \textbf{OLS}.
\item
  \(\boldsymbol{W}_{\text{WLSs}} \propto \operatorname{diag}(\boldsymbol{S} \mathbf{1})\),
  is the \emph{WLS estimator applying structural scaling} proposed by
  \textcite{Athanasopoulos2017-jj}, where \(\mathbf{1}\) is a vector of
  1s of size \(n_b\), and \(\operatorname{diag}(\cdot)\) constructs a
  diagonal matrix using a given vector. This estimator depends only on
  the aggregation structure of the hierarchy. It assumes that the
  variance of each bottom-level base forecast error is equivalent and
  uncorrelated between nodes. We denote this method as \textbf{WLSs}.
\item
  \(\boldsymbol{W}_{\text{WLSv}} \propto \operatorname{Diag}(\hat{\boldsymbol{W}}_1)\),
  is the \emph{WLS estimator applying variance scaling} proposed by
  \textcite{Hyndman2016-cz}, where \(\hat{\boldsymbol{W}}_1\) denotes
  the unbiased covariance estimator based on the in-sample
  one-step-ahead base forecast errors (i.e., residuals), and
  \(\operatorname{Diag}(\cdot)\) forms a diagonal matrix using the
  diagonal elements of the input matrix. We denote this as
  \textbf{WLSv}.
\item
  \(\boldsymbol{W}_{\text{MinT}} \propto \hat{\boldsymbol{W}}_1\), is
  referred to as the \emph{MinT estimator} based on the sample
  covariance matrix proposed by \textcite{Wickramasuriya2019-fc}. We
  denote this method as \textbf{MinT} in the results that follow.
\item
  \(\boldsymbol{W}_{\text{MinTs}} \propto \lambda\operatorname{Diag}(\hat{\boldsymbol{W}}_1) + (1-\lambda)\hat{\boldsymbol{W}}_1\),
  is the \emph{MinT shrinkage estimator} suggested by
  \textcite{Wickramasuriya2019-fc}, in which off-diagonal elements of
  \(\hat{\boldsymbol{W}}_1\) are shrunk towards zero. We refer to this
  method as \textbf{MinTs}.
\end{enumerate}

In practice, it is hard to say which estimator for \(\boldsymbol{W}\)
works better. \textcite{Pritularga2021-lz} demonstrated that the
performance of forecast reconciliation is affected by two sources of
uncertainties, i.e., the base forecast uncertainty and the
reconciliation weight uncertainty. Recall that the uncertainty in the
MinT solution in Equation~\ref{eq-mint} is introduced by the uncertainty
in the weighting matrix as the summing matrix is fixed for a given
structure. This indicates that OLS and WLSs estimators for
\(\boldsymbol{W}\) may lead to less volatile reconciliation performance
compared to WLSv, MinT, and MinTs estimators.
\textcite{Panagiotelis2021-mf} provided a geometric intuition for
reconciliation and showed that, when considering the Euclidean distance
loss function, OLS reconciliation yields results that are at least as
favorable as the base forecasts, whereas MinT reconciliation performs
poorly relative to the base forecasts. However, when considering the
expected Euclidean distance of the reconciled forecast error,
\textcite{Wickramasuriya2021-am} indicated that MinT reconciliation is
better than OLS reconciliation. Therefore, which estimator for
\(\boldsymbol{W}\) to use hinges on the specific time series structure
of interest, the targeted level or series, and the selected loss
function.

\hypertarget{relaxation-of-the-unbiasedness-assumptions}{%
\subsubsection{Relaxation of the unbiasedness
assumptions}\label{relaxation-of-the-unbiasedness-assumptions}}

Both \textcite{Hyndman2011-sd} and \textcite{Wickramasuriya2019-fc}
impose two unbiasedness conditions, i.e., the base forecasts and the
reconciled forecasts are unbiased. \textcite{Ben_Taieb2019-be} proposed
a reconciliation method relaxing the assumption of unbiasedness.
Specifically, by expanding the training window forward by one
observation until \(T-h\), they formulated the reconciliation problem as
a regularized empirical risk minimization (RERM) problem given by \[
\min _{\boldsymbol{G}_h} \frac{1}{(T-T_1-h+1)n}\left\|\boldsymbol{Y}_{h}^{*}-\hat{\boldsymbol{Y}}_{h}^{*} \mathbf{G}_{h}^{\prime} \boldsymbol{S}^{\prime}\right\|_F^2+\lambda\|\operatorname{vec}( \boldsymbol{G}_h)\|_1,
\]

where \(T_1\) denotes the minimum number of observations used for model
training, \(\left\| \cdot \right\|_F\) is the Frobenius norm,
\(\|\cdot\|_1\) is the \(L_1\) norm, \(\operatorname{vec}(\cdot)\)
denotes the vectorization of a matrix, which stacks the columns of the
matrix on top of one another,
\(\boldsymbol{Y}_{h}^{*}=\left[\boldsymbol{y}_{T_1+h}, \ldots, \boldsymbol{y}_T\right]^{\prime} \in \mathbb{R}^{\left(T-T_1-h+1\right) \times n}\),
\(\hat{\boldsymbol{Y}}_{h}^{*}=\left[\hat{\boldsymbol{y}}_{T_1+h \mid T_1}, \ldots, \hat{\boldsymbol{y}}_{T \mid T-h}\right]^{\prime} \in \mathbb{R}^{\left(T-T_1-h+1\right) \times n}\),
and \(\lambda \geq 0\) is a regularization parameter.

When \(\lambda = 0\), the problem reduces to an empirical risk
minimization (ERM) problem without regularization. Assuming that the
series in the structure are jointly weakly stationary and
\(\hat{\boldsymbol{Y}}_{h}^{*\prime}\hat{\boldsymbol{Y}}_{h}^{*}\) is
invertible, it has a closed-form solution given by \[
\hat{\boldsymbol{G}}_h = \boldsymbol{B}_{h}^{*\prime}\hat{\boldsymbol{Y}}_{h}^{*}\left(\hat{\boldsymbol{Y}}_{h}^{*\prime}\hat{\boldsymbol{Y}}_{h}^{*}\right)^{-1},
\]

where
\(\boldsymbol{B}_{h}^{*}=\left[\boldsymbol{b}_{T_1+h}, \ldots, \boldsymbol{b}_T\right]^{\prime} \in \mathbb{R}^{\left(T-T_1-h+1\right) \times n}\).
If \(\hat{\boldsymbol{Y}}_{h}^{*\prime}\hat{\boldsymbol{Y}}_{h}^{*}\) is
not invertible, they suggested using a generalized inverse.

When \(\lambda > 0\), imposing such a \(L_1\) penalty on
\(\boldsymbol{G}_h\) will introduce sparsity and reduce estimation
variance, albeit at the cost of introducing some bias. In addition, they
also proposed another strategy that penalizes the matrix
\(\boldsymbol{G}_h\) towards the solution obtained by bottom-up
(\textbf{BU}) method, i.e.,
\(\boldsymbol{G}_{\text{BU}} = \left[\boldsymbol{0}_{n_b \times n_a} \mid \boldsymbol{I}_{n_b}\right]\).

Following the work, \textcite{Wickramasuriya2021-am} proposed an
empirical MinT (\textbf{EMinT}) without the unbiasedness constraint by
minimizing the trace of the covariance matrix of the reconciled forecast
errors, \(\operatorname{Var}(\tilde{\boldsymbol{e}}_{T+h \mid T})\).
Assuming that the series are jointly weakly stationary, she derived the
solution given by \[
\hat{\boldsymbol{G}}_{h} = \boldsymbol{B}_{h}^{\prime}\hat{\boldsymbol{Y}}_{h}\left(\hat{\boldsymbol{Y}}_{h}^{\prime}\hat{\boldsymbol{Y}}_{h}\right)^{-1},
\]

where
\(\boldsymbol{B}_{h}=\left[\boldsymbol{b}_{h}, \ldots, \boldsymbol{b}_T\right]^{\prime} \in \mathbb{R}^{\left(T-h+1\right) \times n}\),
and
\(\hat{\boldsymbol{Y}}_{h}=\left[\hat{\boldsymbol{y}}_{h \mid 0}, \ldots, \hat{\boldsymbol{y}}_{T \mid T-h}\right]^{\prime} \in \mathbb{R}^{\left(T-h+1\right) \times n}\).
The difference between EMinT and ERM lies in the data sources used, as
EMinT uses in-sample observations and base forecasts, while ERM relies
on observations and base forecasts from a holdout validation set. We
note that both ERM and EMinT consider an estimate of \(\boldsymbol{G}\)
that changes over the forecast horizon, which is why we keep the
subscript \(h\) here. When reporting the EMinT results in following
sections, we assume the weighting matrix \(\boldsymbol{G}\) for \(h=1\)
holds for \(h>1\).

In practice, a prevalent challenge in forecast reconciliation arises
when the base forecasts of some time series within the structure may
perform poorly, especially for large hierarchies. This can be attributed
to either the inherent complexity of forecasting these series or
potential model misspecification. In such cases, the effectiveness of
forecast reconciliation may diminish, as the role of the weighting
matrix \(\boldsymbol{G}\) is to assimilate \emph{all} base forecasts and
map them into bottom-level disaggregated forecasts which are
subsequently summed by \(\boldsymbol{S}\). While the RERM method
proposed by \textcite{Ben_Taieb2019-be} introduces sparsity by shrinking
some elements of \(\boldsymbol{G}\) towards zero, it remains incapable
of mitigating the adverse impact of underperforming base forecasts on
the quality of the reconciled forecasts. Moreover, the method is
time-consuming and might be problematic when there are limited number of
observations because it uses expanding windows to recursively generate
out-of-sample base forecasts, which are then used in the minimization
problem.

We therefore propose two categories of innovative methods, constrained
out-of-sample (under the unbiasedness assumption) and unconstrained
in-sample (without unbiasedness assumption) forecast reconciliation with
time series selection. These methods aim to identify and address the
negative effect of some base forecasts of poor performance in a
structure on the overall performance of the reconciled forecasts.
Additionally, through the incorporation of regularization in our
objective function, our method has the potential to enhance
reconciliation outcomes produced by using a ``poor'' choice of
\(\boldsymbol{W}\), thus reducing the risk of choosing estimator of
\(\boldsymbol{W}\).

\hypertarget{sec-methodology}{%
\section{Forecast reconciliation with time series
selection}\label{sec-methodology}}

In this section, we introduce our methods for keeping forecasts of an
automatically selected set of series, identified as harmful to
reconciliation, unused in forming reconciled forecasts, i.e., forecast
reconciliation with time series selection. Section~\ref{sec-constrained}
introduces constrained reconciliation methods with selection that
formulate the problem based on out-of-sample base forecasts, while
Section~\ref{sec-unconstrained} presents an unconstrained reconciliation
method with selection, where we formulate the problem based on in-sample
observations and base forecasts.

\hypertarget{sec-constrained}{%
\subsection{Series selection with unbiasedness
constraint}\label{sec-constrained}}

As \(\boldsymbol{S}\) is fixed and \(\hat{\boldsymbol{y}}_{T+h \mid T}\)
is given, once we get the estimation of \(\boldsymbol{G}\), the linear
reconciliation performance is determined, as shown in
Equation~\ref{eq-lr}. In this section, the subscript \(h\) is dropped as
we assume \(\boldsymbol{W}\) and \(\boldsymbol{G}\) do not vary with the
forecast horizon. A natural way to keep forecasts of some series unused
in reconciliation is through controlling the number of nonzero column
entries in \(\boldsymbol{G}\). This leads to a generalization of the
MinT optimization problem by applying an additional penalty to the
objective function. More precisely, let
\(\hat{\boldsymbol{y}}:=\hat{\boldsymbol{y}}_{T+1 \mid T}\), we consider
the optimization problem given by
\begin{equation}\protect\hypertarget{eq-op_u}{}{
\begin{aligned}
& \min _{\boldsymbol{G}} \quad \frac{1}{2}\left(\hat{\boldsymbol{y}}-\boldsymbol{SG}\hat{\boldsymbol{y}}\right)^{\prime} \boldsymbol{W}^{-1}\left(\hat{\boldsymbol{y}}-\boldsymbol{SG}\hat{\boldsymbol{y}}\right)
+ \lambda\mathfrak{g}(\boldsymbol{G}) \\
& \text { s.t. } \quad \boldsymbol{GS}=\boldsymbol{I},
\end{aligned}
}\label{eq-op_u}\end{equation}

where \(\mathfrak{g}(\cdot)\) is defined as an exterior penalty function
designed to penalize the columns of \(\boldsymbol{G}\) towards zero,
with \(\lambda\) is the corresponding penalty coefficient. Thus, this
can be considered as \emph{a grouped variable selection problem}, with
each group corresponding to a column of \(\boldsymbol{G}\). Obviously,
these groups are not overlapped. The constraint,
\(\boldsymbol{GS}=\boldsymbol{I}\), reflects the assumption that base
forecasts and reconciled forecasts are unbiased. When \(\lambda = 0\),
\(\forall h\), the problem reduces to the MinT optimization problem in
Equation~\ref{eq-mint_op} with a closed-form solution given by
Equation~\ref{eq-mint}.

\textbf{Proposition 1.} \emph{Under the assumption of unbiasedness, the
count of nonzero column entries of} \(\boldsymbol{G}\) (\emph{i.e., the
number of time series selected for reconciliation}), \emph{derived
through solving Equation~\ref{eq-op_u}, is at least equal to the number
of time series at the bottom level. In addition, we can restore the full
hierarchical structure by aggregating/disaggregating the selected time
series.}

\emph{Proof}. According to the unbiasedness constraint
\(\boldsymbol{GS}=\boldsymbol{I}\), we have \[
\min \left(\operatorname{rank}(\boldsymbol{G}), \operatorname{rank}(\boldsymbol{S})\right) \geq \operatorname{rank}(\boldsymbol{I}_{n_b})=n_b,
\]

which indicates that the count of nonzero column entries of
\(\boldsymbol{G}\) is at least equal to \(n_b\).

Let
\(\boldsymbol{X}_{\cdot \mathbb{S}} \in \mathbb{R}^{r \times |\mathbb{S}|}\)
denote the submatrix of the \(r \times c\) matrix \(\boldsymbol{X}\)
with column indices forming a set \(\mathbb{S}\) (and when
\(\mathbb{S} = \{j\}\), we simply use \(\boldsymbol{X}_{\cdot j}\)),
where \(|\mathbb{S}|\) denotes the size of the set \(\mathbb{S}\).
Similarly, let
\(\boldsymbol{X}_{\mathbb{S}\cdot} \in \mathbb{R}^{|\mathbb{S}| \times c}\)
denote the submatrix of \(\boldsymbol{X}\) whose rows are indexed by a
set \(\mathbb{S}\) (and when \(\mathbb{S} = \{i\}\), we simply use
\(\boldsymbol{X}_{i\cdot}\)). Assuming that the set \(\mathbb{S}\)
consists of the indices of nonzero columns in the solution of
Equation~\ref{eq-op_u}, \(\hat{\boldsymbol{G}}\), the following
equations hold: \[
\begin{aligned}
& \boldsymbol{G}\boldsymbol{S} = \hat{\boldsymbol{G}}_{\cdot \mathbb{S}}\boldsymbol{S}_{\mathbb{S}\cdot} = \boldsymbol{I} \text{, and } \\
& \min \left(\operatorname{rank}(\hat{\boldsymbol{G}}_{\cdot \mathbb{S}}), \operatorname{rank}(\boldsymbol{S}_{\mathbb{S}\cdot})\right) \geq \operatorname{rank}(\boldsymbol{I}_{n_b})=n_b.
\end{aligned}
\]

Additionally, we have
\(\operatorname{rank}(\boldsymbol{S}_{\mathbb{S}\cdot}) \leq n_b\) as
\(\boldsymbol{S}\) has \(n_b\) columns. Therefore, we can conclude that
\(\operatorname{rank}(\boldsymbol{S}_{\mathbb{S}\cdot}) = n_b\).
Moreover, we have \[
\boldsymbol{y}_t = \boldsymbol{S}\boldsymbol{b}_t = \boldsymbol{S}\boldsymbol{G}\boldsymbol{S}\boldsymbol{b}_t=\boldsymbol{S}\hat{\boldsymbol{G}}_{\cdot\mathbb{S}}\boldsymbol{S}_{\mathbb{S}\cdot}\boldsymbol{b}_t=\boldsymbol{S}\hat{\boldsymbol{G}}_{\cdot \mathbb{S}}(\boldsymbol{y}_t)_{\mathbb{S}},
\]

which implies that the hierarchical structure can be fully restored by
aggregating/disaggregating the selected time series denoted by
\((\boldsymbol{y}_{t})_{\mathbb{S}}\).

For example, consider the simple hierarchy shown in
Figure~\ref{fig-hts}, it is not possible for our constrained
reconciliation methods with selection to simultaneously zero out columns
of \(\boldsymbol{G}\) associated with series AA and AB. However, it is
possible to zero out columns related to series AA and BA simultaneously.

\textbf{Proposition 2.} \emph{The optimization problem in
Equation~\ref{eq-op_u} can be reformulated as a least squares problem
with regularization and linear equality constraint as follows:}
\begin{equation}\protect\hypertarget{eq-op_u_reg}{}{
\begin{aligned}
& \min _{\operatorname{vec}(\boldsymbol{G})} \quad \frac{1}{2}\left(\hat{\boldsymbol{y}}-\left(\hat{\boldsymbol{y}}^{\prime} \otimes \boldsymbol{S}\right) \operatorname{vec}(\boldsymbol{G})\right)^{\prime} \boldsymbol{W}^{-1}\left(\hat{\boldsymbol{y}}-\left(\hat{\boldsymbol{y}}^{\prime} \otimes \boldsymbol{S}\right) \operatorname{vec}(\boldsymbol{G})\right) + \lambda\mathfrak{g}\left(\operatorname{vec}(\boldsymbol{G})\right) \\
& \text { s.t. } \quad \left(\boldsymbol{S}^{\prime} \otimes \boldsymbol{I}_{n_b}\right) \operatorname{vec}(\boldsymbol{G})=\operatorname{vec}(\boldsymbol{I}_{n_b}),
\end{aligned}
}\label{eq-op_u_reg}\end{equation}

\emph{which is characterized as a high-dimensional problem in which the
number of features, denoted as} \(p = n_b \times n\)\emph{, is much
larger than the number of observations,} \(n\)\emph{.}

\emph{Proof.} We have \[
\begin{aligned}
& \operatorname{vec}\left(\hat{\boldsymbol{y}}\right) = \hat{\boldsymbol{y}}, \\
& \operatorname{vec}\left(\boldsymbol{SG}\hat{\boldsymbol{y}}\right) = \left(\hat{\boldsymbol{y}}^{\prime} \otimes \boldsymbol{S}\right) \operatorname{vec}(\boldsymbol{G}), \\
& \operatorname{vec}\left(\boldsymbol{GS}\right) = \operatorname{vec}\left(\boldsymbol{I}_{n_b}\boldsymbol{GS}\right) = \left(\boldsymbol{S}^{\prime} \otimes \boldsymbol{I}_{n_b}\right) \operatorname{vec}(\boldsymbol{G}).
\end{aligned}
\]

Substituting the terms in Equation~\ref{eq-op_u} with these expressions,
the previous problem now takes the form of a regression problem with an
additional regularization term and an equality constraint on the
coefficients, as shown in Equation~\ref{eq-op_u_reg}.

Moving forward, we present three classes of regularizations we use to
establish forecast reconciliation with series selection, resulting in
the consideration of three optimization problems: (i) group best-subset
selection with ridge regularization, (ii) intuitive method with \(L_0\)
regularization, and (iii) group lasso method.

\hypertarget{sec-subset}{%
\subsubsection{Group best-subset selection with ridge
regularization}\label{sec-subset}}

In high-dimensional regime with \(p \gg n\), a common desiderata is to
assume that the true regression coefficient (i.e.,
\(\operatorname{vec}(\boldsymbol{G})\) in our problem) is sparse. We
propose to apply a combination of \(L_0\) and \(L_2\) regularization as
the exterior penalty function to control the nonzero column entries in
\(\boldsymbol{G}\): \begin{equation}\protect\hypertarget{eq-subset}{}{
\begin{aligned}
\min _{\operatorname{vec}(\boldsymbol{G})} \quad & \frac{1}{2}\left(\hat{\boldsymbol{y}}-\left(\hat{\boldsymbol{y}}^{\prime} \otimes \boldsymbol{S}\right) \operatorname{vec}(\boldsymbol{G})\right)^{\prime} \boldsymbol{W}^{-1}\left(\hat{\boldsymbol{y}}-\left(\hat{\boldsymbol{y}}^{\prime} \otimes \boldsymbol{S}\right) \operatorname{vec}(\boldsymbol{G})\right) + \lambda_0 \sum_{j=1}^n 1\left(\boldsymbol{G}_{\cdot j} \neq \mathbf{0}\right) + \lambda_2 \left\|\operatorname{vec}\left(\boldsymbol{G}\right)\right\|_2^2 \\
\text { s.t. } \quad & \left(\boldsymbol{S}^{\prime} \otimes \boldsymbol{I}_{n_b}\right) \operatorname{vec}(\boldsymbol{G})=\operatorname{vec}(\boldsymbol{I}_{n_b}),
\end{aligned}
}\label{eq-subset}\end{equation}

where \(1(\cdot)\) is the indicator function, \(\lambda_0 \geq 0\)
controls the number of nonzero columns of \(\boldsymbol{G}\) selected,
\(\lambda_2 \geq 0\) controls the strength of the ridge regularization,
and \(\|\cdot\|_2\) is the \(L_2\) norm. In a hierarchical or grouped
time series context, the parameter of interest in
Equation~\ref{eq-subset}, \(\operatorname{vec}(\boldsymbol{G})\), has an
inherent non-overlapping grouping structure, wherein each group
corresponds to a single column of \(\boldsymbol{G}\), each with a size
of \(n_b\). Therefore, we refer to this reconciliation method as
\emph{group best-subset selection with ridge regularization}. In the
results that follow, we label the \textbf{Subset} method differently
based on various estimators for \(\boldsymbol{W}\), referring to them as
\textbf{OLS-subset}, \textbf{WLSs-subset}, \textbf{WLSv-subset},
\textbf{MinT-subset}, and \textbf{MinTs-subset}, respectively.

The inclusion of the ridge term in Equation~\ref{eq-subset} is motivated
by earlier work on best-subset selection
\autocites[e.g.,][]{Hazimeh2020-xd,Mazumder2022-hx}, which suggests that
additional ridge regularization can mitigate the poor predictive
performance of best-subset selection method in the low signal-to-noise
ratio (SNR) regimes.

We present a Big-M based mixed integer programming (MIP) formulation for
problem in Equation~\ref{eq-subset} given by
\begin{equation}\protect\hypertarget{eq-subset_mip}{}{
\begin{aligned}
\min _{\operatorname{vec}(\boldsymbol{G}), \boldsymbol{z}, \check{\boldsymbol{e}}, \boldsymbol{g}^{+}} & \frac{1}{2}\check{\boldsymbol{e}}^{\prime} \boldsymbol{W}^{-1}\check{\boldsymbol{e}} + \lambda_0 \sum_{j=1}^n z_j + \lambda_2 \boldsymbol{g}^{+\prime}\boldsymbol{g}^{+} \\
\text { s.t. } \quad & \left(\boldsymbol{S}^{\prime} \otimes \boldsymbol{I}_{n_b}\right) \operatorname{vec}(\boldsymbol{G})=\operatorname{vec}\left(\boldsymbol{I}_{n_b}\right) \\
& \hat{\boldsymbol{y}}-\left(\hat{\boldsymbol{y}}^{\prime} \otimes \boldsymbol{S}\right)\operatorname{vec}(\boldsymbol{G}) = \check{\boldsymbol{e}} \\
& \sum_{i=1}^{n_b} g_{i + (j-1) n_b}^{+} \leqslant \mathcal{M} z_j, \quad j \in[n] \\
& \boldsymbol{g}^{+} \geqslant \operatorname{vec}(\boldsymbol{G}) \\
& \boldsymbol{g}^{+} \geqslant-\operatorname{vec}(\boldsymbol{G}) \\
& z_j \in\{0,1\}, \quad j \in[n],
\end{aligned}
}\label{eq-subset_mip}\end{equation}

where \(\mathcal{M}\) is a Big-M parameter (a-priori specified) that is
sufficiently large such that some optimal solution, say
\(\boldsymbol{g}^{+*}\), to Equation~\ref{eq-subset_mip} satisfies
\(\max _{j \in [n]}\sum_{i=1}^{n_b} g_{i + (j-1) n_b}^{+} \leqslant \mathcal{M}\),
the binary variable \(z_j\) controls whether all the regression
coefficients, \(\operatorname{vec}(\boldsymbol{G})\), in group \(j\) are
zero or not, i.e., \(z_j=0\) implies that
\(\boldsymbol{G}_{\cdot j}=\mathbf{0}\), and \(z_j=1\) implies that
\(\sum_{i=1}^{n_b} g_{i + (j-1) n_b}^{+} \leqslant \mathcal{M}\). Such
Big-M formulations are commonly used in MIP problems to model relations
between discrete and continuous variables, and have been recently
explored in regression with \(L_0\) regularization
\autocite{Bertsimas2016-ig}. The problem is a mixed integer quadratic
program (MIQP) that can be solved using commercial MIP solvers, e.g.,
Gurobi and CPLEX.

\textbf{Parameter tuning.} To avoid computationally-expensive
cross-validation, we tune the parameters to minimize the sum of squared
reconciled forecast errors on the truncated training set, comprising
only the \(\max\{h, s\}\) observations closest to the forecast origin,
where \(s\) is the seasonal period for seasonal data and \(s=T\) for
non-seasonal data. Let
\(\lambda_{0}^{1} = \frac{1}{2}\left(\hat{\boldsymbol{y}}-\tilde{\boldsymbol{y}}^{\text{bench}}\right)^{\prime} \boldsymbol{W}^{-1}\left(\hat{\boldsymbol{y}}-\tilde{\boldsymbol{y}}^{\text{bench}}\right)\)
that captures the scale of first term in the objective function, where
\(\tilde{\boldsymbol{y}}^{\text{bench}}\) is a vector of reconciled
forecasts obtained using Equation~\ref{eq-mint} with same estimator of
\(\boldsymbol{W}\), and define
\(\lambda_{0}^{k} = 0.0001\lambda_{0}^{1}\). For the parameter
\(\lambda_0\), we consider a grid of \(k+1\) values,
\(\{\lambda_{0}^{1},...,\lambda_{0}^{k}, 0\}\), where
\(\lambda_{0}^{j} = \lambda_{0}^{1}\left(\lambda_{0}^{k} / \lambda_{0}^{1}\right)^{(j-1) / (k-1)}\)
for \(j \in [k]\). So \(\lambda_{0}^{1},...,\lambda_{0}^{k}\) is a
sequence decreasing on the log scale. We use a grid of six values for
the parameter \(\lambda_2\),
\(\{0, 10^{-2}, 10^{-1}, 10^{0}, 10^{1}, 10^{2}\}\). Therefore, we tune
over a two-dimensional grid of \((k+1) \times 6\) values to find the
optimal combination of \(\lambda_0\) and \(\lambda_2\).

\textbf{Computation details.} The MIQP problem in
Equation~\ref{eq-subset_mip} is NP-Hard and computationally intensive.
\textcite{Bertsimas2016-ig} showed that commercial MIP solvers are
capable of tackling problem instances for \(p\) up to a thousand. To
address larger instances, there has been impressive work on developing
MIP-based approches for solving \(L_0\)-regularized regression problem,
e.g., \textcite{Bertsimas2016-ig}, \textcite{Hazimeh2020-xd}, and
\textcite{Hazimeh2022-hc}. However, it is challenging to extend their
approaches to accommodate additional constraints within the optimization
problem. Despite the potential sluggishness of handling large instances
with commercial MIP solvers, in our experiments, we use Gurobi to solve
our problem in Equation~\ref{eq-subset_mip} by configuring parameters
such as MIPGap = \(0.001\) and TimeLimit = \(600\) seconds for cases
with \(p > 1000\). This enables us to terminate the solver before
reaching the global optimum and return a suboptimal solution instead.
This strategy is motivated by our need to consider numerous parameter
candidates, and the final solution will be validated against the
training set, which prevents the utilization of a very poor estimate of
\(\boldsymbol{G}\).

\hypertarget{sec-intuitive}{%
\subsubsection{\texorpdfstring{Intuitive method with \(L_0\)
regularization}{Intuitive method with L\_0 regularization}}\label{sec-intuitive}}

Instead of estimating the entire matrix \(\boldsymbol{G}\) in
Section~\ref{sec-subset}, we leverage the MinT solution in
Equation~\ref{eq-mint} to streamline the optimization problem under
consideration. Specifically, we define
\(\bar{\boldsymbol{S}} = \boldsymbol{A}\boldsymbol{S}\), where
\(\boldsymbol{A} = \operatorname{diag}(\boldsymbol{z})\) is an
\(n \times n\) diagonal matrix, and \(\boldsymbol{z}\) is an
\(n\)-dimensional vector with elements either equal to 0 or 1. Taking
the MinT solution in Equation~\ref{eq-mint}, we have
\(\bar{\boldsymbol{G}} = (\boldsymbol{S}^{\prime}\boldsymbol{A}^{\prime}\boldsymbol{W}^{-1}\boldsymbol{A}\boldsymbol{S})^{-1}\boldsymbol{S}^{\prime}\boldsymbol{A}^{\prime}\boldsymbol{W}^{-1}\).
Given fixed \(\boldsymbol{S}\) and estimation of \(\boldsymbol{W}\),
\(\bar{\boldsymbol{G}}\) is entirely determined by \(\boldsymbol{A}\).
By this way, when the \(j\)th diagonal element of \(\boldsymbol{A}\)
equals zero, the \(j\)th column of \(\bar{\boldsymbol{G}}\) becomes
entirely composed of zeros. Therefore, the optimization problem can be
reduced to an integer quadratic programming (IQP) problem in which all
of the variables are restricted to be integers: \[
\begin{aligned}
\min _{\boldsymbol{A}} \quad & \frac{1}{2}\left(\hat{\boldsymbol{y}}-\boldsymbol{S}\bar{\boldsymbol{G}}\hat{\boldsymbol{y}}\right)^{\prime} \boldsymbol{W}^{-1}\left(\hat{\boldsymbol{y}}-\boldsymbol{S}\bar{\boldsymbol{G}}\hat{\boldsymbol{y}}\right) + \lambda_0 \sum_{j=1}^n \boldsymbol{A}_{jj} \\
\text { s.t. } \quad & \bar{\boldsymbol{G}} = (\boldsymbol{S}^{\prime}\boldsymbol{A}^{\prime}\boldsymbol{W}^{-1}\boldsymbol{A}\boldsymbol{S})^{-1}\boldsymbol{S}^{\prime}\boldsymbol{A}^{\prime}\boldsymbol{W}^{-1} \\
& \bar{\boldsymbol{G}}\boldsymbol{S} = \boldsymbol{I},
\end{aligned}
\]

where \(\lambda_0 \geq 0\) controls the number of nonzero diagonal
elements in \(\boldsymbol{A}\), consequently affecting the number of
nonzero columns (i.e., selected time series) in \(\boldsymbol{G}\). We
refer to this reconciliation method as \emph{intuitive method with}
\(L_0\) \emph{regularization}. In the results that follow, we label the
\textbf{Intuitive} method differently based on various estimators for
\(\boldsymbol{W}\), referring to them as \textbf{OLS-intuitive},
\textbf{WLSs-intuitive}, \textbf{WLSv-intuitive},
\textbf{MinT-intuitive}, and \textbf{MinTs-intuitive}, respectively.

We should note that implementing grouped variable selection with this
optimization problem can be challenging because it imposes restrictions
on the parameter of interest (\(\bar{\boldsymbol{G}}\)) to ensure it
adheres rigorously to the analytical solution of MinT while making the
selection. Therefore, the resulting solution tends to be dense and may
not have zero columns.

To ensure the invertibility of
\(\boldsymbol{S}^{\prime}\boldsymbol{A}^{\prime}\boldsymbol{W}^{-1}\boldsymbol{A}\boldsymbol{S}\)
and make the problem compatible with Gurobi, we reformulate the problem
as \begin{equation}\protect\hypertarget{eq-intuitive_mip}{}{
\begin{aligned}
\min _{\boldsymbol{A},\bar{\boldsymbol{G}},\boldsymbol{C},\check{\boldsymbol{e}},\boldsymbol{z}} \quad & \frac{1}{2}\check{\boldsymbol{e}}^{\prime} \boldsymbol{W}^{-1}\check{\boldsymbol{e}} + \lambda_0 \sum_{j=1}^n z_j \\
\text { s.t. } \quad & \bar{\boldsymbol{G}}\boldsymbol{S} = \boldsymbol{I} \\
& \hat{\boldsymbol{y}}-\left(\hat{\boldsymbol{y}}^{\prime} \otimes \boldsymbol{S}\right)\operatorname{vec}(\bar{\boldsymbol{G}}) = \check{\boldsymbol{e}} \\
& \bar{\boldsymbol{G}}\boldsymbol{A}\boldsymbol{S} = \boldsymbol{I} \\
& \bar{\boldsymbol{G}} = \boldsymbol{C}\boldsymbol{S}^{\prime}\boldsymbol{A}^{\prime}\boldsymbol{W}^{-1} \\
& z_j \in\{0,1\}, \quad j \in[n].
\end{aligned}
}\label{eq-intuitive_mip}\end{equation}

\textbf{Parameter tuning.} Similarly to the setup in
Section~\ref{sec-subset}, we select the tuning parameter, \(\lambda_0\),
by minimizing the sum of squared reconciled forecast errors on a
truncated training set, comprising only the \(\max\{h, s\}\)
observations occurred prior to the forecast origin. Let
\(\lambda_{0}^{1} = \frac{1}{2}\left(\hat{\boldsymbol{y}}-\tilde{\boldsymbol{y}}^{\text{bench}}\right)^{\prime} \boldsymbol{W}^{-1}\left(\hat{\boldsymbol{y}}-\tilde{\boldsymbol{y}}^{\text{bench}}\right)\),
and \(\lambda_{0}^{k} = 0.0001\lambda_{0}^{1}\), the collection of
candidate values for \(\lambda_0\) we consider is
\(\{\lambda_{0}^{1},...,\lambda_{0}^{k}, 0\}\), where
\(\lambda_{0}^{j} = \lambda_{0}^{1}\left(\lambda_{0}^{k} / \lambda_{0}^{1}\right)^{(j-1) / (k-1)}\)
for \(j \in [k]\).

\textbf{Computation details.} Following a setup akin to that in
Section~\ref{sec-subset}, we employ Gurobi to solve
Equation~\ref{eq-intuitive_mip} by configuring parameters such as MIPGap
= \(0.001\) and TimeLimit = \(600\) seconds for problems with
\(p > 1000\).

\hypertarget{sec-lasso}{%
\subsubsection{Group lasso method}\label{sec-lasso}}

Lasso is another popular method for selection and estimation of
parameters in the context of linear regression. \textcite{Yuan2006-mw}
introduced the group lasso method that can be used when there is a
grouped structure among the variables. Here, we consider \emph{a group
lasso problem under the unbiasedness assumption} given by
\begin{equation}\protect\hypertarget{eq-lasso}{}{
\begin{aligned}
\min _{\boldsymbol{G}} \quad & \frac{1}{2}\left(\hat{\boldsymbol{y}}-\left(\hat{\boldsymbol{y}}^{\prime} \otimes \boldsymbol{S}\right) \operatorname{vec}(\boldsymbol{G})\right)^{\prime} \boldsymbol{W}^{-1}\left(\hat{\boldsymbol{y}}-\left(\hat{\boldsymbol{y}}^{\prime} \otimes \boldsymbol{S}\right) \operatorname{vec}(\boldsymbol{G})\right) + \lambda \sum_{j=1}^n w_j \left\|\boldsymbol{G}_{\cdot j}\right\|_2 \\
\text { s.t. } \quad & \left(\boldsymbol{S}^{\prime} \otimes \boldsymbol{I}_{n_b}\right) \operatorname{vec}(\boldsymbol{G})=\operatorname{vec}\left(\boldsymbol{I}_{n_b}\right),
\end{aligned}
}\label{eq-lasso}\end{equation}

where \(\lambda \geq 0\) is a tuning parameter, \(w_j \neq 0\) is the
penalty weight assigned in \(\boldsymbol{G}_{\cdot j}\) to make model
more flexible, and the second term in the objective is the penalty
function that is intermediate between the \(L_1\)-penalty that is used
in the lasso and the \(L_2\)-penalty that is used in ridge regression.
In the results that follow, we label the \textbf{Lasso} method based on
various estimators for \(\boldsymbol{W}\), referring to them as
\textbf{OLS-lasso}, \textbf{WLSs-lasso}, \textbf{WLSv-lasso},
\textbf{MinT-lasso}, and \textbf{MinTs-lasso}, respectively.

Next, we present the second order cone programming (SOCP) formulation
for the group lasso based estimators given by
\begin{equation}\protect\hypertarget{eq-lasso_socp}{}{
\begin{aligned}
\min _{\operatorname{vec}(\boldsymbol{G}), \check{\boldsymbol{e}}, \boldsymbol{g}^{+}} & \frac{1}{2}\check{\boldsymbol{e}}^{\prime} \boldsymbol{W}_h^{-1}\check{\boldsymbol{e}} + \lambda \sum_{j=1}^n w_j c_j \\
\text { s.t. } \quad & \left(\boldsymbol{S}^{\prime} \otimes \boldsymbol{I}_{n_b}\right) \operatorname{vec}(\boldsymbol{G})=\operatorname{vec}\left(\boldsymbol{I}_{n_b}\right) \\
& \hat{\boldsymbol{y}}-\left(\hat{\boldsymbol{y}}^{\prime} \otimes \boldsymbol{S}\right) \operatorname{vec}(\boldsymbol{G}) = \check{\boldsymbol{e}} \\
& c_j = \sqrt{\sum_{i=1}^{n_b} g_{i + (j-1) n_b}^{+2}}, \quad j \in[n].
\end{aligned}
}\label{eq-lasso_socp}\end{equation}

Equation~\ref{eq-lasso_socp} includes additional auxiliary variables
\(c_j \in \mathbb{R}_{\geq 0}\), \(j \in [n]\), and second order cone
constraints, \(c_j = \sqrt{\sum_{i=1}^{n_b} g_{i + (j-1) n_b}^{+2}}\)
for \(j \in[n]\).

Compared to the previous two methods we proposed, the group lasso method
is computationally friendlier. Nonetheless, \textcite{Hazimeh2023-ie}
demonstrated, both empirically and theoretically, that group
\(L_0\)-regularized method exhibits advantages over its group lasso
counterpart across a range of regimes. Group lasso can either be highly
dense or possess non-zero coefficients that are overly shrunk. This
issue becomes more pronounced when the groups are correlated with each
other as group lasso tends to retain all correlated groups instead of
seeking a more concise model.

\textbf{Penalty weights and parameter tuning.} In the context of group
lasso, the default choice for the penalty weight, \(w_j\), is
\(\sqrt{p_j}\), where \(p_j\) is the size of each group (in our case,
\(p_j = n_b\)). In our experiments, we allocate different penalty
weights to each group using
\(w_j = 1/\left\|\boldsymbol{G}_{\cdot j}^{\text{bench}}\right\|_2\),
which allows us to account for variations in scale across different time
series in the structure.

We compute the group lasso over \(k+1\) values of the tuning parameter
\(\lambda\), and select the tuning parameter by optimizing the sum of
squared reconciled forecast errors on a truncated training set,
consisting only of \(\max\{h, s\}\) observations occurred prior to the
forecast origin. The collection of candidate values for \(\lambda\)
under consideration is \(\{\lambda^{1},...,\lambda^{k}, 0\}\), where
\(\lambda^{1} = \max _{j=1, \ldots, n}\left\|-\left(\left(\hat{\boldsymbol{y}}^{\prime} \otimes \boldsymbol{S}\right)_{\cdot j^{*}}\right)^{\prime} \boldsymbol{W}^{-1} \hat{\boldsymbol{y}}\right\|_2 / w_j\),
\(\lambda^{k} = 0.0001\lambda^{1}\), and
\(\lambda^{j} = \lambda^{1}\left(\lambda^{k} / \lambda^{1}\right)^{(j-1) / (k-1)}\)
for \(j \in [k]\).

\textbf{Proposition 3.} \emph{Ignoring the unbiasedness constraint, we
define} \(\lambda^{1}\) \emph{as the smallest} \(\lambda\) \emph{value
such that all predictors in the group lasso problem have zero
coefficients. Then we have} \[
\lambda^{1} = \max _{j=1, \ldots, n}\left\|-\left(\left(\hat{\boldsymbol{y}}^{\prime} \otimes \boldsymbol{S}\right)_{\cdot j^{*}}\right)^{\prime} \boldsymbol{W}^{-1} \hat{\boldsymbol{y}}\right\|_2 / w_j,
\]

\emph{where} \(j^{*}\) \emph{denotes the column index of}
\(\hat{\boldsymbol{y}}^{\prime} \otimes \boldsymbol{S}\) \emph{that
corresponds to the} \(j\)\emph{th column of} \(\boldsymbol{G}\)\emph{.}

\emph{Proof.} Denote
\(\boldsymbol{\beta} = \operatorname{vec}(\boldsymbol{G})\), and the
first term in the objective of Equation~\ref{eq-lasso} as
\(L\left(\boldsymbol{\beta} \mid \boldsymbol{D}\right)\), where
\(\boldsymbol{D}\) is the working data
\(\{\hat{\boldsymbol{y}} , \hat{\boldsymbol{y}}^{\prime} \otimes \boldsymbol{S}\}\).
Ignoring the unbiasedness constraint, we define \(\lambda^{1}\) as the
smallest \(\lambda\) value such that all predictors in the group lasso
problem have zero coefficients, i.e., the solution at \(\lambda^{1}\) is
\(\hat{\boldsymbol{\beta}}^{1}=\boldsymbol{0}\). (Note that there is no
intercept in our problem.) Under the Karush-Kuhn-Tucker conditions, we
have \[
\begin{aligned}
\lambda^{1} & = \max _{j=1, \ldots, n}\left\|\left[\nabla L\left(\hat{\boldsymbol{\beta}}^{1} \mid \mathbf{D}\right)\right]^{(j)}\right\|_2 / w_j \\
& = \max _{j=1, \ldots, n}\left\|-\left(\left(\hat{\boldsymbol{y}}^{\prime} \otimes \boldsymbol{S}\right)_{\cdot j^{*}}\right)^{\prime} \boldsymbol{W}^{-1} \hat{\boldsymbol{y}}\right\|_2 / w_j.
\end{aligned}
\]

\textbf{Computation details.} Due to the incorporation of the
unbiasedness constraint, we can not directly use some open-source
packages designed for group lasso. Consequently, we employ Gurobi to
solve the SOCP problem, configuring it by setting OptimalityTol =
\(0.0001\).

\hypertarget{sec-unconstrained}{%
\subsection{Series selection method without unbiasedness
constraint}\label{sec-unconstrained}}

In this section, we relax the unbiasedness constraint,
\(\boldsymbol{GS} = \boldsymbol{I}\), and introduce a reconciliation
method with selection that relies on in-sample observations and fitted
values. Let \(\boldsymbol{Y} \in \mathbb{R}^{T \times n}\) denote a
matrix comprising observations from all time series on the training set
in the structure, and
\(\hat{\boldsymbol{Y}} \in \mathbb{R}^{T \times n}\) denote a matrix of
in-sample one-step-ahead forecasts (i.e., fitted values) for all time
series. The proposed \emph{empirical group lasso} method considers the
optimization problem \[
\min _{\boldsymbol{G}} \quad \frac{1}{2 T} \left\|\boldsymbol{Y}-\hat{\boldsymbol{Y}} \boldsymbol{G}^{\prime} \boldsymbol{S}^{\prime}\right\|_F^2 + \lambda \sum_{j=1}^n w_j \left\|\boldsymbol{G}_{\cdot j}\right\|_2,
\]

where \(\lambda \geq 0\) is a tuning parameter, \(w_j \neq 0\) is the
penalty weight assigned in \(\boldsymbol{G}_{\cdot j}\) to make a more
flexible model. We rewrite the problem as \[
\min _{\operatorname{vec}(\boldsymbol{G})} \quad \frac{1}{2 T} \left\|\operatorname{vec}(\boldsymbol{Y})-(\boldsymbol{S} \otimes \hat{\boldsymbol{Y}}) \operatorname{vec}\left(\boldsymbol{G}^{\prime}\right)\right\|_2^2 + \lambda \sum_{j=1}^n w_j \left\|\boldsymbol{G}_{\cdot j}\right\|_2,
\]

which becomes a standard group lasso problem, with
\(\operatorname{vec}(\boldsymbol{Y})\) serving as the dependent variable
and \(\boldsymbol{S} \otimes \hat{\boldsymbol{Y}}\) as the covariate
matrix. We denote this as \textbf{Elasso} in the results that follow.

Upon relaxing the unbiasedness constraint, the number of non-zero column
entries in the solution for \(\boldsymbol{G}\) may be less than the
number of time series at the bottom level. This differs from the series
selection methods with an unbiasedness constraint that we introduced in
Section~\ref{sec-constrained}. In an extreme scenario, it can happen
that the solution takes the form of a top-down
\(\boldsymbol{G}_{TD}=[\boldsymbol{p} \mid \boldsymbol{O}_{n_b \times (n-1)}]\),
where only the column corresponding to the top level (most aggregated
level) retains non-zero values, and
\(\boldsymbol{p} = (p_1, p_2, \ldots, p_{n_b})\) is a proportionality
vector obtained based on in-sample reconciled forecast errors.

We also explored the empirical version of group best-subset selection
with ridge regularization and intuitive method with \(L_0\)
regularization in which we do not impose the unbiasedness constraint. It
is worth mentioning that \textcite{Hazimeh2023-ie} presented a new
algorithmic framework for formulating the group \(L_0\) problem with
ridge regularization and provided the \textbf{L0Group} Python package
for implementation. However, our experiments showed that this algorithm
can not terminate within five hours for typical instances with
\(p \sim 10^4\). Therefore, in this paper, we only present the empirical
group lasso method for series selection without unbiasedness constraint.

\textbf{Penalty weights and parameter tuning.} Similarly to the setup in
Section~\ref{sec-lasso}, we assign different penalty weights to each
group by setting
\(w_j = 1/\left\|\boldsymbol{G}_{\cdot j}^{\text{OLS}}\right\|_2\),
where \(\boldsymbol{G}^{\text{OLS}}\) is the solution obtained by the
OLS estimator of \(\boldsymbol{W}\). Given a fixed tuning parameter
value, we solve the target optimization problem by considering the
initial \(T-T_v\) observations, where \(T_v = \max\{h, s\}\) for
seasonal time series and \(T_v = \lfloor \frac{1}{10}T \rfloor\) for
non-seasonal time series. Then we select the tuning parameter,
\(\lambda\), by minimizing the sum of squared reconciled forecast errors
on a truncated training set, comprising only the \(T_v\) observations
closest to the forecast origin. Specifically, we form the set of
candidate values for \(\lambda\) as
\(\{\lambda^{1},...,\lambda^{k}, 0\}\), where
\(\lambda^{1} = \max _{j=1, \ldots, n}\left\|-\frac{1}{N}\left(\left(\boldsymbol{S} \otimes \hat{\boldsymbol{Y}}\right)_{\cdot j*}\right)^{\prime} \operatorname{vec}(\boldsymbol{Y})\right\|_2 / w_j\),
\(\lambda^{k} = 0.0001\lambda^{1}\), and
\(\lambda^{j} = \lambda^{1}\left(\lambda^{k} / \lambda^{1}\right)^{(j-1) / (k-1)}\)
for \(j \in [k]\). Following the same derivation as in the proof of
\textbf{Proposition 3}, \(\lambda^{1}\) is the smallest \(\lambda\)
value such that all predictors in the empirical group lasso problem have
zero coefficients, i.e., \(\boldsymbol{G} = \boldsymbol{O}\). Note that
we need to resolve the optimization problem based the whole training set
by using the optimal tuning parameter to obtain the final solution.

\textbf{Computation details.} While there are open-source packages
available for solving a group lasso problem, they are still relatively
slow when applied to large instance for practical usage. For example,
given a specific value for the parameter, \(\lambda\), our experiments
observed that, using the \textbf{gglasso} R package, we can not obtain a
solution within five hours for typical instances with \(p \sim 10^4\).
Instead, we use Gurobi to solve the problem based on the SOCP
formulation for the empirical group lasso which aligns with
Equation~\ref{eq-lasso_socp} but omits the unbiasedness constraint.

\hypertarget{sec-simulations}{%
\section{Monte Carlo simulations}\label{sec-simulations}}

To evaluate the performance of various reconciliation methods with time
series selection outlined in Section~\ref{sec-methodology}, we carry out
two simulations with different designs. In both simulations, we consider
a hierarchy comprising two levels of aggregation, as shown in
Figure~\ref{fig-hts}. Specifically, the structure has four series at the
bottom level, and seven series in total, i.e., \(n_b = 4\), and
\(n = 7\). The bottom-level series are first generated and then summed
appropriately to obtain aggregated series at higher levels.

Section~\ref{sec-sim1} considers a setup where the bottom-level series
are generated using a structural time series model, but model
misspecification exists for some series within the structure.
Section~\ref{sec-sim2} explores the impact of correlation between series
on the performance of reconciled forecasts.

\hypertarget{sec-sim1}{%
\subsection{Setup 1: Exploring the effect of model
misspecification}\label{sec-sim1}}

In this simulation design, we follow a simulation setup similar to
\textcite{Wickramasuriya2019-fc}, assuming that the bottom-level time
series are generated using the basic structural time series model \[
\boldsymbol{b}_t=\boldsymbol{\mu}_t+\boldsymbol{\gamma}_t+\boldsymbol{\eta}_t,
\]

where \(\boldsymbol{\mu}_t\), \(\boldsymbol{\gamma}_t\), and
\(\boldsymbol{\eta}_t\) are trend, seasonality, and error components,
respectively. The trend and seasonality components are defined by

\[
\begin{aligned}
\boldsymbol{\mu}_t & =\boldsymbol{\mu}_{t-1}+\boldsymbol{v}_t+\boldsymbol{\varrho}_t, & \boldsymbol{\varrho}_t & \sim \mathcal{N}\left(\boldsymbol{0}, \sigma_{\varrho}^2 \boldsymbol{I}_4\right), \\
\boldsymbol{v}_t & =\boldsymbol{v}_{t-1}+\boldsymbol{\zeta}_t, & \boldsymbol{\zeta}_t & \sim \mathcal{N}\left(\boldsymbol{0}, \sigma_\zeta^2 \boldsymbol{I}_4\right), \\
\boldsymbol{\gamma}_t & =-\sum_{i=1}^{s-1} \boldsymbol{\gamma}_{t-i}+\boldsymbol{\omega}_t, & \boldsymbol{\omega}_t & \sim \mathcal{N}\left(\boldsymbol{0}, \sigma_\omega^2 \boldsymbol{I}_4\right),
\end{aligned}
\]

where \(\boldsymbol{\varrho}_t\), \(\boldsymbol{\zeta}_t\), and
\(\boldsymbol{\omega}_t\) are error terms independent of each other and
over time. The error term \(\boldsymbol{\eta}_t\) is generated
independently from an \(\text{ARIMA}(p,0,q)\) process, where \(p\) and
\(q\) take values of \(0\) or \(1\) with equal probability. The
coefficients for the AR and MA components in the ARIMA process are
sampled randomly from a uniform distribution within the range
\([0.5, 0.7]\), and the contemporaneous error covariance matrix is given
by \[
\left[\begin{array}{llll}
5 & 3 & 2 & 1 \\
3 & 4 & 2 & 1 \\
2 & 2 & 5 & 3 \\
1 & 1 & 3 & 4
\end{array}\right],
\]

which enables correlations among time series in a hierarchical
structure.

We set \(s = 4\) for quarterly data with error variances
\(\sigma_{\varrho}^2=2\), \(\sigma_\zeta^2=0.007\), and
\(\sigma_\omega^2=7\), respectively. The initial values for
\(\boldsymbol{\mu}_0\), \(\boldsymbol{v}_0\), \(\boldsymbol{\gamma}_0\),
\(\boldsymbol{\gamma}_1\), and \(\boldsymbol{\gamma}_2\) are generated
independently from a multivariate normal distribution with zero mean and
identity covariance matrix. For each series at the bottom level, we
generate a total of \(T+h = 180\) observations, with the last \(h = 16\)
observations serving as the test set. Recall that the bottom-level
series are aggregated to obtain the data for the aggregated levels. This
process is repeated \(500\) times.

We use ETS models to generate base forecasts for all time series in the
hierarchy, using the default settings as implemented in the
\textbf{forecast} R package \autocite{Hyndman2023-fc}. To introduce
model misspecification into our experiment, we deliberately undermine
the quality of in-sample and out-of-sample forecasts (i.e., fitted
values and base forecasts) for some specific time series. Specifically,
we investigate three scenarios characterized by artificial model
misspecifications, where a 1.5 multiplier is applied to in-sample and
out-of-sample forecasts for a single series in each scenario, i.e.,
series AA at the bottom level, series A at the middle level, and series
Total at the top level, resulting in Scenario I, Scenario II, and
Scenario III, respectively.

The results for Scenario I, II, and III are presented in
Table~\ref{tbl-s1-rmse}, Table~\ref{tbl-s2-rmse}, and
Table~\ref{tbl-s3-rmse}, respectively. Each table reports the average
root mean squared error (RMSE) for each level as well as the whole
structure (denoted as \emph{Average}). The \emph{Base} row shows the
average RMSE of the base forecasts, while entries below this row
reporting the percentage decrease (negative) or increase (positive) in
the average RMSE of the reconciled forecasts compared to the base
forecasts.

\hypertarget{tbl-s1-rmse}{}
\begin{table}[!h]
\caption{\label{tbl-s1-rmse}Out-of-sample forecast results for the simulated data in Scenario I,
Setup 1. }\tabularnewline

\centering
\resizebox{\linewidth}{!}{
\begin{threeparttable}
\begin{tabular}{lrrrrlrrrrlrrrrlr}
\toprule
\multicolumn{1}{c}{} & \multicolumn{4}{c}{Top} & \multicolumn{4}{c}{Middle} & \multicolumn{4}{c}{Bottom} & \multicolumn{4}{c}{Average} \\
\cmidrule(l{3pt}r{3pt}){2-5} \cmidrule(l{3pt}r{3pt}){6-9} \cmidrule(l{3pt}r{3pt}){10-13} \cmidrule(l{3pt}r{3pt}){14-17}
Method & h=1 & 1-4 & 1-8 & 1-16 & h=1 & 1-4 & 1-8 & 1-16 & h=1 & 1-4 & 1-8 & 1-16 & h=1 & 1-4 & 1-8 & 1-16\\
\midrule
Base & 9.6 & 10.7 & 12.6 & 15.6 & 6.3 & 7.3 & 8.6 & 10.8 & 6.4 & 7.5 & 8.3 & 9.8 & 6.8 & 7.9 & 9.0 & 10.9\\
BU & 57.8 & 68.5 & 53.7 & 38.9 & 58.2 & 61.8 & 48.1 & 34.4 & 0.0 & 0.0 & 0.0 & 0.0 & 27.0 & 29.6 & 23.8 & 17.7\\
\midrule
OLS & 0.6 & 2.2 & 1.8 & 1.4 & 7.1 & 6.4 & 4.6 & 3.1 & -7.6 & -8.6 & -8.2 & -7.3 & -2.1 & -2.5 & -2.7 & -2.6\\
\cellcolor[HTML]{e6e3e3}{OLS-subset} & \cellcolor[HTML]{e6e3e3}{0.6} & \cellcolor[HTML]{e6e3e3}{\textbf{ 1.8}} & \cellcolor[HTML]{e6e3e3}{\textbf{ 1.5}} & \cellcolor[HTML]{e6e3e3}{\textbf{ 1.3}} & \cellcolor[HTML]{e6e3e3}{7.2} & \cellcolor[HTML]{e6e3e3}{\textbf{ 5.2}} & \cellcolor[HTML]{e6e3e3}{\textbf{ 3.8}} & \cellcolor[HTML]{e6e3e3}{\textbf{ 2.6}} & \cellcolor[HTML]{e6e3e3}{\textbf{ -8.3}} & \cellcolor[HTML]{e6e3e3}{\textbf{-12.9}} & \cellcolor[HTML]{e6e3e3}{\textbf{-11.6}} & \cellcolor[HTML]{e6e3e3}{\textbf{ -9.9}} & \cellcolor[HTML]{e6e3e3}{\textbf{ -2.4}} & \cellcolor[HTML]{e6e3e3}{\textbf{ -5.2}} & \cellcolor[HTML]{e6e3e3}{\textbf{ -4.8}} & \cellcolor[HTML]{e6e3e3}{\textbf{ -4.1}}\\
\cellcolor[HTML]{e6e3e3}{OLS-intuitive} & \cellcolor[HTML]{e6e3e3}{0.8} & \cellcolor[HTML]{e6e3e3}{2.6} & \cellcolor[HTML]{e6e3e3}{2.1} & \cellcolor[HTML]{e6e3e3}{1.8} & \cellcolor[HTML]{e6e3e3}{7.5} & \cellcolor[HTML]{e6e3e3}{\textbf{ 6.1}} & \cellcolor[HTML]{e6e3e3}{\textbf{ 4.4}} & \cellcolor[HTML]{e6e3e3}{\textbf{ 3.0}} & \cellcolor[HTML]{e6e3e3}{\textbf{ -9.0}} & \cellcolor[HTML]{e6e3e3}{\textbf{-12.8}} & \cellcolor[HTML]{e6e3e3}{\textbf{-11.6}} & \cellcolor[HTML]{e6e3e3}{\textbf{ -9.9}} & \cellcolor[HTML]{e6e3e3}{\textbf{ -2.7}} & \cellcolor[HTML]{e6e3e3}{\textbf{ -4.8}} & \cellcolor[HTML]{e6e3e3}{\textbf{ -4.5}} & \cellcolor[HTML]{e6e3e3}{\textbf{ -3.8}}\\
\cellcolor[HTML]{e6e3e3}{OLS-lasso} & \cellcolor[HTML]{e6e3e3}{0.6} & \cellcolor[HTML]{e6e3e3}{2.2} & \cellcolor[HTML]{e6e3e3}{1.8} & \cellcolor[HTML]{e6e3e3}{1.6} & \cellcolor[HTML]{e6e3e3}{7.4} & \cellcolor[HTML]{e6e3e3}{6.7} & \cellcolor[HTML]{e6e3e3}{4.8} & \cellcolor[HTML]{e6e3e3}{3.2} & \cellcolor[HTML]{e6e3e3}{-7.6} & \cellcolor[HTML]{e6e3e3}{-8.5} & \cellcolor[HTML]{e6e3e3}{-8.1} & \cellcolor[HTML]{e6e3e3}{-7.2} & \cellcolor[HTML]{e6e3e3}{-2.0} & \cellcolor[HTML]{e6e3e3}{-2.4} & \cellcolor[HTML]{e6e3e3}{-2.6} & \cellcolor[HTML]{e6e3e3}{-2.5}\\
\midrule
WLSs & 7.3 & 10.6 & 8.1 & 5.9 & 15.6 & 16.0 & 11.8 & 8.0 & -6.9 & -7.8 & -7.4 & -6.4 & 1.9 & 2.0 & 1.0 & 0.2\\
\cellcolor[HTML]{e6e3e3}{WLSs-subset} & \cellcolor[HTML]{e6e3e3}{\textbf{ 5.0}} & \cellcolor[HTML]{e6e3e3}{\textbf{ 5.7}} & \cellcolor[HTML]{e6e3e3}{\textbf{ 4.6}} & \cellcolor[HTML]{e6e3e3}{\textbf{ 3.6}} & \cellcolor[HTML]{e6e3e3}{\textbf{12.3}} & \cellcolor[HTML]{e6e3e3}{\textbf{10.0}} & \cellcolor[HTML]{e6e3e3}{\textbf{ 7.5}} & \cellcolor[HTML]{e6e3e3}{\textbf{ 5.2}} & \cellcolor[HTML]{e6e3e3}{\textbf{ -7.6}} & \cellcolor[HTML]{e6e3e3}{\textbf{-10.5}} & \cellcolor[HTML]{e6e3e3}{\textbf{ -9.6}} & \cellcolor[HTML]{e6e3e3}{\textbf{ -8.2}} & \cellcolor[HTML]{e6e3e3}{\textbf{  0.2}} & \cellcolor[HTML]{e6e3e3}{\textbf{ -2.0}} & \cellcolor[HTML]{e6e3e3}{\textbf{ -2.1}} & \cellcolor[HTML]{e6e3e3}{\textbf{ -2.0}}\\
\cellcolor[HTML]{e6e3e3}{WLSs-intuitive} & \cellcolor[HTML]{e6e3e3}{\textbf{ 7.1}} & \cellcolor[HTML]{e6e3e3}{\textbf{ 9.2}} & \cellcolor[HTML]{e6e3e3}{\textbf{ 7.1}} & \cellcolor[HTML]{e6e3e3}{\textbf{ 5.2}} & \cellcolor[HTML]{e6e3e3}{16.5} & \cellcolor[HTML]{e6e3e3}{\textbf{15.5}} & \cellcolor[HTML]{e6e3e3}{\textbf{11.5}} & \cellcolor[HTML]{e6e3e3}{\textbf{ 7.9}} & \cellcolor[HTML]{e6e3e3}{-6.8} & \cellcolor[HTML]{e6e3e3}{\textbf{ -9.2}} & \cellcolor[HTML]{e6e3e3}{\textbf{ -8.4}} & \cellcolor[HTML]{e6e3e3}{\textbf{ -7.3}} & \cellcolor[HTML]{e6e3e3}{2.1} & \cellcolor[HTML]{e6e3e3}{\textbf{  0.9}} & \cellcolor[HTML]{e6e3e3}{\textbf{  0.1}} & \cellcolor[HTML]{e6e3e3}{\textbf{ -0.4}}\\
\cellcolor[HTML]{e6e3e3}{WLSs-lasso} & \cellcolor[HTML]{e6e3e3}{7.3} & \cellcolor[HTML]{e6e3e3}{\textbf{10.3}} & \cellcolor[HTML]{e6e3e3}{\textbf{ 8.0}} & \cellcolor[HTML]{e6e3e3}{5.9} & \cellcolor[HTML]{e6e3e3}{15.7} & \cellcolor[HTML]{e6e3e3}{16.1} & \cellcolor[HTML]{e6e3e3}{11.8} & \cellcolor[HTML]{e6e3e3}{8.1} & \cellcolor[HTML]{e6e3e3}{\textbf{ -7.0}} & \cellcolor[HTML]{e6e3e3}{-7.8} & \cellcolor[HTML]{e6e3e3}{-7.3} & \cellcolor[HTML]{e6e3e3}{-6.4} & \cellcolor[HTML]{e6e3e3}{1.9} & \cellcolor[HTML]{e6e3e3}{2.0} & \cellcolor[HTML]{e6e3e3}{1.0} & \cellcolor[HTML]{e6e3e3}{0.2}\\
\midrule
WLSv & 1.0 & 2.9 & 2.3 & 1.9 & 4.5 & 4.3 & 3.2 & 2.1 & -25.8 & -26.4 & -22.7 & -18.3 & -12.4 & -12.6 & -10.7 & -8.4\\
\cellcolor[HTML]{e6e3e3}{WLSv-subset} & \cellcolor[HTML]{e6e3e3}{\textcolor{blue}{\textbf{-1.0}}} & \cellcolor[HTML]{e6e3e3}{\textbf{ 0.3}} & \cellcolor[HTML]{e6e3e3}{\textbf{ 0.4}} & \cellcolor[HTML]{e6e3e3}{\textbf{ 0.5}} & \cellcolor[HTML]{e6e3e3}{\textbf{ 0.6}} & \cellcolor[HTML]{e6e3e3}{\textbf{ 0.6}} & \cellcolor[HTML]{e6e3e3}{\textbf{ 0.5}} & \cellcolor[HTML]{e6e3e3}{\textbf{ 0.3}} & \cellcolor[HTML]{e6e3e3}{\textbf{-32.3}} & \cellcolor[HTML]{e6e3e3}{\textbf{-32.2}} & \cellcolor[HTML]{e6e3e3}{\textbf{-27.3}} & \cellcolor[HTML]{e6e3e3}{\textbf{-21.7}} & \cellcolor[HTML]{e6e3e3}{\textbf{-17.3}} & \cellcolor[HTML]{e6e3e3}{\textbf{-17.3}} & \cellcolor[HTML]{e6e3e3}{\textbf{-14.2}} & \cellcolor[HTML]{e6e3e3}{\textbf{-10.9}}\\
\cellcolor[HTML]{e6e3e3}{WLSv-intuitive} & \cellcolor[HTML]{e6e3e3}{\textbf{-0.5}} & \cellcolor[HTML]{e6e3e3}{\textcolor{blue}{\textbf{ 0.2}}} & \cellcolor[HTML]{e6e3e3}{\textcolor{blue}{\textbf{ 0.3}}} & \cellcolor[HTML]{e6e3e3}{\textbf{ 0.5}} & \cellcolor[HTML]{e6e3e3}{\textbf{ 0.9}} & \cellcolor[HTML]{e6e3e3}{\textbf{ 0.7}} & \cellcolor[HTML]{e6e3e3}{\textbf{ 0.5}} & \cellcolor[HTML]{e6e3e3}{\textbf{ 0.3}} & \cellcolor[HTML]{e6e3e3}{\textbf{-32.3}} & \cellcolor[HTML]{e6e3e3}{\textbf{-32.3}} & \cellcolor[HTML]{e6e3e3}{\textbf{-27.4}} & \cellcolor[HTML]{e6e3e3}{\textbf{-21.7}} & \cellcolor[HTML]{e6e3e3}{\textbf{-17.1}} & \cellcolor[HTML]{e6e3e3}{\textbf{-17.3}} & \cellcolor[HTML]{e6e3e3}{\textbf{-14.2}} & \cellcolor[HTML]{e6e3e3}{\textbf{-10.9}}\\
\cellcolor[HTML]{e6e3e3}{WLSv-lasso} & \cellcolor[HTML]{e6e3e3}{\textbf{ 0.4}} & \cellcolor[HTML]{e6e3e3}{\textbf{ 1.5}} & \cellcolor[HTML]{e6e3e3}{\textbf{ 1.5}} & \cellcolor[HTML]{e6e3e3}{\textbf{ 1.4}} & \cellcolor[HTML]{e6e3e3}{\textbf{ 3.0}} & \cellcolor[HTML]{e6e3e3}{\textbf{ 2.5}} & \cellcolor[HTML]{e6e3e3}{\textbf{ 2.0}} & \cellcolor[HTML]{e6e3e3}{\textbf{ 1.3}} & \cellcolor[HTML]{e6e3e3}{\textbf{-28.5}} & \cellcolor[HTML]{e6e3e3}{\textbf{-29.2}} & \cellcolor[HTML]{e6e3e3}{\textbf{-24.9}} & \cellcolor[HTML]{e6e3e3}{\textbf{-19.9}} & \cellcolor[HTML]{e6e3e3}{\textbf{-14.4}} & \cellcolor[HTML]{e6e3e3}{\textbf{-14.9}} & \cellcolor[HTML]{e6e3e3}{\textbf{-12.3}} & \cellcolor[HTML]{e6e3e3}{\textbf{ -9.5}}\\
\midrule
MinT & -0.4 & 0.7 & 0.9 & 0.6 & 0.7 & 0.7 & 0.6 & 0.3 & -32.9 & -33.4 & -28.3 & -22.5 & -17.5 & -17.8 & -14.6 & -11.3\\
\cellcolor[HTML]{e6e3e3}{MinT-subset} & \cellcolor[HTML]{e6e3e3}{\textbf{-0.6}} & \cellcolor[HTML]{e6e3e3}{0.7} & \cellcolor[HTML]{e6e3e3}{\textbf{ 0.8}} & \cellcolor[HTML]{e6e3e3}{0.7} & \cellcolor[HTML]{e6e3e3}{\textbf{ 0.6}} & \cellcolor[HTML]{e6e3e3}{0.8} & \cellcolor[HTML]{e6e3e3}{0.6} & \cellcolor[HTML]{e6e3e3}{0.3} & \cellcolor[HTML]{e6e3e3}{\textbf{-33.0}} & \cellcolor[HTML]{e6e3e3}{-33.1} & \cellcolor[HTML]{e6e3e3}{-28.0} & \cellcolor[HTML]{e6e3e3}{-22.3} & \cellcolor[HTML]{e6e3e3}{\textbf{-17.6}} & \cellcolor[HTML]{e6e3e3}{-17.6} & \cellcolor[HTML]{e6e3e3}{-14.5} & \cellcolor[HTML]{e6e3e3}{-11.2}\\
\cellcolor[HTML]{e6e3e3}{MinT-intuitive} & \cellcolor[HTML]{e6e3e3}{-0.4} & \cellcolor[HTML]{e6e3e3}{0.7} & \cellcolor[HTML]{e6e3e3}{0.9} & \cellcolor[HTML]{e6e3e3}{0.6} & \cellcolor[HTML]{e6e3e3}{0.7} & \cellcolor[HTML]{e6e3e3}{0.7} & \cellcolor[HTML]{e6e3e3}{0.6} & \cellcolor[HTML]{e6e3e3}{0.3} & \cellcolor[HTML]{e6e3e3}{-32.9} & \cellcolor[HTML]{e6e3e3}{-33.4} & \cellcolor[HTML]{e6e3e3}{-28.3} & \cellcolor[HTML]{e6e3e3}{-22.5} & \cellcolor[HTML]{e6e3e3}{-17.5} & \cellcolor[HTML]{e6e3e3}{-17.8} & \cellcolor[HTML]{e6e3e3}{-14.6} & \cellcolor[HTML]{e6e3e3}{-11.3}\\
\cellcolor[HTML]{e6e3e3}{MinT-lasso} & \cellcolor[HTML]{e6e3e3}{\textbf{-0.7}} & \cellcolor[HTML]{e6e3e3}{\textbf{ 0.3}} & \cellcolor[HTML]{e6e3e3}{\textbf{ 0.6}} & \cellcolor[HTML]{e6e3e3}{\textcolor{blue}{\textbf{ 0.4}}} & \cellcolor[HTML]{e6e3e3}{\textcolor{blue}{\textbf{ 0.3}}} & \cellcolor[HTML]{e6e3e3}{\textcolor{blue}{\textbf{ 0.4}}} & \cellcolor[HTML]{e6e3e3}{\textcolor{blue}{\textbf{ 0.4}}} & \cellcolor[HTML]{e6e3e3}{\textcolor{blue}{\textbf{ 0.1}}} & \cellcolor[HTML]{e6e3e3}{\textcolor{blue}{\textbf{-33.2}}} & \cellcolor[HTML]{e6e3e3}{\textcolor{blue}{\textbf{-33.7}}} & \cellcolor[HTML]{e6e3e3}{\textcolor{blue}{\textbf{-28.5}}} & \cellcolor[HTML]{e6e3e3}{\textcolor{blue}{\textbf{-22.6}}} & \cellcolor[HTML]{e6e3e3}{\textcolor{blue}{\textbf{-17.8}}} & \cellcolor[HTML]{e6e3e3}{\textcolor{blue}{\textbf{-18.1}}} & \cellcolor[HTML]{e6e3e3}{\textcolor{blue}{\textbf{-14.8}}} & \cellcolor[HTML]{e6e3e3}{\textcolor{blue}{\textbf{-11.4}}}\\
\midrule
MinTs & -0.9 & 0.6 & 0.7 & 0.5 & 0.6 & 0.6 & 0.5 & 0.2 & -32.9 & -33.5 & -28.3 & -22.5 & -17.6 & -17.9 & -14.6 & -11.3\\
\cellcolor[HTML]{e6e3e3}{MinTs-subset} & \cellcolor[HTML]{e6e3e3}{-0.7} & \cellcolor[HTML]{e6e3e3}{0.9} & \cellcolor[HTML]{e6e3e3}{1.1} & \cellcolor[HTML]{e6e3e3}{1.0} & \cellcolor[HTML]{e6e3e3}{0.7} & \cellcolor[HTML]{e6e3e3}{0.8} & \cellcolor[HTML]{e6e3e3}{0.7} & \cellcolor[HTML]{e6e3e3}{0.4} & \cellcolor[HTML]{e6e3e3}{\textbf{-33.0}} & \cellcolor[HTML]{e6e3e3}{-33.1} & \cellcolor[HTML]{e6e3e3}{-27.9} & \cellcolor[HTML]{e6e3e3}{-22.2} & \cellcolor[HTML]{e6e3e3}{-17.6} & \cellcolor[HTML]{e6e3e3}{-17.5} & \cellcolor[HTML]{e6e3e3}{-14.3} & \cellcolor[HTML]{e6e3e3}{-11.0}\\
\cellcolor[HTML]{e6e3e3}{MinTs-intuitive} & \cellcolor[HTML]{e6e3e3}{-0.9} & \cellcolor[HTML]{e6e3e3}{0.6} & \cellcolor[HTML]{e6e3e3}{0.7} & \cellcolor[HTML]{e6e3e3}{0.5} & \cellcolor[HTML]{e6e3e3}{0.6} & \cellcolor[HTML]{e6e3e3}{0.6} & \cellcolor[HTML]{e6e3e3}{0.5} & \cellcolor[HTML]{e6e3e3}{0.2} & \cellcolor[HTML]{e6e3e3}{-32.9} & \cellcolor[HTML]{e6e3e3}{-33.5} & \cellcolor[HTML]{e6e3e3}{-28.3} & \cellcolor[HTML]{e6e3e3}{-22.5} & \cellcolor[HTML]{e6e3e3}{-17.6} & \cellcolor[HTML]{e6e3e3}{-17.9} & \cellcolor[HTML]{e6e3e3}{-14.6} & \cellcolor[HTML]{e6e3e3}{-11.3}\\
\cellcolor[HTML]{e6e3e3}{MinTs-lasso} & \cellcolor[HTML]{e6e3e3}{-0.9} & \cellcolor[HTML]{e6e3e3}{\textbf{ 0.4}} & \cellcolor[HTML]{e6e3e3}{\textbf{ 0.6}} & \cellcolor[HTML]{e6e3e3}{0.5} & \cellcolor[HTML]{e6e3e3}{0.6} & \cellcolor[HTML]{e6e3e3}{\textcolor{blue}{\textbf{ 0.4}}} & \cellcolor[HTML]{e6e3e3}{\textcolor{blue}{\textbf{ 0.4}}} & \cellcolor[HTML]{e6e3e3}{\textcolor{blue}{\textbf{ 0.1}}} & \cellcolor[HTML]{e6e3e3}{\textcolor{blue}{\textbf{-33.2}}} & \cellcolor[HTML]{e6e3e3}{\textbf{-33.6}} & \cellcolor[HTML]{e6e3e3}{\textbf{-28.4}} & \cellcolor[HTML]{e6e3e3}{\textcolor{blue}{\textbf{-22.6}}} & \cellcolor[HTML]{e6e3e3}{\textbf{-17.7}} & \cellcolor[HTML]{e6e3e3}{\textbf{-18.0}} & \cellcolor[HTML]{e6e3e3}{\textcolor{blue}{\textbf{-14.8}}} & \cellcolor[HTML]{e6e3e3}{\textcolor{blue}{\textbf{-11.4}}}\\
\midrule
EMinT & 2.2 & 2.9 & 2.5 & 1.7 & 2.5 & 2.9 & 2.3 & 1.3 & -31.9 & -32.3 & -27.5 & -22.0 & -15.9 & -16.2 & -13.4 & -10.5\\
\cellcolor[HTML]{e6e3e3}{Elasso} & \cellcolor[HTML]{e6e3e3}{\textbf{ 1.5}} & \cellcolor[HTML]{e6e3e3}{\textbf{ 2.8}} & \cellcolor[HTML]{e6e3e3}{\textbf{ 2.4}} & \cellcolor[HTML]{e6e3e3}{1.7} & \cellcolor[HTML]{e6e3e3}{\textbf{ 2.1}} & \cellcolor[HTML]{e6e3e3}{\textbf{ 2.8}} & \cellcolor[HTML]{e6e3e3}{2.3} & \cellcolor[HTML]{e6e3e3}{1.3} & \cellcolor[HTML]{e6e3e3}{\textbf{-32.1}} & \cellcolor[HTML]{e6e3e3}{-32.2} & \cellcolor[HTML]{e6e3e3}{-27.4} & \cellcolor[HTML]{e6e3e3}{-21.9} & \cellcolor[HTML]{e6e3e3}{\textbf{-16.3}} & \cellcolor[HTML]{e6e3e3}{-16.2} & \cellcolor[HTML]{e6e3e3}{-13.3} & \cellcolor[HTML]{e6e3e3}{-10.5}\\
\bottomrule
\end{tabular}
\begin{tablenotes}[para]
\item \underline{\textit{NOTE:}} 
\item The Base row shows the average RMSE of the base forecasts. Entries below this row indicate the percentage decrease (negative) or increase (positive) in the average RMSE of the reconciled forecasts compared to the base forecasts. The entries with the lowest values in each column are highlighted in blue. In each panel, the proposed methods are indicated with a gray background, and methods that outperform the benchmark method are marked in bold.
\end{tablenotes}
\end{threeparttable}}
\end{table}

\hypertarget{tbl-s1-selection}{}
\begin{table}[!h]
\caption{\label{tbl-s1-selection}Proportion of time series being selected after using the proposed
reconciliation methods with selection in Scenario I, Setup 1. }\tabularnewline

\centering\begingroup\fontsize{7}{9}\selectfont

\resizebox{\linewidth}{!}{
\begin{threeparttable}
\begin{tabular}{llrrrrrr>{}r}
\toprule
  & Top & A & B & AA & AB & BA & BB & Summary\\
\midrule
OLS-subset & 0.52 & 0.79 & 0.57 & 0.79 & 1 & 0.91 & 0.85 & \includegraphics[width=0.47in, height=0.1in]{/Users/xwan0362/Git/hfs/paper/_figs/s1_OLS-subset.png}\\
OLS-intuitive & 0.80 & 0.90 & 0.81 & 0.80 & 1 & 0.85 & 0.86 & \includegraphics[width=0.47in, height=0.1in]{/Users/xwan0362/Git/hfs/paper/_figs/s1_OLS-intuitive.png}\\
OLS-lasso & 0.90 & 1.00 & 0.68 & 1.00 & 1 & 1.00 & 1.00 & \includegraphics[width=0.47in, height=0.1in]{/Users/xwan0362/Git/hfs/paper/_figs/s1_OLS-lasso.png}\\
\midrule
WLSs-subset & 0.85 & 0.91 & 0.86 & 0.90 & 1 & 0.97 & 0.97 & \includegraphics[width=0.47in, height=0.1in]{/Users/xwan0362/Git/hfs/paper/_figs/s1_WLSs-subset.png}\\
WLSs-intuitive & 0.92 & 0.95 & 0.67 & 0.92 & 1 & 0.92 & 0.95 & \includegraphics[width=0.47in, height=0.1in]{/Users/xwan0362/Git/hfs/paper/_figs/s1_WLSs-intuitive.png}\\
WLSs-lasso & 0.72 & 1.00 & 0.72 & 1.00 & 1 & 1.00 & 1.00 & \includegraphics[width=0.47in, height=0.1in]{/Users/xwan0362/Git/hfs/paper/_figs/s1_WLSs-lasso.png}\\
\midrule
WLSv-subset & 0.50 & 0.62 & 0.42 & 0.19 & 1 & 0.81 & 0.87 & \includegraphics[width=0.47in, height=0.1in]{/Users/xwan0362/Git/hfs/paper/_figs/s1_WLSv-subset.png}\\
WLSv-intuitive & 0.59 & 0.55 & 0.49 & 0.17 & 1 & 0.76 & 0.86 & \includegraphics[width=0.47in, height=0.1in]{/Users/xwan0362/Git/hfs/paper/_figs/s1_WLSv-intuitive.png}\\
WLSv-lasso & 0.40 & 1.00 & 0.41 & 0.77 & 1 & 1.00 & 1.00 & \includegraphics[width=0.47in, height=0.1in]{/Users/xwan0362/Git/hfs/paper/_figs/s1_WLSv-lasso.png}\\
\midrule
MinT-subset & 0.66 & 0.90 & 0.61 & 0.72 & 1 & 0.91 & 0.93 & \includegraphics[width=0.47in, height=0.1in]{/Users/xwan0362/Git/hfs/paper/_figs/s1_MinT-subset.png}\\
MinT-intuitive & 1.00 & 1.00 & 1.00 & 1.00 & 1 & 1.00 & 1.00 & \includegraphics[width=0.47in, height=0.1in]{/Users/xwan0362/Git/hfs/paper/_figs/s1_MinT-intuitive.png}\\
MinT-lasso & 0.80 & 0.96 & 0.84 & 0.72 & 1 & 0.98 & 0.97 & \includegraphics[width=0.47in, height=0.1in]{/Users/xwan0362/Git/hfs/paper/_figs/s1_MinT-lasso.png}\\
\midrule
MinTs-subset & 0.57 & 0.88 & 0.52 & 0.67 & 1 & 0.89 & 0.92 & \includegraphics[width=0.47in, height=0.1in]{/Users/xwan0362/Git/hfs/paper/_figs/s1_MinTs-subset.png}\\
MinTs-intuitive & 1.00 & 1.00 & 1.00 & 1.00 & 1 & 1.00 & 1.00 & \includegraphics[width=0.47in, height=0.1in]{/Users/xwan0362/Git/hfs/paper/_figs/s1_MinTs-intuitive.png}\\
MinTs-lasso & 0.68 & 1.00 & 0.66 & 0.74 & 1 & 1.00 & 1.00 & \includegraphics[width=0.47in, height=0.1in]{/Users/xwan0362/Git/hfs/paper/_figs/s1_MinTs-lasso.png}\\
\midrule
Elasso & 0.82 & 0.63 & 0.69 & 1.00 & 1 & 1.00 & 1.00 & \includegraphics[width=0.47in, height=0.1in]{/Users/xwan0362/Git/hfs/paper/_figs/s1_Elasso.png}\\
\bottomrule
\end{tabular}
\begin{tablenotes}[para]
\item \underline{\textit{NOTE:}} 
\item The last column displays a stacked barplot for each method, based on the total number of selected series data from 500 simulation instances, with a darker sub-bar indicating a larger number.
\end{tablenotes}
\end{threeparttable}}
\endgroup{}
\end{table}

Focusing on the results of the benchmark reconciliation methods, we find
that the BU approach performs the best in both Scenario II and III but
ranks as the worst overall in Scenario I. This is not surprising, as
bottom-level base forecasts are deteriorated in Scenario I, while
higher-level base forecasts are deteriorated in Scenario II and III.
Moreover, the WLSv, MinT, and MinTs approaches perform especially well
in this simulation design, benefiting from their ability to consider the
in-sample covariance of base forecast errors, allowing for larger range
of adjustments in reconciliation for base forecasts with higher
estimated error variance. EMinT also provides accurate reconciled
forecasts in our setup, where the in-sample forecasts for specific
series are intentionally undermined, a situation that can be detected by
the in-sample information based EMinT method. However, OLS and WLSs
perform much worse than other benchmark methods in this simulation
design.

In all three scenarios, our proposed methods consistently produce either
improved or comparable reconciled forecasts compared to their respective
benchmark methods. The improvements are particularly pronounced when
using OLS and WLSs estimators of \(\boldsymbol{W}\), which do not take
into account the in-sample covariance of base forecast errors. One
advantage of using the forecast reconciliation methods with selection
proposed in this paper is that they can reduce the difference introduced
by using different estimates of \(\boldsymbol{W}\), thereby mitigating
the risk of estimator selection. In some cases, such as Scenarios II and
III, we can align the forecast accuracy achieved using different
estimators, and make them close to the best results we can obtain. When
we drop the unbiasedness assumption, Elasso delivers results on par with
EMinT overall, while achieving improvements at the top level, which is
typically the aspect of greatest concern to practitioners.

In addition, we report the proportion of time series being selected from
the implementation of our proposed methods in 500 simulation instances,
as shown in Table~\ref{tbl-s1-selection}, Table~\ref{tbl-s2-selection},
and Table~\ref{tbl-s3-selection} for each respective scenario. Clearly,
our proposed methods select fewer time series from the hierarchy for
forecast reconciliation, and generally improve forecast accuracy over
the benchmark methods. Furthermore, we observe that the Subset methods
tend to return fewer time series compared to the Intuitive and Lasso
methods, which aligns with our expectations. As discussed in
Section~\ref{sec-methodology}, the Intuitive and Lasso methods tend to
produce dense estimates.

\hypertarget{sec-sim2}{%
\subsection{Setup 2: Exploring the effect of
correlation}\label{sec-sim2}}

We now consider to simulate a hierarchical structure with correlated
series. A similar simulation to \textcite{Wickramasuriya2021-am} is
implemented in this section. Using the same hierarchical structure as
shown in Figure~\ref{fig-hts}, we assume the data generating process for
the time series at the bottom level follows a stationary first-order
vector autoregressive model, i.e., \(\text{VAR}(1)\), given by \[
\boldsymbol{b}_t= \boldsymbol{c} + \left[\begin{array}{cc}
\boldsymbol{A}_1 & \boldsymbol{0} \\
\boldsymbol{0} & \boldsymbol{A}_2
\end{array}\right] \boldsymbol{b}_{t-1} + \boldsymbol{\varepsilon}_t,
\]

where \(\boldsymbol{c}\) is a constant vector with all entries set to
\(1\), \(\boldsymbol{A}_1\) and \(\boldsymbol{A}_2\) are \(2 \times 2\)
matrices with eigenvalues
\(z_{1,2}=0.6[\cos (\pi / 3) \pm i \sin (\pi / 3)]\) and
\(z_{3,4}=0.9[\cos (\pi / 6) \pm i \sin (\pi / 6)]\), respectively, and
\(\boldsymbol{\varepsilon}_t \sim \mathcal{N}(\boldsymbol{0}, \boldsymbol{\Sigma})\),
where \[
\boldsymbol{\Sigma}=\left[\begin{array}{cc}
\boldsymbol{\Sigma}_1 & 0 \\0 & \boldsymbol{\Sigma}_2
\end{array}\right], \text { and } \boldsymbol{\Sigma}_1=\boldsymbol{\Sigma}_2=\left[\begin{array}{cc}2 & \sqrt{6} \rho \\\sqrt{6} \rho & 3\end{array}\right],
\]

and \(\rho \in \{0, \pm 0.2, \pm 0.4, \pm 0.6, \pm 0.8\}\) controls the
error correlation in the simulated hierarchy.

For each time series at the bottom level, we generate a total of \(101\)
observations, with the last one observation serving as the test set,
i.e., \(T=100\) and \(h=1\). Once again, the data at the higher levels
are obtained by aggreating the bottom-level series. The process is
repeated \(500\) times for each candidate correlation, \(\rho\).

For each series in the hierarchy, base forecasts are generated from ARMA
models based on a training data comprising \(100\) observations.
Specifically, we identify the best ARMA model with the minimum AICc
(corrected Akaike information criterion) value for each series by using
the automated algorithm implemented in the \textbf{forecast} R package.
Additionally, when fitting ARMA models for time series Total, A, and BA,
we introduce a slight bias by omitting the constant term, which is a
common scenario in practice. Figure~\ref{fig-corr-data} presents an
illustrative example of a hierarchical time series simulated. The left
panels depict time plots for each series at different levels of the
structure, while right panels show the residuals obtained from
forecasting each series using the fitted ARMA model. Notably, despite
our omission of the constant term when fitting ARMA models to series
Total, A, and BA, the residuals derived from the identified optimal
models still exhibit fluctuations around zero and do not display
significant deviations in comparison to the residuals from other series.
This is because the influence of the constant term is minimal, i.e., it
is much smaller compared to the data variability. Thus, it may be
challenging to identify the ``poor'' base forecasts and exclude them
from reconciliation in this setup.

\begin{figure}

{\centering \includegraphics{hf_selection_files/figure-pdf/fig-corr-data-1.pdf}

}

\caption{\label{fig-corr-data}An example hierarchical time series and
its in-sample residuals in Setup 2.}

\end{figure}

Table~\ref{tbl-corr-rmse} summarizes the average RMSE of the base
forecasts across various error correlations and the percentage relative
improvements in RMSE achieved by reconciliation methods relative to the
base forecasts. The results show that, for OLS, WLSs, WLSv estimators,
our proposed methods consistently dominate their respective benchmark
methods at all levels when the error correlation is \(-0.8\). In
general, as the error correlation ranges from \(-0.8\) to \(0.8\), the
overall improvements in our methods over the benchmark methods show a
slight decrease. Similar pattern is observed in the overall improvements
of all benchmark reconciliation methods compared to the base forecasts.
Of the methods we propose, the Subset methods display more consistently
stable improvements overall. We should highlight the challenge of
identifying the ``poor'' base forecasts in this simulation design, given
that the omission of the constant term has minimal impact relative to
the data variability. In addition, we observe that the MinT and MinTs
methods perform especially well and our methods provide results same
with benchmark methods. This is attributed to the use of in-sample
covariance by MinT and MinTs, which allows for large adjustments in
reconciliation for base forecasts with high estimated error variance.
Elasso forecasts are slightly worse than EMinT, possibly due to the
difficulty of identifying underperforming base forecasts in this
simulation setup.

We have also considered alternative error correlation values,
\(\rho = -0.6, -0.2, 0.2, 0.4\), for this simulation setting, but to
save space, we do not present all results. The omitted results follow a
similar pattern and are available upon request.

\hypertarget{tbl-corr-rmse}{}
\begin{table}[!h]
\caption{\label{tbl-corr-rmse}Out-of-sample forecast results across various error correlations for
simulation in Setup 2. }\tabularnewline

\centering
\resizebox{\linewidth}{!}{
\begin{threeparttable}
\begin{tabular}{lrrrrlrrrrlrrrrlrrrrl}
\toprule
\multicolumn{1}{c}{} & \multicolumn{5}{c}{Top} & \multicolumn{5}{c}{Middle} & \multicolumn{5}{c}{Bottom} & \multicolumn{5}{c}{Average} \\
\cmidrule(l{3pt}r{3pt}){2-6} \cmidrule(l{3pt}r{3pt}){7-11} \cmidrule(l{3pt}r{3pt}){12-16} \cmidrule(l{3pt}r{3pt}){17-21}
Method & $\rho$=-0.8 & -0.4 & 0 & 0.4 & 0.8 & $\rho$=-0.8 & -0.4 & 0 & 0.4 & 0.8 & $\rho$=-0.8 & -0.4 & 0 & 0.4 & 0.8 & $\rho$=-0.8 & -0.4 & 0 & 0.4 & 0.8\\
\midrule
Base & 2.4 & 2.9 & 3.4 & 4.1 & 4.0 & 1.5 & 1.8 & 2.1 & 2.4 & 2.5 & 1.5 & 1.5 & 1.5 & 1.5 & 1.4 & 1.6 & 1.8 & 2.0 & 2.1 & 2.1\\
BU & -17.0 & -9.0 & -6.7 & -7.0 & -7.4 & -6.8 & 0.4 & 4.8 & 5.7 & 2.8 & 0.0 & 0.0 & 0.0 & 0.0 & 0.0 & -5.3 & -1.9 & -0.2 & -0.1 & -1.0\\
\midrule
OLS & -11.0 & -8.2 & -7.7 & -8.2 & -8.0 & -3.5 & -0.7 & 3.1 & 2.5 & 0.8 & 0.7 & -0.6 & -2.0 & -2.3 & -2.1 & -2.8 & -2.4 & -1.8 & -2.4 & -2.7\\
\cellcolor[HTML]{e6e3e3}{OLS-subset} & \cellcolor[HTML]{e6e3e3}{\textbf{-11.4}} & \cellcolor[HTML]{e6e3e3}{\textbf{ -8.4}} & \cellcolor[HTML]{e6e3e3}{\textbf{ -8.1}} & \cellcolor[HTML]{e6e3e3}{\textbf{ -8.4}} & \cellcolor[HTML]{e6e3e3}{\textbf{ -8.8}} & \cellcolor[HTML]{e6e3e3}{\textbf{ -3.7}} & \cellcolor[HTML]{e6e3e3}{-0.7} & \cellcolor[HTML]{e6e3e3}{3.2} & \cellcolor[HTML]{e6e3e3}{2.5} & \cellcolor[HTML]{e6e3e3}{\textbf{ 0.4}} & \cellcolor[HTML]{e6e3e3}{\textbf{ 0.3}} & \cellcolor[HTML]{e6e3e3}{\textbf{-0.8}} & \cellcolor[HTML]{e6e3e3}{-2.0} & \cellcolor[HTML]{e6e3e3}{-1.7} & \cellcolor[HTML]{e6e3e3}{\textbf{-2.6}} & \cellcolor[HTML]{e6e3e3}{\textbf{ -3.2}} & \cellcolor[HTML]{e6e3e3}{\textbf{ -2.5}} & \cellcolor[HTML]{e6e3e3}{\textbf{-1.9}} & \cellcolor[HTML]{e6e3e3}{-2.2} & \cellcolor[HTML]{e6e3e3}{\textbf{-3.2}}\\
\cellcolor[HTML]{e6e3e3}{OLS-intuitive} & \cellcolor[HTML]{e6e3e3}{\textbf{-11.6}} & \cellcolor[HTML]{e6e3e3}{-8.0} & \cellcolor[HTML]{e6e3e3}{\textbf{ -7.8}} & \cellcolor[HTML]{e6e3e3}{-8.0} & \cellcolor[HTML]{e6e3e3}{\textbf{ -8.4}} & \cellcolor[HTML]{e6e3e3}{\textbf{ -3.6}} & \cellcolor[HTML]{e6e3e3}{-0.4} & \cellcolor[HTML]{e6e3e3}{3.7} & \cellcolor[HTML]{e6e3e3}{2.5} & \cellcolor[HTML]{e6e3e3}{\textbf{ 0.3}} & \cellcolor[HTML]{e6e3e3}{\textbf{ 0.6}} & \cellcolor[HTML]{e6e3e3}{-0.2} & \cellcolor[HTML]{e6e3e3}{-1.3} & \cellcolor[HTML]{e6e3e3}{-0.4} & \cellcolor[HTML]{e6e3e3}{-1.5} & \cellcolor[HTML]{e6e3e3}{\textbf{ -3.0}} & \cellcolor[HTML]{e6e3e3}{-2.0} & \cellcolor[HTML]{e6e3e3}{-1.3} & \cellcolor[HTML]{e6e3e3}{-1.6} & \cellcolor[HTML]{e6e3e3}{\textbf{-2.8}}\\
\cellcolor[HTML]{e6e3e3}{OLS-lasso} & \cellcolor[HTML]{e6e3e3}{\textbf{-19.2}} & \cellcolor[HTML]{e6e3e3}{\textbf{ -9.8}} & \cellcolor[HTML]{e6e3e3}{-7.2} & \cellcolor[HTML]{e6e3e3}{\textbf{ -8.7}} & \cellcolor[HTML]{e6e3e3}{\textbf{ -8.2}} & \cellcolor[HTML]{e6e3e3}{\textbf{-10.5}} & \cellcolor[HTML]{e6e3e3}{\textbf{ -1.7}} & \cellcolor[HTML]{e6e3e3}{\textbf{ 2.9}} & \cellcolor[HTML]{e6e3e3}{\textbf{ 2.4}} & \cellcolor[HTML]{e6e3e3}{0.8} & \cellcolor[HTML]{e6e3e3}{\textbf{-0.8}} & \cellcolor[HTML]{e6e3e3}{\textbf{-0.8}} & \cellcolor[HTML]{e6e3e3}{-1.6} & \cellcolor[HTML]{e6e3e3}{-2.3} & \cellcolor[HTML]{e6e3e3}{-2.1} & \cellcolor[HTML]{e6e3e3}{\textbf{ -7.1}} & \cellcolor[HTML]{e6e3e3}{\textbf{ -3.1}} & \cellcolor[HTML]{e6e3e3}{-1.6} & \cellcolor[HTML]{e6e3e3}{\textbf{-2.5}} & \cellcolor[HTML]{e6e3e3}{\textbf{-2.8}}\\
\midrule
WLSs & -16.8 & -11.1 & -9.6 & -10.4 & -10.2 & -8.1 & -2.8 & 1.5 & 1.2 & -0.4 & -0.3 & -1.1 & -2.4 & -2.9 & -2.9 & -5.7 & -3.9 & -3.0 & -3.6 & -4.0\\
\cellcolor[HTML]{e6e3e3}{WLSs-subset} & \cellcolor[HTML]{e6e3e3}{\textbf{-17.3}} & \cellcolor[HTML]{e6e3e3}{\textbf{-11.4}} & \cellcolor[HTML]{e6e3e3}{\textbf{ -9.9}} & \cellcolor[HTML]{e6e3e3}{\textbf{-11.1}} & \cellcolor[HTML]{e6e3e3}{\textbf{-10.8}} & \cellcolor[HTML]{e6e3e3}{\textbf{ -8.3}} & \cellcolor[HTML]{e6e3e3}{-2.8} & \cellcolor[HTML]{e6e3e3}{\textbf{ 1.4}} & \cellcolor[HTML]{e6e3e3}{\textbf{ 0.7}} & \cellcolor[HTML]{e6e3e3}{\textbf{-0.9}} & \cellcolor[HTML]{e6e3e3}{\textbf{-0.7}} & \cellcolor[HTML]{e6e3e3}{\textbf{-1.3}} & \cellcolor[HTML]{e6e3e3}{-2.4} & \cellcolor[HTML]{e6e3e3}{\textbf{-3.2}} & \cellcolor[HTML]{e6e3e3}{\textbf{-3.3}} & \cellcolor[HTML]{e6e3e3}{\textbf{ -6.1}} & \cellcolor[HTML]{e6e3e3}{\textbf{ -4.0}} & \cellcolor[HTML]{e6e3e3}{\textbf{-3.1}} & \cellcolor[HTML]{e6e3e3}{\textbf{-4.1}} & \cellcolor[HTML]{e6e3e3}{\textbf{-4.5}}\\
\cellcolor[HTML]{e6e3e3}{WLSs-intuitive} & \cellcolor[HTML]{e6e3e3}{\textbf{-16.9}} & \cellcolor[HTML]{e6e3e3}{\textbf{-11.5}} & \cellcolor[HTML]{e6e3e3}{\textbf{ -9.8}} & \cellcolor[HTML]{e6e3e3}{-10.0} & \cellcolor[HTML]{e6e3e3}{\textbf{-10.6}} & \cellcolor[HTML]{e6e3e3}{\textbf{ -8.5}} & \cellcolor[HTML]{e6e3e3}{-2.8} & \cellcolor[HTML]{e6e3e3}{\textbf{ 1.4}} & \cellcolor[HTML]{e6e3e3}{1.5} & \cellcolor[HTML]{e6e3e3}{\textbf{-0.7}} & \cellcolor[HTML]{e6e3e3}{\textbf{-0.7}} & \cellcolor[HTML]{e6e3e3}{\textbf{-1.2}} & \cellcolor[HTML]{e6e3e3}{-2.3} & \cellcolor[HTML]{e6e3e3}{-2.7} & \cellcolor[HTML]{e6e3e3}{\textbf{-3.0}} & \cellcolor[HTML]{e6e3e3}{\textbf{ -6.1}} & \cellcolor[HTML]{e6e3e3}{\textbf{ -4.0}} & \cellcolor[HTML]{e6e3e3}{-3.0} & \cellcolor[HTML]{e6e3e3}{-3.3} & \cellcolor[HTML]{e6e3e3}{\textbf{-4.3}}\\
\cellcolor[HTML]{e6e3e3}{WLSs-lasso} & \cellcolor[HTML]{e6e3e3}{\textbf{-18.3}} & \cellcolor[HTML]{e6e3e3}{-11.1} & \cellcolor[HTML]{e6e3e3}{-9.2} & \cellcolor[HTML]{e6e3e3}{\textbf{-10.5}} & \cellcolor[HTML]{e6e3e3}{-9.8} & \cellcolor[HTML]{e6e3e3}{\textbf{ -9.3}} & \cellcolor[HTML]{e6e3e3}{-2.4} & \cellcolor[HTML]{e6e3e3}{\textbf{ 1.4}} & \cellcolor[HTML]{e6e3e3}{1.2} & \cellcolor[HTML]{e6e3e3}{-0.1} & \cellcolor[HTML]{e6e3e3}{\textbf{-0.8}} & \cellcolor[HTML]{e6e3e3}{-1.0} & \cellcolor[HTML]{e6e3e3}{-2.4} & \cellcolor[HTML]{e6e3e3}{-2.9} & \cellcolor[HTML]{e6e3e3}{-2.8} & \cellcolor[HTML]{e6e3e3}{\textbf{ -6.6}} & \cellcolor[HTML]{e6e3e3}{-3.7} & \cellcolor[HTML]{e6e3e3}{-2.9} & \cellcolor[HTML]{e6e3e3}{\textbf{-3.7}} & \cellcolor[HTML]{e6e3e3}{-3.7}\\
\midrule
WLSv & -16.5 & -11.9 & -10.0 & -10.6 & -10.6 & -7.6 & -3.4 & 0.9 & 1.1 & -0.5 & -0.5 & -1.2 & -2.3 & -2.9 & -3.0 & -5.7 & -4.3 & -3.2 & -3.7 & -4.2\\
\cellcolor[HTML]{e6e3e3}{WLSv-subset} & \cellcolor[HTML]{e6e3e3}{\textbf{-16.8}} & \cellcolor[HTML]{e6e3e3}{\textbf{-12.1}} & \cellcolor[HTML]{e6e3e3}{-9.8} & \cellcolor[HTML]{e6e3e3}{\textbf{-10.8}} & \cellcolor[HTML]{e6e3e3}{\textbf{-10.7}} & \cellcolor[HTML]{e6e3e3}{\textbf{ -7.8}} & \cellcolor[HTML]{e6e3e3}{\textbf{ -3.5}} & \cellcolor[HTML]{e6e3e3}{1.1} & \cellcolor[HTML]{e6e3e3}{1.2} & \cellcolor[HTML]{e6e3e3}{\textbf{-1.0}} & \cellcolor[HTML]{e6e3e3}{\textbf{-1.1}} & \cellcolor[HTML]{e6e3e3}{\textbf{-1.3}} & \cellcolor[HTML]{e6e3e3}{-2.2} & \cellcolor[HTML]{e6e3e3}{-2.9} & \cellcolor[HTML]{e6e3e3}{\textbf{-3.2}} & \cellcolor[HTML]{e6e3e3}{\textbf{ -6.1}} & \cellcolor[HTML]{e6e3e3}{\textbf{ -4.4}} & \cellcolor[HTML]{e6e3e3}{-3.0} & \cellcolor[HTML]{e6e3e3}{-3.7} & \cellcolor[HTML]{e6e3e3}{\textbf{-4.4}}\\
\cellcolor[HTML]{e6e3e3}{WLSv-intuitive} & \cellcolor[HTML]{e6e3e3}{\textbf{-17.6}} & \cellcolor[HTML]{e6e3e3}{\textbf{-12.6}} & \cellcolor[HTML]{e6e3e3}{\textbf{-10.1}} & \cellcolor[HTML]{e6e3e3}{-10.5} & \cellcolor[HTML]{e6e3e3}{-10.6} & \cellcolor[HTML]{e6e3e3}{\textbf{ -8.7}} & \cellcolor[HTML]{e6e3e3}{\textbf{ -3.8}} & \cellcolor[HTML]{e6e3e3}{\textbf{ 0.7}} & \cellcolor[HTML]{e6e3e3}{1.1} & \cellcolor[HTML]{e6e3e3}{\textbf{-0.8}} & \cellcolor[HTML]{e6e3e3}{\textbf{-1.9}} & \cellcolor[HTML]{e6e3e3}{\textbf{-1.5}} & \cellcolor[HTML]{e6e3e3}{-2.3} & \cellcolor[HTML]{e6e3e3}{\textbf{-3.0}} & \cellcolor[HTML]{e6e3e3}{-3.0} & \cellcolor[HTML]{e6e3e3}{\textbf{ -7.0}} & \cellcolor[HTML]{e6e3e3}{\textbf{ -4.7}} & \cellcolor[HTML]{e6e3e3}{\textbf{-3.3}} & \cellcolor[HTML]{e6e3e3}{-3.7} & \cellcolor[HTML]{e6e3e3}{\textbf{-4.3}}\\
\cellcolor[HTML]{e6e3e3}{WLSv-lasso} & \cellcolor[HTML]{e6e3e3}{\textbf{-19.8}} & \cellcolor[HTML]{e6e3e3}{-11.6} & \cellcolor[HTML]{e6e3e3}{-9.7} & \cellcolor[HTML]{e6e3e3}{-10.5} & \cellcolor[HTML]{e6e3e3}{-10.6} & \cellcolor[HTML]{e6e3e3}{\textbf{-10.5}} & \cellcolor[HTML]{e6e3e3}{-3.0} & \cellcolor[HTML]{e6e3e3}{1.2} & \cellcolor[HTML]{e6e3e3}{1.2} & \cellcolor[HTML]{e6e3e3}{-0.5} & \cellcolor[HTML]{e6e3e3}{\textbf{-1.2}} & \cellcolor[HTML]{e6e3e3}{-1.1} & \cellcolor[HTML]{e6e3e3}{-2.2} & \cellcolor[HTML]{e6e3e3}{-2.9} & \cellcolor[HTML]{e6e3e3}{-3.0} & \cellcolor[HTML]{e6e3e3}{\textbf{ -7.5}} & \cellcolor[HTML]{e6e3e3}{-4.1} & \cellcolor[HTML]{e6e3e3}{-3.0} & \cellcolor[HTML]{e6e3e3}{-3.7} & \cellcolor[HTML]{e6e3e3}{-4.2}\\
\midrule
MinT & -25.4 & -18.8 & -12.4 & \textcolor{blue}{\textbf{-15.3}} & \textcolor{blue}{\textbf{-12.6}} & -15.5 & -7.0 & 0.0 & -2.0 & -2.0 & -4.0 & -4.6 & -4.3 & -5.8 & -5.1 & -11.4 & -8.5 & -5.0 & -7.2 & -6.0\\
\cellcolor[HTML]{e6e3e3}{MinT-subset} & \cellcolor[HTML]{e6e3e3}{-25.4} & \cellcolor[HTML]{e6e3e3}{-18.8} & \cellcolor[HTML]{e6e3e3}{-12.4} & \cellcolor[HTML]{e6e3e3}{\textcolor{blue}{\textbf{-15.3}}} & \cellcolor[HTML]{e6e3e3}{\textcolor{blue}{\textbf{-12.6}}} & \cellcolor[HTML]{e6e3e3}{-15.5} & \cellcolor[HTML]{e6e3e3}{-7.0} & \cellcolor[HTML]{e6e3e3}{0.0} & \cellcolor[HTML]{e6e3e3}{-2.0} & \cellcolor[HTML]{e6e3e3}{-2.0} & \cellcolor[HTML]{e6e3e3}{-4.0} & \cellcolor[HTML]{e6e3e3}{-4.6} & \cellcolor[HTML]{e6e3e3}{-4.3} & \cellcolor[HTML]{e6e3e3}{-5.8} & \cellcolor[HTML]{e6e3e3}{-5.1} & \cellcolor[HTML]{e6e3e3}{-11.4} & \cellcolor[HTML]{e6e3e3}{-8.5} & \cellcolor[HTML]{e6e3e3}{-5.0} & \cellcolor[HTML]{e6e3e3}{-7.2} & \cellcolor[HTML]{e6e3e3}{-6.0}\\
\cellcolor[HTML]{e6e3e3}{MinT-intuitive} & \cellcolor[HTML]{e6e3e3}{-25.4} & \cellcolor[HTML]{e6e3e3}{-18.8} & \cellcolor[HTML]{e6e3e3}{-12.4} & \cellcolor[HTML]{e6e3e3}{\textcolor{blue}{\textbf{-15.3}}} & \cellcolor[HTML]{e6e3e3}{\textcolor{blue}{\textbf{-12.6}}} & \cellcolor[HTML]{e6e3e3}{-15.5} & \cellcolor[HTML]{e6e3e3}{-7.0} & \cellcolor[HTML]{e6e3e3}{0.0} & \cellcolor[HTML]{e6e3e3}{-2.0} & \cellcolor[HTML]{e6e3e3}{-2.0} & \cellcolor[HTML]{e6e3e3}{-4.0} & \cellcolor[HTML]{e6e3e3}{-4.6} & \cellcolor[HTML]{e6e3e3}{-4.3} & \cellcolor[HTML]{e6e3e3}{-5.8} & \cellcolor[HTML]{e6e3e3}{-5.1} & \cellcolor[HTML]{e6e3e3}{-11.4} & \cellcolor[HTML]{e6e3e3}{-8.5} & \cellcolor[HTML]{e6e3e3}{-5.0} & \cellcolor[HTML]{e6e3e3}{-7.2} & \cellcolor[HTML]{e6e3e3}{-6.0}\\
\cellcolor[HTML]{e6e3e3}{MinT-lasso} & \cellcolor[HTML]{e6e3e3}{-25.4} & \cellcolor[HTML]{e6e3e3}{-18.8} & \cellcolor[HTML]{e6e3e3}{-12.4} & \cellcolor[HTML]{e6e3e3}{\textcolor{blue}{\textbf{-15.3}}} & \cellcolor[HTML]{e6e3e3}{\textcolor{blue}{\textbf{-12.6}}} & \cellcolor[HTML]{e6e3e3}{-15.5} & \cellcolor[HTML]{e6e3e3}{-7.0} & \cellcolor[HTML]{e6e3e3}{0.0} & \cellcolor[HTML]{e6e3e3}{-2.0} & \cellcolor[HTML]{e6e3e3}{-2.0} & \cellcolor[HTML]{e6e3e3}{-4.0} & \cellcolor[HTML]{e6e3e3}{-4.6} & \cellcolor[HTML]{e6e3e3}{-4.3} & \cellcolor[HTML]{e6e3e3}{-5.8} & \cellcolor[HTML]{e6e3e3}{-5.1} & \cellcolor[HTML]{e6e3e3}{-11.4} & \cellcolor[HTML]{e6e3e3}{-8.5} & \cellcolor[HTML]{e6e3e3}{-5.0} & \cellcolor[HTML]{e6e3e3}{-7.2} & \cellcolor[HTML]{e6e3e3}{-6.0}\\
\midrule
MinTs & -25.4 & -17.7 & -12.1 & -14.2 & -12.5 & -16.1 & -6.8 & -0.8 & -1.6 & \textcolor{blue}{\textbf{-2.4}} & -4.0 & -4.6 & -4.9 & -5.9 & \textcolor{blue}{\textbf{-5.2}} & -11.6 & -8.2 & -5.4 & -6.8 & \textcolor{blue}{\textbf{-6.2}}\\
\cellcolor[HTML]{e6e3e3}{MinTs-subset} & \cellcolor[HTML]{e6e3e3}{-25.2} & \cellcolor[HTML]{e6e3e3}{-17.6} & \cellcolor[HTML]{e6e3e3}{-12.1} & \cellcolor[HTML]{e6e3e3}{-14.2} & \cellcolor[HTML]{e6e3e3}{-12.5} & \cellcolor[HTML]{e6e3e3}{-16.1} & \cellcolor[HTML]{e6e3e3}{-6.8} & \cellcolor[HTML]{e6e3e3}{-0.8} & \cellcolor[HTML]{e6e3e3}{-1.6} & \cellcolor[HTML]{e6e3e3}{\textcolor{blue}{\textbf{-2.4}}} & \cellcolor[HTML]{e6e3e3}{-3.9} & \cellcolor[HTML]{e6e3e3}{-4.6} & \cellcolor[HTML]{e6e3e3}{-4.9} & \cellcolor[HTML]{e6e3e3}{-5.9} & \cellcolor[HTML]{e6e3e3}{\textcolor{blue}{\textbf{-5.2}}} & \cellcolor[HTML]{e6e3e3}{-11.5} & \cellcolor[HTML]{e6e3e3}{-8.2} & \cellcolor[HTML]{e6e3e3}{-5.4} & \cellcolor[HTML]{e6e3e3}{-6.8} & \cellcolor[HTML]{e6e3e3}{\textcolor{blue}{\textbf{-6.2}}}\\
\cellcolor[HTML]{e6e3e3}{MinTs-intuitive} & \cellcolor[HTML]{e6e3e3}{-25.4} & \cellcolor[HTML]{e6e3e3}{-17.7} & \cellcolor[HTML]{e6e3e3}{-12.1} & \cellcolor[HTML]{e6e3e3}{-14.2} & \cellcolor[HTML]{e6e3e3}{-12.5} & \cellcolor[HTML]{e6e3e3}{-16.1} & \cellcolor[HTML]{e6e3e3}{-6.8} & \cellcolor[HTML]{e6e3e3}{-0.8} & \cellcolor[HTML]{e6e3e3}{-1.6} & \cellcolor[HTML]{e6e3e3}{\textcolor{blue}{\textbf{-2.4}}} & \cellcolor[HTML]{e6e3e3}{-4.0} & \cellcolor[HTML]{e6e3e3}{-4.6} & \cellcolor[HTML]{e6e3e3}{-4.9} & \cellcolor[HTML]{e6e3e3}{-5.9} & \cellcolor[HTML]{e6e3e3}{\textcolor{blue}{\textbf{-5.2}}} & \cellcolor[HTML]{e6e3e3}{-11.6} & \cellcolor[HTML]{e6e3e3}{-8.2} & \cellcolor[HTML]{e6e3e3}{-5.4} & \cellcolor[HTML]{e6e3e3}{-6.8} & \cellcolor[HTML]{e6e3e3}{\textcolor{blue}{\textbf{-6.2}}}\\
\cellcolor[HTML]{e6e3e3}{MinTs-lasso} & \cellcolor[HTML]{e6e3e3}{-25.4} & \cellcolor[HTML]{e6e3e3}{-17.6} & \cellcolor[HTML]{e6e3e3}{-12.1} & \cellcolor[HTML]{e6e3e3}{-14.2} & \cellcolor[HTML]{e6e3e3}{-12.5} & \cellcolor[HTML]{e6e3e3}{-16.1} & \cellcolor[HTML]{e6e3e3}{-6.7} & \cellcolor[HTML]{e6e3e3}{-0.8} & \cellcolor[HTML]{e6e3e3}{-1.6} & \cellcolor[HTML]{e6e3e3}{\textcolor{blue}{\textbf{-2.4}}} & \cellcolor[HTML]{e6e3e3}{-4.0} & \cellcolor[HTML]{e6e3e3}{-4.6} & \cellcolor[HTML]{e6e3e3}{-4.9} & \cellcolor[HTML]{e6e3e3}{-5.9} & \cellcolor[HTML]{e6e3e3}{\textcolor{blue}{\textbf{-5.2}}} & \cellcolor[HTML]{e6e3e3}{-11.6} & \cellcolor[HTML]{e6e3e3}{-8.2} & \cellcolor[HTML]{e6e3e3}{-5.4} & \cellcolor[HTML]{e6e3e3}{-6.8} & \cellcolor[HTML]{e6e3e3}{\textcolor{blue}{\textbf{-6.2}}}\\
\midrule
EMinT & \textcolor{blue}{\textbf{-31.2}} & \textcolor{blue}{\textbf{-19.8}} & \textcolor{blue}{\textbf{-12.5}} & -14.1 & -11.1 & \textcolor{blue}{\textbf{-22.9}} & \textcolor{blue}{\textbf{-10.9}} & \textcolor{blue}{\textbf{-2.4}} & \textcolor{blue}{\textbf{-3.2}} & -1.0 & \textcolor{blue}{\textbf{-7.4}} & \textcolor{blue}{\textbf{-7.3}} & \textcolor{blue}{\textbf{-6.9}} & \textcolor{blue}{\textbf{-7.5}} & -5.1 & \textcolor{blue}{\textbf{-16.4}} & \textcolor{blue}{\textbf{-11.2}} & \textcolor{blue}{\textbf{-6.9}} & \textcolor{blue}{\textbf{-7.9}} & -5.3\\
\cellcolor[HTML]{e6e3e3}{Elasso} & \cellcolor[HTML]{e6e3e3}{-31.0} & \cellcolor[HTML]{e6e3e3}{-19.1} & \cellcolor[HTML]{e6e3e3}{-11.1} & \cellcolor[HTML]{e6e3e3}{-13.6} & \cellcolor[HTML]{e6e3e3}{\textbf{-11.2}} & \cellcolor[HTML]{e6e3e3}{-22.7} & \cellcolor[HTML]{e6e3e3}{-9.7} & \cellcolor[HTML]{e6e3e3}{-1.8} & \cellcolor[HTML]{e6e3e3}{-2.4} & \cellcolor[HTML]{e6e3e3}{\textbf{-1.7}} & \cellcolor[HTML]{e6e3e3}{\textcolor{blue}{\textbf{-7.4}}} & \cellcolor[HTML]{e6e3e3}{-7.2} & \cellcolor[HTML]{e6e3e3}{-6.1} & \cellcolor[HTML]{e6e3e3}{-5.7} & \cellcolor[HTML]{e6e3e3}{-3.5} & \cellcolor[HTML]{e6e3e3}{-16.3} & \cellcolor[HTML]{e6e3e3}{-10.6} & \cellcolor[HTML]{e6e3e3}{-6.0} & \cellcolor[HTML]{e6e3e3}{-6.8} & \cellcolor[HTML]{e6e3e3}{-4.9}\\
\bottomrule
\end{tabular}
\begin{tablenotes}[para]
\item \underline{\textit{NOTE:}} 
\item The Base row shows the average RMSE of the base forecasts. Entries below this row indicate the percentage decrease (negative) or increase (positive) in the average RMSE of the reconciled forecasts compared to the base forecasts. The entries with the lowest values in each column are highlighted in blue. In each panel, the proposed methods are indicated with a gray background, and methods that outperform the benchmark method are marked in bold.
\end{tablenotes}
\end{threeparttable}}
\end{table}

We present the proportion of time series being selected by applying our
proposed methods in \(500\) simulation instances for error correlation
coefficients of \(-0.8\) and \(0.8\) in
Table~\ref{tbl-corr-selection-neg} and
Table~\ref{tbl-corr-selection-pos}, respectively. Once again, we note
that it is difficult to exclude poor-performing base forecasts in this
simulation design as the constant term omitted is very small compared to
the data variability. As observed in Table~\ref{tbl-corr-selection-neg},
for OLS, WLSs, and WLS estimators, the Subset and Intuitive methods are
still able to exclude the series Total, A, and BA in some instances, in
which small biases are introduced in model fitting, while essentially
retaining the rest series in the hierarchy. The Subset methods perform
superior to the Intuitive method in selection.The Lasso methods
typically select all bottom-level series since they tend to yield dense
estimates. Elasso also select all bottom-level series. When dealing with
a high positive error correlation, Table~\ref{tbl-corr-selection-pos}
shows that our methods still have the potential to do some selection but
it becomes somewhat challenging to identify and exclude the series that
should be omitted in reconciliation. Hence, our methods are preferred,
particularly when the error correlation within the hierarchical
structure is negative.

\hypertarget{tbl-corr-selection-neg}{}
\begin{table}[!h]
\caption{\label{tbl-corr-selection-neg}Proportion of time series being selected after using the proposed
reconciliation methods with selection in Setup 2, with the error
correlation being -0.8. }\tabularnewline

\centering\begingroup\fontsize{7}{9}\selectfont

\resizebox{\linewidth}{!}{
\begin{threeparttable}
\begin{tabular}{llrrrrrr>{}r}
\toprule
  & Top & A & B & AA & AB & BA & BB & Summary\\
\midrule
OLS-subset & 0.32 & 0.34 & 0.95 & 0.98 & 1 & 0.74 & 1.00 & \includegraphics[width=0.47in, height=0.1in]{/Users/xwan0362/Git/hfs/paper/_figs/corr_neg_OLS-subset.png}\\
OLS-intuitive & 0.58 & 0.52 & 0.93 & 0.97 & 1 & 0.61 & 0.97 & \includegraphics[width=0.47in, height=0.1in]{/Users/xwan0362/Git/hfs/paper/_figs/corr_neg_OLS-intuitive.png}\\
OLS-lasso & 0.61 & 0.34 & 0.38 & 1.00 & 1 & 1.00 & 1.00 & \includegraphics[width=0.47in, height=0.1in]{/Users/xwan0362/Git/hfs/paper/_figs/corr_neg_OLS-lasso.png}\\
\midrule
WLSs-subset & 0.27 & 0.40 & 0.98 & 1.00 & 1 & 0.73 & 1.00 & \includegraphics[width=0.47in, height=0.1in]{/Users/xwan0362/Git/hfs/paper/_figs/corr_neg_WLSs-subset.png}\\
WLSs-intuitive & 0.49 & 0.57 & 0.96 & 1.00 & 1 & 0.74 & 0.99 & \includegraphics[width=0.47in, height=0.1in]{/Users/xwan0362/Git/hfs/paper/_figs/corr_neg_WLSs-intuitive.png}\\
WLSs-lasso & 0.48 & 0.62 & 0.72 & 1.00 & 1 & 1.00 & 1.00 & \includegraphics[width=0.47in, height=0.1in]{/Users/xwan0362/Git/hfs/paper/_figs/corr_neg_WLSs-lasso.png}\\
\midrule
WLSv-subset & 0.30 & 0.42 & 1.00 & 1.00 & 1 & 0.68 & 1.00 & \includegraphics[width=0.47in, height=0.1in]{/Users/xwan0362/Git/hfs/paper/_figs/corr_neg_WLSv-subset.png}\\
WLSv-intuitive & 0.49 & 0.53 & 0.99 & 1.00 & 1 & 0.47 & 1.00 & \includegraphics[width=0.47in, height=0.1in]{/Users/xwan0362/Git/hfs/paper/_figs/corr_neg_WLSv-intuitive.png}\\
WLSv-lasso & 0.35 & 0.70 & 0.85 & 1.00 & 1 & 1.00 & 1.00 & \includegraphics[width=0.47in, height=0.1in]{/Users/xwan0362/Git/hfs/paper/_figs/corr_neg_WLSv-lasso.png}\\
\midrule
MinT-subset & 1.00 & 1.00 & 1.00 & 1.00 & 1 & 1.00 & 1.00 & \includegraphics[width=0.47in, height=0.1in]{/Users/xwan0362/Git/hfs/paper/_figs/corr_neg_MinT-subset.png}\\
MinT-intuitive & 1.00 & 1.00 & 1.00 & 1.00 & 1 & 1.00 & 1.00 & \includegraphics[width=0.47in, height=0.1in]{/Users/xwan0362/Git/hfs/paper/_figs/corr_neg_MinT-intuitive.png}\\
MinT-lasso & 1.00 & 1.00 & 1.00 & 1.00 & 1 & 1.00 & 1.00 & \includegraphics[width=0.47in, height=0.1in]{/Users/xwan0362/Git/hfs/paper/_figs/corr_neg_MinT-lasso.png}\\
\midrule
MinTs-subset & 0.87 & 0.85 & 1.00 & 1.00 & 1 & 0.85 & 1.00 & \includegraphics[width=0.47in, height=0.1in]{/Users/xwan0362/Git/hfs/paper/_figs/corr_neg_MinTs-subset.png}\\
MinTs-intuitive & 1.00 & 1.00 & 1.00 & 1.00 & 1 & 1.00 & 1.00 & \includegraphics[width=0.47in, height=0.1in]{/Users/xwan0362/Git/hfs/paper/_figs/corr_neg_MinTs-intuitive.png}\\
MinTs-lasso & 0.86 & 0.84 & 1.00 & 1.00 & 1 & 0.85 & 1.00 & \includegraphics[width=0.47in, height=0.1in]{/Users/xwan0362/Git/hfs/paper/_figs/corr_neg_MinTs-lasso.png}\\
\midrule
Elasso & 0.94 & 0.79 & 0.93 & 1.00 & 1 & 1.00 & 1.00 & \includegraphics[width=0.47in, height=0.1in]{/Users/xwan0362/Git/hfs/paper/_figs/corr_neg_Elasso.png}\\
\bottomrule
\end{tabular}
\begin{tablenotes}[para]
\item \underline{\textit{NOTE:}} 
\item The last column displays a stacked barplot for each method, based on the total number of selected series data from 500 simulation instances, with a darker sub-bar indicating a larger number.
\end{tablenotes}
\end{threeparttable}}
\endgroup{}
\end{table}

\hypertarget{sec-applications}{%
\section{Applications}\label{sec-applications}}

In this section we perform two empirical applications to investigate the
performance of our proposed methods and compare them with
state-of-the-art reconciliation approaches. Section~\ref{sec-labour}
focuses on a grouped hierarchy built using the Australian labour force
survey data released by the Australian Bureau of Statistics, while
Section~\ref{sec-tourism} considers Australian domestic tourism flows
with a natural geographical hierarchy.

\hypertarget{sec-labour}{%
\subsection{Forecasting Australian labour force}\label{sec-labour}}

This section evaluates the performance of the proposed methods using a
grouped hierarchy built using the Australian labour force dataset. The
dataset from the Labour Force Survey are released by the Australian
Bureau of Statistics, consisting of monthly data on the number of
unemployed persons in Australia for the period from January 2010 to July
2023\footnote{The Labour Force Survey data is publicly available at
  \url{https://www.abs.gov.au/statistics/labour/employment-and-unemployment/labour-force-australia-detailed/aug-2023}.}.
There are a few missing values in the dataset. To deal with the missing
observations, we use a random walk to give linear interpolation between
points. Analysis of unemployment data in a country by labor market
region and duration of job search can provide valuable insights into
regional disparities, and the structural nuances underlying
unemployment. Forecast reconciliation is crucial in such a case to
ensure aligned decision making.

We construct a grouped hierarchy by disaggregating the number of
unemployed persons over two independent attributes, duration of job
search (referred to as \emph{Duration}), and State and Territory
(referred to as \emph{STT} ). The two attributes are crossed, but none
are nested within the others. At the bottom level, the data are
disaggregated by both attributes. We refer to the bottom level as the
\emph{Duration} \(\times\) \emph{STT} level. Specifically, there are six
different groups of job search duration, under 1 month, 1-3 months, 3-6
months, 6-12 months, 1-2 years, and 2 years and over. Additionally, the
number of unemployed persons in Australia can be disaggregated by eight
states and territories, i.e., NSW (New South Wales), VIC (Victoria), QLD
(Queensland), SA (South Australia), WA (Western Australia), TAS
(Tasmania), NT (Northern Territory), and ACT (Australian Capital
Territory). So the final grouped hierarchy consists of the top series,
six series at the Duration level, eight series at the STT level, and
\(48\) series at the Duration \(\times\) STT level, giving \(63\) time
series in total, each of length \(163\) observations.

The top panel in Figure~\ref{fig-labour-data} shows the total number of
unemployed persons in Australia from January 2010 to July 2023,
representing the top-level series in the grouped hierarchical structure.
The monthly series shows strong seasonality within each year, marked by
prominent peaks occurring every January, possibly attributable to people
waiting to start new jobs. In addition, lower peaks occur in July,
impacted by the timing of school holidays. Amidst the backdrop of
COVID-19's non-essential service shutdowns and trading restrictions,
March and April of 2020 saw a notable surge in unemployment. However, as
coronavirus cases dwindled significantly and restrictions eased in the
aftermath, employment made a remarkable recovery, leading to a
subsequent decline in unemployment. The bottom-left panel displays the
breakdown of unemployed individuals by state and territory, while the
bottom-right panel presents the breakdown by the duration of job search.
The plots display diverse and rich dynamics both within and between
different levels of the hierarchy. For example, there was noticeable
growth observed during 2020 for some states such as NSW, VIC, and QLD,
whereas other states did not experience such significant growth.
Additionally, there is a resemblance in the seasonal patterns between
NSW and QLD, while the seasonal pattern in VIC appears relatively
different. When comparing the series at the STT level and Duration
level, we notice that the seasonal patterns in the Duration-level series
is more consistent and potentially easier to forecast.

\begin{figure}

{\centering \includegraphics{hf_selection_files/figure-pdf/fig-labour-data-1.pdf}

}

\caption{\label{fig-labour-data}Australia unemployed persons,
disaggregated by state and territory, and by duration of job search.}

\end{figure}

We assess the forecast accuracy of base forecasts and various
reconciliation methods through a rolling forecast origin approach. Our
aim is to generate \(1\)- to \(12\)-steps-ahead forecasts for each of
the \(63\) series while ensuring coherence. Given the limited data
compared to the forecast horizon, we initiate the process with a
training set of \(139\) observations for each series. The training set
is used to select the optimal ETS model with the automatic algorithm
implemented in the \textbf{forecast} package for R. Using these fitted
ETS models, we generate base forecasts, and then perform diverse
forecast reconciliation methods. Following this, we roll the forecast
origin forward by one month and repeat the process until July 2022. We
note that it may be challenging to identify the series with ``poor''
forecasts due to structural changes in the data caused by the COVID-19
pandemic, which affect the accuracy of forecasts across all time series.

The average results are presented in Table~\ref{tbl-labour-rmse-avg}.
The Subset methods using different estimators of \(\boldsymbol{G}\)
generally improve forecast accuracy over their respective benchmark
methods overall, particularly when focusing on aggregation levels, which
are typically of paramount concern to practitioners. The only one
exception is the WLSs-subset method, which returns reduced accuracy for
longer horizons overall. However, it still demonstrates improvements in
top-level forecasts, and other levels remains within a reasonable range.
Moreover, the Intuitive and Lasso methods almost always yield results
identical to the corresponding benchmark methods. This is because they
tend to provide dense estimates, and ETS models typically do not result
in extremely poor forecasts. The only exception is OLS-intuitive, which
shows improved forecast accuracy at the top level but deterioration at
other levels. When we drop the unbiasedness assumption, EMinT is the
worst performing method across all levels because it relies on the
assumption that the series in the hierarchy are jointly weakly
stationary, which is evidently not the case in the application. Elasso
significantly improves the quality of forecasts over EMinT, with the
most accurate coherent forecasts observed at the top level and STT
level. Overall, Elasso performs well for longer forecast horizons, but
it is less effective for one-step-ahead forecasts.

\hypertarget{tbl-labour-rmse-avg}{}
\begin{table}[!h]
\caption{\label{tbl-labour-rmse-avg}Average out-of-sample forecast results for Australian labour force data. }\tabularnewline

\centering
\resizebox{\linewidth}{!}{
\begin{threeparttable}
\begin{tabular}{lrrrrrlrrrrrlrrrrrlrr}
\toprule
\multicolumn{1}{c}{} & \multicolumn{4}{c}{Top} & \multicolumn{4}{c}{Duration} & \multicolumn{4}{c}{STT} & \multicolumn{4}{c}{Duration x STT} & \multicolumn{4}{c}{Average} \\
\cmidrule(l{3pt}r{3pt}){2-5} \cmidrule(l{3pt}r{3pt}){6-9} \cmidrule(l{3pt}r{3pt}){10-13} \cmidrule(l{3pt}r{3pt}){14-17} \cmidrule(l{3pt}r{3pt}){18-21}
Method & h=1 & 1-4 & 1-8 & 1-12 & h=1 & 1-4 & 1-8 & 1-12 & h=1 & 1-4 & 1-8 & 1-12 & h=1 & 1-4 & 1-8 & 1-12 & h=1 & 1-4 & 1-8 & 1-12\\
\midrule
Base & 29.4 & 44.9 & 58.6 & 67.6 & 10.1 & 14.2 & 16.3 & 18.1 & 6.6 & 8.4 & 9.9 & 10.7 & 2.3 & 2.9 & 3.1 & 3.3 & 4.0 & 5.3 & 6.1 & 6.6\\
BU & 46.7 & 34.1 & 29.3 & 24.2 & 7.4 & 2.3 & 0.8 & 0.8 & 5.1 & 9.8 & 10.5 & 10.4 & 0.0 & 0.0 & 0.0 & 0.0 & 8.4 & 7.1 & 6.8 & 6.3\\
\midrule
OLS & 2.0 & 1.7 & 1.5 & 1.0 & 0.6 & -4.2 & -4.3 & -3.3 & -0.7 & 0.4 & 0.0 & -0.1 & 1.9 & 0.7 & 0.8 & 0.7 & 1.0 & -0.5 & -0.6 & -0.5\\
\cellcolor[HTML]{e6e3e3}{OLS-subset} & \cellcolor[HTML]{e6e3e3}{2.1} & \cellcolor[HTML]{e6e3e3}{\textbf{ 1.0}} & \cellcolor[HTML]{e6e3e3}{\textbf{-1.2}} & \cellcolor[HTML]{e6e3e3}{\textbf{-2.0}} & \cellcolor[HTML]{e6e3e3}{0.6} & \cellcolor[HTML]{e6e3e3}{-4.1} & \cellcolor[HTML]{e6e3e3}{\textbf{-5.2}} & \cellcolor[HTML]{e6e3e3}{\textbf{-4.4}} & \cellcolor[HTML]{e6e3e3}{\textbf{ -1.0}} & \cellcolor[HTML]{e6e3e3}{0.6} & \cellcolor[HTML]{e6e3e3}{\textbf{ -0.6}} & \cellcolor[HTML]{e6e3e3}{\textbf{-1.2}} & \cellcolor[HTML]{e6e3e3}{1.9} & \cellcolor[HTML]{e6e3e3}{0.8} & \cellcolor[HTML]{e6e3e3}{\textbf{ 0.3}} & \cellcolor[HTML]{e6e3e3}{\textbf{ 0.2}} & \cellcolor[HTML]{e6e3e3}{1.0} & \cellcolor[HTML]{e6e3e3}{-0.5} & \cellcolor[HTML]{e6e3e3}{\textcolor{blue}{\textbf{-1.5}}} & \cellcolor[HTML]{e6e3e3}{\textbf{-1.6}}\\
\cellcolor[HTML]{e6e3e3}{OLS-intuitive} & \cellcolor[HTML]{e6e3e3}{\textbf{-1.3}} & \cellcolor[HTML]{e6e3e3}{\textbf{ 1.5}} & \cellcolor[HTML]{e6e3e3}{\textbf{ 1.0}} & \cellcolor[HTML]{e6e3e3}{\textbf{ 0.2}} & \cellcolor[HTML]{e6e3e3}{\textbf{-0.9}} & \cellcolor[HTML]{e6e3e3}{-3.9} & \cellcolor[HTML]{e6e3e3}{-4.2} & \cellcolor[HTML]{e6e3e3}{-3.2} & \cellcolor[HTML]{e6e3e3}{\textbf{ -1.3}} & \cellcolor[HTML]{e6e3e3}{0.8} & \cellcolor[HTML]{e6e3e3}{0.5} & \cellcolor[HTML]{e6e3e3}{0.6} & \cellcolor[HTML]{e6e3e3}{\textbf{ 1.8}} & \cellcolor[HTML]{e6e3e3}{1.2} & \cellcolor[HTML]{e6e3e3}{1.3} & \cellcolor[HTML]{e6e3e3}{1.2} & \cellcolor[HTML]{e6e3e3}{\textcolor{blue}{\textbf{ 0.1}}} & \cellcolor[HTML]{e6e3e3}{-0.1} & \cellcolor[HTML]{e6e3e3}{-0.3} & \cellcolor[HTML]{e6e3e3}{-0.2}\\
\cellcolor[HTML]{e6e3e3}{OLS-lasso} & \cellcolor[HTML]{e6e3e3}{2.0} & \cellcolor[HTML]{e6e3e3}{1.7} & \cellcolor[HTML]{e6e3e3}{1.5} & \cellcolor[HTML]{e6e3e3}{1.0} & \cellcolor[HTML]{e6e3e3}{0.6} & \cellcolor[HTML]{e6e3e3}{-4.2} & \cellcolor[HTML]{e6e3e3}{-4.3} & \cellcolor[HTML]{e6e3e3}{-3.3} & \cellcolor[HTML]{e6e3e3}{-0.7} & \cellcolor[HTML]{e6e3e3}{0.4} & \cellcolor[HTML]{e6e3e3}{0.0} & \cellcolor[HTML]{e6e3e3}{-0.1} & \cellcolor[HTML]{e6e3e3}{1.9} & \cellcolor[HTML]{e6e3e3}{0.7} & \cellcolor[HTML]{e6e3e3}{0.8} & \cellcolor[HTML]{e6e3e3}{0.7} & \cellcolor[HTML]{e6e3e3}{1.0} & \cellcolor[HTML]{e6e3e3}{-0.5} & \cellcolor[HTML]{e6e3e3}{-0.6} & \cellcolor[HTML]{e6e3e3}{-0.5}\\
\midrule
WLSs & 16.9 & 10.8 & 9.0 & 7.0 & -0.7 & -4.8 & -5.0 & -4.3 & -3.0 & 0.6 & 1.4 & 1.6 & -1.3 & \textcolor{blue}{\textbf{-1.7}} & \textcolor{blue}{\textbf{-1.5}} & \textcolor{blue}{\textbf{-1.6}} & 0.6 & -0.3 & -0.2 & -0.3\\
\cellcolor[HTML]{e6e3e3}{WLSs-subset} & \cellcolor[HTML]{e6e3e3}{\textbf{14.9}} & \cellcolor[HTML]{e6e3e3}{\textbf{ 9.7}} & \cellcolor[HTML]{e6e3e3}{\textbf{ 6.4}} & \cellcolor[HTML]{e6e3e3}{\textbf{ 4.2}} & \cellcolor[HTML]{e6e3e3}{\textcolor{blue}{\textbf{-1.6}}} & \cellcolor[HTML]{e6e3e3}{-3.7} & \cellcolor[HTML]{e6e3e3}{-4.1} & \cellcolor[HTML]{e6e3e3}{-3.4} & \cellcolor[HTML]{e6e3e3}{-2.3} & \cellcolor[HTML]{e6e3e3}{1.5} & \cellcolor[HTML]{e6e3e3}{1.6} & \cellcolor[HTML]{e6e3e3}{\textbf{ 1.0}} & \cellcolor[HTML]{e6e3e3}{-0.6} & \cellcolor[HTML]{e6e3e3}{-0.2} & \cellcolor[HTML]{e6e3e3}{0.5} & \cellcolor[HTML]{e6e3e3}{0.3} & \cellcolor[HTML]{e6e3e3}{0.6} & \cellcolor[HTML]{e6e3e3}{0.6} & \cellcolor[HTML]{e6e3e3}{0.4} & \cellcolor[HTML]{e6e3e3}{0.1}\\
\cellcolor[HTML]{e6e3e3}{WLSs-intuitive} & \cellcolor[HTML]{e6e3e3}{16.9} & \cellcolor[HTML]{e6e3e3}{10.8} & \cellcolor[HTML]{e6e3e3}{9.0} & \cellcolor[HTML]{e6e3e3}{7.0} & \cellcolor[HTML]{e6e3e3}{-0.7} & \cellcolor[HTML]{e6e3e3}{-4.8} & \cellcolor[HTML]{e6e3e3}{-5.0} & \cellcolor[HTML]{e6e3e3}{-4.3} & \cellcolor[HTML]{e6e3e3}{-3.0} & \cellcolor[HTML]{e6e3e3}{0.6} & \cellcolor[HTML]{e6e3e3}{1.4} & \cellcolor[HTML]{e6e3e3}{1.6} & \cellcolor[HTML]{e6e3e3}{-1.3} & \cellcolor[HTML]{e6e3e3}{\textcolor{blue}{\textbf{-1.7}}} & \cellcolor[HTML]{e6e3e3}{\textcolor{blue}{\textbf{-1.5}}} & \cellcolor[HTML]{e6e3e3}{\textcolor{blue}{\textbf{-1.6}}} & \cellcolor[HTML]{e6e3e3}{0.6} & \cellcolor[HTML]{e6e3e3}{-0.3} & \cellcolor[HTML]{e6e3e3}{-0.2} & \cellcolor[HTML]{e6e3e3}{-0.3}\\
\cellcolor[HTML]{e6e3e3}{WLSs-lasso} & \cellcolor[HTML]{e6e3e3}{16.9} & \cellcolor[HTML]{e6e3e3}{10.8} & \cellcolor[HTML]{e6e3e3}{9.0} & \cellcolor[HTML]{e6e3e3}{7.0} & \cellcolor[HTML]{e6e3e3}{-0.7} & \cellcolor[HTML]{e6e3e3}{-4.8} & \cellcolor[HTML]{e6e3e3}{-5.0} & \cellcolor[HTML]{e6e3e3}{-4.3} & \cellcolor[HTML]{e6e3e3}{-3.0} & \cellcolor[HTML]{e6e3e3}{0.6} & \cellcolor[HTML]{e6e3e3}{1.4} & \cellcolor[HTML]{e6e3e3}{1.6} & \cellcolor[HTML]{e6e3e3}{-1.3} & \cellcolor[HTML]{e6e3e3}{\textcolor{blue}{\textbf{-1.7}}} & \cellcolor[HTML]{e6e3e3}{\textcolor{blue}{\textbf{-1.5}}} & \cellcolor[HTML]{e6e3e3}{\textcolor{blue}{\textbf{-1.6}}} & \cellcolor[HTML]{e6e3e3}{0.6} & \cellcolor[HTML]{e6e3e3}{-0.3} & \cellcolor[HTML]{e6e3e3}{-0.2} & \cellcolor[HTML]{e6e3e3}{-0.3}\\
\midrule
WLSv & 15.6 & 9.8 & 8.4 & 6.6 & 1.1 & -5.0 & \textcolor{blue}{\textbf{-5.6}} & -4.5 & -2.6 & -0.5 & -0.1 & 0.2 & \textcolor{blue}{\textbf{-1.5}} & -1.5 & \textcolor{blue}{\textbf{-1.5}} & -1.3 & 0.9 & -0.7 & -0.8 & -0.6\\
\cellcolor[HTML]{e6e3e3}{WLSv-subset} & \cellcolor[HTML]{e6e3e3}{\textbf{10.2}} & \cellcolor[HTML]{e6e3e3}{\textbf{ 5.1}} & \cellcolor[HTML]{e6e3e3}{\textbf{ 2.9}} & \cellcolor[HTML]{e6e3e3}{\textbf{ 1.8}} & \cellcolor[HTML]{e6e3e3}{\textbf{-0.6}} & \cellcolor[HTML]{e6e3e3}{\textcolor{blue}{\textbf{-5.4}}} & \cellcolor[HTML]{e6e3e3}{-5.5} & \cellcolor[HTML]{e6e3e3}{\textcolor{blue}{\textbf{-4.6}}} & \cellcolor[HTML]{e6e3e3}{-1.8} & \cellcolor[HTML]{e6e3e3}{\textbf{-0.8}} & \cellcolor[HTML]{e6e3e3}{\textbf{ -1.1}} & \cellcolor[HTML]{e6e3e3}{\textbf{-1.0}} & \cellcolor[HTML]{e6e3e3}{-1.1} & \cellcolor[HTML]{e6e3e3}{-1.1} & \cellcolor[HTML]{e6e3e3}{-0.7} & \cellcolor[HTML]{e6e3e3}{-0.5} & \cellcolor[HTML]{e6e3e3}{\textbf{ 0.2}} & \cellcolor[HTML]{e6e3e3}{\textcolor{blue}{\textbf{-1.3}}} & \cellcolor[HTML]{e6e3e3}{\textcolor{blue}{\textbf{-1.5}}} & \cellcolor[HTML]{e6e3e3}{\textbf{-1.3}}\\
\cellcolor[HTML]{e6e3e3}{WLSv-intuitive} & \cellcolor[HTML]{e6e3e3}{15.6} & \cellcolor[HTML]{e6e3e3}{9.8} & \cellcolor[HTML]{e6e3e3}{8.4} & \cellcolor[HTML]{e6e3e3}{6.6} & \cellcolor[HTML]{e6e3e3}{1.1} & \cellcolor[HTML]{e6e3e3}{-5.0} & \cellcolor[HTML]{e6e3e3}{\textcolor{blue}{\textbf{-5.6}}} & \cellcolor[HTML]{e6e3e3}{-4.5} & \cellcolor[HTML]{e6e3e3}{-2.6} & \cellcolor[HTML]{e6e3e3}{-0.5} & \cellcolor[HTML]{e6e3e3}{-0.1} & \cellcolor[HTML]{e6e3e3}{0.2} & \cellcolor[HTML]{e6e3e3}{\textcolor{blue}{\textbf{-1.5}}} & \cellcolor[HTML]{e6e3e3}{-1.5} & \cellcolor[HTML]{e6e3e3}{\textcolor{blue}{\textbf{-1.5}}} & \cellcolor[HTML]{e6e3e3}{-1.3} & \cellcolor[HTML]{e6e3e3}{0.9} & \cellcolor[HTML]{e6e3e3}{-0.7} & \cellcolor[HTML]{e6e3e3}{-0.8} & \cellcolor[HTML]{e6e3e3}{-0.6}\\
\cellcolor[HTML]{e6e3e3}{WLSv-lasso} & \cellcolor[HTML]{e6e3e3}{15.6} & \cellcolor[HTML]{e6e3e3}{9.8} & \cellcolor[HTML]{e6e3e3}{8.4} & \cellcolor[HTML]{e6e3e3}{6.6} & \cellcolor[HTML]{e6e3e3}{1.1} & \cellcolor[HTML]{e6e3e3}{-5.0} & \cellcolor[HTML]{e6e3e3}{\textcolor{blue}{\textbf{-5.6}}} & \cellcolor[HTML]{e6e3e3}{-4.5} & \cellcolor[HTML]{e6e3e3}{-2.6} & \cellcolor[HTML]{e6e3e3}{-0.5} & \cellcolor[HTML]{e6e3e3}{-0.1} & \cellcolor[HTML]{e6e3e3}{0.2} & \cellcolor[HTML]{e6e3e3}{\textcolor{blue}{\textbf{-1.5}}} & \cellcolor[HTML]{e6e3e3}{-1.5} & \cellcolor[HTML]{e6e3e3}{\textcolor{blue}{\textbf{-1.5}}} & \cellcolor[HTML]{e6e3e3}{-1.3} & \cellcolor[HTML]{e6e3e3}{0.9} & \cellcolor[HTML]{e6e3e3}{-0.7} & \cellcolor[HTML]{e6e3e3}{-0.8} & \cellcolor[HTML]{e6e3e3}{-0.6}\\
\midrule
MinTs & 9.9 & 5.8 & 6.4 & 5.3 & 0.6 & -5.0 & -5.0 & -3.7 & -3.4 & -2.4 & -1.4 & -0.8 & -0.4 & -0.9 & -0.8 & -0.7 & 0.4 & \textcolor{blue}{\textbf{-1.3}} & -0.9 & -0.5\\
\cellcolor[HTML]{e6e3e3}{MinTs-subset} & \cellcolor[HTML]{e6e3e3}{\textbf{ 9.1}} & \cellcolor[HTML]{e6e3e3}{6.1} & \cellcolor[HTML]{e6e3e3}{6.5} & \cellcolor[HTML]{e6e3e3}{\textbf{ 4.5}} & \cellcolor[HTML]{e6e3e3}{\textbf{ 0.3}} & \cellcolor[HTML]{e6e3e3}{-4.8} & \cellcolor[HTML]{e6e3e3}{\textbf{-5.1}} & \cellcolor[HTML]{e6e3e3}{\textbf{-3.9}} & \cellcolor[HTML]{e6e3e3}{\textbf{ -3.6}} & \cellcolor[HTML]{e6e3e3}{-2.4} & \cellcolor[HTML]{e6e3e3}{-1.3} & \cellcolor[HTML]{e6e3e3}{\textbf{-0.9}} & \cellcolor[HTML]{e6e3e3}{\textbf{-0.6}} & \cellcolor[HTML]{e6e3e3}{-0.8} & \cellcolor[HTML]{e6e3e3}{-0.7} & \cellcolor[HTML]{e6e3e3}{-0.7} & \cellcolor[HTML]{e6e3e3}{\textcolor{blue}{\textbf{ 0.1}}} & \cellcolor[HTML]{e6e3e3}{-1.2} & \cellcolor[HTML]{e6e3e3}{-0.9} & \cellcolor[HTML]{e6e3e3}{\textbf{-0.7}}\\
\cellcolor[HTML]{e6e3e3}{MinTs-intuitive} & \cellcolor[HTML]{e6e3e3}{9.9} & \cellcolor[HTML]{e6e3e3}{5.8} & \cellcolor[HTML]{e6e3e3}{6.4} & \cellcolor[HTML]{e6e3e3}{5.3} & \cellcolor[HTML]{e6e3e3}{0.6} & \cellcolor[HTML]{e6e3e3}{-5.0} & \cellcolor[HTML]{e6e3e3}{-5.0} & \cellcolor[HTML]{e6e3e3}{-3.7} & \cellcolor[HTML]{e6e3e3}{-3.4} & \cellcolor[HTML]{e6e3e3}{-2.4} & \cellcolor[HTML]{e6e3e3}{-1.4} & \cellcolor[HTML]{e6e3e3}{-0.8} & \cellcolor[HTML]{e6e3e3}{-0.4} & \cellcolor[HTML]{e6e3e3}{-0.9} & \cellcolor[HTML]{e6e3e3}{-0.8} & \cellcolor[HTML]{e6e3e3}{-0.7} & \cellcolor[HTML]{e6e3e3}{0.4} & \cellcolor[HTML]{e6e3e3}{\textcolor{blue}{\textbf{-1.3}}} & \cellcolor[HTML]{e6e3e3}{-0.9} & \cellcolor[HTML]{e6e3e3}{-0.5}\\
\cellcolor[HTML]{e6e3e3}{MinTs-lasso} & \cellcolor[HTML]{e6e3e3}{9.9} & \cellcolor[HTML]{e6e3e3}{5.8} & \cellcolor[HTML]{e6e3e3}{6.4} & \cellcolor[HTML]{e6e3e3}{5.3} & \cellcolor[HTML]{e6e3e3}{0.6} & \cellcolor[HTML]{e6e3e3}{-5.0} & \cellcolor[HTML]{e6e3e3}{-5.0} & \cellcolor[HTML]{e6e3e3}{-3.7} & \cellcolor[HTML]{e6e3e3}{-3.4} & \cellcolor[HTML]{e6e3e3}{-2.4} & \cellcolor[HTML]{e6e3e3}{-1.4} & \cellcolor[HTML]{e6e3e3}{-0.8} & \cellcolor[HTML]{e6e3e3}{-0.4} & \cellcolor[HTML]{e6e3e3}{-0.9} & \cellcolor[HTML]{e6e3e3}{-0.8} & \cellcolor[HTML]{e6e3e3}{-0.7} & \cellcolor[HTML]{e6e3e3}{0.4} & \cellcolor[HTML]{e6e3e3}{\textcolor{blue}{\textbf{-1.3}}} & \cellcolor[HTML]{e6e3e3}{-0.9} & \cellcolor[HTML]{e6e3e3}{-0.5}\\
\midrule
EMinT & 43.1 & 17.6 & 9.9 & 10.2 & 36.8 & 25.2 & 27.9 & 24.1 & 16.8 & 15.1 & 6.2 & 6.2 & 32.3 & 27.9 & 29.8 & 27.9 & 31.4 & 23.3 & 21.4 & 19.6\\
\cellcolor[HTML]{e6e3e3}{Elasso} & \cellcolor[HTML]{e6e3e3}{\textcolor{blue}{\textbf{-5.8}}} & \cellcolor[HTML]{e6e3e3}{\textcolor{blue}{\textbf{-2.0}}} & \cellcolor[HTML]{e6e3e3}{\textcolor{blue}{\textbf{-2.5}}} & \cellcolor[HTML]{e6e3e3}{\textcolor{blue}{\textbf{-2.3}}} & \cellcolor[HTML]{e6e3e3}{\textbf{33.5}} & \cellcolor[HTML]{e6e3e3}{\textbf{11.1}} & \cellcolor[HTML]{e6e3e3}{\textbf{ 1.1}} & \cellcolor[HTML]{e6e3e3}{\textbf{-3.6}} & \cellcolor[HTML]{e6e3e3}{\textcolor{blue}{\textbf{-17.4}}} & \cellcolor[HTML]{e6e3e3}{\textcolor{blue}{\textbf{-8.9}}} & \cellcolor[HTML]{e6e3e3}{\textcolor{blue}{\textbf{-10.0}}} & \cellcolor[HTML]{e6e3e3}{\textcolor{blue}{\textbf{-8.8}}} & \cellcolor[HTML]{e6e3e3}{\textbf{20.6}} & \cellcolor[HTML]{e6e3e3}{\textbf{ 6.4}} & \cellcolor[HTML]{e6e3e3}{\textbf{ 2.6}} & \cellcolor[HTML]{e6e3e3}{\textbf{ 0.5}} & \cellcolor[HTML]{e6e3e3}{\textbf{12.7}} & \cellcolor[HTML]{e6e3e3}{\textbf{ 3.4}} & \cellcolor[HTML]{e6e3e3}{\textbf{-1.1}} & \cellcolor[HTML]{e6e3e3}{\textcolor{blue}{\textbf{-2.9}}}\\
\bottomrule
\end{tabular}
\begin{tablenotes}[para]
\item \underline{\textit{NOTE:}} 
\item The Base row shows the average RMSE of the base forecasts. Entries below this row indicate the percentage decrease (negative) or increase (positive) in the average RMSE of the reconciled forecasts compared to the base forecasts. The entries with the lowest values in each column are highlighted in blue. In each panel, the proposed methods are indicated with a gray background, and methods that outperform the benchmark method are marked in bold.
\end{tablenotes}
\end{threeparttable}}
\end{table}

We also provide the results based on the final test set spanning from
August 2022 to July 2023 in Table~\ref{tbl-labour-rmse}. The results
indicate that all Subset methods using different estimators of
\(\boldsymbol{G}\), i.e., OLS-subset, WLSs-subset, WLSv-subset, and
MinTs-subset, produce improved or comparable reconciled forecasts
compared to their respective benchmark methods. The improvements in
forecast accuracy become more noticeable for longer forecast horizons.
Similar to the average results in Table~\ref{tbl-labour-rmse-avg}, the
Intuitive and Lasso methods yield results identical to the benchmark
methods due to their tendency to offer dense estimates. Surprisingly,
when relaxing the unbiasedness constraint, the Elasso method ranks the
best and demonstrates significant improvement compared to the EMinT
method, and outperforms other methods across almost all levels except
for the top level.

\hypertarget{tbl-labour-rmse}{}
\begin{table}[!h]
\caption{\label{tbl-labour-rmse}Out-of-sample forecast results on a single test set (from August 2022 to
July 2023) for Australian labour force data. }\tabularnewline

\centering
\resizebox{\linewidth}{!}{
\begin{threeparttable}
\begin{tabular}{lrrrrrlrrrrrlrrrrrlrr}
\toprule
\multicolumn{1}{c}{} & \multicolumn{4}{c}{Top} & \multicolumn{4}{c}{Duration} & \multicolumn{4}{c}{STT} & \multicolumn{4}{c}{Duration x STT} & \multicolumn{4}{c}{Average} \\
\cmidrule(l{3pt}r{3pt}){2-5} \cmidrule(l{3pt}r{3pt}){6-9} \cmidrule(l{3pt}r{3pt}){10-13} \cmidrule(l{3pt}r{3pt}){14-17} \cmidrule(l{3pt}r{3pt}){18-21}
Method & h=1 & 1-4 & 1-8 & 1-12 & h=1 & 1-4 & 1-8 & 1-12 & h=1 & 1-4 & 1-8 & 1-12 & h=1 & 1-4 & 1-8 & 1-12 & h=1 & 1-4 & 1-8 & 1-12\\
\midrule
Base & 18.5 & 13.6 & 18.3 & 28.3 & 11.8 & 12.7 & 13.9 & 16.9 & 6.7 & 6.0 & 6.0 & 6.3 & 2.3 & 2.6 & 2.7 & 2.9 & 4.1 & 4.1 & 4.4 & 5.1\\
BU & \textcolor{blue}{\textbf{-81.5}} & 33.4 & -19.9 & -45.0 & \textcolor{blue}{\textbf{-30.7}} & -9.2 & -7.9 & -10.1 & -12.9 & -10.4 & -13.4 & -13.5 & 0.0 & 0.0 & 0.0 & 0.0 & \textcolor{blue}{\textbf{-17.1}} & -2.8 & -5.9 & -9.3\\
\midrule
OLS & -16.2 & -14.2 & -13.4 & -10.4 & 2.5 & -2.6 & -2.7 & -0.6 & -1.8 & -0.9 & -1.9 & 0.3 & 6.7 & 5.1 & 5.1 & 4.9 & 2.1 & 0.7 & 0.4 & 1.1\\
\cellcolor[HTML]{e6e3e3}{OLS-subset} & \cellcolor[HTML]{e6e3e3}{\textbf{-17.0}} & \cellcolor[HTML]{e6e3e3}{-2.1} & \cellcolor[HTML]{e6e3e3}{\textbf{-31.2}} & \cellcolor[HTML]{e6e3e3}{\textbf{-38.4}} & \cellcolor[HTML]{e6e3e3}{\textbf{  2.0}} & \cellcolor[HTML]{e6e3e3}{-1.7} & \cellcolor[HTML]{e6e3e3}{\textbf{ -5.0}} & \cellcolor[HTML]{e6e3e3}{\textbf{ -2.7}} & \cellcolor[HTML]{e6e3e3}{\textbf{ -2.7}} & \cellcolor[HTML]{e6e3e3}{\textbf{ -4.2}} & \cellcolor[HTML]{e6e3e3}{\textbf{ -8.6}} & \cellcolor[HTML]{e6e3e3}{\textbf{ -7.3}} & \cellcolor[HTML]{e6e3e3}{6.7} & \cellcolor[HTML]{e6e3e3}{5.2} & \cellcolor[HTML]{e6e3e3}{\textbf{ 3.3}} & \cellcolor[HTML]{e6e3e3}{\textbf{ 3.7}} & \cellcolor[HTML]{e6e3e3}{\textbf{  1.7}} & \cellcolor[HTML]{e6e3e3}{1.1} & \cellcolor[HTML]{e6e3e3}{\textbf{ -3.5}} & \cellcolor[HTML]{e6e3e3}{\textbf{ -3.8}}\\
\cellcolor[HTML]{e6e3e3}{OLS-intuitive} & \cellcolor[HTML]{e6e3e3}{\textbf{-79.6}} & \cellcolor[HTML]{e6e3e3}{\textbf{-23.9}} & \cellcolor[HTML]{e6e3e3}{\textbf{-31.9}} & \cellcolor[HTML]{e6e3e3}{\textbf{-32.1}} & \cellcolor[HTML]{e6e3e3}{\textbf{-13.0}} & \cellcolor[HTML]{e6e3e3}{0.4} & \cellcolor[HTML]{e6e3e3}{-0.8} & \cellcolor[HTML]{e6e3e3}{0.3} & \cellcolor[HTML]{e6e3e3}{\textbf{ -8.9}} & \cellcolor[HTML]{e6e3e3}{4.9} & \cellcolor[HTML]{e6e3e3}{7.0} & \cellcolor[HTML]{e6e3e3}{13.2} & \cellcolor[HTML]{e6e3e3}{\textbf{  6.3}} & \cellcolor[HTML]{e6e3e3}{12.3} & \cellcolor[HTML]{e6e3e3}{12.4} & \cellcolor[HTML]{e6e3e3}{11.6} & \cellcolor[HTML]{e6e3e3}{\textbf{ -8.5}} & \cellcolor[HTML]{e6e3e3}{5.6} & \cellcolor[HTML]{e6e3e3}{4.7} & \cellcolor[HTML]{e6e3e3}{4.4}\\
\cellcolor[HTML]{e6e3e3}{OLS-lasso} & \cellcolor[HTML]{e6e3e3}{-16.2} & \cellcolor[HTML]{e6e3e3}{-14.2} & \cellcolor[HTML]{e6e3e3}{-13.4} & \cellcolor[HTML]{e6e3e3}{-10.4} & \cellcolor[HTML]{e6e3e3}{2.5} & \cellcolor[HTML]{e6e3e3}{-2.6} & \cellcolor[HTML]{e6e3e3}{-2.7} & \cellcolor[HTML]{e6e3e3}{-0.6} & \cellcolor[HTML]{e6e3e3}{-1.8} & \cellcolor[HTML]{e6e3e3}{-0.9} & \cellcolor[HTML]{e6e3e3}{-1.9} & \cellcolor[HTML]{e6e3e3}{0.3} & \cellcolor[HTML]{e6e3e3}{6.7} & \cellcolor[HTML]{e6e3e3}{5.1} & \cellcolor[HTML]{e6e3e3}{5.1} & \cellcolor[HTML]{e6e3e3}{4.9} & \cellcolor[HTML]{e6e3e3}{2.1} & \cellcolor[HTML]{e6e3e3}{0.7} & \cellcolor[HTML]{e6e3e3}{0.4} & \cellcolor[HTML]{e6e3e3}{1.1}\\
\midrule
WLSs & -60.6 & -29.4 & -44.0 & -38.6 & -12.0 & -7.6 & -6.9 & -5.9 & -6.5 & -8.1 & -9.4 & -8.0 & 3.2 & 1.7 & 1.7 & 1.6 & -7.7 & -4.4 & -5.8 & -5.9\\
\cellcolor[HTML]{e6e3e3}{WLSs-subset} & \cellcolor[HTML]{e6e3e3}{\textbf{-61.6}} & \cellcolor[HTML]{e6e3e3}{-22.4} & \cellcolor[HTML]{e6e3e3}{\textcolor{blue}{\textbf{-47.3}}} & \cellcolor[HTML]{e6e3e3}{\textcolor{blue}{\textbf{-50.4}}} & \cellcolor[HTML]{e6e3e3}{-12.0} & \cellcolor[HTML]{e6e3e3}{\textbf{-8.0}} & \cellcolor[HTML]{e6e3e3}{\textbf{-10.5}} & \cellcolor[HTML]{e6e3e3}{\textbf{ -7.8}} & \cellcolor[HTML]{e6e3e3}{\textbf{ -6.6}} & \cellcolor[HTML]{e6e3e3}{\textbf{-10.7}} & \cellcolor[HTML]{e6e3e3}{\textbf{-14.3}} & \cellcolor[HTML]{e6e3e3}{\textbf{-12.8}} & \cellcolor[HTML]{e6e3e3}{3.2} & \cellcolor[HTML]{e6e3e3}{5.6} & \cellcolor[HTML]{e6e3e3}{4.1} & \cellcolor[HTML]{e6e3e3}{5.9} & \cellcolor[HTML]{e6e3e3}{\textbf{ -7.8}} & \cellcolor[HTML]{e6e3e3}{-2.8} & \cellcolor[HTML]{e6e3e3}{\textbf{ -6.7}} & \cellcolor[HTML]{e6e3e3}{\textbf{ -6.4}}\\
\cellcolor[HTML]{e6e3e3}{WLSs-intuitive} & \cellcolor[HTML]{e6e3e3}{-60.6} & \cellcolor[HTML]{e6e3e3}{-29.4} & \cellcolor[HTML]{e6e3e3}{-44.0} & \cellcolor[HTML]{e6e3e3}{-38.6} & \cellcolor[HTML]{e6e3e3}{-12.0} & \cellcolor[HTML]{e6e3e3}{-7.6} & \cellcolor[HTML]{e6e3e3}{-6.9} & \cellcolor[HTML]{e6e3e3}{-5.9} & \cellcolor[HTML]{e6e3e3}{-6.5} & \cellcolor[HTML]{e6e3e3}{-8.1} & \cellcolor[HTML]{e6e3e3}{-9.4} & \cellcolor[HTML]{e6e3e3}{-8.0} & \cellcolor[HTML]{e6e3e3}{3.2} & \cellcolor[HTML]{e6e3e3}{1.7} & \cellcolor[HTML]{e6e3e3}{1.7} & \cellcolor[HTML]{e6e3e3}{1.6} & \cellcolor[HTML]{e6e3e3}{-7.7} & \cellcolor[HTML]{e6e3e3}{-4.4} & \cellcolor[HTML]{e6e3e3}{-5.8} & \cellcolor[HTML]{e6e3e3}{-5.9}\\
\cellcolor[HTML]{e6e3e3}{WLSs-lasso} & \cellcolor[HTML]{e6e3e3}{-60.6} & \cellcolor[HTML]{e6e3e3}{-29.4} & \cellcolor[HTML]{e6e3e3}{-44.0} & \cellcolor[HTML]{e6e3e3}{-38.6} & \cellcolor[HTML]{e6e3e3}{-12.0} & \cellcolor[HTML]{e6e3e3}{-7.6} & \cellcolor[HTML]{e6e3e3}{-6.9} & \cellcolor[HTML]{e6e3e3}{-5.9} & \cellcolor[HTML]{e6e3e3}{-6.5} & \cellcolor[HTML]{e6e3e3}{-8.1} & \cellcolor[HTML]{e6e3e3}{-9.4} & \cellcolor[HTML]{e6e3e3}{-8.0} & \cellcolor[HTML]{e6e3e3}{3.2} & \cellcolor[HTML]{e6e3e3}{1.7} & \cellcolor[HTML]{e6e3e3}{1.7} & \cellcolor[HTML]{e6e3e3}{1.6} & \cellcolor[HTML]{e6e3e3}{-7.7} & \cellcolor[HTML]{e6e3e3}{-4.4} & \cellcolor[HTML]{e6e3e3}{-5.8} & \cellcolor[HTML]{e6e3e3}{-5.9}\\
\midrule
WLSv & -60.6 & -29.1 & -41.4 & -36.6 & -14.5 & -8.7 & -5.6 & -4.8 & -3.3 & -6.8 & -8.0 & -7.0 & 5.5 & 2.6 & 2.6 & 3.1 & -6.7 & -4.0 & -4.5 & -4.6\\
\cellcolor[HTML]{e6e3e3}{WLSv-subset} & \cellcolor[HTML]{e6e3e3}{-51.6} & \cellcolor[HTML]{e6e3e3}{\textcolor{blue}{\textbf{-32.7}}} & \cellcolor[HTML]{e6e3e3}{-36.6} & \cellcolor[HTML]{e6e3e3}{-29.6} & \cellcolor[HTML]{e6e3e3}{\textbf{-18.3}} & \cellcolor[HTML]{e6e3e3}{\textbf{-9.8}} & \cellcolor[HTML]{e6e3e3}{\textbf{-10.5}} & \cellcolor[HTML]{e6e3e3}{\textbf{-10.9}} & \cellcolor[HTML]{e6e3e3}{-1.1} & \cellcolor[HTML]{e6e3e3}{-4.3} & \cellcolor[HTML]{e6e3e3}{\textbf{ -8.1}} & \cellcolor[HTML]{e6e3e3}{\textbf{ -7.3}} & \cellcolor[HTML]{e6e3e3}{\textbf{  2.5}} & \cellcolor[HTML]{e6e3e3}{\textbf{ 2.1}} & \cellcolor[HTML]{e6e3e3}{\textbf{ 2.3}} & \cellcolor[HTML]{e6e3e3}{\textbf{ 1.8}} & \cellcolor[HTML]{e6e3e3}{\textbf{ -7.9}} & \cellcolor[HTML]{e6e3e3}{\textbf{-4.3}} & \cellcolor[HTML]{e6e3e3}{\textbf{ -5.8}} & \cellcolor[HTML]{e6e3e3}{\textbf{ -6.5}}\\
\cellcolor[HTML]{e6e3e3}{WLSv-intuitive} & \cellcolor[HTML]{e6e3e3}{-60.6} & \cellcolor[HTML]{e6e3e3}{-29.1} & \cellcolor[HTML]{e6e3e3}{-41.4} & \cellcolor[HTML]{e6e3e3}{-36.6} & \cellcolor[HTML]{e6e3e3}{-14.5} & \cellcolor[HTML]{e6e3e3}{-8.7} & \cellcolor[HTML]{e6e3e3}{-5.6} & \cellcolor[HTML]{e6e3e3}{-4.8} & \cellcolor[HTML]{e6e3e3}{-3.3} & \cellcolor[HTML]{e6e3e3}{-6.8} & \cellcolor[HTML]{e6e3e3}{-8.0} & \cellcolor[HTML]{e6e3e3}{-7.0} & \cellcolor[HTML]{e6e3e3}{5.5} & \cellcolor[HTML]{e6e3e3}{2.6} & \cellcolor[HTML]{e6e3e3}{2.6} & \cellcolor[HTML]{e6e3e3}{3.1} & \cellcolor[HTML]{e6e3e3}{-6.7} & \cellcolor[HTML]{e6e3e3}{-4.0} & \cellcolor[HTML]{e6e3e3}{-4.5} & \cellcolor[HTML]{e6e3e3}{-4.6}\\
\cellcolor[HTML]{e6e3e3}{WLSv-lasso} & \cellcolor[HTML]{e6e3e3}{-60.6} & \cellcolor[HTML]{e6e3e3}{-29.1} & \cellcolor[HTML]{e6e3e3}{-41.4} & \cellcolor[HTML]{e6e3e3}{-36.6} & \cellcolor[HTML]{e6e3e3}{-14.5} & \cellcolor[HTML]{e6e3e3}{-8.7} & \cellcolor[HTML]{e6e3e3}{-5.6} & \cellcolor[HTML]{e6e3e3}{-4.8} & \cellcolor[HTML]{e6e3e3}{-3.3} & \cellcolor[HTML]{e6e3e3}{-6.8} & \cellcolor[HTML]{e6e3e3}{-8.0} & \cellcolor[HTML]{e6e3e3}{-7.0} & \cellcolor[HTML]{e6e3e3}{5.5} & \cellcolor[HTML]{e6e3e3}{2.6} & \cellcolor[HTML]{e6e3e3}{2.6} & \cellcolor[HTML]{e6e3e3}{3.1} & \cellcolor[HTML]{e6e3e3}{-6.7} & \cellcolor[HTML]{e6e3e3}{-4.0} & \cellcolor[HTML]{e6e3e3}{-4.5} & \cellcolor[HTML]{e6e3e3}{-4.6}\\
\midrule
MinTs & -27.0 & -21.1 & -22.9 & -21.8 & -9.1 & -7.9 & -6.7 & -4.6 & -3.6 & -9.0 & -10.5 & -7.6 & 7.7 & 3.7 & 3.0 & 3.4 & -1.9 & -3.3 & -3.9 & -3.1\\
\cellcolor[HTML]{e6e3e3}{MinTs-subset} & \cellcolor[HTML]{e6e3e3}{\textbf{-41.4}} & \cellcolor[HTML]{e6e3e3}{-9.3} & \cellcolor[HTML]{e6e3e3}{-17.8} & \cellcolor[HTML]{e6e3e3}{\textbf{-45.1}} & \cellcolor[HTML]{e6e3e3}{\textbf{-12.2}} & \cellcolor[HTML]{e6e3e3}{-5.0} & \cellcolor[HTML]{e6e3e3}{\textbf{ -8.2}} & \cellcolor[HTML]{e6e3e3}{\textbf{ -6.8}} & \cellcolor[HTML]{e6e3e3}{\textbf{ -6.1}} & \cellcolor[HTML]{e6e3e3}{\textbf{ -9.8}} & \cellcolor[HTML]{e6e3e3}{-7.1} & \cellcolor[HTML]{e6e3e3}{\textbf{ -9.3}} & \cellcolor[HTML]{e6e3e3}{\textbf{  5.3}} & \cellcolor[HTML]{e6e3e3}{4.7} & \cellcolor[HTML]{e6e3e3}{3.8} & \cellcolor[HTML]{e6e3e3}{3.6} & \cellcolor[HTML]{e6e3e3}{\textbf{ -5.3}} & \cellcolor[HTML]{e6e3e3}{-1.5} & \cellcolor[HTML]{e6e3e3}{-3.0} & \cellcolor[HTML]{e6e3e3}{\textbf{ -6.1}}\\
\cellcolor[HTML]{e6e3e3}{MinTs-intuitive} & \cellcolor[HTML]{e6e3e3}{-27.0} & \cellcolor[HTML]{e6e3e3}{-21.1} & \cellcolor[HTML]{e6e3e3}{-22.9} & \cellcolor[HTML]{e6e3e3}{-21.8} & \cellcolor[HTML]{e6e3e3}{-9.1} & \cellcolor[HTML]{e6e3e3}{-7.9} & \cellcolor[HTML]{e6e3e3}{-6.7} & \cellcolor[HTML]{e6e3e3}{-4.6} & \cellcolor[HTML]{e6e3e3}{-3.6} & \cellcolor[HTML]{e6e3e3}{-9.0} & \cellcolor[HTML]{e6e3e3}{-10.5} & \cellcolor[HTML]{e6e3e3}{-7.6} & \cellcolor[HTML]{e6e3e3}{7.7} & \cellcolor[HTML]{e6e3e3}{3.7} & \cellcolor[HTML]{e6e3e3}{3.0} & \cellcolor[HTML]{e6e3e3}{3.4} & \cellcolor[HTML]{e6e3e3}{-1.9} & \cellcolor[HTML]{e6e3e3}{-3.3} & \cellcolor[HTML]{e6e3e3}{-3.9} & \cellcolor[HTML]{e6e3e3}{-3.1}\\
\cellcolor[HTML]{e6e3e3}{MinTs-lasso} & \cellcolor[HTML]{e6e3e3}{-27.0} & \cellcolor[HTML]{e6e3e3}{-21.1} & \cellcolor[HTML]{e6e3e3}{-22.9} & \cellcolor[HTML]{e6e3e3}{-21.8} & \cellcolor[HTML]{e6e3e3}{-9.1} & \cellcolor[HTML]{e6e3e3}{-7.9} & \cellcolor[HTML]{e6e3e3}{-6.7} & \cellcolor[HTML]{e6e3e3}{-4.6} & \cellcolor[HTML]{e6e3e3}{-3.6} & \cellcolor[HTML]{e6e3e3}{-9.0} & \cellcolor[HTML]{e6e3e3}{-10.5} & \cellcolor[HTML]{e6e3e3}{-7.6} & \cellcolor[HTML]{e6e3e3}{7.7} & \cellcolor[HTML]{e6e3e3}{3.7} & \cellcolor[HTML]{e6e3e3}{3.0} & \cellcolor[HTML]{e6e3e3}{3.4} & \cellcolor[HTML]{e6e3e3}{-1.9} & \cellcolor[HTML]{e6e3e3}{-3.3} & \cellcolor[HTML]{e6e3e3}{-3.9} & \cellcolor[HTML]{e6e3e3}{-3.1}\\
\midrule
EMinT & -60.4 & -14.0 & 1.4 & -29.9 & -6.0 & 12.0 & 10.7 & -6.7 & 16.7 & -0.9 & -12.4 & \textcolor{blue}{\textbf{-21.0}} & 23.3 & 17.2 & 16.7 & 10.1 & 7.7 & 10.8 & 9.0 & -3.7\\
\cellcolor[HTML]{e6e3e3}{Elasso} & \cellcolor[HTML]{e6e3e3}{-4.2} & \cellcolor[HTML]{e6e3e3}{-3.3} & \cellcolor[HTML]{e6e3e3}{\textbf{-22.3}} & \cellcolor[HTML]{e6e3e3}{-8.0} & \cellcolor[HTML]{e6e3e3}{\textbf{-19.7}} & \cellcolor[HTML]{e6e3e3}{\textcolor{blue}{\textbf{-9.9}}} & \cellcolor[HTML]{e6e3e3}{\textcolor{blue}{\textbf{-19.9}}} & \cellcolor[HTML]{e6e3e3}{\textcolor{blue}{\textbf{-25.3}}} & \cellcolor[HTML]{e6e3e3}{\textcolor{blue}{\textbf{-24.6}}} & \cellcolor[HTML]{e6e3e3}{\textcolor{blue}{\textbf{-24.3}}} & \cellcolor[HTML]{e6e3e3}{\textcolor{blue}{\textbf{-22.6}}} & \cellcolor[HTML]{e6e3e3}{-14.6} & \cellcolor[HTML]{e6e3e3}{\textcolor{blue}{\textbf{-10.8}}} & \cellcolor[HTML]{e6e3e3}{\textcolor{blue}{\textbf{-3.8}}} & \cellcolor[HTML]{e6e3e3}{\textcolor{blue}{\textbf{-0.2}}} & \cellcolor[HTML]{e6e3e3}{\textcolor{blue}{\textbf{-4.9}}} & \cellcolor[HTML]{e6e3e3}{\textbf{-15.7}} & \cellcolor[HTML]{e6e3e3}{\textcolor{blue}{\textbf{-9.3}}} & \cellcolor[HTML]{e6e3e3}{\textcolor{blue}{\textbf{-11.4}}} & \cellcolor[HTML]{e6e3e3}{\textcolor{blue}{\textbf{-13.2}}}\\
\bottomrule
\end{tabular}
\begin{tablenotes}[para]
\item \underline{\textit{NOTE:}} 
\item The Base row shows the average RMSE of the base forecasts. Entries below this row indicate the percentage decrease (negative) or increase (positive) in the average RMSE of the reconciled forecasts compared to the base forecasts. The entries with the lowest values in each column are highlighted in blue. In each panel, the proposed methods are indicated with a gray background, and methods that outperform the benchmark method are marked in bold.
\end{tablenotes}
\end{threeparttable}}
\end{table}

Furthermore, based on the final test set, we present the number of
series selected at each level and the optimal tuning parameter values
obtained using different proposed methods, as shown in
Table~\ref{tbl-labour-info}. Here, we only showcase results from the
Subset and Elasso methods, as they prove to be valuable in the labour
force application in terms of the RMSE results. Note that the variation
in the scale of the optimal parameters for different methods comes from
the difference in the scales of objective. Table~\ref{tbl-labour-info}
shows that all Subset methods exclude some series when performing
forecast reconciliation. Remarkably, the Elasso method consistently
outperforms the others overall, even though it uses only \(11\) series
for forecast reconciliation. Additionally, it is worth noting that most
of the series at the STT level are removed, while the majority of series
at the Duration level are retained. This aligns with our data
description, highlighting that the seasonal patterns in the Duration
level series is more consistent and potentially easier to forecast
compared to those at the STT level.

\hypertarget{tbl-labour-info}{}
\begin{table}[!h]
\caption{\label{tbl-labour-info}Number of time series selected using different proposed methods and the
optimal parameter values identified in the labour application,
considering a single test set (from August 2022 to July 2023). }\tabularnewline

\centering
\resizebox{\linewidth}{!}{
\begin{tabular}{lrrrrrrrr}
\toprule
\multicolumn{1}{c}{} & \multicolumn{5}{c}{Number of time series retained} & \multicolumn{3}{c}{Optimal parameters} \\
\cmidrule(l{3pt}r{3pt}){2-6} \cmidrule(l{3pt}r{3pt}){7-9}
  & Top & Duration & STT & Duration x STT & Total & $\lambda$ & $\lambda_0$ & $\lambda_2$\\
\midrule
None & 1 & 6 & 8 & 48 & 63 & - & - & -\\
OLS-subset & 0 & 5 & 1 & 48 & 54 & - & 4.16 & 1.00\\
WLSs-subset & 0 & 5 & 1 & 46 & 52 & - & 0.38 & 0.10\\
WLSv-subset & 1 & 5 & 7 & 48 & 61 & - & 0.51 & 1.00\\
MinTs-subset & 0 & 1 & 1 & 47 & 49 & - & 0.03 & 0.01\\
Elasso & 1 & 5 & 2 & 3 & 11 & 213.59 & - & -\\
\bottomrule
\end{tabular}}
\end{table}

\hypertarget{sec-tourism}{%
\subsection{Forecasting Australian domestic tourism}\label{sec-tourism}}

In this section we consider Australian domestic tourism flows, measured
as the number of overnight trips Australians spend away from home, and
create a hierarchical structure using geographic divisions. The data are
sourced from the National Visitor Survey and collected through
computer-assisted telephone interviews involving approximately
\(120,000\) Australian residents aged \(15\) years and older. The
hierarchical structure starts with the national total tourism flow as
the top-level aggregation, then disaggregates it into seven states and
territories (referred to as \emph{State} level hereafter), further
divides them into \(27\) zones, and finally, into \(76\) regions, thus
forming a natural geographical hierarchy.

Therefore, the hierarchy under consideration involves \(76\) monthly
time series at the bottom level and \(111\) monthly series in total,
i.e., \(n_b=76\) and \(n=111\). Each series in the hierarchy spans the
period from January 1998 to December 2017, with a total of \(240\)
observations.

Figure~\ref{fig-tourism-data} shows the aggregate tourism flows for
Australia as well as individual states, revealing pronounced seasonal
patterns across the national total and states, albeit with varying
seasonal patterns among the series. Notably, there was a significant
growth starting from around 2010 for the national total flow and some
states such as NSW, VIC, QLD, and WA. While flows are relatively flat
for SA, TAS, and NT. Moreover, the time plot displays that there was a
large decrease in tourism flows for WA occurred in 2016.

\begin{figure}

{\centering \includegraphics{hf_selection_files/figure-pdf/fig-tourism-data-1.pdf}

}

\caption{\label{fig-tourism-data}Domestic tourism flows from January
1998 to December 2017 for the whole of Australia as well as the states.}

\end{figure}

Our objective is to forecast tourism flows for each series in the
geographical hierarchy while ensuring the coherence of forecasts across
all levels. We adopt the rolling forecast origin approach to evaluate
the forecast accuracy of different methods. We start with a training set
of \(216\) months for each series to generate base forecasts by fitting
the optimal ETS model. Following this, we roll the forecast origin
forward by one month and repeat the process until December 2016. The
base forecasts are reconciled using our proposed methods and some
state-of-the-art reconciliation methods.

Table~\ref{tbl-tourism-rmse-avg} reports the average RMSE values for
base forecasts generated by ETS models, along with the percentage
relative improvements in average RMSE obtained by a particular
reconciliation method relative to the base forecasts. The results show
that the OLS method outperforms other benchmark methods like WLSs, WLSv
and MinTs, despite the fact that WLSv and MinTs account for the
in-sample covariance of base forecast errors. This highlights the
effectiveness of the OLS method despite its simplicity.

Overall, the Subset methods outperform their respective benchmark
methods, especially for aggregation levels and for longer forecast
horizons. The only exception is the OLS-subset method, which slightly
reduces overall accuracy while still improving top-level forecasts.
Moreover, the Intuitive and Lasso methods produce results almost
identical to the corresponding benchmark methods, which is not
surprising as ETS models typically do not yield extremely poor
forecasts, making them challenging to be selected out using methods that
tend to return dense estimates. When we relax the unbiasedness
constraint, EMinT consistently performs the worst across all levels due
to the evident lack of joint weak stationarity among the series in the
hierarchy. The Elasso method presents significant improvement compared
to the EMinT method, and it also outperforms other methods across almost
all levels except for the bottom level.

\hypertarget{tbl-tourism-rmse-avg}{}
\begin{table}[!h]
\caption{\label{tbl-tourism-rmse-avg}Average out-of-sample forecast results for Australian domestic tourism
data. }\tabularnewline

\centering
\resizebox{\linewidth}{!}{
\begin{threeparttable}
\begin{tabular}{lrrrrrlrrrrrlrrrrrlrr}
\toprule
\multicolumn{1}{c}{} & \multicolumn{4}{c}{Top} & \multicolumn{4}{c}{State} & \multicolumn{4}{c}{Zone} & \multicolumn{4}{c}{Region} & \multicolumn{4}{c}{Average} \\
\cmidrule(l{3pt}r{3pt}){2-5} \cmidrule(l{3pt}r{3pt}){6-9} \cmidrule(l{3pt}r{3pt}){10-13} \cmidrule(l{3pt}r{3pt}){14-17} \cmidrule(l{3pt}r{3pt}){18-21}
Method & h=1 & 1-4 & 1-8 & 1-12 & h=1 & 1-4 & 1-8 & 1-12 & h=1 & 1-4 & 1-8 & 1-12 & h=1 & 1-4 & 1-8 & 1-12 & h=1 & 1-4 & 1-8 & 1-12\\
\midrule
Base & 1565.8 & 1520.2 & 1548.3 & 1773.1 & 366.0 & 406.1 & 421.4 & 442.0 & 142.5 & 170.8 & 178.5 & 185.3 & 72.1 & 86.4 & 90.9 & 94.4 & 121.2 & 140.0 & 146.2 & 153.6\\
BU & 14.3 & 38.8 & 42.0 & 38.8 & 4.6 & 10.3 & 13.8 & 15.7 & -0.1 & 0.9 & 1.3 & 1.6 & 0.0 & 0.0 & 0.0 & 0.0 & 2.5 & 6.0 & 6.9 & 7.4\\
\midrule
OLS & -0.6 & 1.1 & 1.8 & 1.9 & -1.2 & -1.0 & -1.0 & -1.3 & -2.8 & -4.0 & -4.8 & -5.6 & -0.1 & -0.8 & -1.6 & -2.4 & -1.1 & -1.6 & -2.1 & -2.7\\
\cellcolor[HTML]{e6e3e3}{OLS-subset} & \cellcolor[HTML]{e6e3e3}{-0.6} & \cellcolor[HTML]{e6e3e3}{\textbf{  -1.9}} & \cellcolor[HTML]{e6e3e3}{\textbf{  -4.9}} & \cellcolor[HTML]{e6e3e3}{\textbf{  -3.0}} & \cellcolor[HTML]{e6e3e3}{-1.2} & \cellcolor[HTML]{e6e3e3}{\textbf{ -1.2}} & \cellcolor[HTML]{e6e3e3}{1.5} & \cellcolor[HTML]{e6e3e3}{0.9} & \cellcolor[HTML]{e6e3e3}{-2.6} & \cellcolor[HTML]{e6e3e3}{-0.5} & \cellcolor[HTML]{e6e3e3}{-0.2} & \cellcolor[HTML]{e6e3e3}{-1.1} & \cellcolor[HTML]{e6e3e3}{0.2} & \cellcolor[HTML]{e6e3e3}{2.1} & \cellcolor[HTML]{e6e3e3}{1.7} & \cellcolor[HTML]{e6e3e3}{0.6} & \cellcolor[HTML]{e6e3e3}{-1.0} & \cellcolor[HTML]{e6e3e3}{0.3} & \cellcolor[HTML]{e6e3e3}{0.5} & \cellcolor[HTML]{e6e3e3}{-0.2}\\
\cellcolor[HTML]{e6e3e3}{OLS-intuitive} & \cellcolor[HTML]{e6e3e3}{-0.6} & \cellcolor[HTML]{e6e3e3}{1.1} & \cellcolor[HTML]{e6e3e3}{1.8} & \cellcolor[HTML]{e6e3e3}{1.9} & \cellcolor[HTML]{e6e3e3}{-1.2} & \cellcolor[HTML]{e6e3e3}{-1.0} & \cellcolor[HTML]{e6e3e3}{-1.0} & \cellcolor[HTML]{e6e3e3}{-1.3} & \cellcolor[HTML]{e6e3e3}{-2.8} & \cellcolor[HTML]{e6e3e3}{-4.0} & \cellcolor[HTML]{e6e3e3}{-4.8} & \cellcolor[HTML]{e6e3e3}{-5.6} & \cellcolor[HTML]{e6e3e3}{-0.1} & \cellcolor[HTML]{e6e3e3}{-0.8} & \cellcolor[HTML]{e6e3e3}{-1.6} & \cellcolor[HTML]{e6e3e3}{-2.4} & \cellcolor[HTML]{e6e3e3}{-1.1} & \cellcolor[HTML]{e6e3e3}{-1.6} & \cellcolor[HTML]{e6e3e3}{-2.1} & \cellcolor[HTML]{e6e3e3}{-2.7}\\
\cellcolor[HTML]{e6e3e3}{OLS-lasso} & \cellcolor[HTML]{e6e3e3}{\textbf{  -0.8}} & \cellcolor[HTML]{e6e3e3}{2.1} & \cellcolor[HTML]{e6e3e3}{2.8} & \cellcolor[HTML]{e6e3e3}{2.9} & \cellcolor[HTML]{e6e3e3}{\textbf{ -1.3}} & \cellcolor[HTML]{e6e3e3}{-0.4} & \cellcolor[HTML]{e6e3e3}{0.5} & \cellcolor[HTML]{e6e3e3}{0.3} & \cellcolor[HTML]{e6e3e3}{-2.3} & \cellcolor[HTML]{e6e3e3}{-3.5} & \cellcolor[HTML]{e6e3e3}{-4.2} & \cellcolor[HTML]{e6e3e3}{-4.9} & \cellcolor[HTML]{e6e3e3}{0.0} & \cellcolor[HTML]{e6e3e3}{\textbf{-0.9}} & \cellcolor[HTML]{e6e3e3}{-1.5} & \cellcolor[HTML]{e6e3e3}{-2.2} & \cellcolor[HTML]{e6e3e3}{-1.0} & \cellcolor[HTML]{e6e3e3}{-1.3} & \cellcolor[HTML]{e6e3e3}{-1.5} & \cellcolor[HTML]{e6e3e3}{-2.0}\\
\midrule
WLSs & 4.1 & 16.5 & 19.0 & 18.1 & 0.6 & 2.0 & 4.1 & 5.2 & -2.7 & -3.1 & -3.3 & -3.4 & -0.5 & -1.0 & -1.4 & -1.8 & -0.4 & 0.7 & 1.0 & 1.1\\
\cellcolor[HTML]{e6e3e3}{WLSs-subset} & \cellcolor[HTML]{e6e3e3}{4.1} & \cellcolor[HTML]{e6e3e3}{\textbf{   6.9}} & \cellcolor[HTML]{e6e3e3}{\textbf{   8.9}} & \cellcolor[HTML]{e6e3e3}{\textbf{  10.2}} & \cellcolor[HTML]{e6e3e3}{0.6} & \cellcolor[HTML]{e6e3e3}{\textbf{  1.7}} & \cellcolor[HTML]{e6e3e3}{\textbf{  4.0}} & \cellcolor[HTML]{e6e3e3}{\textbf{  4.3}} & \cellcolor[HTML]{e6e3e3}{-2.7} & \cellcolor[HTML]{e6e3e3}{-3.0} & \cellcolor[HTML]{e6e3e3}{-2.1} & \cellcolor[HTML]{e6e3e3}{-2.1} & \cellcolor[HTML]{e6e3e3}{-0.5} & \cellcolor[HTML]{e6e3e3}{-0.1} & \cellcolor[HTML]{e6e3e3}{-0.1} & \cellcolor[HTML]{e6e3e3}{-0.5} & \cellcolor[HTML]{e6e3e3}{-0.4} & \cellcolor[HTML]{e6e3e3}{\textbf{  0.0}} & \cellcolor[HTML]{e6e3e3}{\textbf{  0.9}} & \cellcolor[HTML]{e6e3e3}{\textbf{  1.0}}\\
\cellcolor[HTML]{e6e3e3}{WLSs-intuitive} & \cellcolor[HTML]{e6e3e3}{4.1} & \cellcolor[HTML]{e6e3e3}{16.5} & \cellcolor[HTML]{e6e3e3}{19.0} & \cellcolor[HTML]{e6e3e3}{18.1} & \cellcolor[HTML]{e6e3e3}{0.6} & \cellcolor[HTML]{e6e3e3}{2.0} & \cellcolor[HTML]{e6e3e3}{4.1} & \cellcolor[HTML]{e6e3e3}{5.2} & \cellcolor[HTML]{e6e3e3}{-2.7} & \cellcolor[HTML]{e6e3e3}{-3.1} & \cellcolor[HTML]{e6e3e3}{-3.3} & \cellcolor[HTML]{e6e3e3}{-3.4} & \cellcolor[HTML]{e6e3e3}{-0.5} & \cellcolor[HTML]{e6e3e3}{-1.0} & \cellcolor[HTML]{e6e3e3}{-1.4} & \cellcolor[HTML]{e6e3e3}{-1.8} & \cellcolor[HTML]{e6e3e3}{-0.4} & \cellcolor[HTML]{e6e3e3}{0.7} & \cellcolor[HTML]{e6e3e3}{1.0} & \cellcolor[HTML]{e6e3e3}{1.1}\\
\cellcolor[HTML]{e6e3e3}{WLSs-lasso} & \cellcolor[HTML]{e6e3e3}{\textbf{   3.6}} & \cellcolor[HTML]{e6e3e3}{17.1} & \cellcolor[HTML]{e6e3e3}{19.5} & \cellcolor[HTML]{e6e3e3}{18.5} & \cellcolor[HTML]{e6e3e3}{\textbf{  0.3}} & \cellcolor[HTML]{e6e3e3}{2.1} & \cellcolor[HTML]{e6e3e3}{4.3} & \cellcolor[HTML]{e6e3e3}{5.5} & \cellcolor[HTML]{e6e3e3}{-2.7} & \cellcolor[HTML]{e6e3e3}{-2.9} & \cellcolor[HTML]{e6e3e3}{-3.2} & \cellcolor[HTML]{e6e3e3}{-3.3} & \cellcolor[HTML]{e6e3e3}{\textbf{-0.6}} & \cellcolor[HTML]{e6e3e3}{-1.0} & \cellcolor[HTML]{e6e3e3}{-1.4} & \cellcolor[HTML]{e6e3e3}{-1.7} & \cellcolor[HTML]{e6e3e3}{\textbf{ -0.5}} & \cellcolor[HTML]{e6e3e3}{0.8} & \cellcolor[HTML]{e6e3e3}{1.1} & \cellcolor[HTML]{e6e3e3}{1.2}\\
\midrule
WLSv & 6.2 & 22.3 & 25.0 & 23.6 & 1.1 & 3.7 & 6.3 & 7.9 & -2.6 & -2.4 & -2.4 & -2.2 & -1.6 & -1.6 & -1.8 & -1.9 & -0.5 & 1.5 & 2.1 & 2.4\\
\cellcolor[HTML]{e6e3e3}{WLSv-subset} & \cellcolor[HTML]{e6e3e3}{\textbf{   0.9}} & \cellcolor[HTML]{e6e3e3}{\textbf{   4.5}} & \cellcolor[HTML]{e6e3e3}{\textbf{   4.9}} & \cellcolor[HTML]{e6e3e3}{\textbf{   5.7}} & \cellcolor[HTML]{e6e3e3}{\textbf{ -1.2}} & \cellcolor[HTML]{e6e3e3}{\textbf{ -0.3}} & \cellcolor[HTML]{e6e3e3}{\textbf{ -0.1}} & \cellcolor[HTML]{e6e3e3}{\textbf{  0.7}} & \cellcolor[HTML]{e6e3e3}{\textbf{ -3.2}} & \cellcolor[HTML]{e6e3e3}{\textbf{ -3.8}} & \cellcolor[HTML]{e6e3e3}{\textbf{ -4.5}} & \cellcolor[HTML]{e6e3e3}{\textbf{ -4.9}} & \cellcolor[HTML]{e6e3e3}{-1.3} & \cellcolor[HTML]{e6e3e3}{-1.2} & \cellcolor[HTML]{e6e3e3}{-1.7} & \cellcolor[HTML]{e6e3e3}{\textbf{-2.3}} & \cellcolor[HTML]{e6e3e3}{\textbf{ -1.6}} & \cellcolor[HTML]{e6e3e3}{\textbf{ -1.2}} & \cellcolor[HTML]{e6e3e3}{\textbf{ -1.6}} & \cellcolor[HTML]{e6e3e3}{\textbf{ -1.7}}\\
\cellcolor[HTML]{e6e3e3}{WLSv-intuitive} & \cellcolor[HTML]{e6e3e3}{6.2} & \cellcolor[HTML]{e6e3e3}{22.3} & \cellcolor[HTML]{e6e3e3}{25.0} & \cellcolor[HTML]{e6e3e3}{23.6} & \cellcolor[HTML]{e6e3e3}{1.1} & \cellcolor[HTML]{e6e3e3}{3.7} & \cellcolor[HTML]{e6e3e3}{6.3} & \cellcolor[HTML]{e6e3e3}{7.9} & \cellcolor[HTML]{e6e3e3}{-2.6} & \cellcolor[HTML]{e6e3e3}{-2.4} & \cellcolor[HTML]{e6e3e3}{-2.4} & \cellcolor[HTML]{e6e3e3}{-2.2} & \cellcolor[HTML]{e6e3e3}{-1.6} & \cellcolor[HTML]{e6e3e3}{-1.6} & \cellcolor[HTML]{e6e3e3}{-1.8} & \cellcolor[HTML]{e6e3e3}{-1.9} & \cellcolor[HTML]{e6e3e3}{-0.5} & \cellcolor[HTML]{e6e3e3}{1.5} & \cellcolor[HTML]{e6e3e3}{2.1} & \cellcolor[HTML]{e6e3e3}{2.4}\\
\cellcolor[HTML]{e6e3e3}{WLSv-lasso} & \cellcolor[HTML]{e6e3e3}{6.2} & \cellcolor[HTML]{e6e3e3}{22.3} & \cellcolor[HTML]{e6e3e3}{25.0} & \cellcolor[HTML]{e6e3e3}{23.6} & \cellcolor[HTML]{e6e3e3}{1.1} & \cellcolor[HTML]{e6e3e3}{3.7} & \cellcolor[HTML]{e6e3e3}{6.3} & \cellcolor[HTML]{e6e3e3}{7.9} & \cellcolor[HTML]{e6e3e3}{-2.6} & \cellcolor[HTML]{e6e3e3}{-2.4} & \cellcolor[HTML]{e6e3e3}{-2.4} & \cellcolor[HTML]{e6e3e3}{-2.2} & \cellcolor[HTML]{e6e3e3}{-1.6} & \cellcolor[HTML]{e6e3e3}{-1.6} & \cellcolor[HTML]{e6e3e3}{-1.8} & \cellcolor[HTML]{e6e3e3}{-1.9} & \cellcolor[HTML]{e6e3e3}{-0.5} & \cellcolor[HTML]{e6e3e3}{1.5} & \cellcolor[HTML]{e6e3e3}{2.1} & \cellcolor[HTML]{e6e3e3}{2.4}\\
\midrule
MinTs & 5.1 & 17.2 & 19.7 & 18.6 & 0.1 & 1.9 & 4.4 & 5.7 & -3.5 & -3.3 & -3.4 & -3.4 & \textcolor{blue}{\textbf{-1.9}} & \textcolor{blue}{\textbf{-2.0}} & \textcolor{blue}{\textbf{-2.4}} & \textcolor{blue}{\textbf{-2.7}} & -1.2 & 0.2 & 0.7 & 0.9\\
\cellcolor[HTML]{e6e3e3}{MinTs-subset} & \cellcolor[HTML]{e6e3e3}{\textbf{   1.8}} & \cellcolor[HTML]{e6e3e3}{\textbf{   1.3}} & \cellcolor[HTML]{e6e3e3}{\textbf{   2.0}} & \cellcolor[HTML]{e6e3e3}{\textbf{   3.2}} & \cellcolor[HTML]{e6e3e3}{\textbf{ -2.2}} & \cellcolor[HTML]{e6e3e3}{\textbf{ -2.1}} & \cellcolor[HTML]{e6e3e3}{\textbf{ -1.3}} & \cellcolor[HTML]{e6e3e3}{\textbf{ -0.7}} & \cellcolor[HTML]{e6e3e3}{\textbf{ -4.2}} & \cellcolor[HTML]{e6e3e3}{\textbf{ -4.5}} & \cellcolor[HTML]{e6e3e3}{\textbf{ -4.9}} & \cellcolor[HTML]{e6e3e3}{\textbf{ -5.4}} & \cellcolor[HTML]{e6e3e3}{-1.5} & \cellcolor[HTML]{e6e3e3}{-1.3} & \cellcolor[HTML]{e6e3e3}{-1.9} & \cellcolor[HTML]{e6e3e3}{-2.5} & \cellcolor[HTML]{e6e3e3}{\textbf{ -2.0}} & \cellcolor[HTML]{e6e3e3}{\textbf{ -2.2}} & \cellcolor[HTML]{e6e3e3}{\textbf{ -2.3}} & \cellcolor[HTML]{e6e3e3}{\textbf{ -2.5}}\\
\cellcolor[HTML]{e6e3e3}{MinTs-intuitive} & \cellcolor[HTML]{e6e3e3}{5.1} & \cellcolor[HTML]{e6e3e3}{17.2} & \cellcolor[HTML]{e6e3e3}{19.7} & \cellcolor[HTML]{e6e3e3}{18.6} & \cellcolor[HTML]{e6e3e3}{0.1} & \cellcolor[HTML]{e6e3e3}{1.9} & \cellcolor[HTML]{e6e3e3}{4.4} & \cellcolor[HTML]{e6e3e3}{5.7} & \cellcolor[HTML]{e6e3e3}{-3.5} & \cellcolor[HTML]{e6e3e3}{-3.3} & \cellcolor[HTML]{e6e3e3}{-3.4} & \cellcolor[HTML]{e6e3e3}{-3.4} & \cellcolor[HTML]{e6e3e3}{\textcolor{blue}{\textbf{-1.9}}} & \cellcolor[HTML]{e6e3e3}{\textcolor{blue}{\textbf{-2.0}}} & \cellcolor[HTML]{e6e3e3}{\textcolor{blue}{\textbf{-2.4}}} & \cellcolor[HTML]{e6e3e3}{\textcolor{blue}{\textbf{-2.7}}} & \cellcolor[HTML]{e6e3e3}{-1.2} & \cellcolor[HTML]{e6e3e3}{0.2} & \cellcolor[HTML]{e6e3e3}{0.7} & \cellcolor[HTML]{e6e3e3}{0.9}\\
\cellcolor[HTML]{e6e3e3}{MinTs-lasso} & \cellcolor[HTML]{e6e3e3}{5.1} & \cellcolor[HTML]{e6e3e3}{17.2} & \cellcolor[HTML]{e6e3e3}{19.7} & \cellcolor[HTML]{e6e3e3}{18.6} & \cellcolor[HTML]{e6e3e3}{0.1} & \cellcolor[HTML]{e6e3e3}{1.9} & \cellcolor[HTML]{e6e3e3}{4.4} & \cellcolor[HTML]{e6e3e3}{5.7} & \cellcolor[HTML]{e6e3e3}{-3.5} & \cellcolor[HTML]{e6e3e3}{-3.3} & \cellcolor[HTML]{e6e3e3}{-3.4} & \cellcolor[HTML]{e6e3e3}{-3.4} & \cellcolor[HTML]{e6e3e3}{\textcolor{blue}{\textbf{-1.9}}} & \cellcolor[HTML]{e6e3e3}{\textcolor{blue}{\textbf{-2.0}}} & \cellcolor[HTML]{e6e3e3}{\textcolor{blue}{\textbf{-2.4}}} & \cellcolor[HTML]{e6e3e3}{\textcolor{blue}{\textbf{-2.7}}} & \cellcolor[HTML]{e6e3e3}{-1.2} & \cellcolor[HTML]{e6e3e3}{0.2} & \cellcolor[HTML]{e6e3e3}{0.7} & \cellcolor[HTML]{e6e3e3}{0.9}\\
\midrule
EMinT & -2.3 & 24.3 & 58.8 & 59.7 & 36.9 & 56.0 & 68.4 & 70.4 & 51.4 & 64.6 & 75.8 & 81.4 & 65.9 & 72.3 & 81.9 & 85.9 & 48.3 & 62.3 & 75.4 & 79.0\\
\cellcolor[HTML]{e6e3e3}{Elasso} & \cellcolor[HTML]{e6e3e3}{\textcolor{blue}{\textbf{ -17.0}}} & \cellcolor[HTML]{e6e3e3}{\textcolor{blue}{\textbf{ -19.4}}} & \cellcolor[HTML]{e6e3e3}{\textcolor{blue}{\textbf{ -19.8}}} & \cellcolor[HTML]{e6e3e3}{\textcolor{blue}{\textbf{ -18.7}}} & \cellcolor[HTML]{e6e3e3}{\textcolor{blue}{\textbf{-21.6}}} & \cellcolor[HTML]{e6e3e3}{\textcolor{blue}{\textbf{-17.3}}} & \cellcolor[HTML]{e6e3e3}{\textcolor{blue}{\textbf{-19.3}}} & \cellcolor[HTML]{e6e3e3}{\textcolor{blue}{\textbf{-19.6}}} & \cellcolor[HTML]{e6e3e3}{\textcolor{blue}{\textbf{ -6.5}}} & \cellcolor[HTML]{e6e3e3}{\textcolor{blue}{\textbf{ -9.4}}} & \cellcolor[HTML]{e6e3e3}{\textcolor{blue}{\textbf{-11.5}}} & \cellcolor[HTML]{e6e3e3}{\textcolor{blue}{\textbf{-12.6}}} & \cellcolor[HTML]{e6e3e3}{\textbf{ 2.2}} & \cellcolor[HTML]{e6e3e3}{\textbf{ 0.4}} & \cellcolor[HTML]{e6e3e3}{\textbf{-1.0}} & \cellcolor[HTML]{e6e3e3}{\textbf{-1.8}} & \cellcolor[HTML]{e6e3e3}{\textcolor{blue}{\textbf{ -7.0}}} & \cellcolor[HTML]{e6e3e3}{\textcolor{blue}{\textbf{ -7.7}}} & \cellcolor[HTML]{e6e3e3}{\textcolor{blue}{\textbf{ -9.2}}} & \cellcolor[HTML]{e6e3e3}{\textcolor{blue}{\textbf{ -9.9}}}\\
\bottomrule
\end{tabular}
\begin{tablenotes}[para]
\item \underline{\textit{NOTE:}} 
\item The Base row shows the average RMSE of the base forecasts. Entries below this row indicate the percentage decrease (negative) or increase (positive) in the average RMSE of the reconciled forecasts compared to the base forecasts. The entries with the lowest values in each column are highlighted in blue. In each panel, the proposed methods are indicated with a gray background, and methods that outperform the benchmark method are marked in bold.
\end{tablenotes}
\end{threeparttable}}
\end{table}

We also present the results based on the last one training set spanning
from January 2017 to December 2017 in Table~\ref{tbl-tourism-rmse}. The
results shows a similar performance to the average results described
above, indicating relatively high-quality forecasts from the Subset and
Elasso methods. The reconciliation errors across each of the \(111\)
series and across the four levels in the hierarchy are displayed in
Figure~\ref{fig-tourism-rmse}.

\hypertarget{tbl-tourism-rmse}{}
\begin{table}[!h]
\caption{\label{tbl-tourism-rmse}Out-of-sample forecast results on a single test set (from January 2017
to December 2017) for Australian domestic tourism data. }\tabularnewline

\centering
\resizebox{\linewidth}{!}{
\begin{threeparttable}
\begin{tabular}{lrrrrrlrrrrrlrrrrrlrr}
\toprule
\multicolumn{1}{c}{} & \multicolumn{4}{c}{Top} & \multicolumn{4}{c}{State} & \multicolumn{4}{c}{Zone} & \multicolumn{4}{c}{Region} & \multicolumn{4}{c}{Average} \\
\cmidrule(l{3pt}r{3pt}){2-5} \cmidrule(l{3pt}r{3pt}){6-9} \cmidrule(l{3pt}r{3pt}){10-13} \cmidrule(l{3pt}r{3pt}){14-17} \cmidrule(l{3pt}r{3pt}){18-21}
Method & h=1 & 1-4 & 1-8 & 1-12 & h=1 & 1-4 & 1-8 & 1-12 & h=1 & 1-4 & 1-8 & 1-12 & h=1 & 1-4 & 1-8 & 1-12 & h=1 & 1-4 & 1-8 & 1-12\\
\midrule
Base & 1158.2 & 716.6 & 1279.5 & 1907.6 & 452.7 & 323.3 & 349.9 & 424.8 & 165.5 & 163.6 & 160.7 & 179.7 & 100.8 & 89.4 & 88.2 & 94.1 & 148.3 & 127.9 & 133.1 & 152.1\\
BU & 89.1 & 132.8 & 53.4 & 42.0 & -4.6 & 10.3 & 17.0 & 19.7 & 1.1 & -2.4 & 0.4 & 1.0 & 0.0 & 0.0 & 0.0 & 0.0 & 5.7 & 7.6 & 7.6 & 8.5\\
\midrule
OLS & -4.7 & -0.4 & 0.5 & 1.4 & -3.0 & -3.9 & -1.6 & -1.5 & -2.1 & -4.2 & -5.6 & -7.5 & 1.0 & -0.4 & -1.9 & -3.2 & -1.0 & -2.1 & -2.7 & -3.6\\
\cellcolor[HTML]{e6e3e3}{OLS-subset} & \cellcolor[HTML]{e6e3e3}{-4.7} & \cellcolor[HTML]{e6e3e3}{8.0} & \cellcolor[HTML]{e6e3e3}{\textbf{  -1.4}} & \cellcolor[HTML]{e6e3e3}{\textbf{ -14.1}} & \cellcolor[HTML]{e6e3e3}{-3.0} & \cellcolor[HTML]{e6e3e3}{5.5} & \cellcolor[HTML]{e6e3e3}{0.3} & \cellcolor[HTML]{e6e3e3}{\textbf{ -7.9}} & \cellcolor[HTML]{e6e3e3}{-2.1} & \cellcolor[HTML]{e6e3e3}{-1.5} & \cellcolor[HTML]{e6e3e3}{-3.7} & \cellcolor[HTML]{e6e3e3}{\textbf{ -8.7}} & \cellcolor[HTML]{e6e3e3}{1.0} & \cellcolor[HTML]{e6e3e3}{1.7} & \cellcolor[HTML]{e6e3e3}{-0.1} & \cellcolor[HTML]{e6e3e3}{-2.3} & \cellcolor[HTML]{e6e3e3}{-1.0} & \cellcolor[HTML]{e6e3e3}{1.7} & \cellcolor[HTML]{e6e3e3}{-1.2} & \cellcolor[HTML]{e6e3e3}{\textbf{ -6.5}}\\
\cellcolor[HTML]{e6e3e3}{OLS-intuitive} & \cellcolor[HTML]{e6e3e3}{-4.7} & \cellcolor[HTML]{e6e3e3}{-0.4} & \cellcolor[HTML]{e6e3e3}{0.5} & \cellcolor[HTML]{e6e3e3}{1.4} & \cellcolor[HTML]{e6e3e3}{-3.0} & \cellcolor[HTML]{e6e3e3}{-3.9} & \cellcolor[HTML]{e6e3e3}{-1.6} & \cellcolor[HTML]{e6e3e3}{-1.5} & \cellcolor[HTML]{e6e3e3}{-2.1} & \cellcolor[HTML]{e6e3e3}{-4.2} & \cellcolor[HTML]{e6e3e3}{-5.6} & \cellcolor[HTML]{e6e3e3}{-7.5} & \cellcolor[HTML]{e6e3e3}{1.0} & \cellcolor[HTML]{e6e3e3}{-0.4} & \cellcolor[HTML]{e6e3e3}{-1.9} & \cellcolor[HTML]{e6e3e3}{-3.2} & \cellcolor[HTML]{e6e3e3}{-1.0} & \cellcolor[HTML]{e6e3e3}{-2.1} & \cellcolor[HTML]{e6e3e3}{-2.7} & \cellcolor[HTML]{e6e3e3}{-3.6}\\
\cellcolor[HTML]{e6e3e3}{OLS-lasso} & \cellcolor[HTML]{e6e3e3}{-4.7} & \cellcolor[HTML]{e6e3e3}{-0.4} & \cellcolor[HTML]{e6e3e3}{0.5} & \cellcolor[HTML]{e6e3e3}{1.4} & \cellcolor[HTML]{e6e3e3}{-3.0} & \cellcolor[HTML]{e6e3e3}{-3.9} & \cellcolor[HTML]{e6e3e3}{-1.6} & \cellcolor[HTML]{e6e3e3}{-1.5} & \cellcolor[HTML]{e6e3e3}{-2.1} & \cellcolor[HTML]{e6e3e3}{-4.2} & \cellcolor[HTML]{e6e3e3}{-5.6} & \cellcolor[HTML]{e6e3e3}{-7.5} & \cellcolor[HTML]{e6e3e3}{1.0} & \cellcolor[HTML]{e6e3e3}{-0.4} & \cellcolor[HTML]{e6e3e3}{-1.9} & \cellcolor[HTML]{e6e3e3}{-3.2} & \cellcolor[HTML]{e6e3e3}{-1.0} & \cellcolor[HTML]{e6e3e3}{-2.1} & \cellcolor[HTML]{e6e3e3}{-2.7} & \cellcolor[HTML]{e6e3e3}{-3.6}\\
\midrule
WLSs & 25.1 & 55.2 & 20.8 & 19.1 & -15.8 & -5.0 & 3.5 & 6.2 & -5.9 & -5.4 & -4.7 & -5.0 & -0.2 & -0.8 & -1.6 & -2.2 & -3.0 & -0.1 & 0.3 & 0.9\\
\cellcolor[HTML]{e6e3e3}{WLSs-subset} & \cellcolor[HTML]{e6e3e3}{25.1} & \cellcolor[HTML]{e6e3e3}{\textbf{ 18.7}} & \cellcolor[HTML]{e6e3e3}{\textbf{   0.8}} & \cellcolor[HTML]{e6e3e3}{\textbf{  -7.8}} & \cellcolor[HTML]{e6e3e3}{-15.8} & \cellcolor[HTML]{e6e3e3}{-2.7} & \cellcolor[HTML]{e6e3e3}{\textbf{ -2.1}} & \cellcolor[HTML]{e6e3e3}{\textbf{ -6.2}} & \cellcolor[HTML]{e6e3e3}{-5.9} & \cellcolor[HTML]{e6e3e3}{-4.1} & \cellcolor[HTML]{e6e3e3}{\textbf{ -4.8}} & \cellcolor[HTML]{e6e3e3}{\textbf{ -8.5}} & \cellcolor[HTML]{e6e3e3}{-0.2} & \cellcolor[HTML]{e6e3e3}{0.3} & \cellcolor[HTML]{e6e3e3}{-1.0} & \cellcolor[HTML]{e6e3e3}{\textbf{-2.5}} & \cellcolor[HTML]{e6e3e3}{-3.0} & \cellcolor[HTML]{e6e3e3}{\textbf{ -0.6}} & \cellcolor[HTML]{e6e3e3}{\textbf{ -2.1}} & \cellcolor[HTML]{e6e3e3}{\textbf{ -5.5}}\\
\cellcolor[HTML]{e6e3e3}{WLSs-intuitive} & \cellcolor[HTML]{e6e3e3}{25.1} & \cellcolor[HTML]{e6e3e3}{55.2} & \cellcolor[HTML]{e6e3e3}{20.8} & \cellcolor[HTML]{e6e3e3}{19.1} & \cellcolor[HTML]{e6e3e3}{-15.8} & \cellcolor[HTML]{e6e3e3}{-5.0} & \cellcolor[HTML]{e6e3e3}{3.5} & \cellcolor[HTML]{e6e3e3}{6.2} & \cellcolor[HTML]{e6e3e3}{-5.9} & \cellcolor[HTML]{e6e3e3}{-5.4} & \cellcolor[HTML]{e6e3e3}{-4.7} & \cellcolor[HTML]{e6e3e3}{-5.0} & \cellcolor[HTML]{e6e3e3}{-0.2} & \cellcolor[HTML]{e6e3e3}{-0.8} & \cellcolor[HTML]{e6e3e3}{-1.6} & \cellcolor[HTML]{e6e3e3}{-2.2} & \cellcolor[HTML]{e6e3e3}{-3.0} & \cellcolor[HTML]{e6e3e3}{-0.1} & \cellcolor[HTML]{e6e3e3}{0.3} & \cellcolor[HTML]{e6e3e3}{0.9}\\
\cellcolor[HTML]{e6e3e3}{WLSs-lasso} & \cellcolor[HTML]{e6e3e3}{25.1} & \cellcolor[HTML]{e6e3e3}{55.2} & \cellcolor[HTML]{e6e3e3}{20.8} & \cellcolor[HTML]{e6e3e3}{19.1} & \cellcolor[HTML]{e6e3e3}{-15.8} & \cellcolor[HTML]{e6e3e3}{-5.0} & \cellcolor[HTML]{e6e3e3}{3.5} & \cellcolor[HTML]{e6e3e3}{6.2} & \cellcolor[HTML]{e6e3e3}{-5.9} & \cellcolor[HTML]{e6e3e3}{-5.4} & \cellcolor[HTML]{e6e3e3}{-4.7} & \cellcolor[HTML]{e6e3e3}{-5.0} & \cellcolor[HTML]{e6e3e3}{-0.2} & \cellcolor[HTML]{e6e3e3}{-0.8} & \cellcolor[HTML]{e6e3e3}{-1.6} & \cellcolor[HTML]{e6e3e3}{-2.2} & \cellcolor[HTML]{e6e3e3}{-3.0} & \cellcolor[HTML]{e6e3e3}{-0.1} & \cellcolor[HTML]{e6e3e3}{0.3} & \cellcolor[HTML]{e6e3e3}{0.9}\\
\midrule
WLSv & 38.2 & 76.2 & 29.6 & 25.6 & -17.4 & -3.1 & 7.0 & 9.9 & -5.0 & -4.3 & -3.1 & -3.2 & -4.2 & -1.6 & -1.8 & -2.1 & -3.9 & 1.3 & 2.0 & 2.8\\
\cellcolor[HTML]{e6e3e3}{WLSv-subset} & \cellcolor[HTML]{e6e3e3}{38.2} & \cellcolor[HTML]{e6e3e3}{\textbf{ 34.5}} & \cellcolor[HTML]{e6e3e3}{\textbf{  10.7}} & \cellcolor[HTML]{e6e3e3}{\textbf{   8.5}} & \cellcolor[HTML]{e6e3e3}{-17.4} & \cellcolor[HTML]{e6e3e3}{\textbf{ -8.8}} & \cellcolor[HTML]{e6e3e3}{\textbf{ -0.8}} & \cellcolor[HTML]{e6e3e3}{\textbf{  1.4}} & \cellcolor[HTML]{e6e3e3}{-5.0} & \cellcolor[HTML]{e6e3e3}{\textbf{ -5.5}} & \cellcolor[HTML]{e6e3e3}{\textbf{ -5.3}} & \cellcolor[HTML]{e6e3e3}{\textbf{ -6.7}} & \cellcolor[HTML]{e6e3e3}{-4.1} & \cellcolor[HTML]{e6e3e3}{\textbf{ -2.0}} & \cellcolor[HTML]{e6e3e3}{\textbf{-2.6}} & \cellcolor[HTML]{e6e3e3}{\textbf{-3.4}} & \cellcolor[HTML]{e6e3e3}{-3.9} & \cellcolor[HTML]{e6e3e3}{\textbf{ -2.3}} & \cellcolor[HTML]{e6e3e3}{\textbf{ -2.0}} & \cellcolor[HTML]{e6e3e3}{\textbf{ -2.2}}\\
\cellcolor[HTML]{e6e3e3}{WLSv-intuitive} & \cellcolor[HTML]{e6e3e3}{38.2} & \cellcolor[HTML]{e6e3e3}{76.2} & \cellcolor[HTML]{e6e3e3}{29.6} & \cellcolor[HTML]{e6e3e3}{25.6} & \cellcolor[HTML]{e6e3e3}{-17.4} & \cellcolor[HTML]{e6e3e3}{-3.1} & \cellcolor[HTML]{e6e3e3}{7.0} & \cellcolor[HTML]{e6e3e3}{9.9} & \cellcolor[HTML]{e6e3e3}{-5.0} & \cellcolor[HTML]{e6e3e3}{-4.3} & \cellcolor[HTML]{e6e3e3}{-3.1} & \cellcolor[HTML]{e6e3e3}{-3.2} & \cellcolor[HTML]{e6e3e3}{-4.2} & \cellcolor[HTML]{e6e3e3}{-1.6} & \cellcolor[HTML]{e6e3e3}{-1.8} & \cellcolor[HTML]{e6e3e3}{-2.1} & \cellcolor[HTML]{e6e3e3}{-3.9} & \cellcolor[HTML]{e6e3e3}{1.3} & \cellcolor[HTML]{e6e3e3}{2.0} & \cellcolor[HTML]{e6e3e3}{2.8}\\
\cellcolor[HTML]{e6e3e3}{WLSv-lasso} & \cellcolor[HTML]{e6e3e3}{38.2} & \cellcolor[HTML]{e6e3e3}{76.2} & \cellcolor[HTML]{e6e3e3}{29.6} & \cellcolor[HTML]{e6e3e3}{25.6} & \cellcolor[HTML]{e6e3e3}{-17.4} & \cellcolor[HTML]{e6e3e3}{-3.1} & \cellcolor[HTML]{e6e3e3}{7.0} & \cellcolor[HTML]{e6e3e3}{9.9} & \cellcolor[HTML]{e6e3e3}{-5.0} & \cellcolor[HTML]{e6e3e3}{-4.3} & \cellcolor[HTML]{e6e3e3}{-3.1} & \cellcolor[HTML]{e6e3e3}{-3.2} & \cellcolor[HTML]{e6e3e3}{-4.2} & \cellcolor[HTML]{e6e3e3}{-1.6} & \cellcolor[HTML]{e6e3e3}{-1.8} & \cellcolor[HTML]{e6e3e3}{-2.1} & \cellcolor[HTML]{e6e3e3}{-3.9} & \cellcolor[HTML]{e6e3e3}{1.3} & \cellcolor[HTML]{e6e3e3}{2.0} & \cellcolor[HTML]{e6e3e3}{2.8}\\
\midrule
MinTs & 20.6 & 53.6 & 21.6 & 19.0 & \textcolor{blue}{\textbf{-22.2}} & -7.2 & 3.5 & 6.3 & \textcolor{blue}{\textbf{-12.1}} & -6.6 & -5.1 & -5.3 & \textcolor{blue}{\textbf{ -5.3}} & -2.6 & -2.8 & -3.1 & -8.6 & -1.8 & -0.3 & 0.4\\
\cellcolor[HTML]{e6e3e3}{MinTs-subset} & \cellcolor[HTML]{e6e3e3}{20.6} & \cellcolor[HTML]{e6e3e3}{\textbf{ 20.0}} & \cellcolor[HTML]{e6e3e3}{\textbf{   6.4}} & \cellcolor[HTML]{e6e3e3}{\textbf{   5.6}} & \cellcolor[HTML]{e6e3e3}{\textcolor{blue}{\textbf{-22.2}}} & \cellcolor[HTML]{e6e3e3}{\textcolor{blue}{\textbf{-11.3}}} & \cellcolor[HTML]{e6e3e3}{\textbf{ -2.5}} & \cellcolor[HTML]{e6e3e3}{\textbf{ -0.1}} & \cellcolor[HTML]{e6e3e3}{\textcolor{blue}{\textbf{-12.1}}} & \cellcolor[HTML]{e6e3e3}{\textbf{ -7.5}} & \cellcolor[HTML]{e6e3e3}{\textbf{ -6.4}} & \cellcolor[HTML]{e6e3e3}{\textbf{ -7.8}} & \cellcolor[HTML]{e6e3e3}{\textcolor{blue}{\textbf{ -5.3}}} & \cellcolor[HTML]{e6e3e3}{\textcolor{blue}{\textbf{ -2.9}}} & \cellcolor[HTML]{e6e3e3}{\textcolor{blue}{\textbf{-3.2}}} & \cellcolor[HTML]{e6e3e3}{\textcolor{blue}{\textbf{-3.9}}} & \cellcolor[HTML]{e6e3e3}{-8.6} & \cellcolor[HTML]{e6e3e3}{\textcolor{blue}{\textbf{ -4.5}}} & \cellcolor[HTML]{e6e3e3}{\textcolor{blue}{\textbf{ -3.2}}} & \cellcolor[HTML]{e6e3e3}{\textbf{ -3.3}}\\
\cellcolor[HTML]{e6e3e3}{MinTs-intuitive} & \cellcolor[HTML]{e6e3e3}{20.6} & \cellcolor[HTML]{e6e3e3}{53.6} & \cellcolor[HTML]{e6e3e3}{21.6} & \cellcolor[HTML]{e6e3e3}{19.0} & \cellcolor[HTML]{e6e3e3}{\textcolor{blue}{\textbf{-22.2}}} & \cellcolor[HTML]{e6e3e3}{-7.2} & \cellcolor[HTML]{e6e3e3}{3.5} & \cellcolor[HTML]{e6e3e3}{6.3} & \cellcolor[HTML]{e6e3e3}{\textcolor{blue}{\textbf{-12.1}}} & \cellcolor[HTML]{e6e3e3}{-6.6} & \cellcolor[HTML]{e6e3e3}{-5.1} & \cellcolor[HTML]{e6e3e3}{-5.3} & \cellcolor[HTML]{e6e3e3}{\textcolor{blue}{\textbf{ -5.3}}} & \cellcolor[HTML]{e6e3e3}{-2.6} & \cellcolor[HTML]{e6e3e3}{-2.8} & \cellcolor[HTML]{e6e3e3}{-3.1} & \cellcolor[HTML]{e6e3e3}{-8.6} & \cellcolor[HTML]{e6e3e3}{-1.8} & \cellcolor[HTML]{e6e3e3}{-0.3} & \cellcolor[HTML]{e6e3e3}{0.4}\\
\cellcolor[HTML]{e6e3e3}{MinTs-lasso} & \cellcolor[HTML]{e6e3e3}{20.6} & \cellcolor[HTML]{e6e3e3}{53.6} & \cellcolor[HTML]{e6e3e3}{21.6} & \cellcolor[HTML]{e6e3e3}{19.0} & \cellcolor[HTML]{e6e3e3}{\textcolor{blue}{\textbf{-22.2}}} & \cellcolor[HTML]{e6e3e3}{-7.2} & \cellcolor[HTML]{e6e3e3}{3.5} & \cellcolor[HTML]{e6e3e3}{6.3} & \cellcolor[HTML]{e6e3e3}{\textcolor{blue}{\textbf{-12.1}}} & \cellcolor[HTML]{e6e3e3}{-6.6} & \cellcolor[HTML]{e6e3e3}{-5.1} & \cellcolor[HTML]{e6e3e3}{-5.3} & \cellcolor[HTML]{e6e3e3}{\textcolor{blue}{\textbf{ -5.3}}} & \cellcolor[HTML]{e6e3e3}{-2.6} & \cellcolor[HTML]{e6e3e3}{-2.8} & \cellcolor[HTML]{e6e3e3}{-3.1} & \cellcolor[HTML]{e6e3e3}{-8.6} & \cellcolor[HTML]{e6e3e3}{-1.8} & \cellcolor[HTML]{e6e3e3}{-0.3} & \cellcolor[HTML]{e6e3e3}{0.4}\\
\midrule
EMinT & 116.5 & 97.8 & -15.8 & -13.7 & 149.4 & 114.5 & 63.5 & 47.5 & 108.4 & 68.4 & 60.6 & 54.2 & 122.1 & 103.1 & 90.2 & 78.2 & 123.2 & 93.9 & 67.9 & 55.5\\
\cellcolor[HTML]{e6e3e3}{Elasso} & \cellcolor[HTML]{e6e3e3}{\textcolor{blue}{\textbf{ -84.5}}} & \cellcolor[HTML]{e6e3e3}{\textcolor{blue}{\textbf{-50.4}}} & \cellcolor[HTML]{e6e3e3}{\textcolor{blue}{\textbf{ -16.3}}} & \cellcolor[HTML]{e6e3e3}{\textcolor{blue}{\textbf{ -16.4}}} & \cellcolor[HTML]{e6e3e3}{\textbf{-18.3}} & \cellcolor[HTML]{e6e3e3}{\textbf{  0.6}} & \cellcolor[HTML]{e6e3e3}{\textcolor{blue}{\textbf{ -9.0}}} & \cellcolor[HTML]{e6e3e3}{\textcolor{blue}{\textbf{-11.4}}} & \cellcolor[HTML]{e6e3e3}{\textbf{ -7.8}} & \cellcolor[HTML]{e6e3e3}{\textcolor{blue}{\textbf{ -8.8}}} & \cellcolor[HTML]{e6e3e3}{\textcolor{blue}{\textbf{ -7.5}}} & \cellcolor[HTML]{e6e3e3}{\textcolor{blue}{\textbf{-10.4}}} & \cellcolor[HTML]{e6e3e3}{\textbf{  2.9}} & \cellcolor[HTML]{e6e3e3}{\textbf{  1.6}} & \cellcolor[HTML]{e6e3e3}{\textbf{ 4.1}} & \cellcolor[HTML]{e6e3e3}{\textbf{ 0.3}} & \cellcolor[HTML]{e6e3e3}{\textcolor{blue}{\textbf{-10.2}}} & \cellcolor[HTML]{e6e3e3}{\textbf{ -4.4}} & \cellcolor[HTML]{e6e3e3}{\textcolor{blue}{\textbf{ -3.2}}} & \cellcolor[HTML]{e6e3e3}{\textcolor{blue}{\textbf{ -6.7}}}\\
\bottomrule
\end{tabular}
\begin{tablenotes}[para]
\item \underline{\textit{NOTE:}} 
\item The Base row shows the average RMSE of the base forecasts. Entries below this row indicate the percentage decrease (negative) or increase (positive) in the average RMSE of the reconciled forecasts compared to the base forecasts. The entries with the lowest values in each column are highlighted in blue. In each panel, the proposed methods are indicated with a gray background, and methods that outperform the benchmark method are marked in bold.
\end{tablenotes}
\end{threeparttable}}
\end{table}

\begin{figure}

{\centering \includegraphics{hf_selection_files/figure-pdf/fig-tourism-rmse-1.pdf}

}

\caption{\label{fig-tourism-rmse}Average out-of-sample forecasting
performance, measured in terms of RMSE (from 1- to 12-step-ahead), for
each series across different reconciliation methods. Time series are
arranged along the horizontal axis.}

\end{figure}

Additionally, Table~\ref{tbl-tourism-info} presents a summary of the
number of series selected using different proposed methods for each
level as well as the optimal tuning parameter values identified. Here we
only give the results of the Subset and Elasso methods since they are
useful in the tourism application. Note that the variation in the scale
of the optimal parameters for different methods comes from the
difference in the scales of objective. We observe that the OLS-subset
and WLSs-subset methods exclude some series at the State and Zone levels
for forecast reconciliation. In contrast, the WLSv and MinTs methods
retain all series, which is reasonable because they take into account
the in-sample covariance, making themselves allow for larger adjustments
made to series with large in-sample forecast error variances in forecast
reconciliation. Nonetheless, the WLSv and MinTs methods can still
enhance the quality of reconciled forecasts due to the inclusion of
shrinkage through additional ridge regularization. It is surprising that
Elasso performs exceptionally well despite using only \(13\) series for
reconciliation.

\hypertarget{tbl-tourism-info}{}
\begin{table}[!h]
\caption{\label{tbl-tourism-info}Number of time series selected using different proposed methods and the
optimal parameter values identified in the tourism application,
considering a single test set (from January 2017 to December 2017). }\tabularnewline

\centering
\resizebox{\linewidth}{!}{
\begin{tabular}{lrrrrrrrr}
\toprule
\multicolumn{1}{c}{} & \multicolumn{5}{c}{Number of time series retained} & \multicolumn{3}{c}{Optimal parameters} \\
\cmidrule(l{3pt}r{3pt}){2-6} \cmidrule(l{3pt}r{3pt}){7-9}
  & Top & State & Zone & Region & Total & $\lambda$ & $\lambda_0$ & $\lambda_2$\\
\midrule
None & 1 & 7 & 27 & 76 & 111 & - & - & -\\
OLS-subset & 1 & 2 & 13 & 76 & 92 & - & 27.98 & 10.00\\
WLSs-subset & 1 & 1 & 15 & 76 & 93 & - & 18.73 & 10.00\\
WLSv-subset & 1 & 7 & 27 & 76 & 111 & - & 0.03 & 0.01\\
MinTs-subset & 1 & 7 & 27 & 76 & 111 & - & 0.05 & 0.01\\
Elasso & 1 & 4 & 0 & 8 & 13 & 71759.21 & - & -\\
\bottomrule
\end{tabular}}
\end{table}

\hypertarget{sec-conclusion}{%
\section{Conclusion}\label{sec-conclusion}}

In the existing literature on hierarchical time series and linear
forecast reconciliation, we map all base forecasts into bottom-level
disaggregated forecasts, which are then summed up by a summing matrix to
yield coherent forecasts for the entire structure. Hence, the mapping
step in forecast reconciliation can be conceptually regarded as a
forecast combination. In practical applications, it is common that the
base forecasts for some time series in the hierarchical structure may
perform poorly, especially in the context of large hierarchies. This may
reduce the overall effectiveness of forecast reconciliation methods. In
this paper, we aimed to address this issue by introducing a selection
mechanism in forecast reconciliation, i.e., involving time series
selection when reconciling forecasts for hierarchical time series, while
ensuring the generation of coherent forecasts for all series.

Under the unbiasedness constraint, we developed three reconciliation
methods with selection mechanisms to keep forecasts for an automatically
selected set of series unused in forming reconciled forecasts. These
methods include group best-subset selection with ridge regularization
(Subset), intuitive method with \(L_0\) regularization (Intuitive), and
group lasso method (Lasso). These methods formulated the problem based
on out-of-sample base forecasts using different penalty functions
designed to penalize the columns of the weighting matrix,
\(\boldsymbol{G}\), towards zero. Additionally, we relaxed the
unbiasedness constraint and proposed the empirical group lasso method
(Elasso) which achieves series selection based on in-sample observations
and fitted values.

The simulation experiments and two empirical applications demonstrated
the superiority of the proposed methods over the reconciliation methods
that do not involve series selection. In particular, our methods were
preferred, particularly when the error correlation within the
hierarchical structure is negative. Furthermore, when model
misspecification was introduced for some series in the hierarchy, our
proposed methods guaranteed coherent forecasts that outperformed or, at
the very least, matched their respective benchmark methods in the
minimum trace reconciliation framework. In both empirical applications,
where no apparent model misspecification was present, the Subset and
Elasso methods were always preferred, particularly for aggregation
levels and longer forecast horizons, while the Intuitive and Lasso
methods yield results identical to the corresponding benchmark methods,
as they tend to provide dense estimates.

A remarkable feature of the proposed methods is their ability to reduce
the disparities arising from using different estimates of the base
forecast error covariance matrix, thereby mitigating the challenges
associated with estimator selection, which is a prominent issue within
the field of forecast reconciliation research.

\hypertarget{references}{%
\section*{References}\label{references}}
\addcontentsline{toc}{section}{References}

\printbibliography[heading=none]

\newpage
\appendix
\setcounter{section}{0}
\renewcommand{\thesection}{\Alph{section}}
\renewcommand{\thefigure}{A\arabic{figure}}
\renewcommand{\thetable}{A\arabic{table}}
\setcounter{figure}{0}
\setcounter{table}{0}

\hypertarget{appendix}{%
\section*{Appendix}\label{appendix}}
\addcontentsline{toc}{section}{Appendix}

AAA

\hypertarget{tbl-s2-rmse}{}
\begin{table}[!h]
\caption{\label{tbl-s2-rmse}Out-of-sample forecast results for the simulated data in Scenario II,
Setup 1. }\tabularnewline

\centering
\resizebox{\linewidth}{!}{
\begin{threeparttable}
\begin{tabular}{lrrrrlrrrrlrrrrlr}
\toprule
\multicolumn{1}{c}{} & \multicolumn{4}{c}{Top} & \multicolumn{4}{c}{Middle} & \multicolumn{4}{c}{Bottom} & \multicolumn{4}{c}{Average} \\
\cmidrule(l{3pt}r{3pt}){2-5} \cmidrule(l{3pt}r{3pt}){6-9} \cmidrule(l{3pt}r{3pt}){10-13} \cmidrule(l{3pt}r{3pt}){14-17}
Method & h=1 & 1-4 & 1-8 & 1-16 & h=1 & 1-4 & 1-8 & 1-16 & h=1 & 1-4 & 1-8 & 1-16 & h=1 & 1-4 & 1-8 & 1-16\\
\midrule
Base & 9.6 & 10.7 & 12.6 & 15.6 & 12.1 & 14.4 & 15.3 & 17.0 & 4.2 & 4.9 & 5.9 & 7.5 & 7.2 & 8.5 & 9.6 & 11.4\\
BU & \textcolor{blue}{\textbf{-1.0}} & 0.4 & 0.6 & 0.7 & \textcolor{blue}{\textbf{-47.7}} & \textcolor{blue}{\textbf{-49.6}} & \textcolor{blue}{\textbf{-43.6}} & \textcolor{blue}{\textbf{-36.2}} & \textcolor{blue}{\textbf{ 0.0}} & \textcolor{blue}{\textbf{ 0.0}} & \textcolor{blue}{\textbf{ 0.0}} & \textcolor{blue}{\textbf{ 0.0}} & \textcolor{blue}{\textbf{-23.0}} & \textcolor{blue}{\textbf{-24.0}} & \textcolor{blue}{\textbf{-19.8}} & \textcolor{blue}{\textbf{-15.3}}\\
\midrule
OLS & 8.5 & 13.9 & 10.4 & 7.6 & -28.2 & -29.4 & -26.7 & -23.1 & 22.9 & 23.9 & 17.0 & 11.3 & -4.2 & -3.8 & -4.2 & -4.1\\
\cellcolor[HTML]{e6e3e3}{OLS-subset} & \cellcolor[HTML]{e6e3e3}{\textbf{-0.5}} & \cellcolor[HTML]{e6e3e3}{\textbf{ 0.5}} & \cellcolor[HTML]{e6e3e3}{\textbf{ 0.6}} & \cellcolor[HTML]{e6e3e3}{\textbf{ 0.7}} & \cellcolor[HTML]{e6e3e3}{\textbf{-46.3}} & \cellcolor[HTML]{e6e3e3}{\textbf{-49.0}} & \cellcolor[HTML]{e6e3e3}{\textbf{-43.2}} & \cellcolor[HTML]{e6e3e3}{\textbf{-35.9}} & \cellcolor[HTML]{e6e3e3}{\textbf{ 2.2}} & \cellcolor[HTML]{e6e3e3}{\textbf{ 1.0}} & \cellcolor[HTML]{e6e3e3}{\textbf{ 0.7}} & \cellcolor[HTML]{e6e3e3}{\textbf{ 0.5}} & \cellcolor[HTML]{e6e3e3}{\textbf{-21.5}} & \cellcolor[HTML]{e6e3e3}{\textbf{-23.4}} & \cellcolor[HTML]{e6e3e3}{\textbf{-19.4}} & \cellcolor[HTML]{e6e3e3}{\textbf{-15.0}}\\
\cellcolor[HTML]{e6e3e3}{OLS-intuitive} & \cellcolor[HTML]{e6e3e3}{\textbf{-0.5}} & \cellcolor[HTML]{e6e3e3}{\textbf{ 0.5}} & \cellcolor[HTML]{e6e3e3}{\textbf{ 0.6}} & \cellcolor[HTML]{e6e3e3}{\textbf{ 0.6}} & \cellcolor[HTML]{e6e3e3}{\textbf{-46.5}} & \cellcolor[HTML]{e6e3e3}{\textbf{-49.0}} & \cellcolor[HTML]{e6e3e3}{\textbf{-43.2}} & \cellcolor[HTML]{e6e3e3}{\textbf{-36.0}} & \cellcolor[HTML]{e6e3e3}{\textbf{ 2.2}} & \cellcolor[HTML]{e6e3e3}{\textbf{ 1.2}} & \cellcolor[HTML]{e6e3e3}{\textbf{ 0.7}} & \cellcolor[HTML]{e6e3e3}{\textbf{ 0.5}} & \cellcolor[HTML]{e6e3e3}{\textbf{-21.6}} & \cellcolor[HTML]{e6e3e3}{\textbf{-23.4}} & \cellcolor[HTML]{e6e3e3}{\textbf{-19.4}} & \cellcolor[HTML]{e6e3e3}{\textbf{-15.0}}\\
\cellcolor[HTML]{e6e3e3}{OLS-lasso} & \cellcolor[HTML]{e6e3e3}{\textbf{-0.2}} & \cellcolor[HTML]{e6e3e3}{\textbf{ 1.5}} & \cellcolor[HTML]{e6e3e3}{\textbf{ 1.4}} & \cellcolor[HTML]{e6e3e3}{\textbf{ 1.3}} & \cellcolor[HTML]{e6e3e3}{\textbf{-46.9}} & \cellcolor[HTML]{e6e3e3}{\textbf{-48.9}} & \cellcolor[HTML]{e6e3e3}{\textbf{-43.1}} & \cellcolor[HTML]{e6e3e3}{\textbf{-35.8}} & \cellcolor[HTML]{e6e3e3}{\textbf{ 0.9}} & \cellcolor[HTML]{e6e3e3}{\textbf{ 0.8}} & \cellcolor[HTML]{e6e3e3}{\textbf{ 0.5}} & \cellcolor[HTML]{e6e3e3}{\textbf{ 0.3}} & \cellcolor[HTML]{e6e3e3}{\textbf{-22.1}} & \cellcolor[HTML]{e6e3e3}{\textbf{-23.3}} & \cellcolor[HTML]{e6e3e3}{\textbf{-19.3}} & \cellcolor[HTML]{e6e3e3}{\textbf{-14.9}}\\
\midrule
WLSs & 12.1 & 18.6 & 14.0 & 10.2 & -34.4 & -35.1 & -31.7 & -26.9 & 15.6 & 17.0 & 12.0 & 8.0 & -9.0 & -8.0 & -7.6 & -6.5\\
\cellcolor[HTML]{e6e3e3}{WLSs-subset} & \cellcolor[HTML]{e6e3e3}{\textbf{-0.1}} & \cellcolor[HTML]{e6e3e3}{\textbf{ 1.2}} & \cellcolor[HTML]{e6e3e3}{\textbf{ 1.1}} & \cellcolor[HTML]{e6e3e3}{\textbf{ 1.1}} & \cellcolor[HTML]{e6e3e3}{\textbf{-46.7}} & \cellcolor[HTML]{e6e3e3}{\textbf{-48.8}} & \cellcolor[HTML]{e6e3e3}{\textbf{-43.1}} & \cellcolor[HTML]{e6e3e3}{\textbf{-35.8}} & \cellcolor[HTML]{e6e3e3}{\textbf{ 1.5}} & \cellcolor[HTML]{e6e3e3}{\textbf{ 1.1}} & \cellcolor[HTML]{e6e3e3}{\textbf{ 0.8}} & \cellcolor[HTML]{e6e3e3}{\textbf{ 0.6}} & \cellcolor[HTML]{e6e3e3}{\textbf{-21.8}} & \cellcolor[HTML]{e6e3e3}{\textbf{-23.2}} & \cellcolor[HTML]{e6e3e3}{\textbf{-19.2}} & \cellcolor[HTML]{e6e3e3}{\textbf{-14.8}}\\
\cellcolor[HTML]{e6e3e3}{WLSs-intuitive} & \cellcolor[HTML]{e6e3e3}{\textbf{ 0.0}} & \cellcolor[HTML]{e6e3e3}{\textbf{ 1.2}} & \cellcolor[HTML]{e6e3e3}{\textbf{ 1.0}} & \cellcolor[HTML]{e6e3e3}{\textbf{ 0.9}} & \cellcolor[HTML]{e6e3e3}{\textbf{-46.5}} & \cellcolor[HTML]{e6e3e3}{\textbf{-48.8}} & \cellcolor[HTML]{e6e3e3}{\textbf{-43.1}} & \cellcolor[HTML]{e6e3e3}{\textbf{-35.9}} & \cellcolor[HTML]{e6e3e3}{\textbf{ 1.7}} & \cellcolor[HTML]{e6e3e3}{\textbf{ 1.3}} & \cellcolor[HTML]{e6e3e3}{\textbf{ 0.9}} & \cellcolor[HTML]{e6e3e3}{\textbf{ 0.6}} & \cellcolor[HTML]{e6e3e3}{\textbf{-21.6}} & \cellcolor[HTML]{e6e3e3}{\textbf{-23.1}} & \cellcolor[HTML]{e6e3e3}{\textbf{-19.2}} & \cellcolor[HTML]{e6e3e3}{\textbf{-14.9}}\\
\cellcolor[HTML]{e6e3e3}{WLSs-lasso} & \cellcolor[HTML]{e6e3e3}{\textbf{-0.1}} & \cellcolor[HTML]{e6e3e3}{\textbf{ 1.5}} & \cellcolor[HTML]{e6e3e3}{\textbf{ 1.5}} & \cellcolor[HTML]{e6e3e3}{\textbf{ 1.3}} & \cellcolor[HTML]{e6e3e3}{\textbf{-46.7}} & \cellcolor[HTML]{e6e3e3}{\textbf{-48.9}} & \cellcolor[HTML]{e6e3e3}{\textbf{-43.1}} & \cellcolor[HTML]{e6e3e3}{\textbf{-35.8}} & \cellcolor[HTML]{e6e3e3}{\textbf{ 0.9}} & \cellcolor[HTML]{e6e3e3}{\textbf{ 0.8}} & \cellcolor[HTML]{e6e3e3}{\textbf{ 0.5}} & \cellcolor[HTML]{e6e3e3}{\textbf{ 0.3}} & \cellcolor[HTML]{e6e3e3}{\textbf{-22.0}} & \cellcolor[HTML]{e6e3e3}{\textbf{-23.2}} & \cellcolor[HTML]{e6e3e3}{\textbf{-19.3}} & \cellcolor[HTML]{e6e3e3}{\textbf{-14.9}}\\
\midrule
WLSv & -0.8 & 2.3 & 1.8 & 1.6 & -46.3 & -47.9 & -42.3 & -35.2 & 1.6 & 1.9 & 1.2 & 0.8 & -21.7 & -22.2 & -18.6 & -14.4\\
\cellcolor[HTML]{e6e3e3}{WLSv-subset} & \cellcolor[HTML]{e6e3e3}{-0.7} & \cellcolor[HTML]{e6e3e3}{\textbf{ 1.3}} & \cellcolor[HTML]{e6e3e3}{\textbf{ 1.4}} & \cellcolor[HTML]{e6e3e3}{\textbf{ 1.4}} & \cellcolor[HTML]{e6e3e3}{\textbf{-46.9}} & \cellcolor[HTML]{e6e3e3}{\textbf{-48.7}} & \cellcolor[HTML]{e6e3e3}{\textbf{-42.9}} & \cellcolor[HTML]{e6e3e3}{\textbf{-35.6}} & \cellcolor[HTML]{e6e3e3}{\textbf{ 1.0}} & \cellcolor[HTML]{e6e3e3}{\textbf{ 1.0}} & \cellcolor[HTML]{e6e3e3}{\textbf{ 0.8}} & \cellcolor[HTML]{e6e3e3}{\textbf{ 0.6}} & \cellcolor[HTML]{e6e3e3}{\textbf{-22.2}} & \cellcolor[HTML]{e6e3e3}{\textbf{-23.1}} & \cellcolor[HTML]{e6e3e3}{\textbf{-19.1}} & \cellcolor[HTML]{e6e3e3}{\textbf{-14.7}}\\
\cellcolor[HTML]{e6e3e3}{WLSv-intuitive} & \cellcolor[HTML]{e6e3e3}{-0.4} & \cellcolor[HTML]{e6e3e3}{\textbf{ 1.5}} & \cellcolor[HTML]{e6e3e3}{\textbf{ 1.4}} & \cellcolor[HTML]{e6e3e3}{\textbf{ 1.2}} & \cellcolor[HTML]{e6e3e3}{\textbf{-46.9}} & \cellcolor[HTML]{e6e3e3}{\textbf{-48.6}} & \cellcolor[HTML]{e6e3e3}{\textbf{-42.8}} & \cellcolor[HTML]{e6e3e3}{\textbf{-35.6}} & \cellcolor[HTML]{e6e3e3}{\textbf{ 0.9}} & \cellcolor[HTML]{e6e3e3}{\textbf{ 1.2}} & \cellcolor[HTML]{e6e3e3}{\textbf{ 0.9}} & \cellcolor[HTML]{e6e3e3}{\textbf{ 0.7}} & \cellcolor[HTML]{e6e3e3}{\textbf{-22.2}} & \cellcolor[HTML]{e6e3e3}{\textbf{-23.0}} & \cellcolor[HTML]{e6e3e3}{\textbf{-19.0}} & \cellcolor[HTML]{e6e3e3}{\textbf{-14.7}}\\
\cellcolor[HTML]{e6e3e3}{WLSv-lasso} & \cellcolor[HTML]{e6e3e3}{-0.6} & \cellcolor[HTML]{e6e3e3}{\textbf{ 1.3}} & \cellcolor[HTML]{e6e3e3}{\textbf{ 1.3}} & \cellcolor[HTML]{e6e3e3}{\textbf{ 1.3}} & \cellcolor[HTML]{e6e3e3}{\textbf{-47.2}} & \cellcolor[HTML]{e6e3e3}{\textbf{-48.9}} & \cellcolor[HTML]{e6e3e3}{\textbf{-43.0}} & \cellcolor[HTML]{e6e3e3}{\textbf{-35.7}} & \cellcolor[HTML]{e6e3e3}{\textbf{ 0.6}} & \cellcolor[HTML]{e6e3e3}{\textbf{ 0.8}} & \cellcolor[HTML]{e6e3e3}{\textbf{ 0.5}} & \cellcolor[HTML]{e6e3e3}{\textbf{ 0.4}} & \cellcolor[HTML]{e6e3e3}{\textbf{-22.4}} & \cellcolor[HTML]{e6e3e3}{\textbf{-23.3}} & \cellcolor[HTML]{e6e3e3}{\textbf{-19.2}} & \cellcolor[HTML]{e6e3e3}{\textbf{-14.8}}\\
\midrule
MinT & 0.2 & 0.5 & 0.6 & 0.5 & -47.5 & -49.4 & -43.5 & -36.1 & 1.1 & 0.5 & 0.3 & 0.1 & -22.3 & -23.7 & -19.6 & \textcolor{blue}{\textbf{-15.3}}\\
\cellcolor[HTML]{e6e3e3}{MinT-subset} & \cellcolor[HTML]{e6e3e3}{\textbf{-0.1}} & \cellcolor[HTML]{e6e3e3}{0.8} & \cellcolor[HTML]{e6e3e3}{0.9} & \cellcolor[HTML]{e6e3e3}{0.9} & \cellcolor[HTML]{e6e3e3}{-46.9} & \cellcolor[HTML]{e6e3e3}{-49.1} & \cellcolor[HTML]{e6e3e3}{-43.3} & \cellcolor[HTML]{e6e3e3}{-36.0} & \cellcolor[HTML]{e6e3e3}{1.7} & \cellcolor[HTML]{e6e3e3}{0.9} & \cellcolor[HTML]{e6e3e3}{0.5} & \cellcolor[HTML]{e6e3e3}{0.3} & \cellcolor[HTML]{e6e3e3}{-21.9} & \cellcolor[HTML]{e6e3e3}{-23.4} & \cellcolor[HTML]{e6e3e3}{-19.4} & \cellcolor[HTML]{e6e3e3}{-15.1}\\
\cellcolor[HTML]{e6e3e3}{MinT-intuitive} & \cellcolor[HTML]{e6e3e3}{0.2} & \cellcolor[HTML]{e6e3e3}{0.5} & \cellcolor[HTML]{e6e3e3}{0.6} & \cellcolor[HTML]{e6e3e3}{0.5} & \cellcolor[HTML]{e6e3e3}{-47.5} & \cellcolor[HTML]{e6e3e3}{-49.4} & \cellcolor[HTML]{e6e3e3}{-43.5} & \cellcolor[HTML]{e6e3e3}{-36.1} & \cellcolor[HTML]{e6e3e3}{1.1} & \cellcolor[HTML]{e6e3e3}{0.5} & \cellcolor[HTML]{e6e3e3}{0.3} & \cellcolor[HTML]{e6e3e3}{0.1} & \cellcolor[HTML]{e6e3e3}{-22.3} & \cellcolor[HTML]{e6e3e3}{-23.7} & \cellcolor[HTML]{e6e3e3}{-19.6} & \cellcolor[HTML]{e6e3e3}{\textcolor{blue}{\textbf{-15.3}}}\\
\cellcolor[HTML]{e6e3e3}{MinT-lasso} & \cellcolor[HTML]{e6e3e3}{\textbf{-0.3}} & \cellcolor[HTML]{e6e3e3}{\textbf{ 0.3}} & \cellcolor[HTML]{e6e3e3}{0.6} & \cellcolor[HTML]{e6e3e3}{0.5} & \cellcolor[HTML]{e6e3e3}{\textbf{-47.6}} & \cellcolor[HTML]{e6e3e3}{-49.4} & \cellcolor[HTML]{e6e3e3}{-43.5} & \cellcolor[HTML]{e6e3e3}{-36.1} & \cellcolor[HTML]{e6e3e3}{\textbf{ 0.8}} & \cellcolor[HTML]{e6e3e3}{\textbf{ 0.3}} & \cellcolor[HTML]{e6e3e3}{\textbf{ 0.2}} & \cellcolor[HTML]{e6e3e3}{0.1} & \cellcolor[HTML]{e6e3e3}{\textbf{-22.5}} & \cellcolor[HTML]{e6e3e3}{\textbf{-23.9}} & \cellcolor[HTML]{e6e3e3}{\textbf{-19.7}} & \cellcolor[HTML]{e6e3e3}{\textcolor{blue}{\textbf{-15.3}}}\\
\midrule
MinTs & -0.3 & 0.3 & \textcolor{blue}{\textbf{ 0.4}} & \textcolor{blue}{\textbf{ 0.4}} & -47.6 & -49.5 & \textcolor{blue}{\textbf{-43.6}} & \textcolor{blue}{\textbf{-36.2}} & 0.7 & 0.2 & 0.1 & \textcolor{blue}{\textbf{ 0.0}} & -22.6 & -23.9 & \textcolor{blue}{\textbf{-19.8}} & \textcolor{blue}{\textbf{-15.3}}\\
\cellcolor[HTML]{e6e3e3}{MinTs-subset} & \cellcolor[HTML]{e6e3e3}{\textbf{-0.8}} & \cellcolor[HTML]{e6e3e3}{0.5} & \cellcolor[HTML]{e6e3e3}{0.8} & \cellcolor[HTML]{e6e3e3}{0.8} & \cellcolor[HTML]{e6e3e3}{-47.2} & \cellcolor[HTML]{e6e3e3}{-49.2} & \cellcolor[HTML]{e6e3e3}{-43.4} & \cellcolor[HTML]{e6e3e3}{-36.0} & \cellcolor[HTML]{e6e3e3}{1.0} & \cellcolor[HTML]{e6e3e3}{0.7} & \cellcolor[HTML]{e6e3e3}{0.4} & \cellcolor[HTML]{e6e3e3}{0.3} & \cellcolor[HTML]{e6e3e3}{-22.3} & \cellcolor[HTML]{e6e3e3}{-23.6} & \cellcolor[HTML]{e6e3e3}{-19.5} & \cellcolor[HTML]{e6e3e3}{-15.1}\\
\cellcolor[HTML]{e6e3e3}{MinTs-intuitive} & \cellcolor[HTML]{e6e3e3}{-0.3} & \cellcolor[HTML]{e6e3e3}{0.3} & \cellcolor[HTML]{e6e3e3}{\textcolor{blue}{\textbf{ 0.4}}} & \cellcolor[HTML]{e6e3e3}{\textcolor{blue}{\textbf{ 0.4}}} & \cellcolor[HTML]{e6e3e3}{-47.6} & \cellcolor[HTML]{e6e3e3}{-49.5} & \cellcolor[HTML]{e6e3e3}{\textcolor{blue}{\textbf{-43.6}}} & \cellcolor[HTML]{e6e3e3}{\textcolor{blue}{\textbf{-36.2}}} & \cellcolor[HTML]{e6e3e3}{0.7} & \cellcolor[HTML]{e6e3e3}{0.2} & \cellcolor[HTML]{e6e3e3}{0.1} & \cellcolor[HTML]{e6e3e3}{\textcolor{blue}{\textbf{ 0.0}}} & \cellcolor[HTML]{e6e3e3}{-22.6} & \cellcolor[HTML]{e6e3e3}{-23.9} & \cellcolor[HTML]{e6e3e3}{\textcolor{blue}{\textbf{-19.8}}} & \cellcolor[HTML]{e6e3e3}{\textcolor{blue}{\textbf{-15.3}}}\\
\cellcolor[HTML]{e6e3e3}{MinTs-lasso} & \cellcolor[HTML]{e6e3e3}{\textbf{-0.9}} & \cellcolor[HTML]{e6e3e3}{\textcolor{blue}{\textbf{ 0.2}}} & \cellcolor[HTML]{e6e3e3}{0.5} & \cellcolor[HTML]{e6e3e3}{0.5} & \cellcolor[HTML]{e6e3e3}{\textcolor{blue}{\textbf{-47.7}}} & \cellcolor[HTML]{e6e3e3}{-49.5} & \cellcolor[HTML]{e6e3e3}{\textcolor{blue}{\textbf{-43.6}}} & \cellcolor[HTML]{e6e3e3}{\textcolor{blue}{\textbf{-36.2}}} & \cellcolor[HTML]{e6e3e3}{\textbf{ 0.5}} & \cellcolor[HTML]{e6e3e3}{0.2} & \cellcolor[HTML]{e6e3e3}{0.1} & \cellcolor[HTML]{e6e3e3}{0.1} & \cellcolor[HTML]{e6e3e3}{\textbf{-22.8}} & \cellcolor[HTML]{e6e3e3}{\textcolor{blue}{\textbf{-24.0}}} & \cellcolor[HTML]{e6e3e3}{\textcolor{blue}{\textbf{-19.8}}} & \cellcolor[HTML]{e6e3e3}{\textcolor{blue}{\textbf{-15.3}}}\\
\midrule
EMinT & 2.2 & 2.9 & 2.5 & 1.7 & -46.2 & -48.1 & -42.4 & -35.3 & 3.6 & 2.9 & 2.0 & 1.1 & -20.5 & -21.9 & -18.2 & -14.3\\
\cellcolor[HTML]{e6e3e3}{Elasso} & \cellcolor[HTML]{e6e3e3}{\textbf{ 1.4}} & \cellcolor[HTML]{e6e3e3}{\textbf{ 2.7}} & \cellcolor[HTML]{e6e3e3}{\textbf{ 2.4}} & \cellcolor[HTML]{e6e3e3}{\textbf{ 1.6}} & \cellcolor[HTML]{e6e3e3}{\textbf{-46.4}} & \cellcolor[HTML]{e6e3e3}{\textbf{-48.2}} & \cellcolor[HTML]{e6e3e3}{-42.4} & \cellcolor[HTML]{e6e3e3}{\textbf{-35.4}} & \cellcolor[HTML]{e6e3e3}{\textbf{ 3.1}} & \cellcolor[HTML]{e6e3e3}{3.2} & \cellcolor[HTML]{e6e3e3}{2.1} & \cellcolor[HTML]{e6e3e3}{1.2} & \cellcolor[HTML]{e6e3e3}{\textbf{-20.9}} & \cellcolor[HTML]{e6e3e3}{-21.9} & \cellcolor[HTML]{e6e3e3}{-18.2} & \cellcolor[HTML]{e6e3e3}{-14.3}\\
\bottomrule
\end{tabular}
\begin{tablenotes}[para]
\item \underline{\textit{NOTE:}} 
\item The Base row shows the average RMSE of the base forecasts. Entries below this row indicate the percentage decrease (negative) or increase (positive) in the average RMSE of the reconciled forecasts compared to the base forecasts. The entries with the lowest values in each column are highlighted in blue. In each panel, the proposed methods are indicated with a gray background, and methods that outperform the benchmark method are marked in bold.
\end{tablenotes}
\end{threeparttable}}
\end{table}

\hypertarget{tbl-s3-rmse}{}
\begin{table}[!h]
\caption{\label{tbl-s3-rmse}Out-of-sample forecast results for the simulated data in Scenario III,
Setup 1. }\tabularnewline

\centering
\resizebox{\linewidth}{!}{
\begin{threeparttable}
\begin{tabular}{lrrrrlrrrrlrrrrlr}
\toprule
\multicolumn{1}{c}{} & \multicolumn{4}{c}{Top} & \multicolumn{4}{c}{Middle} & \multicolumn{4}{c}{Bottom} & \multicolumn{4}{c}{Average} \\
\cmidrule(l{3pt}r{3pt}){2-5} \cmidrule(l{3pt}r{3pt}){6-9} \cmidrule(l{3pt}r{3pt}){10-13} \cmidrule(l{3pt}r{3pt}){14-17}
Method & h=1 & 1-4 & 1-8 & 1-16 & h=1 & 1-4 & 1-8 & 1-16 & h=1 & 1-4 & 1-8 & 1-16 & h=1 & 1-4 & 1-8 & 1-16\\
\midrule
Base & 25.0 & 30.3 & 30.9 & 32.3 & 6.3 & 7.3 & 8.6 & 10.8 & 4.2 & 4.9 & 5.9 & 7.5 & 7.8 & 9.2 & 10.3 & 12.0\\
BU & -62.0 & \textcolor{blue}{\textbf{-64.4}} & \textcolor{blue}{\textbf{-59.0}} & -51.5 & \textcolor{blue}{\textbf{-0.3}} & \textcolor{blue}{\textbf{ 0.0}} & \textcolor{blue}{\textbf{ 0.1}} & \textcolor{blue}{\textbf{ 0.0}} & \textcolor{blue}{\textbf{ 0.0}} & \textcolor{blue}{\textbf{ 0.0}} & \textcolor{blue}{\textbf{ 0.0}} & \textcolor{blue}{\textbf{ 0.0}} & \textcolor{blue}{\textbf{-28.5}} & \textcolor{blue}{\textbf{-30.2}} & \textcolor{blue}{\textbf{-25.3}} & \textcolor{blue}{\textbf{-19.8}}\\
\midrule
OLS & -34.8 & -35.5 & -33.5 & -30.1 & 45.3 & 50.6 & 37.7 & 25.1 & 27.7 & 29.9 & 21.2 & 13.7 & 3.1 & 3.8 & 1.6 & -0.2\\
\cellcolor[HTML]{e6e3e3}{OLS-subset} & \cellcolor[HTML]{e6e3e3}{\textbf{-35.3}} & \cellcolor[HTML]{e6e3e3}{\textbf{-41.9}} & \cellcolor[HTML]{e6e3e3}{\textbf{-39.2}} & \cellcolor[HTML]{e6e3e3}{\textbf{-35.0}} & \cellcolor[HTML]{e6e3e3}{\textbf{43.9}} & \cellcolor[HTML]{e6e3e3}{\textbf{39.5}} & \cellcolor[HTML]{e6e3e3}{\textbf{29.5}} & \cellcolor[HTML]{e6e3e3}{\textbf{19.6}} & \cellcolor[HTML]{e6e3e3}{\textbf{27.1}} & \cellcolor[HTML]{e6e3e3}{\textbf{23.6}} & \cellcolor[HTML]{e6e3e3}{\textbf{16.8}} & \cellcolor[HTML]{e6e3e3}{\textbf{10.9}} & \cellcolor[HTML]{e6e3e3}{\textbf{  2.4}} & \cellcolor[HTML]{e6e3e3}{\textbf{ -3.5}} & \cellcolor[HTML]{e6e3e3}{\textbf{ -4.2}} & \cellcolor[HTML]{e6e3e3}{\textbf{ -4.5}}\\
\cellcolor[HTML]{e6e3e3}{OLS-intuitive} & \cellcolor[HTML]{e6e3e3}{\textbf{-41.2}} & \cellcolor[HTML]{e6e3e3}{\textbf{-49.2}} & \cellcolor[HTML]{e6e3e3}{\textbf{-45.5}} & \cellcolor[HTML]{e6e3e3}{\textbf{-40.0}} & \cellcolor[HTML]{e6e3e3}{\textbf{35.1}} & \cellcolor[HTML]{e6e3e3}{\textbf{26.8}} & \cellcolor[HTML]{e6e3e3}{\textbf{20.3}} & \cellcolor[HTML]{e6e3e3}{\textbf{13.7}} & \cellcolor[HTML]{e6e3e3}{\textbf{21.9}} & \cellcolor[HTML]{e6e3e3}{\textbf{15.9}} & \cellcolor[HTML]{e6e3e3}{\textbf{11.5}} & \cellcolor[HTML]{e6e3e3}{\textbf{ 7.6}} & \cellcolor[HTML]{e6e3e3}{\textbf{ -4.0}} & \cellcolor[HTML]{e6e3e3}{\textbf{-12.2}} & \cellcolor[HTML]{e6e3e3}{\textbf{-10.9}} & \cellcolor[HTML]{e6e3e3}{\textbf{ -9.1}}\\
\cellcolor[HTML]{e6e3e3}{OLS-lasso} & \cellcolor[HTML]{e6e3e3}{\textbf{-61.8}} & \cellcolor[HTML]{e6e3e3}{\textbf{-63.6}} & \cellcolor[HTML]{e6e3e3}{\textbf{-58.1}} & \cellcolor[HTML]{e6e3e3}{\textbf{-50.9}} & \cellcolor[HTML]{e6e3e3}{\textbf{ 0.4}} & \cellcolor[HTML]{e6e3e3}{\textbf{ 1.3}} & \cellcolor[HTML]{e6e3e3}{\textbf{ 1.3}} & \cellcolor[HTML]{e6e3e3}{\textbf{ 0.7}} & \cellcolor[HTML]{e6e3e3}{\textbf{ 0.3}} & \cellcolor[HTML]{e6e3e3}{\textbf{ 0.8}} & \cellcolor[HTML]{e6e3e3}{\textbf{ 0.6}} & \cellcolor[HTML]{e6e3e3}{\textbf{ 0.4}} & \cellcolor[HTML]{e6e3e3}{\textbf{-28.2}} & \cellcolor[HTML]{e6e3e3}{\textbf{-29.3}} & \cellcolor[HTML]{e6e3e3}{\textbf{-24.5}} & \cellcolor[HTML]{e6e3e3}{\textbf{-19.2}}\\
\midrule
WLSs & -50.9 & -52.4 & -48.7 & -43.3 & 17.6 & 20.0 & 14.5 & 9.3 & 9.6 & 11.3 & 7.7 & 4.9 & -16.3 & -16.7 & -14.9 & -12.5\\
\cellcolor[HTML]{e6e3e3}{WLSs-subset} & \cellcolor[HTML]{e6e3e3}{\textbf{-61.8}} & \cellcolor[HTML]{e6e3e3}{\textbf{-63.6}} & \cellcolor[HTML]{e6e3e3}{\textbf{-58.1}} & \cellcolor[HTML]{e6e3e3}{\textbf{-50.7}} & \cellcolor[HTML]{e6e3e3}{\textbf{ 0.3}} & \cellcolor[HTML]{e6e3e3}{\textbf{ 1.4}} & \cellcolor[HTML]{e6e3e3}{\textbf{ 1.4}} & \cellcolor[HTML]{e6e3e3}{\textbf{ 0.9}} & \cellcolor[HTML]{e6e3e3}{\textbf{ 0.3}} & \cellcolor[HTML]{e6e3e3}{\textbf{ 0.9}} & \cellcolor[HTML]{e6e3e3}{\textbf{ 0.7}} & \cellcolor[HTML]{e6e3e3}{\textbf{ 0.6}} & \cellcolor[HTML]{e6e3e3}{\textbf{-28.2}} & \cellcolor[HTML]{e6e3e3}{\textbf{-29.3}} & \cellcolor[HTML]{e6e3e3}{\textbf{-24.4}} & \cellcolor[HTML]{e6e3e3}{\textbf{-19.0}}\\
\cellcolor[HTML]{e6e3e3}{WLSs-intuitive} & \cellcolor[HTML]{e6e3e3}{\textbf{-61.8}} & \cellcolor[HTML]{e6e3e3}{\textbf{-63.8}} & \cellcolor[HTML]{e6e3e3}{\textbf{-58.3}} & \cellcolor[HTML]{e6e3e3}{\textbf{-50.9}} & \cellcolor[HTML]{e6e3e3}{\textbf{ 0.0}} & \cellcolor[HTML]{e6e3e3}{\textbf{ 1.0}} & \cellcolor[HTML]{e6e3e3}{\textbf{ 1.0}} & \cellcolor[HTML]{e6e3e3}{\textbf{ 0.7}} & \cellcolor[HTML]{e6e3e3}{\textbf{ 0.3}} & \cellcolor[HTML]{e6e3e3}{\textbf{ 0.7}} & \cellcolor[HTML]{e6e3e3}{\textbf{ 0.6}} & \cellcolor[HTML]{e6e3e3}{\textbf{ 0.5}} & \cellcolor[HTML]{e6e3e3}{\textbf{-28.3}} & \cellcolor[HTML]{e6e3e3}{\textbf{-29.5}} & \cellcolor[HTML]{e6e3e3}{\textbf{-24.6}} & \cellcolor[HTML]{e6e3e3}{\textbf{-19.2}}\\
\cellcolor[HTML]{e6e3e3}{WLSs-lasso} & \cellcolor[HTML]{e6e3e3}{\textbf{-61.7}} & \cellcolor[HTML]{e6e3e3}{\textbf{-63.5}} & \cellcolor[HTML]{e6e3e3}{\textbf{-58.0}} & \cellcolor[HTML]{e6e3e3}{\textbf{-50.7}} & \cellcolor[HTML]{e6e3e3}{\textbf{ 0.5}} & \cellcolor[HTML]{e6e3e3}{\textbf{ 1.5}} & \cellcolor[HTML]{e6e3e3}{\textbf{ 1.4}} & \cellcolor[HTML]{e6e3e3}{\textbf{ 0.9}} & \cellcolor[HTML]{e6e3e3}{\textbf{ 0.3}} & \cellcolor[HTML]{e6e3e3}{\textbf{ 0.9}} & \cellcolor[HTML]{e6e3e3}{\textbf{ 0.7}} & \cellcolor[HTML]{e6e3e3}{\textbf{ 0.5}} & \cellcolor[HTML]{e6e3e3}{\textbf{-28.1}} & \cellcolor[HTML]{e6e3e3}{\textbf{-29.2}} & \cellcolor[HTML]{e6e3e3}{\textbf{-24.4}} & \cellcolor[HTML]{e6e3e3}{\textbf{-19.1}}\\
\midrule
WLSv & -61.1 & -63.4 & -58.1 & -50.8 & 1.0 & 1.7 & 1.3 & 0.8 & 0.7 & 1.0 & 0.6 & 0.4 & -27.6 & -29.1 & -24.5 & -19.2\\
\cellcolor[HTML]{e6e3e3}{WLSv-subset} & \cellcolor[HTML]{e6e3e3}{\textbf{-61.9}} & \cellcolor[HTML]{e6e3e3}{\textbf{-63.6}} & \cellcolor[HTML]{e6e3e3}{\textbf{-58.2}} & \cellcolor[HTML]{e6e3e3}{\textbf{-50.9}} & \cellcolor[HTML]{e6e3e3}{\textbf{ 0.2}} & \cellcolor[HTML]{e6e3e3}{\textbf{ 1.3}} & \cellcolor[HTML]{e6e3e3}{\textbf{ 1.2}} & \cellcolor[HTML]{e6e3e3}{0.8} & \cellcolor[HTML]{e6e3e3}{\textbf{ 0.1}} & \cellcolor[HTML]{e6e3e3}{\textbf{ 0.8}} & \cellcolor[HTML]{e6e3e3}{0.6} & \cellcolor[HTML]{e6e3e3}{0.5} & \cellcolor[HTML]{e6e3e3}{\textbf{-28.3}} & \cellcolor[HTML]{e6e3e3}{\textbf{-29.3}} & \cellcolor[HTML]{e6e3e3}{-24.5} & \cellcolor[HTML]{e6e3e3}{-19.2}\\
\cellcolor[HTML]{e6e3e3}{WLSv-intuitive} & \cellcolor[HTML]{e6e3e3}{\textbf{-61.8}} & \cellcolor[HTML]{e6e3e3}{\textbf{-63.8}} & \cellcolor[HTML]{e6e3e3}{\textbf{-58.3}} & \cellcolor[HTML]{e6e3e3}{\textbf{-51.0}} & \cellcolor[HTML]{e6e3e3}{\textbf{ 0.0}} & \cellcolor[HTML]{e6e3e3}{\textbf{ 1.1}} & \cellcolor[HTML]{e6e3e3}{\textbf{ 1.1}} & \cellcolor[HTML]{e6e3e3}{\textbf{ 0.6}} & \cellcolor[HTML]{e6e3e3}{\textbf{ 0.1}} & \cellcolor[HTML]{e6e3e3}{\textbf{ 0.6}} & \cellcolor[HTML]{e6e3e3}{\textbf{ 0.5}} & \cellcolor[HTML]{e6e3e3}{0.4} & \cellcolor[HTML]{e6e3e3}{\textbf{-28.4}} & \cellcolor[HTML]{e6e3e3}{\textbf{-29.5}} & \cellcolor[HTML]{e6e3e3}{\textbf{-24.7}} & \cellcolor[HTML]{e6e3e3}{\textbf{-19.3}}\\
\cellcolor[HTML]{e6e3e3}{WLSv-lasso} & \cellcolor[HTML]{e6e3e3}{\textbf{-61.8}} & \cellcolor[HTML]{e6e3e3}{\textbf{-63.9}} & \cellcolor[HTML]{e6e3e3}{\textbf{-58.4}} & \cellcolor[HTML]{e6e3e3}{\textbf{-51.1}} & \cellcolor[HTML]{e6e3e3}{\textbf{ 0.2}} & \cellcolor[HTML]{e6e3e3}{\textbf{ 0.9}} & \cellcolor[HTML]{e6e3e3}{\textbf{ 0.9}} & \cellcolor[HTML]{e6e3e3}{\textbf{ 0.5}} & \cellcolor[HTML]{e6e3e3}{\textbf{ 0.1}} & \cellcolor[HTML]{e6e3e3}{\textbf{ 0.5}} & \cellcolor[HTML]{e6e3e3}{\textbf{ 0.4}} & \cellcolor[HTML]{e6e3e3}{\textbf{ 0.3}} & \cellcolor[HTML]{e6e3e3}{\textbf{-28.3}} & \cellcolor[HTML]{e6e3e3}{\textbf{-29.6}} & \cellcolor[HTML]{e6e3e3}{\textbf{-24.8}} & \cellcolor[HTML]{e6e3e3}{\textbf{-19.4}}\\
\midrule
MinT & -62.1 & -64.3 & -58.9 & \textcolor{blue}{\textbf{-51.6}} & -0.2 & 0.6 & 0.5 & 0.2 & 0.8 & 0.5 & 0.3 & 0.1 & -28.3 & -29.9 & -25.1 & \textcolor{blue}{\textbf{-19.8}}\\
\cellcolor[HTML]{e6e3e3}{MinT-subset} & \cellcolor[HTML]{e6e3e3}{-61.8} & \cellcolor[HTML]{e6e3e3}{-63.7} & \cellcolor[HTML]{e6e3e3}{-58.2} & \cellcolor[HTML]{e6e3e3}{-50.9} & \cellcolor[HTML]{e6e3e3}{0.4} & \cellcolor[HTML]{e6e3e3}{1.2} & \cellcolor[HTML]{e6e3e3}{1.3} & \cellcolor[HTML]{e6e3e3}{0.8} & \cellcolor[HTML]{e6e3e3}{0.8} & \cellcolor[HTML]{e6e3e3}{1.0} & \cellcolor[HTML]{e6e3e3}{0.7} & \cellcolor[HTML]{e6e3e3}{0.5} & \cellcolor[HTML]{e6e3e3}{-28.0} & \cellcolor[HTML]{e6e3e3}{-29.3} & \cellcolor[HTML]{e6e3e3}{-24.5} & \cellcolor[HTML]{e6e3e3}{-19.2}\\
\cellcolor[HTML]{e6e3e3}{MinT-intuitive} & \cellcolor[HTML]{e6e3e3}{-62.1} & \cellcolor[HTML]{e6e3e3}{-64.3} & \cellcolor[HTML]{e6e3e3}{-58.9} & \cellcolor[HTML]{e6e3e3}{\textcolor{blue}{\textbf{-51.6}}} & \cellcolor[HTML]{e6e3e3}{-0.2} & \cellcolor[HTML]{e6e3e3}{0.6} & \cellcolor[HTML]{e6e3e3}{0.5} & \cellcolor[HTML]{e6e3e3}{0.2} & \cellcolor[HTML]{e6e3e3}{0.8} & \cellcolor[HTML]{e6e3e3}{0.5} & \cellcolor[HTML]{e6e3e3}{0.3} & \cellcolor[HTML]{e6e3e3}{0.1} & \cellcolor[HTML]{e6e3e3}{-28.3} & \cellcolor[HTML]{e6e3e3}{-29.9} & \cellcolor[HTML]{e6e3e3}{-25.1} & \cellcolor[HTML]{e6e3e3}{\textcolor{blue}{\textbf{-19.8}}}\\
\cellcolor[HTML]{e6e3e3}{MinT-lasso} & \cellcolor[HTML]{e6e3e3}{-62.1} & \cellcolor[HTML]{e6e3e3}{\textcolor{blue}{\textbf{-64.4}}} & \cellcolor[HTML]{e6e3e3}{-58.9} & \cellcolor[HTML]{e6e3e3}{-51.5} & \cellcolor[HTML]{e6e3e3}{\textcolor{blue}{\textbf{-0.3}}} & \cellcolor[HTML]{e6e3e3}{\textbf{ 0.3}} & \cellcolor[HTML]{e6e3e3}{\textbf{ 0.4}} & \cellcolor[HTML]{e6e3e3}{\textbf{ 0.1}} & \cellcolor[HTML]{e6e3e3}{\textbf{ 0.6}} & \cellcolor[HTML]{e6e3e3}{\textbf{ 0.3}} & \cellcolor[HTML]{e6e3e3}{\textbf{ 0.1}} & \cellcolor[HTML]{e6e3e3}{0.1} & \cellcolor[HTML]{e6e3e3}{\textbf{-28.4}} & \cellcolor[HTML]{e6e3e3}{\textbf{-30.1}} & \cellcolor[HTML]{e6e3e3}{\textbf{-25.2}} & \cellcolor[HTML]{e6e3e3}{\textcolor{blue}{\textbf{-19.8}}}\\
\midrule
MinTs & \textcolor{blue}{\textbf{-62.2}} & \textcolor{blue}{\textbf{-64.4}} & \textcolor{blue}{\textbf{-59.0}} & \textcolor{blue}{\textbf{-51.6}} & \textcolor{blue}{\textbf{-0.3}} & 0.3 & 0.4 & 0.1 & 0.4 & 0.3 & 0.1 & \textcolor{blue}{\textbf{ 0.0}} & \textcolor{blue}{\textbf{-28.5}} & -30.1 & -25.2 & \textcolor{blue}{\textbf{-19.8}}\\
\cellcolor[HTML]{e6e3e3}{MinTs-subset} & \cellcolor[HTML]{e6e3e3}{-62.0} & \cellcolor[HTML]{e6e3e3}{-63.8} & \cellcolor[HTML]{e6e3e3}{-58.4} & \cellcolor[HTML]{e6e3e3}{-51.1} & \cellcolor[HTML]{e6e3e3}{0.4} & \cellcolor[HTML]{e6e3e3}{1.1} & \cellcolor[HTML]{e6e3e3}{1.2} & \cellcolor[HTML]{e6e3e3}{0.7} & \cellcolor[HTML]{e6e3e3}{0.5} & \cellcolor[HTML]{e6e3e3}{0.9} & \cellcolor[HTML]{e6e3e3}{0.7} & \cellcolor[HTML]{e6e3e3}{0.5} & \cellcolor[HTML]{e6e3e3}{-28.2} & \cellcolor[HTML]{e6e3e3}{-29.5} & \cellcolor[HTML]{e6e3e3}{-24.6} & \cellcolor[HTML]{e6e3e3}{-19.3}\\
\cellcolor[HTML]{e6e3e3}{MinTs-intuitive} & \cellcolor[HTML]{e6e3e3}{\textcolor{blue}{\textbf{-62.2}}} & \cellcolor[HTML]{e6e3e3}{\textcolor{blue}{\textbf{-64.4}}} & \cellcolor[HTML]{e6e3e3}{\textcolor{blue}{\textbf{-59.0}}} & \cellcolor[HTML]{e6e3e3}{\textcolor{blue}{\textbf{-51.6}}} & \cellcolor[HTML]{e6e3e3}{\textcolor{blue}{\textbf{-0.3}}} & \cellcolor[HTML]{e6e3e3}{0.3} & \cellcolor[HTML]{e6e3e3}{0.4} & \cellcolor[HTML]{e6e3e3}{0.1} & \cellcolor[HTML]{e6e3e3}{0.4} & \cellcolor[HTML]{e6e3e3}{0.3} & \cellcolor[HTML]{e6e3e3}{0.1} & \cellcolor[HTML]{e6e3e3}{\textcolor{blue}{\textbf{ 0.0}}} & \cellcolor[HTML]{e6e3e3}{\textcolor{blue}{\textbf{-28.5}}} & \cellcolor[HTML]{e6e3e3}{-30.1} & \cellcolor[HTML]{e6e3e3}{-25.2} & \cellcolor[HTML]{e6e3e3}{\textcolor{blue}{\textbf{-19.8}}}\\
\cellcolor[HTML]{e6e3e3}{MinTs-lasso} & \cellcolor[HTML]{e6e3e3}{\textcolor{blue}{\textbf{-62.2}}} & \cellcolor[HTML]{e6e3e3}{\textcolor{blue}{\textbf{-64.4}}} & \cellcolor[HTML]{e6e3e3}{-58.9} & \cellcolor[HTML]{e6e3e3}{-51.5} & \cellcolor[HTML]{e6e3e3}{-0.2} & \cellcolor[HTML]{e6e3e3}{0.3} & \cellcolor[HTML]{e6e3e3}{0.4} & \cellcolor[HTML]{e6e3e3}{0.1} & \cellcolor[HTML]{e6e3e3}{\textbf{ 0.2}} & \cellcolor[HTML]{e6e3e3}{\textbf{ 0.2}} & \cellcolor[HTML]{e6e3e3}{0.1} & \cellcolor[HTML]{e6e3e3}{\textcolor{blue}{\textbf{ 0.0}}} & \cellcolor[HTML]{e6e3e3}{\textcolor{blue}{\textbf{-28.5}}} & \cellcolor[HTML]{e6e3e3}{-30.1} & \cellcolor[HTML]{e6e3e3}{-25.2} & \cellcolor[HTML]{e6e3e3}{\textcolor{blue}{\textbf{-19.8}}}\\
\midrule
EMinT & -60.7 & -63.5 & -58.2 & -51.0 & 2.5 & 2.9 & 2.3 & 1.3 & 3.6 & 2.9 & 2.0 & 1.1 & -26.2 & -28.3 & -23.8 & -18.9\\
\cellcolor[HTML]{e6e3e3}{Elasso} & \cellcolor[HTML]{e6e3e3}{\textbf{-60.9}} & \cellcolor[HTML]{e6e3e3}{\textbf{-63.6}} & \cellcolor[HTML]{e6e3e3}{-58.2} & \cellcolor[HTML]{e6e3e3}{\textbf{-51.1}} & \cellcolor[HTML]{e6e3e3}{\textbf{ 2.3}} & \cellcolor[HTML]{e6e3e3}{\textbf{ 2.8}} & \cellcolor[HTML]{e6e3e3}{2.3} & \cellcolor[HTML]{e6e3e3}{1.3} & \cellcolor[HTML]{e6e3e3}{\textbf{ 3.1}} & \cellcolor[HTML]{e6e3e3}{3.1} & \cellcolor[HTML]{e6e3e3}{2.1} & \cellcolor[HTML]{e6e3e3}{1.2} & \cellcolor[HTML]{e6e3e3}{\textbf{-26.5}} & \cellcolor[HTML]{e6e3e3}{-28.3} & \cellcolor[HTML]{e6e3e3}{-23.8} & \cellcolor[HTML]{e6e3e3}{-18.9}\\
\bottomrule
\end{tabular}
\begin{tablenotes}[para]
\item \underline{\textit{NOTE:}} 
\item The Base row shows the average RMSE of the base forecasts. Entries below this row indicate the percentage decrease (negative) or increase (positive) in the average RMSE of the reconciled forecasts compared to the base forecasts. The entries with the lowest values in each column are highlighted in blue. In each panel, the proposed methods are indicated with a gray background, and methods that outperform the benchmark method are marked in bold.
\end{tablenotes}
\end{threeparttable}}
\end{table}

\hypertarget{tbl-s2-selection}{}
\begin{table}[!h]
\caption{\label{tbl-s2-selection}Proportion of time series being selected after using the proposed
reconciliation methods with selection in Scenario II, Setup 1. }\tabularnewline

\centering\begingroup\fontsize{7}{9}\selectfont

\resizebox{\linewidth}{!}{
\begin{threeparttable}
\begin{tabular}{llrrrrrr>{}r}
\toprule
  & Top & A & B & AA & AB & BA & BB & Summary\\
\midrule
OLS-subset & 0.55 & 0.04 & 0.41 & 0.74 & 0.78 & 0.79 & 0.83 & \includegraphics[width=0.47in, height=0.1in]{/Users/xwan0362/Git/hfs/paper/_figs/s2_OLS-subset.png}\\
OLS-intuitive & 0.61 & 0.04 & 0.52 & 0.75 & 0.69 & 0.69 & 0.83 & \includegraphics[width=0.47in, height=0.1in]{/Users/xwan0362/Git/hfs/paper/_figs/s2_OLS-intuitive.png}\\
OLS-lasso & 0.04 & 0.35 & 0.02 & 1.00 & 1.00 & 1.00 & 1.00 & \includegraphics[width=0.47in, height=0.1in]{/Users/xwan0362/Git/hfs/paper/_figs/s2_OLS-lasso.png}\\
\midrule
WLSs-subset & 0.45 & 0.06 & 0.36 & 0.81 & 0.84 & 0.81 & 0.87 & \includegraphics[width=0.47in, height=0.1in]{/Users/xwan0362/Git/hfs/paper/_figs/s2_WLSs-subset.png}\\
WLSs-intuitive & 0.61 & 0.06 & 0.48 & 0.75 & 0.71 & 0.73 & 0.84 & \includegraphics[width=0.47in, height=0.1in]{/Users/xwan0362/Git/hfs/paper/_figs/s2_WLSs-intuitive.png}\\
WLSs-lasso & 0.02 & 0.33 & 0.02 & 1.00 & 1.00 & 1.00 & 1.00 & \includegraphics[width=0.47in, height=0.1in]{/Users/xwan0362/Git/hfs/paper/_figs/s2_WLSs-lasso.png}\\
\midrule
WLSv-subset & 0.54 & 0.29 & 0.46 & 0.91 & 0.94 & 0.86 & 0.89 & \includegraphics[width=0.47in, height=0.1in]{/Users/xwan0362/Git/hfs/paper/_figs/s2_WLSv-subset.png}\\
WLSv-intuitive & 0.59 & 0.32 & 0.53 & 0.82 & 0.86 & 0.77 & 0.86 & \includegraphics[width=0.47in, height=0.1in]{/Users/xwan0362/Git/hfs/paper/_figs/s2_WLSv-intuitive.png}\\
WLSv-lasso & 0.27 & 0.42 & 0.26 & 1.00 & 1.00 & 1.00 & 1.00 & \includegraphics[width=0.47in, height=0.1in]{/Users/xwan0362/Git/hfs/paper/_figs/s2_WLSv-lasso.png}\\
\midrule
MinT-subset & 0.69 & 0.64 & 0.66 & 0.95 & 0.96 & 0.90 & 0.90 & \includegraphics[width=0.47in, height=0.1in]{/Users/xwan0362/Git/hfs/paper/_figs/s2_MinT-subset.png}\\
MinT-intuitive & 1.00 & 1.00 & 1.00 & 1.00 & 1.00 & 1.00 & 1.00 & \includegraphics[width=0.47in, height=0.1in]{/Users/xwan0362/Git/hfs/paper/_figs/s2_MinT-intuitive.png}\\
MinT-lasso & 0.82 & 0.74 & 0.83 & 1.00 & 0.99 & 0.97 & 0.97 & \includegraphics[width=0.47in, height=0.1in]{/Users/xwan0362/Git/hfs/paper/_figs/s2_MinT-lasso.png}\\
\midrule
MinTs-subset & 0.62 & 0.63 & 0.58 & 0.95 & 0.96 & 0.90 & 0.86 & \includegraphics[width=0.47in, height=0.1in]{/Users/xwan0362/Git/hfs/paper/_figs/s2_MinTs-subset.png}\\
MinTs-intuitive & 1.00 & 1.00 & 1.00 & 1.00 & 1.00 & 1.00 & 1.00 & \includegraphics[width=0.47in, height=0.1in]{/Users/xwan0362/Git/hfs/paper/_figs/s2_MinTs-intuitive.png}\\
MinTs-lasso & 0.68 & 0.75 & 0.68 & 1.00 & 1.00 & 1.00 & 1.00 & \includegraphics[width=0.47in, height=0.1in]{/Users/xwan0362/Git/hfs/paper/_figs/s2_MinTs-lasso.png}\\
\midrule
Elasso & 0.78 & 0.95 & 0.68 & 1.00 & 1.00 & 1.00 & 1.00 & \includegraphics[width=0.47in, height=0.1in]{/Users/xwan0362/Git/hfs/paper/_figs/s2_Elasso.png}\\
\bottomrule
\end{tabular}
\begin{tablenotes}[para]
\item \underline{\textit{NOTE:}} 
\item The last column displays a stacked barplot for each method, based on the total number of selected series data from 500 simulation instances, with a darker sub-bar indicating a larger number.
\end{tablenotes}
\end{threeparttable}}
\endgroup{}
\end{table}

\hypertarget{tbl-s3-selection}{}
\begin{table}[!h]
\caption{\label{tbl-s3-selection}Proportion of time series being selected after using the proposed
reconciliation methods with selection in Scenario III, Setup 1. }\tabularnewline

\centering\begingroup\fontsize{7}{9}\selectfont

\resizebox{\linewidth}{!}{
\begin{threeparttable}
\begin{tabular}{llrrrrrr>{}r}
\toprule
  & Top & A & B & AA & AB & BA & BB & Summary\\
\midrule
OLS-subset & 0.75 & 0.45 & 0.44 & 0.82 & 0.79 & 0.83 & 0.80 & \includegraphics[width=0.47in, height=0.1in]{/Users/xwan0362/Git/hfs/paper/_figs/s3_OLS-subset.png}\\
OLS-intuitive & 0.47 & 0.70 & 0.69 & 0.86 & 0.92 & 0.90 & 0.89 & \includegraphics[width=0.47in, height=0.1in]{/Users/xwan0362/Git/hfs/paper/_figs/s3_OLS-intuitive.png}\\
OLS-lasso & 0.38 & 0.01 & 0.01 & 1.00 & 1.00 & 1.00 & 1.00 & \includegraphics[width=0.47in, height=0.1in]{/Users/xwan0362/Git/hfs/paper/_figs/s3_OLS-lasso.png}\\
\midrule
WLSs-subset & 0.08 & 0.42 & 0.41 & 0.87 & 0.85 & 0.84 & 0.89 & \includegraphics[width=0.47in, height=0.1in]{/Users/xwan0362/Git/hfs/paper/_figs/s3_WLSs-subset.png}\\
WLSs-intuitive & 0.06 & 0.55 & 0.50 & 0.66 & 0.87 & 0.69 & 0.88 & \includegraphics[width=0.47in, height=0.1in]{/Users/xwan0362/Git/hfs/paper/_figs/s3_WLSs-intuitive.png}\\
WLSs-lasso & 0.35 & 0.03 & 0.03 & 1.00 & 1.00 & 1.00 & 1.00 & \includegraphics[width=0.47in, height=0.1in]{/Users/xwan0362/Git/hfs/paper/_figs/s3_WLSs-lasso.png}\\
\midrule
WLSv-subset & 0.31 & 0.67 & 0.65 & 0.88 & 0.90 & 0.91 & 0.90 & \includegraphics[width=0.47in, height=0.1in]{/Users/xwan0362/Git/hfs/paper/_figs/s3_WLSv-subset.png}\\
WLSv-intuitive & 0.34 & 0.63 & 0.60 & 0.80 & 0.89 & 0.84 & 0.87 & \includegraphics[width=0.47in, height=0.1in]{/Users/xwan0362/Git/hfs/paper/_figs/s3_WLSv-intuitive.png}\\
WLSv-lasso & 0.45 & 0.35 & 0.36 & 1.00 & 1.00 & 1.00 & 1.00 & \includegraphics[width=0.47in, height=0.1in]{/Users/xwan0362/Git/hfs/paper/_figs/s3_WLSv-lasso.png}\\
\midrule
MinT-subset & 0.69 & 0.78 & 0.80 & 0.91 & 0.91 & 0.91 & 0.91 & \includegraphics[width=0.47in, height=0.1in]{/Users/xwan0362/Git/hfs/paper/_figs/s3_MinT-subset.png}\\
MinT-intuitive & 1.00 & 1.00 & 1.00 & 1.00 & 1.00 & 1.00 & 1.00 & \includegraphics[width=0.47in, height=0.1in]{/Users/xwan0362/Git/hfs/paper/_figs/s3_MinT-intuitive.png}\\
MinT-lasso & 0.75 & 0.89 & 0.86 & 0.97 & 0.97 & 0.97 & 0.97 & \includegraphics[width=0.47in, height=0.1in]{/Users/xwan0362/Git/hfs/paper/_figs/s3_MinT-lasso.png}\\
\midrule
MinTs-subset & 0.67 & 0.74 & 0.76 & 0.90 & 0.89 & 0.88 & 0.91 & \includegraphics[width=0.47in, height=0.1in]{/Users/xwan0362/Git/hfs/paper/_figs/s3_MinTs-subset.png}\\
MinTs-intuitive & 1.00 & 1.00 & 1.00 & 1.00 & 1.00 & 1.00 & 1.00 & \includegraphics[width=0.47in, height=0.1in]{/Users/xwan0362/Git/hfs/paper/_figs/s3_MinTs-intuitive.png}\\
MinTs-lasso & 0.77 & 0.72 & 0.73 & 1.00 & 1.00 & 1.00 & 1.00 & \includegraphics[width=0.47in, height=0.1in]{/Users/xwan0362/Git/hfs/paper/_figs/s3_MinTs-lasso.png}\\
\midrule
Elasso & 0.95 & 0.64 & 0.64 & 1.00 & 1.00 & 1.00 & 1.00 & \includegraphics[width=0.47in, height=0.1in]{/Users/xwan0362/Git/hfs/paper/_figs/s3_Elasso.png}\\
\bottomrule
\end{tabular}
\begin{tablenotes}[para]
\item \underline{\textit{NOTE:}} 
\item The last column displays a stacked barplot for each method, based on the total number of selected series data from 500 simulation instances, with a darker sub-bar indicating a larger number.
\end{tablenotes}
\end{threeparttable}}
\endgroup{}
\end{table}

\hypertarget{tbl-corr-selection-pos}{}
\begin{table}[!h]
\caption{\label{tbl-corr-selection-pos}Proportion of time series being selected after using the proposed
reconciliation methods with selection in Setup 2, with the error
correlation being 0.8. }\tabularnewline

\centering\begingroup\fontsize{7}{9}\selectfont

\resizebox{\linewidth}{!}{
\begin{threeparttable}
\begin{tabular}{llrrrrrr>{}r}
\toprule
  & Top & A & B & AA & AB & BA & BB & Summary\\
\midrule
OLS-subset & 0.33 & 0.52 & 0.96 & 0.95 & 0.98 & 0.96 & 0.78 & \includegraphics[width=0.47in, height=0.1in]{/Users/xwan0362/Git/hfs/paper/_figs/corr_pos_OLS-subset.png}\\
OLS-intuitive & 0.54 & 0.77 & 0.93 & 0.89 & 0.97 & 0.83 & 0.85 & \includegraphics[width=0.47in, height=0.1in]{/Users/xwan0362/Git/hfs/paper/_figs/corr_pos_OLS-intuitive.png}\\
OLS-lasso & 0.69 & 0.53 & 0.60 & 1.00 & 1.00 & 1.00 & 1.00 & \includegraphics[width=0.47in, height=0.1in]{/Users/xwan0362/Git/hfs/paper/_figs/corr_pos_OLS-lasso.png}\\
\midrule
WLSs-subset & 0.29 & 0.60 & 1.00 & 1.00 & 1.00 & 0.98 & 0.86 & \includegraphics[width=0.47in, height=0.1in]{/Users/xwan0362/Git/hfs/paper/_figs/corr_pos_WLSs-subset.png}\\
WLSs-intuitive & 0.63 & 0.67 & 0.99 & 0.98 & 1.00 & 0.93 & 0.86 & \includegraphics[width=0.47in, height=0.1in]{/Users/xwan0362/Git/hfs/paper/_figs/corr_pos_WLSs-intuitive.png}\\
WLSs-lasso & 0.69 & 0.76 & 0.91 & 1.00 & 1.00 & 1.00 & 1.00 & \includegraphics[width=0.47in, height=0.1in]{/Users/xwan0362/Git/hfs/paper/_figs/corr_pos_WLSs-lasso.png}\\
\midrule
WLSv-subset & 0.32 & 0.55 & 1.00 & 1.00 & 1.00 & 0.99 & 0.76 & \includegraphics[width=0.47in, height=0.1in]{/Users/xwan0362/Git/hfs/paper/_figs/corr_pos_WLSv-subset.png}\\
WLSv-intuitive & 0.58 & 0.56 & 1.00 & 1.00 & 0.98 & 1.00 & 0.75 & \includegraphics[width=0.47in, height=0.1in]{/Users/xwan0362/Git/hfs/paper/_figs/corr_pos_WLSv-intuitive.png}\\
WLSv-lasso & 0.77 & 0.84 & 0.99 & 1.00 & 1.00 & 1.00 & 1.00 & \includegraphics[width=0.47in, height=0.1in]{/Users/xwan0362/Git/hfs/paper/_figs/corr_pos_WLSv-lasso.png}\\
\midrule
MinT-subset & 1.00 & 1.00 & 1.00 & 1.00 & 1.00 & 1.00 & 1.00 & \includegraphics[width=0.47in, height=0.1in]{/Users/xwan0362/Git/hfs/paper/_figs/corr_pos_MinT-subset.png}\\
MinT-intuitive & 1.00 & 1.00 & 1.00 & 1.00 & 1.00 & 1.00 & 1.00 & \includegraphics[width=0.47in, height=0.1in]{/Users/xwan0362/Git/hfs/paper/_figs/corr_pos_MinT-intuitive.png}\\
MinT-lasso & 1.00 & 1.00 & 1.00 & 1.00 & 1.00 & 1.00 & 1.00 & \includegraphics[width=0.47in, height=0.1in]{/Users/xwan0362/Git/hfs/paper/_figs/corr_pos_MinT-lasso.png}\\
\midrule
MinTs-subset & 1.00 & 1.00 & 1.00 & 1.00 & 1.00 & 1.00 & 1.00 & \includegraphics[width=0.47in, height=0.1in]{/Users/xwan0362/Git/hfs/paper/_figs/corr_pos_MinTs-subset.png}\\
MinTs-intuitive & 1.00 & 1.00 & 1.00 & 1.00 & 1.00 & 1.00 & 1.00 & \includegraphics[width=0.47in, height=0.1in]{/Users/xwan0362/Git/hfs/paper/_figs/corr_pos_MinTs-intuitive.png}\\
MinTs-lasso & 1.00 & 1.00 & 1.00 & 1.00 & 1.00 & 1.00 & 1.00 & \includegraphics[width=0.47in, height=0.1in]{/Users/xwan0362/Git/hfs/paper/_figs/corr_pos_MinTs-lasso.png}\\
\midrule
Elasso & 0.73 & 0.65 & 0.98 & 0.98 & 0.86 & 1.00 & 0.99 & \includegraphics[width=0.47in, height=0.1in]{/Users/xwan0362/Git/hfs/paper/_figs/corr_pos_Elasso.png}\\
\bottomrule
\end{tabular}
\begin{tablenotes}[para]
\item \underline{\textit{NOTE:}} 
\item The last column displays a stacked barplot for each method, based on the total number of selected series data from 500 simulation instances, with a darker sub-bar indicating a larger number.
\end{tablenotes}
\end{threeparttable}}
\endgroup{}
\end{table}


\end{document}
